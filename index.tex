% Options for packages loaded elsewhere
% Options for packages loaded elsewhere
\PassOptionsToPackage{unicode}{hyperref}
\PassOptionsToPackage{hyphens}{url}
\PassOptionsToPackage{dvipsnames,svgnames,x11names}{xcolor}
%
\documentclass[
  11pt,
  letterpaper,
  oneside,
  openany]{scrbook}
\usepackage{xcolor}
\usepackage[margin=1in,heightrounded]{geometry}
\usepackage{amsmath,amssymb}
\setcounter{secnumdepth}{3}
\usepackage{iftex}
\ifPDFTeX
  \usepackage[T1]{fontenc}
  \usepackage[utf8]{inputenc}
  \usepackage{textcomp} % provide euro and other symbols
\else % if luatex or xetex
  \usepackage{unicode-math} % this also loads fontspec
  \defaultfontfeatures{Scale=MatchLowercase}
  \defaultfontfeatures[\rmfamily]{Ligatures=TeX,Scale=1}
\fi
\usepackage[]{libertinus}
\ifPDFTeX\else
  % xetex/luatex font selection
\fi
% Use upquote if available, for straight quotes in verbatim environments
\IfFileExists{upquote.sty}{\usepackage{upquote}}{}
\IfFileExists{microtype.sty}{% use microtype if available
  \usepackage[]{microtype}
  \UseMicrotypeSet[protrusion]{basicmath} % disable protrusion for tt fonts
}{}
\usepackage{setspace}
\makeatletter
\@ifundefined{KOMAClassName}{% if non-KOMA class
  \IfFileExists{parskip.sty}{%
    \usepackage{parskip}
  }{% else
    \setlength{\parindent}{0pt}
    \setlength{\parskip}{6pt plus 2pt minus 1pt}}
}{% if KOMA class
  \KOMAoptions{parskip=half}}
\makeatother
% Make \paragraph and \subparagraph free-standing
\makeatletter
\ifx\paragraph\undefined\else
  \let\oldparagraph\paragraph
  \renewcommand{\paragraph}{
    \@ifstar
      \xxxParagraphStar
      \xxxParagraphNoStar
  }
  \newcommand{\xxxParagraphStar}[1]{\oldparagraph*{#1}\mbox{}}
  \newcommand{\xxxParagraphNoStar}[1]{\oldparagraph{#1}\mbox{}}
\fi
\ifx\subparagraph\undefined\else
  \let\oldsubparagraph\subparagraph
  \renewcommand{\subparagraph}{
    \@ifstar
      \xxxSubParagraphStar
      \xxxSubParagraphNoStar
  }
  \newcommand{\xxxSubParagraphStar}[1]{\oldsubparagraph*{#1}\mbox{}}
  \newcommand{\xxxSubParagraphNoStar}[1]{\oldsubparagraph{#1}\mbox{}}
\fi
\makeatother

\usepackage{color}
\usepackage{fancyvrb}
\newcommand{\VerbBar}{|}
\newcommand{\VERB}{\Verb[commandchars=\\\{\}]}
\DefineVerbatimEnvironment{Highlighting}{Verbatim}{commandchars=\\\{\}}
% Add ',fontsize=\small' for more characters per line
\newenvironment{Shaded}{}{}
\newcommand{\AlertTok}[1]{\textcolor[rgb]{1.00,0.33,0.33}{\textbf{#1}}}
\newcommand{\AnnotationTok}[1]{\textcolor[rgb]{0.42,0.45,0.49}{#1}}
\newcommand{\AttributeTok}[1]{\textcolor[rgb]{0.84,0.23,0.29}{#1}}
\newcommand{\BaseNTok}[1]{\textcolor[rgb]{0.00,0.36,0.77}{#1}}
\newcommand{\BuiltInTok}[1]{\textcolor[rgb]{0.84,0.23,0.29}{#1}}
\newcommand{\CharTok}[1]{\textcolor[rgb]{0.01,0.18,0.38}{#1}}
\newcommand{\CommentTok}[1]{\textcolor[rgb]{0.42,0.45,0.49}{#1}}
\newcommand{\CommentVarTok}[1]{\textcolor[rgb]{0.42,0.45,0.49}{#1}}
\newcommand{\ConstantTok}[1]{\textcolor[rgb]{0.00,0.36,0.77}{#1}}
\newcommand{\ControlFlowTok}[1]{\textcolor[rgb]{0.84,0.23,0.29}{#1}}
\newcommand{\DataTypeTok}[1]{\textcolor[rgb]{0.84,0.23,0.29}{#1}}
\newcommand{\DecValTok}[1]{\textcolor[rgb]{0.00,0.36,0.77}{#1}}
\newcommand{\DocumentationTok}[1]{\textcolor[rgb]{0.42,0.45,0.49}{#1}}
\newcommand{\ErrorTok}[1]{\textcolor[rgb]{1.00,0.33,0.33}{\underline{#1}}}
\newcommand{\ExtensionTok}[1]{\textcolor[rgb]{0.84,0.23,0.29}{\textbf{#1}}}
\newcommand{\FloatTok}[1]{\textcolor[rgb]{0.00,0.36,0.77}{#1}}
\newcommand{\FunctionTok}[1]{\textcolor[rgb]{0.44,0.26,0.76}{#1}}
\newcommand{\ImportTok}[1]{\textcolor[rgb]{0.01,0.18,0.38}{#1}}
\newcommand{\InformationTok}[1]{\textcolor[rgb]{0.42,0.45,0.49}{#1}}
\newcommand{\KeywordTok}[1]{\textcolor[rgb]{0.84,0.23,0.29}{#1}}
\newcommand{\NormalTok}[1]{\textcolor[rgb]{0.14,0.16,0.18}{#1}}
\newcommand{\OperatorTok}[1]{\textcolor[rgb]{0.14,0.16,0.18}{#1}}
\newcommand{\OtherTok}[1]{\textcolor[rgb]{0.44,0.26,0.76}{#1}}
\newcommand{\PreprocessorTok}[1]{\textcolor[rgb]{0.84,0.23,0.29}{#1}}
\newcommand{\RegionMarkerTok}[1]{\textcolor[rgb]{0.42,0.45,0.49}{#1}}
\newcommand{\SpecialCharTok}[1]{\textcolor[rgb]{0.00,0.36,0.77}{#1}}
\newcommand{\SpecialStringTok}[1]{\textcolor[rgb]{0.01,0.18,0.38}{#1}}
\newcommand{\StringTok}[1]{\textcolor[rgb]{0.01,0.18,0.38}{#1}}
\newcommand{\VariableTok}[1]{\textcolor[rgb]{0.89,0.38,0.04}{#1}}
\newcommand{\VerbatimStringTok}[1]{\textcolor[rgb]{0.01,0.18,0.38}{#1}}
\newcommand{\WarningTok}[1]{\textcolor[rgb]{1.00,0.33,0.33}{#1}}

\usepackage{longtable,booktabs,array}
\usepackage{calc} % for calculating minipage widths
% Correct order of tables after \paragraph or \subparagraph
\usepackage{etoolbox}
\makeatletter
\patchcmd\longtable{\par}{\if@noskipsec\mbox{}\fi\par}{}{}
\makeatother
% Allow footnotes in longtable head/foot
\IfFileExists{footnotehyper.sty}{\usepackage{footnotehyper}}{\usepackage{footnote}}
\makesavenoteenv{longtable}
\usepackage{graphicx}
\makeatletter
\newsavebox\pandoc@box
\newcommand*\pandocbounded[1]{% scales image to fit in text height/width
  \sbox\pandoc@box{#1}%
  \Gscale@div\@tempa{\textheight}{\dimexpr\ht\pandoc@box+\dp\pandoc@box\relax}%
  \Gscale@div\@tempb{\linewidth}{\wd\pandoc@box}%
  \ifdim\@tempb\p@<\@tempa\p@\let\@tempa\@tempb\fi% select the smaller of both
  \ifdim\@tempa\p@<\p@\scalebox{\@tempa}{\usebox\pandoc@box}%
  \else\usebox{\pandoc@box}%
  \fi%
}
% Set default figure placement to htbp
\def\fps@figure{htbp}
\makeatother





\setlength{\emergencystretch}{3em} % prevent overfull lines

\providecommand{\tightlist}{%
  \setlength{\itemsep}{0pt}\setlength{\parskip}{0pt}}



 


\makeatletter
\@ifpackageloaded{tcolorbox}{}{\usepackage[skins,breakable]{tcolorbox}}
\@ifpackageloaded{fontawesome5}{}{\usepackage{fontawesome5}}
\definecolor{quarto-callout-color}{HTML}{909090}
\definecolor{quarto-callout-note-color}{HTML}{0758E5}
\definecolor{quarto-callout-important-color}{HTML}{CC1914}
\definecolor{quarto-callout-warning-color}{HTML}{EB9113}
\definecolor{quarto-callout-tip-color}{HTML}{00A047}
\definecolor{quarto-callout-caution-color}{HTML}{FC5300}
\definecolor{quarto-callout-color-frame}{HTML}{acacac}
\definecolor{quarto-callout-note-color-frame}{HTML}{4582ec}
\definecolor{quarto-callout-important-color-frame}{HTML}{d9534f}
\definecolor{quarto-callout-warning-color-frame}{HTML}{f0ad4e}
\definecolor{quarto-callout-tip-color-frame}{HTML}{02b875}
\definecolor{quarto-callout-caution-color-frame}{HTML}{fd7e14}
\makeatother
\makeatletter
\@ifpackageloaded{bookmark}{}{\usepackage{bookmark}}
\makeatother
\makeatletter
\@ifpackageloaded{caption}{}{\usepackage{caption}}
\AtBeginDocument{%
\ifdefined\contentsname
  \renewcommand*\contentsname{Table of contents}
\else
  \newcommand\contentsname{Table of contents}
\fi
\ifdefined\listfigurename
  \renewcommand*\listfigurename{List of Figures}
\else
  \newcommand\listfigurename{List of Figures}
\fi
\ifdefined\listtablename
  \renewcommand*\listtablename{List of Tables}
\else
  \newcommand\listtablename{List of Tables}
\fi
\ifdefined\figurename
  \renewcommand*\figurename{Figure}
\else
  \newcommand\figurename{Figure}
\fi
\ifdefined\tablename
  \renewcommand*\tablename{Table}
\else
  \newcommand\tablename{Table}
\fi
}
\@ifpackageloaded{float}{}{\usepackage{float}}
\floatstyle{ruled}
\@ifundefined{c@chapter}{\newfloat{codelisting}{h}{lop}}{\newfloat{codelisting}{h}{lop}[chapter]}
\floatname{codelisting}{Listing}
\newcommand*\listoflistings{\listof{codelisting}{List of Listings}}
\makeatother
\makeatletter
\makeatother
\makeatletter
\@ifpackageloaded{caption}{}{\usepackage{caption}}
\@ifpackageloaded{subcaption}{}{\usepackage{subcaption}}
\makeatother
\makeatletter
\@ifpackageloaded{tcolorbox}{}{\usepackage[skins,breakable]{tcolorbox}}
\makeatother
\makeatletter
\@ifundefined{shadecolor}{\definecolor{shadecolor}{HTML}{31BAE9}}{}
\makeatother
\makeatletter
\makeatother
\makeatletter
\ifdefined\Shaded\renewenvironment{Shaded}{\begin{tcolorbox}[borderline west={3pt}{0pt}{shadecolor}, frame hidden, boxrule=0pt, enhanced, interior hidden, breakable, sharp corners]}{\end{tcolorbox}}\fi
\makeatother
\usepackage{bookmark}
\IfFileExists{xurl.sty}{\usepackage{xurl}}{} % add URL line breaks if available
\urlstyle{same}
\hypersetup{
  pdftitle={Guía Práctica de Herramientas CLI con Homebrew},
  pdfauthor={Luis Aguilera},
  colorlinks=true,
  linkcolor={black},
  filecolor={Maroon},
  citecolor={gray},
  urlcolor={blue},
  pdfcreator={LaTeX via pandoc}}


\title{Guía Práctica de Herramientas CLI con Homebrew}
\usepackage{etoolbox}
\makeatletter
\providecommand{\subtitle}[1]{% add subtitle to \maketitle
  \apptocmd{\@title}{\par {\large #1 \par}}{}{}
}
\makeatother
\subtitle{Manual completo de herramientas de línea de comandos}
\author{Luis Aguilera}
\date{30 September 2025}
\begin{document}
\frontmatter
\maketitle

\renewcommand*\contentsname{Table of contents}
{
\hypersetup{linkcolor=}
\setcounter{tocdepth}{2}
\tableofcontents
}

\setstretch{1.2}
\mainmatter
\bookmarksetup{startatroot}

\chapter*{Guía Práctica de Herramientas CLI con
Homebrew}\label{guuxeda-pruxe1ctica-de-herramientas-cli-con-homebrew}
\addcontentsline{toc}{chapter}{Guía Práctica de Herramientas CLI con
Homebrew}

\markboth{Guía Práctica de Herramientas CLI con Homebrew}{Guía Práctica
de Herramientas CLI con Homebrew}

Bienvenido a la guía completa de herramientas de línea de comandos
instaladas con Homebrew.

\section*{Acerca de esta guía}\label{acerca-de-esta-guuxeda}
\addcontentsline{toc}{section}{Acerca de esta guía}

\markright{Acerca de esta guía}

Esta guía documenta herramientas de línea de comandos (CLI) instaladas
mediante Homebrew, con:

\begin{itemize}
\tightlist
\item
  \textbf{Descripciones claras} de cada herramienta
\item
  \textbf{Ejemplos prácticos} listos para usar
\item
  \textbf{Casos de uso reales} del día a día
\item
  \textbf{Combinaciones útiles} entre herramientas
\item
  \textbf{Funcionalidad interactiva} para experimentar
\end{itemize}

\section*{Características
interactivas}\label{caracteruxedsticas-interactivas}
\addcontentsline{toc}{section}{Características interactivas}

\markright{Características interactivas}

\begin{tcolorbox}[enhanced jigsaw, toprule=.15mm, bottomrule=.15mm, opacityback=0, coltitle=black, rightrule=.15mm, colframe=quarto-callout-note-color-frame, titlerule=0mm, opacitybacktitle=0.6, left=2mm, colback=white, bottomtitle=1mm, arc=.35mm, leftrule=.75mm, title=\textcolor{quarto-callout-note-color}{\faInfo}\hspace{0.5em}{Note}, colbacktitle=quarto-callout-note-color!10!white, breakable, toptitle=1mm]

Esta guía incluye funcionalidades interactivas que te permiten:

\begin{itemize}
\tightlist
\item
  Editar código directamente en el navegador
\item
  Probar comandos de forma simulada
\item
  Guardar tus snippets favoritos
\item
  Conectar con API local para funciones avanzadas
\end{itemize}

\end{tcolorbox}

\section*{Contenido de la guía}\label{contenido-de-la-guuxeda}
\addcontentsline{toc}{section}{Contenido de la guía}

\markright{Contenido de la guía}

Esta guía está organizada en secciones temáticas:

\subsection*{Navegación y
Exploración}\label{navegaciuxf3n-y-exploraciuxf3n}
\addcontentsline{toc}{subsection}{Navegación y Exploración}

Herramientas para moverte y explorar tu sistema de archivos de manera
eficiente.

\begin{itemize}
\tightlist
\item
  \textbf{\texttt{eza}} - Listado de archivos moderno y colorido
\item
  \textbf{\texttt{tree}} - Visualización de estructura de directorios
\item
  \textbf{\texttt{ranger}} - Navegador de archivos interactivo\\
\item
  \textbf{\texttt{zoxide}} - Navegación inteligente con memoria
\end{itemize}

\subsection*{Gestión de Archivos}\label{gestiuxf3n-de-archivos}
\addcontentsline{toc}{subsection}{Gestión de Archivos}

Manipulación, organización y sincronización de archivos y directorios.

\begin{itemize}
\tightlist
\item
  \textbf{\texttt{rename}} / \textbf{\texttt{renameutils}} - Renombrado
  masivo de archivos
\item
  \textbf{\texttt{rsync}} - Sincronización y copia avanzada
\item
  \textbf{\texttt{mmv}} - Movimiento múltiple de archivos
\end{itemize}

\subsection*{Búsqueda y Filtrado}\label{buxfasqueda-y-filtrado}
\addcontentsline{toc}{subsection}{Búsqueda y Filtrado}

Herramientas para encontrar información rápidamente en archivos y datos.

\begin{itemize}
\tightlist
\item
  \textbf{\texttt{ripgrep}} (rg) - Búsqueda ultrarrápida en texto
\item
  \textbf{\texttt{fzf}} - Filtro difuso interactivo
\item
  \textbf{\texttt{jq}} - Procesador y consultor JSON
\end{itemize}

\subsection*{Desarrollo y Git}\label{desarrollo-y-git}
\addcontentsline{toc}{subsection}{Desarrollo y Git}

Herramientas esenciales para desarrollo de software y control de
versiones.

\begin{itemize}
\tightlist
\item
  \textbf{\texttt{git}} - Control de versiones distribuido
\item
  \textbf{\texttt{gh}} - CLI oficial de GitHub\\
\item
  \textbf{\texttt{node}} - Runtime de JavaScript
\end{itemize}

\subsection*{Multimedia}\label{multimedia}
\addcontentsline{toc}{subsection}{Multimedia}

Procesamiento de archivos de video, audio e imágenes desde la línea de
comandos.

\begin{itemize}
\tightlist
\item
  \textbf{\texttt{ffmpeg}} - Suite completa de procesamiento multimedia
\item
  \textbf{\texttt{yt-dlp}} - Descargador de videos de múltiples sitios\\
\item
  \textbf{\texttt{imagemagick}} - Manipulación avanzada de imágenes
\end{itemize}

\subsection*{Red y Descargas}\label{red-y-descargas}
\addcontentsline{toc}{subsection}{Red y Descargas}

Herramientas para transferencia de datos, APIs y descargas de internet.

\begin{itemize}
\tightlist
\item
  \textbf{\texttt{curl}} - Cliente HTTP versátil
\item
  \textbf{\texttt{wget}} - Descargador web no interactivo
\item
  \textbf{\texttt{aria2}} - Descargador multihilo avanzado
\item
  \textbf{\texttt{httpie}} - Cliente HTTP amigable
\end{itemize}

\subsection*{Monitoreo del Sistema}\label{monitoreo-del-sistema}
\addcontentsline{toc}{subsection}{Monitoreo del Sistema}

Supervisión del rendimiento y estado del sistema.

\begin{itemize}
\tightlist
\item
  \textbf{\texttt{htop}} - Monitor de procesos interactivo
\item
  \textbf{\texttt{fastfetch}} - Información del sistema con estilo
  (reemplaza neofetch)
\item
  \textbf{\texttt{btop}} - Monitor moderno con visualizaciones avanzadas
\end{itemize}

\subsection*{Texto y Documentos}\label{texto-y-documentos}
\addcontentsline{toc}{subsection}{Texto y Documentos}

Herramientas para procesamiento, edición y análisis de texto.

\begin{itemize}
\tightlist
\item
  \textbf{\texttt{bat}} - Visualizador de archivos con sintaxis
  highlighting
\item
  \textbf{\texttt{pandoc}} - Conversor universal de documentos
\item
  \textbf{\texttt{vale}} - Linter de escritura y estilo
\end{itemize}

\subsection*{Utilidades Diversas}\label{utilidades-diversas}
\addcontentsline{toc}{subsection}{Utilidades Diversas}

Herramientas útiles que no encajan en otras categorías.

\begin{itemize}
\tightlist
\item
  \textbf{\texttt{thefuck}} - Corrector de comandos inteligente
\item
  \textbf{\texttt{tealdeer}} - Páginas de manual simplificadas
  (reemplaza tldr)
\item
  \textbf{\texttt{navi}} - Hoja de comandos interactiva
\end{itemize}

\section*{Funcionalidades Especiales}\label{funcionalidades-especiales}
\addcontentsline{toc}{section}{Funcionalidades Especiales}

\markright{Funcionalidades Especiales}

\subsection*{Playground Interactivo}\label{playground-interactivo}
\addcontentsline{toc}{subsection}{Playground Interactivo}

Experimenta con código editable, ejecución simulada y gestión de
snippets en el \href{interactive-playground.qmd}{capítulo interactivo}.

\subsection*{Integración del
Ecosistema}\label{integraciuxf3n-del-ecosistema}
\addcontentsline{toc}{subsection}{Integración del Ecosistema}

Descubre las diferentes formas de acceder a esta guía en la
\href{ecosystem-integration.qmd}{sección de integración}.

\subsection*{Combinaciones Avanzadas}\label{combinaciones-avanzadas}
\addcontentsline{toc}{subsection}{Combinaciones Avanzadas}

Aprende a combinar herramientas para crear flujos de trabajo poderosos
en \href{combinaciones.qmd}{combinaciones}.

\subsection*{Configuración y
Mantenimiento}\label{configuraciuxf3n-y-mantenimiento}
\addcontentsline{toc}{subsection}{Configuración y Mantenimiento}

Optimiza tu instalación de Homebrew en la
\href{configuracion.qmd}{sección de configuración}.

\section*{Formatos Disponibles}\label{formatos-disponibles}
\addcontentsline{toc}{section}{Formatos Disponibles}

\markright{Formatos Disponibles}

Esta guía está disponible en múltiples formatos:

\begin{itemize}
\tightlist
\item
  \textbf{HTML Interactivo} - Navegación completa con funcionalidades
  interactivas
\item
  \textbf{PDF} - Formato optimizado para impresión y lectura offline
\item
  \textbf{EPUB} - Compatible con lectores de eBooks
\item
  \textbf{DOCX} - Editable en Microsoft Word
\end{itemize}

\section*{Comenzar}\label{comenzar}
\addcontentsline{toc}{section}{Comenzar}

\markright{Comenzar}

Para empezar a usar la guía:

\begin{enumerate}
\def\labelenumi{\arabic{enumi}.}
\tightlist
\item
  \textbf{Navega por las secciones} - Encuentra herramientas para tus
  necesidades específicas\\
\item
  \textbf{Experimenta con el código} - Usa las funciones interactivas
  para aprender
\item
  \textbf{Instala herramientas} - Prueba los comandos en tu sistema
\item
  \textbf{Contribuye} - Añade tus herramientas favoritas y mejoras
\end{enumerate}

\begin{center}\rule{0.5\linewidth}{0.5pt}\end{center}

\textbf{Última actualización:} \texttt{r\ Sys.Date()}\\
\textbf{Versión:} 2.0.0\\
\textbf{GitHub:}
\href{https://github.com/laguileracl/homebrew-cli-guide}{laguileracl/homebrew-cli-guide}

\subsection*{\texorpdfstring{📝 \href{texto.qmd}{Texto y
Documentos}}{📝 Texto y Documentos}}\label{texto-y-documentos-1}
\addcontentsline{toc}{subsection}{📝 \href{texto.qmd}{Texto y
Documentos}}

Herramientas para visualización, edición y conversión de documentos.

\begin{itemize}
\tightlist
\item
  \textbf{\texttt{bat}} - Visualizador de código con sintaxis
\item
  \textbf{\texttt{pandoc}} - Conversor universal de documentos
\item
  \textbf{\texttt{glow}} - Renderizador de Markdown
\end{itemize}

\subsection*{\texorpdfstring{🛠️ \href{utilidades.qmd}{Utilidades
Diversas}}{🛠️ Utilidades Diversas}}\label{utilidades-diversas-1}
\addcontentsline{toc}{subsection}{🛠️ \href{utilidades.qmd}{Utilidades
Diversas}}

Herramientas que mejoran la experiencia general en la terminal.

\begin{itemize}
\tightlist
\item
  \textbf{\texttt{tealdeer}} - Páginas de manual simplificadas
  (reemplaza tldr)
\item
  \textbf{\texttt{thefuck}} - Corrector automático de comandos
\item
  \textbf{\texttt{cowsay}} - Arte ASCII divertido
\item
  \textbf{\texttt{direnv}} - Variables de entorno por directorio
\item
  \textbf{\texttt{starship}} - Prompt personalizable
\end{itemize}

\section*{Recursos adicionales}\label{recursos-adicionales}
\addcontentsline{toc}{section}{Recursos adicionales}

\markright{Recursos adicionales}

\begin{itemize}
\tightlist
\item
  \textbf{\href{combinaciones.qmd}{Combinaciones Útiles}} - Workflows
  avanzados combinando múltiples herramientas
\item
  \textbf{\href{configuracion.qmd}{Configuración}} - Tips de
  configuración y personalización
\end{itemize}

\section*{Cómo usar este libro}\label{cuxf3mo-usar-este-libro}
\addcontentsline{toc}{section}{Cómo usar este libro}

\markright{Cómo usar este libro}

\subsection*{Para principiantes}\label{para-principiantes}
\addcontentsline{toc}{subsection}{Para principiantes}

\begin{itemize}
\tightlist
\item
  Comienza con la sección de \textbf{Navegación y Exploración}
\item
  Experimenta con cada comando en un directorio de prueba
\item
  Lee los ejemplos paso a paso
\end{itemize}

\subsection*{Para usuarios intermedios}\label{para-usuarios-intermedios}
\addcontentsline{toc}{subsection}{Para usuarios intermedios}

\begin{itemize}
\tightlist
\item
  Ve directamente a las secciones que te interesen
\item
  Prueba las \textbf{combinaciones avanzadas}
\item
  Adapta los ejemplos a tus casos de uso
\end{itemize}

\subsection*{Para usuarios avanzados}\label{para-usuarios-avanzados}
\addcontentsline{toc}{subsection}{Para usuarios avanzados}

\begin{itemize}
\tightlist
\item
  Usa el libro como \textbf{referencia rápida}
\item
  Explora las secciones de \textbf{configuración} y
  \textbf{combinaciones}
\item
  Contribuye con tus propios workflows
\end{itemize}

\section*{Convenciones}\label{convenciones}
\addcontentsline{toc}{section}{Convenciones}

\markright{Convenciones}

\begin{tcolorbox}[enhanced jigsaw, toprule=.15mm, bottomrule=.15mm, opacityback=0, coltitle=black, rightrule=.15mm, colframe=quarto-callout-note-color-frame, titlerule=0mm, opacitybacktitle=0.6, left=2mm, colback=white, bottomtitle=1mm, arc=.35mm, leftrule=.75mm, title=\textcolor{quarto-callout-note-color}{\faInfo}\hspace{0.5em}{Nota}, colbacktitle=quarto-callout-note-color!10!white, breakable, toptitle=1mm]

Las notas contienen información adicional útil.

\end{tcolorbox}

\begin{tcolorbox}[enhanced jigsaw, toprule=.15mm, bottomrule=.15mm, opacityback=0, coltitle=black, rightrule=.15mm, colframe=quarto-callout-tip-color-frame, titlerule=0mm, opacitybacktitle=0.6, left=2mm, colback=white, bottomtitle=1mm, arc=.35mm, leftrule=.75mm, title=\textcolor{quarto-callout-tip-color}{\faLightbulb}\hspace{0.5em}{Tip}, colbacktitle=quarto-callout-tip-color!10!white, breakable, toptitle=1mm]

Los tips incluyen trucos y atajos para ser más productivo.

\end{tcolorbox}

\begin{tcolorbox}[enhanced jigsaw, toprule=.15mm, bottomrule=.15mm, opacityback=0, coltitle=black, rightrule=.15mm, colframe=quarto-callout-warning-color-frame, titlerule=0mm, opacitybacktitle=0.6, left=2mm, colback=white, bottomtitle=1mm, arc=.35mm, leftrule=.75mm, title=\textcolor{quarto-callout-warning-color}{\faExclamationTriangle}\hspace{0.5em}{Advertencia}, colbacktitle=quarto-callout-warning-color!10!white, breakable, toptitle=1mm]

Las advertencias indican comandos que pueden ser destructivos o requerir
precaución.

\end{tcolorbox}

\begin{tcolorbox}[enhanced jigsaw, toprule=.15mm, bottomrule=.15mm, opacityback=0, coltitle=black, rightrule=.15mm, colframe=quarto-callout-important-color-frame, titlerule=0mm, opacitybacktitle=0.6, left=2mm, colback=white, bottomtitle=1mm, arc=.35mm, leftrule=.75mm, title=\textcolor{quarto-callout-important-color}{\faExclamation}\hspace{0.5em}{Importante}, colbacktitle=quarto-callout-important-color!10!white, breakable, toptitle=1mm]

Información crítica que debes recordar.

\end{tcolorbox}

\subsection*{Ejemplos de código}\label{ejemplos-de-cuxf3digo}
\addcontentsline{toc}{subsection}{Ejemplos de código}

\begin{Shaded}
\begin{Highlighting}[]
\CommentTok{\# Los comentarios explican qué hace cada comando}
\ExtensionTok{comando} \AttributeTok{{-}{-}opcion}\NormalTok{ archivo.txt}
\end{Highlighting}
\end{Shaded}

\begin{Shaded}
\begin{Highlighting}[]
\CommentTok{\# Ejemplo de salida esperada}
\ExtensionTok{$}\NormalTok{ ls }\AttributeTok{{-}la}
\ExtensionTok{total}\NormalTok{ 8}
\ExtensionTok{drwxr{-}xr{-}x}\NormalTok{  3 usuario  staff   96 Aug  6 16:00 .}
\ExtensionTok{drwxr{-}xr{-}x}\NormalTok{  4 usuario  staff  128 Aug  6 15:59 ..}
\ExtensionTok{{-}rw{-}r{-}{-}r{-}{-}}\NormalTok{  1 usuario  staff  151 Aug  6 16:00 archivo.txt}
\end{Highlighting}
\end{Shaded}

¡Comencemos a explorar el poder de las herramientas CLI! 🚀

\part{Primeros Pasos}

\chapter{🚀 Playground Interactivo}\label{playground-interactivo-1}

\begin{tcolorbox}[enhanced jigsaw, toprule=.15mm, bottomrule=.15mm, opacityback=0, coltitle=black, rightrule=.15mm, colframe=quarto-callout-tip-color-frame, titlerule=0mm, opacitybacktitle=0.6, left=2mm, colback=white, bottomtitle=1mm, arc=.35mm, leftrule=.75mm, title=\textcolor{quarto-callout-tip-color}{\faLightbulb}\hspace{0.5em}{¡Bienvenido al Playground Interactivo!}, colbacktitle=quarto-callout-tip-color!10!white, breakable, toptitle=1mm]

Esta sección te permite \textbf{experimentar directamente} con las
herramientas CLI sin salir del navegador. Aquí puedes:

\begin{itemize}
\tightlist
\item
  ✏️ \textbf{Editar código} en tiempo real
\item
  ▶️ \textbf{Ejecutar comandos} de forma simulada\\
\item
  💾 \textbf{Guardar snippets} para uso posterior
\item
  🔗 \textbf{Conectar con API local} para funcionalidades avanzadas
\end{itemize}

\end{tcolorbox}

\section{Funcionalidades
Interactivas}\label{funcionalidades-interactivas}

\subsection{🎯 Toolbar Flotante}\label{toolbar-flotante}

En la parte derecha de la pantalla encontrarás un \textbf{toolbar
flotante} con estas opciones:

\begin{longtable}[]{@{}
  >{\raggedright\arraybackslash}p{(\linewidth - 4\tabcolsep) * \real{0.3043}}
  >{\raggedright\arraybackslash}p{(\linewidth - 4\tabcolsep) * \real{0.3913}}
  >{\raggedright\arraybackslash}p{(\linewidth - 4\tabcolsep) * \real{0.3043}}@{}}
\toprule\noalign{}
\begin{minipage}[b]{\linewidth}\raggedright
Botón
\end{minipage} & \begin{minipage}[b]{\linewidth}\raggedright
Función
\end{minipage} & \begin{minipage}[b]{\linewidth}\raggedright
Atajo
\end{minipage} \\
\midrule\noalign{}
\endhead
\bottomrule\noalign{}
\endlastfoot
& Editar código en línea & \texttt{Ctrl/Cmd\ +\ E} \\
& Ejecutar código simulado & \texttt{Ctrl/Cmd\ +\ R} \\
& Guardar snippet & \texttt{Ctrl/Cmd\ +\ S} \\
& Probar con API local & - \\
\end{longtable}

\subsection{📝 Edición de Código}\label{ediciuxf3n-de-cuxf3digo}

Cada bloque de código en esta guía es \textbf{completamente editable}.
Simplemente:

\begin{enumerate}
\def\labelenumi{\arabic{enumi}.}
\tightlist
\item
  Haz clic en el botón de cualquier bloque de código
\item
  Modifica el código según tus necesidades
\item
  Haz clic en ``Aplicar'' para ver los cambios
\end{enumerate}

\textbf{Ejemplo interactivo:}

\begin{Shaded}
\begin{Highlighting}[]
\CommentTok{\# Lista archivos con eza (modificable)}
\ExtensionTok{eza} \AttributeTok{{-}la} \AttributeTok{{-}{-}icons} \AttributeTok{{-}{-}git}
\end{Highlighting}
\end{Shaded}

\subsection{⚡ Ejecución Simulada}\label{ejecuciuxf3n-simulada}

Puedes ejecutar comandos de forma simulada para ver su estructura y
salida esperada:

\begin{Shaded}
\begin{Highlighting}[]
\CommentTok{\# Buscar archivos con ripgrep}
\ExtensionTok{rg} \StringTok{"función"} \AttributeTok{{-}{-}type}\NormalTok{ py}
\end{Highlighting}
\end{Shaded}

\subsection{💾 Gestión de Snippets}\label{gestiuxf3n-de-snippets}

Guarda tus comandos favoritos y configuraciones personalizadas:

\begin{Shaded}
\begin{Highlighting}[]
\CommentTok{\# Tu configuración personalizada de git}
\FunctionTok{git}\NormalTok{ config }\AttributeTok{{-}{-}global}\NormalTok{ user.name }\StringTok{"Tu Nombre"}
\FunctionTok{git}\NormalTok{ config }\AttributeTok{{-}{-}global}\NormalTok{ user.email }\StringTok{"tu.email@ejemplo.com"}
\end{Highlighting}
\end{Shaded}

\section{🔗 Integración con el
Ecosistema}\label{integraciuxf3n-con-el-ecosistema}

\subsection{API Local}\label{api-local}

Si tienes el servidor API ejecutándose (\texttt{npm\ run\ start} en
\texttt{api-server/}), podrás:

\begin{itemize}
\tightlist
\item
  🔍 Buscar herramientas en tiempo real
\item
  📊 Ver estadísticas de uso
\item
  🎯 Obtener recomendaciones personalizadas
\end{itemize}

\textbf{Iniciar API:}

\begin{Shaded}
\begin{Highlighting}[]
\BuiltInTok{cd}\NormalTok{ api{-}server}
\ExtensionTok{npm}\NormalTok{ install}
\ExtensionTok{npm}\NormalTok{ start}
\end{Highlighting}
\end{Shaded}

\subsection{CLI Offline}\label{cli-offline}

Usa la herramienta CLI local para consultas rápidas:

\begin{Shaded}
\begin{Highlighting}[]
\CommentTok{\# Buscar herramientas de desarrollo}
\ExtensionTok{./scripts/cli{-}guide}\NormalTok{ search desarrollo}

\CommentTok{\# Modo interactivo con fzf}
\ExtensionTok{./scripts/cli{-}guide}\NormalTok{ interactive}
\end{Highlighting}
\end{Shaded}

\subsection{Dashboard Web}\label{dashboard-web}

Accede al dashboard interactivo:

\begin{Shaded}
\begin{Highlighting}[]
\CommentTok{\# Abrir dashboard en el navegador}
\ExtensionTok{open}\NormalTok{ tools{-}explorer.html}
\end{Highlighting}
\end{Shaded}

\section{🎨 Ejemplos Interactivos
Avanzados}\label{ejemplos-interactivos-avanzados}

\subsection{Combinaciones Poderosas}\label{combinaciones-poderosas}

\begin{Shaded}
\begin{Highlighting}[]
\CommentTok{\# Pipeline complejo: buscar, filtrar y procesar}
\ExtensionTok{rg} \StringTok{"export"} \AttributeTok{{-}{-}type}\NormalTok{ js }\KeywordTok{|} \ExtensionTok{fzf} \AttributeTok{{-}{-}preview} \StringTok{\textquotesingle{}head {-}20 \{\}\textquotesingle{}} \KeywordTok{|} \FunctionTok{xargs} \AttributeTok{{-}I}\NormalTok{ \{\} head }\AttributeTok{{-}5}\NormalTok{ \{\}}
\end{Highlighting}
\end{Shaded}

\subsection{Configuración de
Desarrollo}\label{configuraciuxf3n-de-desarrollo}

\begin{Shaded}
\begin{Highlighting}[]
\CommentTok{\# Setup completo de entorno de desarrollo}
\ExtensionTok{brew}\NormalTok{ install node python@3.11 git gh}
\ExtensionTok{gh}\NormalTok{ auth login}
\ExtensionTok{npm}\NormalTok{ config set prefix \textasciitilde{}/.npm{-}global}
\end{Highlighting}
\end{Shaded}

\subsection{Análisis de Sistema}\label{anuxe1lisis-de-sistema}

\begin{Shaded}
\begin{Highlighting}[]
\CommentTok{\# Monitoreo completo del sistema}
\ExtensionTok{btop} \KeywordTok{\&} 
\FunctionTok{du} \AttributeTok{{-}h}\NormalTok{ \textasciitilde{} }\KeywordTok{|} \FunctionTok{head} \AttributeTok{{-}20}
\FunctionTok{df} \AttributeTok{{-}h}
\end{Highlighting}
\end{Shaded}

\section{🎯 Consejos de Uso}\label{consejos-de-uso}

\begin{tcolorbox}[enhanced jigsaw, toprule=.15mm, bottomrule=.15mm, opacityback=0, coltitle=black, rightrule=.15mm, colframe=quarto-callout-note-color-frame, titlerule=0mm, opacitybacktitle=0.6, left=2mm, colback=white, bottomtitle=1mm, arc=.35mm, leftrule=.75mm, title=\textcolor{quarto-callout-note-color}{\faInfo}\hspace{0.5em}{Mejores Prácticas}, colbacktitle=quarto-callout-note-color!10!white, breakable, toptitle=1mm]

\begin{enumerate}
\def\labelenumi{\arabic{enumi}.}
\tightlist
\item
  \textbf{Experimenta libremente} - todos los cambios son temporales
\item
  \textbf{Guarda snippets útiles} - para reutilizar configuraciones
\item
  \textbf{Usa atajos de teclado} - para mayor productividad
\item
  \textbf{Conecta la API} - para funcionalidades completas
\end{enumerate}

\end{tcolorbox}

\subsection{Personalización}\label{personalizaciuxf3n}

Puedes personalizar la experiencia:

\begin{itemize}
\tightlist
\item
  \textbf{Temas}: Cambia entre modo claro y oscuro
\item
  \textbf{Fuentes}: Optimizadas para programación
\item
  \textbf{Atajos}: Personaliza según tus preferencias
\end{itemize}

\subsection{Resolución de Problemas}\label{resoluciuxf3n-de-problemas}

Si algo no funciona:

\begin{enumerate}
\def\labelenumi{\arabic{enumi}.}
\tightlist
\item
  Recarga la página
\item
  Verifica que JavaScript esté habilitado
\item
  Comprueba la consola del navegador
\item
  Asegúrate de que la API local esté ejecutándose
\end{enumerate}

\section{🚀 Próximos Pasos}\label{pruxf3ximos-pasos}

Una vez que domines el playground:

\begin{enumerate}
\def\labelenumi{\arabic{enumi}.}
\tightlist
\item
  \textbf{Explora las secciones temáticas} del libro
\item
  \textbf{Prueba combinaciones avanzadas} de herramientas
\item
  \textbf{Contribuye} con tus propios ejemplos
\item
  \textbf{Comparte} configuraciones útiles con la comunidad
\end{enumerate}

\begin{center}\rule{0.5\linewidth}{0.5pt}\end{center}

\begin{tcolorbox}[enhanced jigsaw, toprule=.15mm, bottomrule=.15mm, opacityback=0, coltitle=black, rightrule=.15mm, colframe=quarto-callout-important-color-frame, titlerule=0mm, opacitybacktitle=0.6, left=2mm, colback=white, bottomtitle=1mm, arc=.35mm, leftrule=.75mm, title=\textcolor{quarto-callout-important-color}{\faExclamation}\hspace{0.5em}{¡Importante!}, colbacktitle=quarto-callout-important-color!10!white, breakable, toptitle=1mm]

Este playground funciona mejor con:

\begin{itemize}
\tightlist
\item
  \textbf{Navegadores modernos} (Chrome, Firefox, Safari)
\item
  \textbf{JavaScript habilitado}
\item
  \textbf{API local ejecutándose} (opcional pero recomendado)
\end{itemize}

\end{tcolorbox}

¡Disfruta explorando y experimentando con las herramientas CLI! 🎉

\part{Navegación y Exploración}

\chapter{Navegación y
Exploración}\label{navegaciuxf3n-y-exploraciuxf3n-2}

La navegación eficiente por el sistema de archivos es fundamental para
cualquier usuario de terminal. Esta sección cubre las herramientas más
poderosas para moverte, explorar y entender la estructura de tus
directorios de forma productiva.

\section{eza - Listado moderno de archivos}\label{sec-eza}

\texttt{eza} es un reemplazo moderno del comando \texttt{ls}
tradicional, ofreciendo colores, iconos y mejor formato por defecto.

\subsection{Instalación y
configuración}\label{instalaciuxf3n-y-configuraciuxf3n}

\begin{Shaded}
\begin{Highlighting}[]
\CommentTok{\# Ya está instalado con Homebrew}
\FunctionTok{which}\NormalTok{ eza}
\end{Highlighting}
\end{Shaded}

\subsection{Uso básico}\label{uso-buxe1sico}

\begin{Shaded}
\begin{Highlighting}[]
\CommentTok{\# Listado básico con iconos y colores}
\ExtensionTok{eza}
\end{Highlighting}
\end{Shaded}

\begin{tcolorbox}[enhanced jigsaw, toprule=.15mm, bottomrule=.15mm, opacityback=0, coltitle=black, rightrule=.15mm, colframe=quarto-callout-tip-color-frame, titlerule=0mm, opacitybacktitle=0.6, left=2mm, colback=white, bottomtitle=1mm, arc=.35mm, leftrule=.75mm, title=\textcolor{quarto-callout-tip-color}{\faLightbulb}\hspace{0.5em}{Tip}, colbacktitle=quarto-callout-tip-color!10!white, breakable, toptitle=1mm]

\texttt{eza} muestra colores e iconos por defecto, making it much more
readable than traditional \texttt{ls}.

\end{tcolorbox}

\subsection{Ejemplos prácticos}\label{ejemplos-pruxe1cticos}

\subsubsection{Listado detallado}\label{listado-detallado}

\begin{Shaded}
\begin{Highlighting}[]
\CommentTok{\# Equivalente a \textquotesingle{}ls {-}la\textquotesingle{} pero mucho más visual}
\ExtensionTok{eza} \AttributeTok{{-}la}

\CommentTok{\# Con fechas en formato legible}
\ExtensionTok{eza} \AttributeTok{{-}la} \AttributeTok{{-}{-}time{-}style}\OperatorTok{=}\NormalTok{long{-}iso}

\CommentTok{\# Mostrar tamaños en formato legible (KB, MB, GB)}
\ExtensionTok{eza} \AttributeTok{{-}lah}
\end{Highlighting}
\end{Shaded}

\subsubsection{Vista en árbol}\label{vista-en-uxe1rbol}

\begin{Shaded}
\begin{Highlighting}[]
\CommentTok{\# Vista de árbol del directorio actual}
\ExtensionTok{eza} \AttributeTok{{-}{-}tree}

\CommentTok{\# Limitar profundidad a 2 niveles}
\ExtensionTok{eza} \AttributeTok{{-}{-}tree} \AttributeTok{{-}{-}level}\OperatorTok{=}\NormalTok{2}

\CommentTok{\# Árbol con información detallada}
\ExtensionTok{eza} \AttributeTok{{-}{-}tree} \AttributeTok{{-}la}

\CommentTok{\# Ignorar directorios específicos}
\ExtensionTok{eza} \AttributeTok{{-}{-}tree} \AttributeTok{{-}{-}ignore{-}glob}\OperatorTok{=}\StringTok{"node\_modules|.git"}
\end{Highlighting}
\end{Shaded}

\subsubsection{Filtrado y ordenamiento}\label{filtrado-y-ordenamiento}

\begin{Shaded}
\begin{Highlighting}[]
\CommentTok{\# Mostrar solo directorios}
\ExtensionTok{eza} \AttributeTok{{-}D}

\CommentTok{\# Mostrar solo archivos}
\ExtensionTok{eza} \AttributeTok{{-}f}

\CommentTok{\# Ordenar por tamaño (mayor a menor)}
\ExtensionTok{eza} \AttributeTok{{-}la} \AttributeTok{{-}{-}sort}\OperatorTok{=}\NormalTok{size }\AttributeTok{{-}{-}reverse}

\CommentTok{\# Ordenar por fecha de modificación}
\ExtensionTok{eza} \AttributeTok{{-}la} \AttributeTok{{-}{-}sort}\OperatorTok{=}\NormalTok{modified}

\CommentTok{\# Ordenar por extensión}
\ExtensionTok{eza} \AttributeTok{{-}la} \AttributeTok{{-}{-}sort}\OperatorTok{=}\NormalTok{extension}
\end{Highlighting}
\end{Shaded}

\subsubsection{Información de Git}\label{informaciuxf3n-de-git}

\begin{Shaded}
\begin{Highlighting}[]
\CommentTok{\# Mostrar estado de Git para cada archivo}
\ExtensionTok{eza} \AttributeTok{{-}la} \AttributeTok{{-}{-}git}

\CommentTok{\# Solo archivos con cambios en Git}
\ExtensionTok{eza} \AttributeTok{{-}la} \AttributeTok{{-}{-}git} \KeywordTok{|} \FunctionTok{grep} \AttributeTok{{-}E} \StringTok{\textquotesingle{}\textbackslash{}s[MAD?]\textbackslash{}s\textquotesingle{}}
\end{Highlighting}
\end{Shaded}

\subsection{Casos de uso avanzados}\label{casos-de-uso-avanzados}

\subsubsection{Análisis de proyectos}\label{anuxe1lisis-de-proyectos}

\begin{Shaded}
\begin{Highlighting}[]
\CommentTok{\# Ver estructura completa de un proyecto con Git info}
\ExtensionTok{eza} \AttributeTok{{-}{-}tree} \AttributeTok{{-}{-}git} \AttributeTok{{-}{-}ignore{-}glob}\OperatorTok{=}\StringTok{"node\_modules|.git|dist|build"}

\CommentTok{\# Ver archivos grandes en el proyecto}
\ExtensionTok{eza} \AttributeTok{{-}la} \AttributeTok{{-}{-}sort}\OperatorTok{=}\NormalTok{size }\AttributeTok{{-}{-}reverse} \KeywordTok{|} \FunctionTok{head} \AttributeTok{{-}20}
\end{Highlighting}
\end{Shaded}

\subsubsection{Alias útiles}\label{alias-uxfatiles}

Agrega estos a tu \texttt{.zshrc} o \texttt{.bashrc}:

\begin{Shaded}
\begin{Highlighting}[]
\BuiltInTok{alias}\NormalTok{ ll=}\StringTok{\textquotesingle{}eza {-}la {-}{-}git\textquotesingle{}}
\BuiltInTok{alias}\NormalTok{ lt=}\StringTok{\textquotesingle{}eza {-}{-}tree\textquotesingle{}}
\BuiltInTok{alias}\NormalTok{ llt=}\StringTok{\textquotesingle{}eza {-}{-}tree {-}la\textquotesingle{}}
\BuiltInTok{alias}\NormalTok{ ls=}\StringTok{\textquotesingle{}eza\textquotesingle{}}
\end{Highlighting}
\end{Shaded}

\begin{center}\rule{0.5\linewidth}{0.5pt}\end{center}

\section{tree - Estructura visual de directorios}\label{sec-tree}

\texttt{tree} genera una representación visual en forma de árbol de la
estructura de directorios.

\subsection{Uso básico}\label{uso-buxe1sico-1}

\begin{Shaded}
\begin{Highlighting}[]
\CommentTok{\# Árbol básico del directorio actual}
\ExtensionTok{tree}

\CommentTok{\# Limitar a 2 niveles de profundidad}
\ExtensionTok{tree} \AttributeTok{{-}L}\NormalTok{ 2}

\CommentTok{\# Mostrar archivos ocultos}
\ExtensionTok{tree} \AttributeTok{{-}a}
\end{Highlighting}
\end{Shaded}

\subsection{Ejemplos prácticos}\label{ejemplos-pruxe1cticos-1}

\subsubsection{Control de profundidad y
filtros}\label{control-de-profundidad-y-filtros}

\begin{Shaded}
\begin{Highlighting}[]
\CommentTok{\# Solo directorios (sin archivos)}
\ExtensionTok{tree} \AttributeTok{{-}d}

\CommentTok{\# Límite de 3 niveles, solo directorios}
\ExtensionTok{tree} \AttributeTok{{-}d} \AttributeTok{{-}L}\NormalTok{ 3}

\CommentTok{\# Ignorar patrones específicos}
\ExtensionTok{tree} \AttributeTok{{-}I} \StringTok{\textquotesingle{}node\_modules|.git|*.pyc\textquotesingle{}}

\CommentTok{\# Múltiples patrones de exclusión}
\ExtensionTok{tree} \AttributeTok{{-}I} \StringTok{\textquotesingle{}node\_modules|.git|dist|build|coverage\textquotesingle{}}
\end{Highlighting}
\end{Shaded}

\subsubsection{Información adicional}\label{informaciuxf3n-adicional}

\begin{Shaded}
\begin{Highlighting}[]
\CommentTok{\# Mostrar tamaños de archivos}
\ExtensionTok{tree} \AttributeTok{{-}s}

\CommentTok{\# Mostrar fechas de modificación}
\ExtensionTok{tree} \AttributeTok{{-}D}

\CommentTok{\# Mostrar permisos}
\ExtensionTok{tree} \AttributeTok{{-}p}

\CommentTok{\# Combinación: tamaños, fechas y permisos}
\ExtensionTok{tree} \AttributeTok{{-}sDp}
\end{Highlighting}
\end{Shaded}

\subsubsection{Salida formateada}\label{salida-formateada}

\begin{Shaded}
\begin{Highlighting}[]
\CommentTok{\# Solo los nombres de archivos (sin estructura de árbol)}
\ExtensionTok{tree} \AttributeTok{{-}i}

\CommentTok{\# Mostrar rutas completas}
\ExtensionTok{tree} \AttributeTok{{-}f}

\CommentTok{\# Colorear salida según tipo de archivo}
\ExtensionTok{tree} \AttributeTok{{-}C}
\end{Highlighting}
\end{Shaded}

\subsection{Casos de uso avanzados}\label{casos-de-uso-avanzados-1}

\subsubsection{Documentación de
proyectos}\label{documentaciuxf3n-de-proyectos}

\begin{Shaded}
\begin{Highlighting}[]
\CommentTok{\# Generar documentación HTML del proyecto}
\ExtensionTok{tree} \AttributeTok{{-}H}\NormalTok{ . }\AttributeTok{{-}o}\NormalTok{ project{-}structure.html}

\CommentTok{\# Con título personalizado}
\ExtensionTok{tree} \AttributeTok{{-}H}\NormalTok{ . }\AttributeTok{{-}T} \StringTok{"Mi Proyecto"} \AttributeTok{{-}o}\NormalTok{ structure.html}

\CommentTok{\# Incluir CSS personalizado}
\ExtensionTok{tree} \AttributeTok{{-}H}\NormalTok{ . }\AttributeTok{{-}C} \AttributeTok{{-}o}\NormalTok{ structure.html}
\end{Highlighting}
\end{Shaded}

\subsubsection{Análisis de
directorios}\label{anuxe1lisis-de-directorios}

\begin{Shaded}
\begin{Highlighting}[]
\CommentTok{\# Contar archivos por tipo}
\ExtensionTok{tree} \KeywordTok{|} \FunctionTok{grep} \AttributeTok{{-}E} \StringTok{\textquotesingle{}\textbackslash{}.(js|py|md)$\textquotesingle{}} \KeywordTok{|} \FunctionTok{wc} \AttributeTok{{-}l}

\CommentTok{\# Ver directorios más profundos}
\ExtensionTok{tree} \AttributeTok{{-}d} \KeywordTok{|} \FunctionTok{tail} \AttributeTok{{-}20}

\CommentTok{\# Encontrar directorios vacíos}
\ExtensionTok{tree} \AttributeTok{{-}d} \AttributeTok{{-}{-}prune}
\end{Highlighting}
\end{Shaded}

\subsubsection{Scripts de backup}\label{scripts-de-backup}

\begin{Shaded}
\begin{Highlighting}[]
\CommentTok{\# Crear lista de archivos para backup}
\ExtensionTok{tree} \AttributeTok{{-}if} \OperatorTok{\textgreater{}}\NormalTok{ backup{-}list.txt}

\CommentTok{\# Solo archivos modificados en últimos 7 días}
\FunctionTok{find}\NormalTok{ . }\AttributeTok{{-}mtime} \AttributeTok{{-}7} \KeywordTok{|} \ExtensionTok{tree} \AttributeTok{{-}{-}fromfile}
\end{Highlighting}
\end{Shaded}

\begin{center}\rule{0.5\linewidth}{0.5pt}\end{center}

\section{ranger - Navegador interactivo}\label{sec-ranger}

\texttt{ranger} es un navegador de archivos en terminal con interfaz
visual y vista previa.

\subsection{Iniciando ranger}\label{iniciando-ranger}

\begin{Shaded}
\begin{Highlighting}[]
\CommentTok{\# Abrir ranger en directorio actual}
\ExtensionTok{ranger}

\CommentTok{\# Abrir en directorio específico}
\ExtensionTok{ranger}\NormalTok{ \textasciitilde{}/Documents}

\CommentTok{\# Abrir y ejecutar comando al salir}
\ExtensionTok{ranger} \AttributeTok{{-}{-}cmd}\OperatorTok{=}\StringTok{"cd /tmp"}
\end{Highlighting}
\end{Shaded}

\subsection{Navegación básica}\label{navegaciuxf3n-buxe1sica}

\begin{longtable}[]{@{}ll@{}}
\toprule\noalign{}
Tecla & Acción \\
\midrule\noalign{}
\endhead
\bottomrule\noalign{}
\endlastfoot
\texttt{h,j,k,l} & Navegar (izq, abajo, arriba, der) \\
\texttt{Enter} & Entrar a directorio / abrir archivo \\
\texttt{Backspace} & Directorio padre \\
\texttt{gg} & Ir al inicio \\
\texttt{G} & Ir al final \\
\texttt{H} & Historial hacia atrás \\
\texttt{L} & Historial hacia adelante \\
\end{longtable}

\subsection{Comandos esenciales}\label{comandos-esenciales}

\subsubsection{Gestión de archivos}\label{gestiuxf3n-de-archivos-1}

\begin{Shaded}
\begin{Highlighting}[]
\CommentTok{\# Dentro de ranger:}
\ExtensionTok{yy}    \CommentTok{\# Copiar archivo/directorio}
\FunctionTok{dd}    \CommentTok{\# Cortar archivo/directorio}
\ExtensionTok{pp}    \CommentTok{\# Pegar}
\ExtensionTok{dD}    \CommentTok{\# Eliminar permanentemente}
\ExtensionTok{:delete} \CommentTok{\# Eliminar con confirmación}
\end{Highlighting}
\end{Shaded}

\subsubsection{Búsqueda y filtros}\label{buxfasqueda-y-filtros}

\begin{Shaded}
\begin{Highlighting}[]
\CommentTok{\# Buscar archivos}
\ExtensionTok{/nombre\_archivo}

\CommentTok{\# Búsqueda hacia atrás}
\ExtensionTok{?nombre}

\CommentTok{\# Filtrar archivos mostrados}
\ExtensionTok{zf}

\CommentTok{\# Mostrar archivos ocultos}
\ExtensionTok{zh}

\CommentTok{\# Mostrar solo directorios}
\ExtensionTok{zd}
\end{Highlighting}
\end{Shaded}

\subsubsection{Vista previa y
configuración}\label{vista-previa-y-configuraciuxf3n}

\begin{Shaded}
\begin{Highlighting}[]
\CommentTok{\# Alternar vista previa}
\ExtensionTok{zv}

\CommentTok{\# Cambiar modo de vista}
\ExtensionTok{zP}

\CommentTok{\# Información del archivo}
\ExtensionTok{=}
\end{Highlighting}
\end{Shaded}

\subsection{Personalización}\label{personalizaciuxf3n-1}

\subsubsection{Archivo de
configuración}\label{archivo-de-configuraciuxf3n}

Crea \texttt{\textasciitilde{}/.config/ranger/rc.conf}:

\begin{Shaded}
\begin{Highlighting}[]
\CommentTok{\# Mostrar archivos ocultos por defecto}
\BuiltInTok{set}\NormalTok{ show\_hidden true}

\CommentTok{\# Vista previa activada}
\BuiltInTok{set}\NormalTok{ preview\_files true}

\CommentTok{\# Usar colores}
\BuiltInTok{set}\NormalTok{ colorscheme default}

\CommentTok{\# Integración con git}
\BuiltInTok{set}\NormalTok{ vcs\_aware true}
\end{Highlighting}
\end{Shaded}

\subsubsection{Comandos personalizados}\label{comandos-personalizados}

\begin{Shaded}
\begin{Highlighting}[]
\CommentTok{\# En rc.conf, agregar comandos personalizados:}
\ExtensionTok{map} \OperatorTok{\textless{}}\NormalTok{C{-}t}\OperatorTok{\textgreater{}}\NormalTok{ shell tmux new{-}window}
\ExtensionTok{map} \OperatorTok{\textless{}}\NormalTok{C{-}e}\OperatorTok{\textgreater{}}\NormalTok{ shell code \%s}
\ExtensionTok{map} \OperatorTok{\textless{}}\NormalTok{C{-}g}\OperatorTok{\textgreater{}}\NormalTok{ shell git status}
\end{Highlighting}
\end{Shaded}

\subsection{Casos de uso avanzados}\label{casos-de-uso-avanzados-2}

\subsubsection{Integración con otros
programas}\label{integraciuxf3n-con-otros-programas}

\begin{Shaded}
\begin{Highlighting}[]
\CommentTok{\# Abrir archivo con aplicación específica}
\ExtensionTok{r}    \CommentTok{\# Menú de aplicaciones}
\ExtensionTok{1}    \CommentTok{\# Aplicación por defecto}
\ExtensionTok{2}    \CommentTok{\# Editor de texto}
\ExtensionTok{3}    \CommentTok{\# Visor de imágenes}
\end{Highlighting}
\end{Shaded}

\subsubsection{Marcadores y tabs}\label{marcadores-y-tabs}

\begin{Shaded}
\begin{Highlighting}[]
\CommentTok{\# Crear marcador}
\ExtensionTok{m}\NormalTok{ + letra}

\CommentTok{\# Ir a marcador}
\StringTok{\textquotesingle{} + letra  (comilla simple)}

\StringTok{\# Nuevo tab}
\StringTok{Ctrl+n}

\StringTok{\# Cambiar tab}
\StringTok{Tab / Shift+Tab}

\StringTok{\# Cerrar tab}
\StringTok{Ctrl+w}
\end{Highlighting}
\end{Shaded}

\begin{center}\rule{0.5\linewidth}{0.5pt}\end{center}

\section{zoxide - Navegación inteligente}\label{sec-zoxide}

\texttt{zoxide} es un reemplazo inteligente para \texttt{cd} que
recuerda los directorios que visitas frecuentemente.

\subsection{Configuración inicial}\label{configuraciuxf3n-inicial}

\begin{Shaded}
\begin{Highlighting}[]
\CommentTok{\# Agregar a tu .zshrc}
\BuiltInTok{eval} \StringTok{"}\VariableTok{$(}\ExtensionTok{zoxide}\NormalTok{ init zsh}\VariableTok{)}\StringTok{"}

\CommentTok{\# Para bash}
\BuiltInTok{eval} \StringTok{"}\VariableTok{$(}\ExtensionTok{zoxide}\NormalTok{ init bash}\VariableTok{)}\StringTok{"}

\CommentTok{\# Reiniciar shell o ejecutar}
\BuiltInTok{source}\NormalTok{ \textasciitilde{}/.zshrc}
\end{Highlighting}
\end{Shaded}

\subsection{Uso básico}\label{uso-buxe1sico-2}

\begin{Shaded}
\begin{Highlighting}[]
\CommentTok{\# Después de la configuración, \textquotesingle{}z\textquotesingle{} reemplaza a \textquotesingle{}cd\textquotesingle{}}
\ExtensionTok{z}\NormalTok{ Documents}
\ExtensionTok{z}\NormalTok{ proj    }\CommentTok{\# Si has visitado \textasciitilde{}/Projects antes}
\ExtensionTok{z}\NormalTok{ down    }\CommentTok{\# Para \textasciitilde{}/Downloads}
\end{Highlighting}
\end{Shaded}

\subsection{Ejemplos prácticos}\label{ejemplos-pruxe1cticos-2}

\subsubsection{Navegación rápida}\label{navegaciuxf3n-ruxe1pida}

\begin{Shaded}
\begin{Highlighting}[]
\CommentTok{\# Ir a directorio que contiene "react" en el nombre}
\ExtensionTok{z}\NormalTok{ react}

\CommentTok{\# Si hay múltiples, zoxide elegirá el más frecuente/reciente}
\ExtensionTok{z}\NormalTok{ proj}

\CommentTok{\# Ver todas las opciones disponibles}
\ExtensionTok{zi}   \CommentTok{\# Modo interactivo con fzf}
\end{Highlighting}
\end{Shaded}

\subsubsection{Gestión de la base de
datos}\label{gestiuxf3n-de-la-base-de-datos}

\begin{Shaded}
\begin{Highlighting}[]
\CommentTok{\# Ver directorios rastreados con puntuaciones}
\ExtensionTok{zoxide}\NormalTok{ query }\AttributeTok{{-}{-}list}

\CommentTok{\# Ver estadísticas detalladas}
\ExtensionTok{zoxide}\NormalTok{ query }\AttributeTok{{-}{-}stats}

\CommentTok{\# Agregar directorio manualmente}
\ExtensionTok{zoxide}\NormalTok{ add /ruta/especial}

\CommentTok{\# Eliminar directorio de la base de datos}
\ExtensionTok{zoxide}\NormalTok{ remove /ruta/vieja}
\end{Highlighting}
\end{Shaded}

\subsection{Casos de uso avanzados}\label{casos-de-uso-avanzados-3}

\subsubsection{Integración con
scripts}\label{integraciuxf3n-con-scripts}

\begin{Shaded}
\begin{Highlighting}[]
\CommentTok{\#!/bin/bash}
\CommentTok{\# Script para navegar a proyecto y activar entorno}

\VariableTok{PROJECT}\OperatorTok{=}\VariableTok{$(}\ExtensionTok{zoxide}\NormalTok{ query }\StringTok{"}\VariableTok{$1}\StringTok{"}\VariableTok{)}
\ControlFlowTok{if} \BuiltInTok{[} \OtherTok{{-}n} \StringTok{"}\VariableTok{$PROJECT}\StringTok{"} \BuiltInTok{]}\KeywordTok{;} \ControlFlowTok{then}
    \BuiltInTok{cd} \StringTok{"}\VariableTok{$PROJECT}\StringTok{"}
    \ControlFlowTok{if} \BuiltInTok{[} \OtherTok{{-}f} \StringTok{".venv/bin/activate"} \BuiltInTok{]}\KeywordTok{;} \ControlFlowTok{then}
        \BuiltInTok{source}\NormalTok{ .venv/bin/activate}
    \ControlFlowTok{fi}
\ControlFlowTok{fi}
\end{Highlighting}
\end{Shaded}

\subsubsection{Aliases personalizados}\label{aliases-personalizados}

\begin{Shaded}
\begin{Highlighting}[]
\CommentTok{\# En .zshrc}
\BuiltInTok{alias}\NormalTok{ zz=}\StringTok{\textquotesingle{}zi\textquotesingle{}}  \CommentTok{\# Navegación interactiva}
\BuiltInTok{alias}\NormalTok{ zl=}\StringTok{\textquotesingle{}zoxide query {-}{-}list\textquotesingle{}}  \CommentTok{\# Listar directorios}
\BuiltInTok{alias}\NormalTok{ zs=}\StringTok{\textquotesingle{}zoxide query {-}{-}stats\textquotesingle{}}  \CommentTok{\# Estadísticas}
\end{Highlighting}
\end{Shaded}

\begin{center}\rule{0.5\linewidth}{0.5pt}\end{center}

\section{Combinaciones poderosas}\label{combinaciones-poderosas-1}

\subsection{Exploración completa de
proyectos}\label{exploraciuxf3n-completa-de-proyectos}

\begin{Shaded}
\begin{Highlighting}[]
\CommentTok{\# 1. Navegar al proyecto}
\ExtensionTok{z}\NormalTok{ myproject}

\CommentTok{\# 2. Ver estructura general}
\ExtensionTok{eza} \AttributeTok{{-}{-}tree} \AttributeTok{{-}{-}level}\OperatorTok{=}\NormalTok{3 }\AttributeTok{{-}{-}ignore{-}glob}\OperatorTok{=}\StringTok{"node\_modules|.git"}

\CommentTok{\# 3. Ver archivos modificados recientemente}
\ExtensionTok{eza} \AttributeTok{{-}la} \AttributeTok{{-}{-}sort}\OperatorTok{=}\NormalTok{modified }\KeywordTok{|} \FunctionTok{head} \AttributeTok{{-}10}

\CommentTok{\# 4. Abrir navegador interactivo para explorar}
\ExtensionTok{ranger}
\end{Highlighting}
\end{Shaded}

\subsection{Análisis rápido de
directorios}\label{anuxe1lisis-ruxe1pido-de-directorios}

\begin{Shaded}
\begin{Highlighting}[]
\CommentTok{\# Pipeline para analizar un directorio desconocido}
\ExtensionTok{z}\NormalTok{ unknown{-}dir }\KeywordTok{\&\&} \DataTypeTok{\textbackslash{}}
\ExtensionTok{tree} \AttributeTok{{-}L}\NormalTok{ 2 }\KeywordTok{\&\&} \DataTypeTok{\textbackslash{}}
\BuiltInTok{echo} \StringTok{"{-}{-}{-} Archivos más grandes {-}{-}{-}"} \KeywordTok{\&\&} \DataTypeTok{\textbackslash{}}
\ExtensionTok{eza} \AttributeTok{{-}la} \AttributeTok{{-}{-}sort}\OperatorTok{=}\NormalTok{size }\AttributeTok{{-}{-}reverse} \KeywordTok{|} \FunctionTok{head} \AttributeTok{{-}5} \KeywordTok{\&\&} \DataTypeTok{\textbackslash{}}
\BuiltInTok{echo} \StringTok{"{-}{-}{-} Tipos de archivo {-}{-}{-}"} \KeywordTok{\&\&} \DataTypeTok{\textbackslash{}}
\FunctionTok{find}\NormalTok{ . }\AttributeTok{{-}type}\NormalTok{ f }\KeywordTok{|} \FunctionTok{sed} \StringTok{\textquotesingle{}s/.*\textbackslash{}.//\textquotesingle{}} \KeywordTok{|} \FunctionTok{sort} \KeywordTok{|} \FunctionTok{uniq} \AttributeTok{{-}c} \KeywordTok{|} \FunctionTok{sort} \AttributeTok{{-}nr}
\end{Highlighting}
\end{Shaded}

\subsection{Navegación con contexto}\label{navegaciuxf3n-con-contexto}

\begin{Shaded}
\begin{Highlighting}[]
\CommentTok{\# Función para navegar mostrando contexto}
\FunctionTok{nav()} \KeywordTok{\{}
    \ExtensionTok{z} \StringTok{"}\VariableTok{$1}\StringTok{"} \KeywordTok{\&\&} \BuiltInTok{pwd} \KeywordTok{\&\&} \ExtensionTok{eza} \AttributeTok{{-}la} \AttributeTok{{-}{-}git} \KeywordTok{|} \FunctionTok{head} \AttributeTok{{-}20}
\KeywordTok{\}}

\CommentTok{\# Uso: nav myproject}
\end{Highlighting}
\end{Shaded}

\begin{tcolorbox}[enhanced jigsaw, toprule=.15mm, bottomrule=.15mm, opacityback=0, coltitle=black, rightrule=.15mm, colframe=quarto-callout-important-color-frame, titlerule=0mm, opacitybacktitle=0.6, left=2mm, colback=white, bottomtitle=1mm, arc=.35mm, leftrule=.75mm, title=\textcolor{quarto-callout-important-color}{\faExclamation}\hspace{0.5em}{Importante para nuevos usuarios}, colbacktitle=quarto-callout-important-color!10!white, breakable, toptitle=1mm]

\begin{itemize}
\tightlist
\item
  Practica con \texttt{eza} antes de crear aliases que reemplacen
  \texttt{ls}
\item
  \texttt{zoxide} necesita tiempo para ``aprender'' tus patrones de
  navegación
\item
  \texttt{ranger} tiene una curva de aprendizaje, pero vale la pena la
  inversión de tiempo
\end{itemize}

\end{tcolorbox}

Estas herramientas de navegación forman la base de un workflow eficiente
en terminal. En el siguiente capítulo exploraremos herramientas para la
gestión avanzada de archivos.

\part{Gestión de Archivos}

\chapter{Gestión de Archivos}\label{gestiuxf3n-de-archivos-3}

La gestión eficiente de archivos va más allá de simples operaciones de
copia y movimiento. Esta sección cubre herramientas avanzadas para
renombrado masivo, sincronización inteligente y manipulación sofisticada
de archivos y directorios.

\section{rename / renameutils - Renombrado masivo}\label{sec-rename}

Las herramientas de renombrado masivo permiten modificar nombres de
múltiples archivos usando expresiones regulares y patrones.

\subsection{rename (Perl-based)}\label{rename-perl-based}

\begin{Shaded}
\begin{Highlighting}[]
\CommentTok{\# Verificar que tienes la versión correcta}
\ExtensionTok{rename} \AttributeTok{{-}{-}version}
\end{Highlighting}
\end{Shaded}

\subsection{Ejemplos básicos}\label{ejemplos-buxe1sicos}

\subsubsection{Cambios de extensión}\label{cambios-de-extensiuxf3n}

\begin{Shaded}
\begin{Highlighting}[]
\CommentTok{\# Cambiar .jpeg a .jpg}
\ExtensionTok{rename} \StringTok{\textquotesingle{}s/\textbackslash{}.jpeg$/\textbackslash{}.jpg/\textquotesingle{}} \PreprocessorTok{*}\NormalTok{.jpeg}

\CommentTok{\# Cambiar .txt a .md}
\ExtensionTok{rename} \StringTok{\textquotesingle{}s/\textbackslash{}.txt$/\textbackslash{}.md/\textquotesingle{}} \PreprocessorTok{*}\NormalTok{.txt}

\CommentTok{\# Agregar extensión faltante}
\ExtensionTok{rename} \StringTok{\textquotesingle{}s/$/.bak/\textquotesingle{}}\NormalTok{ archivo1 archivo2}
\end{Highlighting}
\end{Shaded}

\subsubsection{Modificaciones de texto}\label{modificaciones-de-texto}

\begin{Shaded}
\begin{Highlighting}[]
\CommentTok{\# Cambiar espacios por guiones bajos}
\ExtensionTok{rename} \StringTok{\textquotesingle{}s/ /\_/g\textquotesingle{}} \PreprocessorTok{*}\NormalTok{.txt}

\CommentTok{\# Cambiar guiones por puntos}
\ExtensionTok{rename} \StringTok{\textquotesingle{}s/{-}/./g\textquotesingle{}} \PreprocessorTok{*}\NormalTok{.md}

\CommentTok{\# Eliminar caracteres especiales}
\ExtensionTok{rename} \StringTok{\textquotesingle{}s/[\^{}a{-}zA{-}Z0{-}9.\_{-}]//g\textquotesingle{}} \PreprocessorTok{*}

\CommentTok{\# Cambiar a minúsculas}
\ExtensionTok{rename} \StringTok{\textquotesingle{}y/A{-}Z/a{-}z/\textquotesingle{}} \PreprocessorTok{*}\NormalTok{.TXT}
\end{Highlighting}
\end{Shaded}

\subsection{Ejemplos avanzados}\label{ejemplos-avanzados}

\subsubsection{Renombrado con fecha}\label{renombrado-con-fecha}

\begin{Shaded}
\begin{Highlighting}[]
\CommentTok{\# Agregar fecha actual al final}
\ExtensionTok{rename} \StringTok{\textquotesingle{}s/(.*)\textbackslash{}.(.*)/$1\_$(date +\%Y\%m\%d).$2/e\textquotesingle{}} \PreprocessorTok{*}\NormalTok{.log}

\CommentTok{\# Renombrar fotos con timestamp}
\ControlFlowTok{for}\NormalTok{ file }\KeywordTok{in} \PreprocessorTok{*}\NormalTok{.jpg}\KeywordTok{;} \ControlFlowTok{do}
    \VariableTok{timestamp}\OperatorTok{=}\VariableTok{$(}\FunctionTok{date} \AttributeTok{{-}r} \StringTok{"}\VariableTok{$file}\StringTok{"}\NormalTok{ +\%Y\%m\%d\_\%H\%M\%S}\VariableTok{)}
    \ExtensionTok{rename} \StringTok{"s/.*/}\VariableTok{$\{timestamp\}}\StringTok{.jpg/"} \StringTok{"}\VariableTok{$file}\StringTok{"}
\ControlFlowTok{done}
\end{Highlighting}
\end{Shaded}

\subsubsection{Renombrado condicional}\label{renombrado-condicional}

\begin{Shaded}
\begin{Highlighting}[]
\CommentTok{\# Solo archivos que contengan "old" en el nombre}
\ExtensionTok{rename} \StringTok{\textquotesingle{}s/old/new/g\textquotesingle{}} \PreprocessorTok{*}\NormalTok{old}\PreprocessorTok{*}

\CommentTok{\# Renombrar solo si el archivo es mayor a 1MB}
\FunctionTok{find}\NormalTok{ . }\AttributeTok{{-}size}\NormalTok{ +1M }\AttributeTok{{-}name} \StringTok{"*.log"} \AttributeTok{{-}exec}\NormalTok{ rename }\StringTok{\textquotesingle{}s/\textbackslash{}.log$/.big.log/\textquotesingle{}}\NormalTok{ \{\} }\DataTypeTok{\textbackslash{};}
\end{Highlighting}
\end{Shaded}

\subsection{mmv - Renombrado múltiple
visual}\label{mmv---renombrado-muxfaltiple-visual}

\texttt{mmv} proporciona una interfaz más visual para renombrado masivo.

\begin{Shaded}
\begin{Highlighting}[]
\CommentTok{\# Renombrar múltiples archivos con patrón}
\ExtensionTok{mmv} \StringTok{\textquotesingle{}*.jpeg\textquotesingle{}} \StringTok{\textquotesingle{}*.jpg\textquotesingle{}}

\CommentTok{\# Mover archivos con renombrado}
\ExtensionTok{mmv} \StringTok{\textquotesingle{}dir1/*.txt\textquotesingle{}} \StringTok{\textquotesingle{}dir2/\#1.backup\textquotesingle{}}

\CommentTok{\# Intercambiar extensiones}
\ExtensionTok{mmv} \StringTok{\textquotesingle{}*.\{txt,md\}\textquotesingle{}} \StringTok{\textquotesingle{}*.\{md,txt\}\textquotesingle{}}
\end{Highlighting}
\end{Shaded}

\subsection{Casos de uso prácticos}\label{casos-de-uso-pruxe1cticos}

\subsubsection{Organización de fotos}\label{organizaciuxf3n-de-fotos}

\begin{Shaded}
\begin{Highlighting}[]
\CommentTok{\# Renombrar fotos por fecha de creación}
\CommentTok{\#!/bin/bash}
\ControlFlowTok{for}\NormalTok{ img }\KeywordTok{in} \PreprocessorTok{*}\NormalTok{.jpg }\PreprocessorTok{*}\NormalTok{.png}\KeywordTok{;} \ControlFlowTok{do}
    \ControlFlowTok{if} \BuiltInTok{[} \OtherTok{{-}f} \StringTok{"}\VariableTok{$img}\StringTok{"} \BuiltInTok{]}\KeywordTok{;} \ControlFlowTok{then}
        \CommentTok{\# Obtener fecha de creación}
        \VariableTok{date\_taken}\OperatorTok{=}\VariableTok{$(}\ExtensionTok{exiftool} \AttributeTok{{-}d} \StringTok{"\%Y\%m\%d\_\%H\%M\%S"} \AttributeTok{{-}DateTimeOriginal} \AttributeTok{{-}s} \AttributeTok{{-}s} \AttributeTok{{-}s} \StringTok{"}\VariableTok{$img}\StringTok{"}\VariableTok{)}
        \ControlFlowTok{if} \BuiltInTok{[} \OtherTok{!} \OtherTok{{-}z} \StringTok{"}\VariableTok{$date\_taken}\StringTok{"} \BuiltInTok{]}\KeywordTok{;} \ControlFlowTok{then}
            \VariableTok{ext}\OperatorTok{=}\StringTok{"}\VariableTok{$\{img}\OperatorTok{\#\#}\PreprocessorTok{*}\NormalTok{.}\VariableTok{\}}\StringTok{"}
            \FunctionTok{mv} \StringTok{"}\VariableTok{$img}\StringTok{"} \StringTok{"photo\_}\VariableTok{$\{date\_taken\}}\StringTok{.}\VariableTok{$\{ext\}}\StringTok{"}
        \ControlFlowTok{fi}
    \ControlFlowTok{fi}
\ControlFlowTok{done}
\end{Highlighting}
\end{Shaded}

\subsubsection{Limpieza de downloads}\label{limpieza-de-downloads}

\begin{Shaded}
\begin{Highlighting}[]
\CommentTok{\# Normalizar nombres de archivos descargados}
\ExtensionTok{rename} \StringTok{\textquotesingle{}s/[\^{}a{-}zA{-}Z0{-}9.\_{-}]/\_/g\textquotesingle{}}\NormalTok{ \textasciitilde{}/Downloads/}\PreprocessorTok{*}
\ExtensionTok{rename} \StringTok{\textquotesingle{}s/\_\_+/\_/g\textquotesingle{}}\NormalTok{ \textasciitilde{}/Downloads/}\PreprocessorTok{*}  \CommentTok{\# Eliminar guiones bajos dobles}
\ExtensionTok{rename} \StringTok{\textquotesingle{}s/\^{}\_|\_$//g\textquotesingle{}}\NormalTok{ \textasciitilde{}/Downloads/}\PreprocessorTok{*}  \CommentTok{\# Eliminar guiones al inicio/final}
\end{Highlighting}
\end{Shaded}

\begin{center}\rule{0.5\linewidth}{0.5pt}\end{center}

\section{rsync - Sincronización avanzada}\label{sec-rsync}

\texttt{rsync} es mucho más que una herramienta de copia; es un sistema
completo de sincronización y backup.

\subsection{Sintaxis básica}\label{sintaxis-buxe1sica}

\begin{Shaded}
\begin{Highlighting}[]
\FunctionTok{rsync} \PreprocessorTok{[}\SpecialStringTok{opciones}\PreprocessorTok{]}\NormalTok{ origen destino}
\end{Highlighting}
\end{Shaded}

\subsection{Opciones esenciales}\label{opciones-esenciales}

\begin{longtable}[]{@{}ll@{}}
\toprule\noalign{}
Opción & Descripción \\
\midrule\noalign{}
\endhead
\bottomrule\noalign{}
\endlastfoot
\texttt{-a} & Modo archivo (preserva todo) \\
\texttt{-v} & Verboso \\
\texttt{-h} & Formato legible \\
\texttt{-z} & Comprimir durante transferencia \\
\texttt{-P} & Mostrar progreso y permitir resumir \\
\texttt{-\/-delete} & Eliminar archivos extra en destino \\
\texttt{-\/-dry-run} & Solo mostrar qué haría \\
\end{longtable}

\subsection{Ejemplos básicos}\label{ejemplos-buxe1sicos-1}

\subsubsection{Copia local}\label{copia-local}

\begin{Shaded}
\begin{Highlighting}[]
\CommentTok{\# Copia básica con progreso}
\FunctionTok{rsync} \AttributeTok{{-}avh} \AttributeTok{{-}{-}progress}\NormalTok{ origen/ destino/}

\CommentTok{\# Nota: la barra final en origen/ es importante}
\CommentTok{\# Con barra: copia CONTENIDO de origen a destino}
\CommentTok{\# Sin barra: copia DIRECTORIO origen dentro de destino}
\end{Highlighting}
\end{Shaded}

\subsubsection{Sincronización}\label{sincronizaciuxf3n}

\begin{Shaded}
\begin{Highlighting}[]
\CommentTok{\# Sincronizar eliminando archivos extra}
\FunctionTok{rsync} \AttributeTok{{-}avh} \AttributeTok{{-}{-}delete}\NormalTok{ origen/ destino/}

\CommentTok{\# Vista previa de sincronización}
\FunctionTok{rsync} \AttributeTok{{-}avh} \AttributeTok{{-}{-}delete} \AttributeTok{{-}{-}dry{-}run}\NormalTok{ origen/ destino/}

\CommentTok{\# Sincronizar solo archivos más nuevos}
\FunctionTok{rsync} \AttributeTok{{-}avh} \AttributeTok{{-}{-}update}\NormalTok{ origen/ destino/}
\end{Highlighting}
\end{Shaded}

\subsection{Ejemplos avanzados}\label{ejemplos-avanzados-1}

\subsubsection{Filtros y exclusiones}\label{filtros-y-exclusiones}

\begin{Shaded}
\begin{Highlighting}[]
\CommentTok{\# Excluir archivos específicos}
\FunctionTok{rsync} \AttributeTok{{-}avh} \AttributeTok{{-}{-}exclude}\OperatorTok{=}\StringTok{\textquotesingle{}*.log\textquotesingle{}} \AttributeTok{{-}{-}exclude}\OperatorTok{=}\StringTok{\textquotesingle{}temp/\textquotesingle{}}\NormalTok{ origen/ destino/}

\CommentTok{\# Usar archivo de exclusiones}
\BuiltInTok{echo} \StringTok{"*.log"} \OperatorTok{\textgreater{}}\NormalTok{ exclude.txt}
\BuiltInTok{echo} \StringTok{"node\_modules/"} \OperatorTok{\textgreater{}\textgreater{}}\NormalTok{ exclude.txt}
\FunctionTok{rsync} \AttributeTok{{-}avh} \AttributeTok{{-}{-}exclude{-}from}\OperatorTok{=}\NormalTok{exclude.txt origen/ destino/}

\CommentTok{\# Incluir solo tipos específicos}
\FunctionTok{rsync} \AttributeTok{{-}avh} \AttributeTok{{-}{-}include}\OperatorTok{=}\StringTok{\textquotesingle{}*.txt\textquotesingle{}} \AttributeTok{{-}{-}exclude}\OperatorTok{=}\StringTok{\textquotesingle{}*\textquotesingle{}}\NormalTok{ origen/ destino/}
\end{Highlighting}
\end{Shaded}

\subsubsection{Backup incremental}\label{backup-incremental}

\begin{Shaded}
\begin{Highlighting}[]
\CommentTok{\# Backup con hardlinks para ahorrar espacio}
\FunctionTok{rsync} \AttributeTok{{-}avh} \AttributeTok{{-}{-}link{-}dest}\OperatorTok{=}\NormalTok{../backup.previous origen/ backup.current/}

\CommentTok{\# Script de backup rotativo}
\CommentTok{\#!/bin/bash}
\VariableTok{BACKUP\_DIR}\OperatorTok{=}\StringTok{"/backups"}
\VariableTok{SOURCE}\OperatorTok{=}\StringTok{"/home/user"}
\VariableTok{DATE}\OperatorTok{=}\VariableTok{$(}\FunctionTok{date}\NormalTok{ +\%Y\%m\%d}\VariableTok{)}

\CommentTok{\# Crear backup con link a anterior}
\FunctionTok{rsync} \AttributeTok{{-}avh} \AttributeTok{{-}{-}delete} \AttributeTok{{-}{-}link{-}dest}\OperatorTok{=}\StringTok{"}\VariableTok{$BACKUP\_DIR}\StringTok{/latest"} \DataTypeTok{\textbackslash{}}
      \StringTok{"}\VariableTok{$SOURCE}\StringTok{/"} \StringTok{"}\VariableTok{$BACKUP\_DIR}\StringTok{/backup{-}}\VariableTok{$DATE}\StringTok{/"}

\CommentTok{\# Actualizar enlace \textquotesingle{}latest\textquotesingle{}}
\FunctionTok{rm} \AttributeTok{{-}f} \StringTok{"}\VariableTok{$BACKUP\_DIR}\StringTok{/latest"}
\FunctionTok{ln} \AttributeTok{{-}s} \StringTok{"backup{-}}\VariableTok{$DATE}\StringTok{"} \StringTok{"}\VariableTok{$BACKUP\_DIR}\StringTok{/latest"}
\end{Highlighting}
\end{Shaded}

\subsubsection{Transferencia remota}\label{transferencia-remota}

\begin{Shaded}
\begin{Highlighting}[]
\CommentTok{\# Subir archivos por SSH}
\FunctionTok{rsync} \AttributeTok{{-}avh} \AttributeTok{{-}e}\NormalTok{ ssh archivo.txt user@server:/ruta/destino/}

\CommentTok{\# Descargar desde servidor}
\FunctionTok{rsync} \AttributeTok{{-}avh}\NormalTok{ user@server:/ruta/origen/ ./local/}

\CommentTok{\# Con puerto SSH personalizado}
\FunctionTok{rsync} \AttributeTok{{-}avh} \AttributeTok{{-}e} \StringTok{"ssh {-}p 2222"}\NormalTok{ local/ user@server:/remoto/}

\CommentTok{\# Con compresión adicional}
\FunctionTok{rsync} \AttributeTok{{-}avhz}\NormalTok{ local/ user@server:/remoto/}
\end{Highlighting}
\end{Shaded}

\subsection{Casos de uso
especializados}\label{casos-de-uso-especializados}

\subsubsection{Sincronización de
desarrollo}\label{sincronizaciuxf3n-de-desarrollo}

\begin{Shaded}
\begin{Highlighting}[]
\CommentTok{\# Sincronizar código excluyendo archivos de build}
\FunctionTok{rsync} \AttributeTok{{-}avh} \AttributeTok{{-}{-}exclude}\OperatorTok{=}\StringTok{\textquotesingle{}.git/\textquotesingle{}} \AttributeTok{{-}{-}exclude}\OperatorTok{=}\StringTok{\textquotesingle{}node\_modules/\textquotesingle{}} \DataTypeTok{\textbackslash{}}
           \AttributeTok{{-}{-}exclude}\OperatorTok{=}\StringTok{\textquotesingle{}dist/\textquotesingle{}} \AttributeTok{{-}{-}exclude}\OperatorTok{=}\StringTok{\textquotesingle{}*.log\textquotesingle{}} \DataTypeTok{\textbackslash{}}
\NormalTok{           ./proyecto/ servidor:/var/www/proyecto/}
\end{Highlighting}
\end{Shaded}

\subsubsection{Backup de bases de datos}\label{backup-de-bases-de-datos}

\begin{Shaded}
\begin{Highlighting}[]
\CommentTok{\# Backup de directorio de PostgreSQL}
\FunctionTok{sudo}\NormalTok{ rsync }\AttributeTok{{-}avh} \AttributeTok{{-}{-}delete}\NormalTok{ /var/lib/postgresql/data/ /backup/postgres/}

\CommentTok{\# Con parada y reinicio de servicio}
\FunctionTok{sudo}\NormalTok{ systemctl stop postgresql}
\FunctionTok{sudo}\NormalTok{ rsync }\AttributeTok{{-}avh}\NormalTok{ /var/lib/postgresql/data/ /backup/postgres/}
\FunctionTok{sudo}\NormalTok{ systemctl start postgresql}
\end{Highlighting}
\end{Shaded}

\subsubsection{Monitoreo de cambios}\label{monitoreo-de-cambios}

\begin{Shaded}
\begin{Highlighting}[]
\CommentTok{\# Ver qué cambiaría sin ejecutar}
\FunctionTok{rsync} \AttributeTok{{-}avh} \AttributeTok{{-}{-}delete} \AttributeTok{{-}{-}dry{-}run}\NormalTok{ origen/ destino/ }\KeywordTok{|} \FunctionTok{tee}\NormalTok{ cambios.log}

\CommentTok{\# Solo mostrar archivos nuevos/modificados}
\FunctionTok{rsync} \AttributeTok{{-}avh} \AttributeTok{{-}{-}itemize{-}changes}\NormalTok{ origen/ destino/}
\end{Highlighting}
\end{Shaded}

\begin{center}\rule{0.5\linewidth}{0.5pt}\end{center}

\section{Herramientas
complementarias}\label{herramientas-complementarias}

\subsection{find - Búsqueda avanzada de
archivos}\label{find---buxfasqueda-avanzada-de-archivos}

\begin{Shaded}
\begin{Highlighting}[]
\CommentTok{\# Encontrar archivos para renombrar}
\FunctionTok{find}\NormalTok{ . }\AttributeTok{{-}name} \StringTok{"*.jpeg"} \AttributeTok{{-}exec}\NormalTok{ rename }\StringTok{\textquotesingle{}s/\textbackslash{}.jpeg$/\textbackslash{}.jpg/\textquotesingle{}}\NormalTok{ \{\} }\DataTypeTok{\textbackslash{};}

\CommentTok{\# Encontrar archivos duplicados por tamaño}
\FunctionTok{find}\NormalTok{ . }\AttributeTok{{-}type}\NormalTok{ f }\AttributeTok{{-}exec}\NormalTok{ du }\AttributeTok{{-}h}\NormalTok{ \{\} + }\KeywordTok{|} \FunctionTok{sort} \AttributeTok{{-}h} \KeywordTok{|} \FunctionTok{uniq} \AttributeTok{{-}d} \AttributeTok{{-}w}\NormalTok{ 8}

\CommentTok{\# Encontrar archivos grandes}
\FunctionTok{find}\NormalTok{ . }\AttributeTok{{-}type}\NormalTok{ f }\AttributeTok{{-}size}\NormalTok{ +100M }\AttributeTok{{-}exec}\NormalTok{ ls }\AttributeTok{{-}lh}\NormalTok{ \{\} }\DataTypeTok{\textbackslash{};}
\end{Highlighting}
\end{Shaded}

\subsection{xargs - Procesamiento en
lotes}\label{xargs---procesamiento-en-lotes}

\begin{Shaded}
\begin{Highlighting}[]
\CommentTok{\# Renombrar múltiples archivos encontrados}
\FunctionTok{find}\NormalTok{ . }\AttributeTok{{-}name} \StringTok{"*.tmp"} \KeywordTok{|} \FunctionTok{xargs} \AttributeTok{{-}I}\NormalTok{ \{\} mv \{\} \{\}.backup}

\CommentTok{\# Aplicar comando a archivos seleccionados}
\FunctionTok{ls} \PreprocessorTok{*}\NormalTok{.txt }\KeywordTok{|} \FunctionTok{xargs} \AttributeTok{{-}I}\NormalTok{ \{\} cp \{\} backup/\{\}}
\end{Highlighting}
\end{Shaded}

\subsection{chmod y chown - Permisos
masivos}\label{chmod-y-chown---permisos-masivos}

\begin{Shaded}
\begin{Highlighting}[]
\CommentTok{\# Cambiar permisos recursivamente}
\FunctionTok{find}\NormalTok{ ./proyecto }\AttributeTok{{-}type}\NormalTok{ f }\AttributeTok{{-}name} \StringTok{"*.sh"} \AttributeTok{{-}exec}\NormalTok{ chmod +x \{\} }\DataTypeTok{\textbackslash{};}

\CommentTok{\# Cambiar propietario de archivos específicos}
\FunctionTok{find}\NormalTok{ ./uploads }\AttributeTok{{-}name} \StringTok{"*.jpg"} \AttributeTok{{-}exec}\NormalTok{ chown www{-}data:www{-}data \{\} }\DataTypeTok{\textbackslash{};}
\end{Highlighting}
\end{Shaded}

\begin{center}\rule{0.5\linewidth}{0.5pt}\end{center}

\section{Workflows complejos}\label{workflows-complejos}

\subsection{Organización automática de
descargas}\label{organizaciuxf3n-automuxe1tica-de-descargas}

\begin{Shaded}
\begin{Highlighting}[]
\CommentTok{\#!/bin/bash}
\CommentTok{\# Script para organizar Downloads por tipo}

\VariableTok{DOWNLOADS}\OperatorTok{=}\StringTok{"}\VariableTok{$HOME}\StringTok{/Downloads"}
\BuiltInTok{cd} \StringTok{"}\VariableTok{$DOWNLOADS}\StringTok{"}

\CommentTok{\# Crear directorios si no existen}
\FunctionTok{mkdir} \AttributeTok{{-}p} \DataTypeTok{\{Images}\OperatorTok{,}\DataTypeTok{Documents}\OperatorTok{,}\DataTypeTok{Videos}\OperatorTok{,}\DataTypeTok{Audio}\OperatorTok{,}\DataTypeTok{Archives}\OperatorTok{,}\DataTypeTok{Code\}}

\CommentTok{\# Mover archivos por tipo}
\FunctionTok{find}\NormalTok{ . }\AttributeTok{{-}maxdepth}\NormalTok{ 1 }\AttributeTok{{-}name} \StringTok{"*.\{jpg,jpeg,png,gif,bmp\}"} \AttributeTok{{-}exec}\NormalTok{ mv \{\} Images/ }\DataTypeTok{\textbackslash{};}
\FunctionTok{find}\NormalTok{ . }\AttributeTok{{-}maxdepth}\NormalTok{ 1 }\AttributeTok{{-}name} \StringTok{"*.\{pdf,doc,docx,txt,rtf\}"} \AttributeTok{{-}exec}\NormalTok{ mv \{\} Documents/ }\DataTypeTok{\textbackslash{};}
\FunctionTok{find}\NormalTok{ . }\AttributeTok{{-}maxdepth}\NormalTok{ 1 }\AttributeTok{{-}name} \StringTok{"*.\{mp4,avi,mkv,mov\}"} \AttributeTok{{-}exec}\NormalTok{ mv \{\} Videos/ }\DataTypeTok{\textbackslash{};}
\FunctionTok{find}\NormalTok{ . }\AttributeTok{{-}maxdepth}\NormalTok{ 1 }\AttributeTok{{-}name} \StringTok{"*.\{mp3,wav,flac,m4a\}"} \AttributeTok{{-}exec}\NormalTok{ mv \{\} Audio/ }\DataTypeTok{\textbackslash{};}
\FunctionTok{find}\NormalTok{ . }\AttributeTok{{-}maxdepth}\NormalTok{ 1 }\AttributeTok{{-}name} \StringTok{"*.\{zip,tar,gz,rar,7z\}"} \AttributeTok{{-}exec}\NormalTok{ mv \{\} Archives/ }\DataTypeTok{\textbackslash{};}
\FunctionTok{find}\NormalTok{ . }\AttributeTok{{-}maxdepth}\NormalTok{ 1 }\AttributeTok{{-}name} \StringTok{"*.\{py,js,html,css,json\}"} \AttributeTok{{-}exec}\NormalTok{ mv \{\} Code/ }\DataTypeTok{\textbackslash{};}

\CommentTok{\# Renombrar archivos problemáticos}
\ExtensionTok{rename} \StringTok{\textquotesingle{}s/[\^{}a{-}zA{-}Z0{-}9.\_{-}]/\_/g\textquotesingle{}} \PreprocessorTok{*}
\end{Highlighting}
\end{Shaded}

\subsection{Backup diferencial
inteligente}\label{backup-diferencial-inteligente}

\begin{Shaded}
\begin{Highlighting}[]
\CommentTok{\#!/bin/bash}
\CommentTok{\# Backup que solo copia archivos modificados}

\VariableTok{SOURCE}\OperatorTok{=}\StringTok{"/home/user/important"}
\VariableTok{BACKUP}\OperatorTok{=}\StringTok{"/backup/incremental"}
\VariableTok{LOGFILE}\OperatorTok{=}\StringTok{"/var/log/backup.log"}

\CommentTok{\# Crear timestamp}
\VariableTok{TIMESTAMP}\OperatorTok{=}\VariableTok{$(}\FunctionTok{date}\NormalTok{ +}\StringTok{"\%Y\%m\%d\_\%H\%M\%S"}\VariableTok{)}
\BuiltInTok{echo} \StringTok{"[}\VariableTok{$TIMESTAMP}\StringTok{] Iniciando backup..."} \OperatorTok{\textgreater{}\textgreater{}} \StringTok{"}\VariableTok{$LOGFILE}\StringTok{"}

\CommentTok{\# Backup con checksums para verificar integridad}
\FunctionTok{rsync} \AttributeTok{{-}avh} \AttributeTok{{-}{-}checksum} \AttributeTok{{-}{-}delete} \DataTypeTok{\textbackslash{}}
      \AttributeTok{{-}{-}backup} \AttributeTok{{-}{-}backup{-}dir}\OperatorTok{=}\StringTok{"../deleted{-}}\VariableTok{$TIMESTAMP}\StringTok{"} \DataTypeTok{\textbackslash{}}
      \AttributeTok{{-}{-}log{-}file}\OperatorTok{=}\StringTok{"}\VariableTok{$LOGFILE}\StringTok{"} \DataTypeTok{\textbackslash{}}
      \StringTok{"}\VariableTok{$SOURCE}\StringTok{/"} \StringTok{"}\VariableTok{$BACKUP}\StringTok{/current/"}

\CommentTok{\# Verificar integridad}
\ControlFlowTok{if} \BuiltInTok{[} \VariableTok{$?} \OtherTok{{-}eq}\NormalTok{ 0 }\BuiltInTok{]}\KeywordTok{;} \ControlFlowTok{then}
    \BuiltInTok{echo} \StringTok{"[}\VariableTok{$TIMESTAMP}\StringTok{] Backup completado exitosamente"} \OperatorTok{\textgreater{}\textgreater{}} \StringTok{"}\VariableTok{$LOGFILE}\StringTok{"}
\ControlFlowTok{else}
    \BuiltInTok{echo} \StringTok{"[}\VariableTok{$TIMESTAMP}\StringTok{] ERROR en backup"} \OperatorTok{\textgreater{}\textgreater{}} \StringTok{"}\VariableTok{$LOGFILE}\StringTok{"}
    \BuiltInTok{exit}\NormalTok{ 1}
\ControlFlowTok{fi}
\end{Highlighting}
\end{Shaded}

\subsection{Sincronización
bidireccional}\label{sincronizaciuxf3n-bidireccional}

\begin{Shaded}
\begin{Highlighting}[]
\CommentTok{\#!/bin/bash}
\CommentTok{\# Sincronizar dos directorios en ambas direcciones}

\VariableTok{DIR1}\OperatorTok{=}\StringTok{"/ruta/directorio1"}
\VariableTok{DIR2}\OperatorTok{=}\StringTok{"/ruta/directorio2"}
\VariableTok{TEMP}\OperatorTok{=}\StringTok{"/tmp/sync\_temp"}

\CommentTok{\# Crear directorio temporal}
\FunctionTok{mkdir} \AttributeTok{{-}p} \StringTok{"}\VariableTok{$TEMP}\StringTok{"}

\CommentTok{\# Sincronizar DIR1 {-}\textgreater{} TEMP}
\FunctionTok{rsync} \AttributeTok{{-}avh} \AttributeTok{{-}{-}delete} \StringTok{"}\VariableTok{$DIR1}\StringTok{/"} \StringTok{"}\VariableTok{$TEMP}\StringTok{/"}

\CommentTok{\# Sincronizar DIR2 {-}\textgreater{} DIR1}
\FunctionTok{rsync} \AttributeTok{{-}avh} \AttributeTok{{-}{-}delete} \StringTok{"}\VariableTok{$DIR2}\StringTok{/"} \StringTok{"}\VariableTok{$DIR1}\StringTok{/"}

\CommentTok{\# Sincronizar TEMP {-}\textgreater{} DIR2}
\FunctionTok{rsync} \AttributeTok{{-}avh} \AttributeTok{{-}{-}delete} \StringTok{"}\VariableTok{$TEMP}\StringTok{/"} \StringTok{"}\VariableTok{$DIR2}\StringTok{/"}

\CommentTok{\# Limpiar}
\FunctionTok{rm} \AttributeTok{{-}rf} \StringTok{"}\VariableTok{$TEMP}\StringTok{"}
\end{Highlighting}
\end{Shaded}

\begin{tcolorbox}[enhanced jigsaw, toprule=.15mm, bottomrule=.15mm, opacityback=0, coltitle=black, rightrule=.15mm, colframe=quarto-callout-warning-color-frame, titlerule=0mm, opacitybacktitle=0.6, left=2mm, colback=white, bottomtitle=1mm, arc=.35mm, leftrule=.75mm, title=\textcolor{quarto-callout-warning-color}{\faExclamationTriangle}\hspace{0.5em}{Advertencia importante}, colbacktitle=quarto-callout-warning-color!10!white, breakable, toptitle=1mm]

\begin{itemize}
\tightlist
\item
  Siempre usa \texttt{-\/-dry-run} antes de operaciones destructivas
\item
  Ten backups antes de renombrados masivos
\item
  Verifica rutas cuidadosamente con \texttt{rsync} (especialmente las
  barras finales)
\item
  Prueba scripts en directorios de test antes de aplicar a datos
  importantes
\end{itemize}

\end{tcolorbox}

\begin{tcolorbox}[enhanced jigsaw, toprule=.15mm, bottomrule=.15mm, opacityback=0, coltitle=black, rightrule=.15mm, colframe=quarto-callout-tip-color-frame, titlerule=0mm, opacitybacktitle=0.6, left=2mm, colback=white, bottomtitle=1mm, arc=.35mm, leftrule=.75mm, title=\textcolor{quarto-callout-tip-color}{\faLightbulb}\hspace{0.5em}{Tips para gestión de archivos}, colbacktitle=quarto-callout-tip-color!10!white, breakable, toptitle=1mm]

\begin{itemize}
\tightlist
\item
  Combina \texttt{find} con \texttt{xargs} para operaciones en lotes
  eficientes
\item
  Usa \texttt{rsync} en lugar de \texttt{cp} para copias grandes o
  frecuentes
\item
  Automatiza tareas repetitivas con scripts que usen estas herramientas
\item
  Mantén logs de operaciones importantes para auditoría
\end{itemize}

\end{tcolorbox}

En el próximo capítulo exploraremos herramientas de búsqueda y filtrado
que te permitirán encontrar información rápidamente en grandes volúmenes
de datos.

\part{Búsqueda y Filtrado}

\chapter{Búsqueda y Filtrado}\label{buxfasqueda-y-filtrado-2}

La capacidad de encontrar información rápidamente en archivos y datos es
crucial para la productividad. Esta sección cubre las herramientas más
potentes para búsqueda de texto ultrarrápida, filtrado interactivo y
procesamiento de datos estructurados.

\section{ripgrep (rg) - Búsqueda ultrarrápida}\label{sec-ripgrep}

\texttt{ripgrep} es una herramienta de búsqueda que combina la potencia
de \texttt{grep} con la velocidad moderna y características inteligentes
como respeto automático de \texttt{.gitignore}.

\subsection{Características
principales}\label{caracteruxedsticas-principales}

\begin{itemize}
\tightlist
\item
  🚀 \textbf{Extremadamente rápido} - Hasta 10x más rápido que grep
\item
  🎯 \textbf{Respeta .gitignore} automáticamente
\item
  🌈 \textbf{Colores por defecto} para mejor legibilidad
\item
  🔍 \textbf{Búsqueda recursiva} por defecto
\item
  📁 \textbf{Filtros por tipo de archivo} inteligentes
\end{itemize}

\subsection{Uso básico}\label{uso-buxe1sico-3}

\begin{Shaded}
\begin{Highlighting}[]
\CommentTok{\# Búsqueda simple en directorio actual}
\ExtensionTok{rg} \StringTok{"función"}

\CommentTok{\# Búsqueda ignorando mayúsculas/minúsculas}
\ExtensionTok{rg} \AttributeTok{{-}i} \StringTok{"error"}

\CommentTok{\# Buscar palabra completa}
\ExtensionTok{rg} \AttributeTok{{-}w} \StringTok{"class"}

\CommentTok{\# Búsqueda literal (sin regex)}
\ExtensionTok{rg} \AttributeTok{{-}F} \StringTok{"console.log("}
\end{Highlighting}
\end{Shaded}

\subsection{Ejemplos por tipo de
archivo}\label{ejemplos-por-tipo-de-archivo}

\subsubsection{Desarrollo web}\label{desarrollo-web}

\begin{Shaded}
\begin{Highlighting}[]
\CommentTok{\# Buscar imports en archivos JavaScript}
\ExtensionTok{rg} \AttributeTok{{-}t}\NormalTok{ js }\StringTok{"import.*from"}

\CommentTok{\# Encontrar funciones en TypeScript}
\ExtensionTok{rg} \AttributeTok{{-}t}\NormalTok{ ts }\StringTok{"function\textbackslash{}s+\textbackslash{}w+"} 

\CommentTok{\# Buscar clases CSS}
\ExtensionTok{rg} \AttributeTok{{-}t}\NormalTok{ css }\StringTok{"\textbackslash{}.[\textbackslash{}w{-}]+\textbackslash{}s*\textbackslash{}\{"}

\CommentTok{\# APIs en archivos de configuración}
\ExtensionTok{rg} \AttributeTok{{-}t}\NormalTok{ json }\StringTok{"api.*url"}
\end{Highlighting}
\end{Shaded}

\subsubsection{Python}\label{python}

\begin{Shaded}
\begin{Highlighting}[]
\CommentTok{\# Buscar definiciones de clase}
\ExtensionTok{rg} \AttributeTok{{-}t}\NormalTok{ py }\StringTok{"\^{}class\textbackslash{}s+\textbackslash{}w+"}

\CommentTok{\# Encontrar imports específicos}
\ExtensionTok{rg} \AttributeTok{{-}t}\NormalTok{ py }\StringTok{"from django import"}

\CommentTok{\# Buscar decoradores}
\ExtensionTok{rg} \AttributeTok{{-}t}\NormalTok{ py }\StringTok{"@\textbackslash{}w+"}

\CommentTok{\# Variables de entorno}
\ExtensionTok{rg} \AttributeTok{{-}t}\NormalTok{ py }\StringTok{"os\textbackslash{}.environ"}
\end{Highlighting}
\end{Shaded}

\subsection{Filtros avanzados}\label{filtros-avanzados}

\subsubsection{Por ubicación}\label{por-ubicaciuxf3n}

\begin{Shaded}
\begin{Highlighting}[]
\CommentTok{\# Buscar solo en archivos de configuración}
\ExtensionTok{rg} \StringTok{"database"} \AttributeTok{{-}{-}glob}\OperatorTok{=}\StringTok{"*.\{conf,cfg,ini,yaml,yml\}"}

\CommentTok{\# Excluir directorios específicos}
\ExtensionTok{rg} \StringTok{"TODO"} \AttributeTok{{-}{-}glob}\OperatorTok{=}\StringTok{"!node\_modules/**"}

\CommentTok{\# Buscar en archivos modificados recientemente}
\FunctionTok{find}\NormalTok{ . }\AttributeTok{{-}mtime} \AttributeTok{{-}1} \AttributeTok{{-}name} \StringTok{"*.py"} \KeywordTok{|} \FunctionTok{xargs}\NormalTok{ rg }\StringTok{"import"}
\end{Highlighting}
\end{Shaded}

\subsubsection{Por contexto}\label{por-contexto}

\begin{Shaded}
\begin{Highlighting}[]
\CommentTok{\# Mostrar 3 líneas antes y después}
\ExtensionTok{rg} \AttributeTok{{-}A}\NormalTok{ 3 }\AttributeTok{{-}B}\NormalTok{ 3 }\StringTok{"error"}

\CommentTok{\# Mostrar 5 líneas de contexto}
\ExtensionTok{rg} \AttributeTok{{-}C}\NormalTok{ 5 }\StringTok{"function main"}

\CommentTok{\# Mostrar número de línea}
\ExtensionTok{rg} \AttributeTok{{-}n} \StringTok{"import React"}

\CommentTok{\# Mostrar solo nombres de archivos que contienen el patrón}
\ExtensionTok{rg} \AttributeTok{{-}l} \StringTok{"password"}
\end{Highlighting}
\end{Shaded}

\subsection{Casos de uso avanzados}\label{casos-de-uso-avanzados-4}

\subsubsection{Auditoría de código}\label{auditoruxeda-de-cuxf3digo}

\begin{Shaded}
\begin{Highlighting}[]
\CommentTok{\# Encontrar hardcoded passwords/secrets}
\ExtensionTok{rg} \AttributeTok{{-}i} \StringTok{"(password|secret|key|token)\textbackslash{}s*[:=]"} \AttributeTok{{-}{-}type{-}not}\OperatorTok{=}\NormalTok{json}

\CommentTok{\# Buscar código duplicado (funciones similares)}
\ExtensionTok{rg} \StringTok{"function \textbackslash{}w+\textbackslash{}("} \AttributeTok{{-}A}\NormalTok{ 10 }\KeywordTok{|} \FunctionTok{grep} \AttributeTok{{-}A}\NormalTok{ 10 }\StringTok{"function.*("} \KeywordTok{|} \FunctionTok{sort} \KeywordTok{|} \FunctionTok{uniq} \AttributeTok{{-}d}

\CommentTok{\# Encontrar imports no utilizados en Python}
\ExtensionTok{rg} \StringTok{"\^{}import \textbackslash{}w+"} \AttributeTok{{-}t}\NormalTok{ py }\KeywordTok{|} \FunctionTok{cut} \AttributeTok{{-}d:} \AttributeTok{{-}f2} \KeywordTok{|} \FunctionTok{sort} \KeywordTok{|} \FunctionTok{uniq} \OperatorTok{\textgreater{}}\NormalTok{ imports.txt}
\CommentTok{\# Luego verificar cada import en el código}

\CommentTok{\# URLs hardcodeadas}
\ExtensionTok{rg} \StringTok{"https?://[\^{}\textbackslash{}s}\DataTypeTok{\textbackslash{}"}\StringTok{\textquotesingle{}]+"}
\end{Highlighting}
\end{Shaded}

\subsubsection{Análisis de logs}\label{anuxe1lisis-de-logs}

\begin{Shaded}
\begin{Highlighting}[]
\CommentTok{\# Errores por severidad}
\ExtensionTok{rg} \StringTok{"(ERROR|WARN|INFO)"}\NormalTok{ /var/log/app.log}

\CommentTok{\# IPs sospechosas (muchas peticiones)}
\ExtensionTok{rg} \AttributeTok{{-}o} \StringTok{"\textbackslash{}b\textbackslash{}d\{1,3\}\textbackslash{}.\textbackslash{}d\{1,3\}\textbackslash{}.\textbackslash{}d\{1,3\}\textbackslash{}.\textbackslash{}d\{1,3\}\textbackslash{}b"}\NormalTok{ access.log }\KeywordTok{|} \DataTypeTok{\textbackslash{}}
\FunctionTok{sort} \KeywordTok{|} \FunctionTok{uniq} \AttributeTok{{-}c} \KeywordTok{|} \FunctionTok{sort} \AttributeTok{{-}nr} \KeywordTok{|} \FunctionTok{head} \AttributeTok{{-}10}

\CommentTok{\# Códigos de error HTTP}
\ExtensionTok{rg} \StringTok{" [45]\textbackslash{}d\{2\} "}\NormalTok{ access.log }\AttributeTok{{-}o} \KeywordTok{|} \FunctionTok{sort} \KeywordTok{|} \FunctionTok{uniq} \AttributeTok{{-}c}
\end{Highlighting}
\end{Shaded}

\subsubsection{Migración de código}\label{migraciuxf3n-de-cuxf3digo}

\begin{Shaded}
\begin{Highlighting}[]
\CommentTok{\# Encontrar uso de API deprecated}
\ExtensionTok{rg} \StringTok{"\textbackslash{}.oldFunction\textbackslash{}("} \AttributeTok{{-}t}\NormalTok{ js}

\CommentTok{\# Buscar patterns específicos para refactoring}
\ExtensionTok{rg} \StringTok{"var\textbackslash{}s+\textbackslash{}w+\textbackslash{}s*="} \AttributeTok{{-}t}\NormalTok{ js  }\CommentTok{\# Variables con \textquotesingle{}var\textquotesingle{} para cambiar a \textquotesingle{}let/const\textquotesingle{}}

\CommentTok{\# Encontrar comentarios }\AlertTok{TODO}\CommentTok{/}\AlertTok{FIXME}\CommentTok{ con contexto}
\ExtensionTok{rg} \AttributeTok{{-}C}\NormalTok{ 2 }\StringTok{"(TODO|FIXME|HACK)"}
\end{Highlighting}
\end{Shaded}

\subsection{Configuración
personalizada}\label{configuraciuxf3n-personalizada}

\subsubsection{Archivo de
configuración}\label{archivo-de-configuraciuxf3n-1}

Crear \texttt{\textasciitilde{}/.ripgreprc}:

\begin{Shaded}
\begin{Highlighting}[]
\CommentTok{\# Siempre mostrar números de línea}
\ExtensionTok{{-}{-}line{-}number}

\CommentTok{\# Búsqueda inteligente de mayúsculas}
\ExtensionTok{{-}{-}smart{-}case}

\CommentTok{\# Mostrar tipos de archivo soportados}
\ExtensionTok{{-}{-}type{-}list}

\CommentTok{\# Colores personalizados}
\ExtensionTok{{-}{-}colors=line:fg:yellow}
\ExtensionTok{{-}{-}colors=line:style:bold}
\ExtensionTok{{-}{-}colors=path:fg:green}
\ExtensionTok{{-}{-}colors=match:fg:red}
\ExtensionTok{{-}{-}colors=match:style:bold}
\end{Highlighting}
\end{Shaded}

\begin{center}\rule{0.5\linewidth}{0.5pt}\end{center}

\section{fzf - Filtro difuso interactivo}\label{sec-fzf}

\texttt{fzf} es un filtro de línea de comandos que permite búsqueda
difusa interactiva en cualquier lista de elementos.

\subsection{Configuración inicial}\label{configuraciuxf3n-inicial-1}

\begin{Shaded}
\begin{Highlighting}[]
\CommentTok{\# Configurar para zsh (agregar a .zshrc)}
\BuiltInTok{source} \OperatorTok{\textless{}(}\ExtensionTok{fzf} \AttributeTok{{-}{-}zsh}\OperatorTok{)}

\CommentTok{\# Variables de entorno útiles}
\BuiltInTok{export} \VariableTok{FZF\_DEFAULT\_COMMAND}\OperatorTok{=}\StringTok{\textquotesingle{}rg {-}{-}files {-}{-}hidden {-}{-}follow {-}{-}glob "!.git/*"\textquotesingle{}}
\BuiltInTok{export} \VariableTok{FZF\_DEFAULT\_OPTS}\OperatorTok{=}\StringTok{\textquotesingle{}{-}{-}height 40\% {-}{-}layout=reverse {-}{-}border\textquotesingle{}}
\end{Highlighting}
\end{Shaded}

\subsection{Uso básico interactivo}\label{uso-buxe1sico-interactivo}

\begin{Shaded}
\begin{Highlighting}[]
\CommentTok{\# Buscar archivos en directorio actual}
\FunctionTok{find}\NormalTok{ . }\AttributeTok{{-}type}\NormalTok{ f }\KeywordTok{|} \ExtensionTok{fzf}

\CommentTok{\# Buscar en historial de comandos}
\BuiltInTok{history} \KeywordTok{|} \ExtensionTok{fzf}

\CommentTok{\# Buscar procesos}
\FunctionTok{ps}\NormalTok{ aux }\KeywordTok{|} \ExtensionTok{fzf}
\end{Highlighting}
\end{Shaded}

\subsection{Ejemplos prácticos}\label{ejemplos-pruxe1cticos-3}

\subsubsection{Navegación de archivos}\label{navegaciuxf3n-de-archivos}

\begin{Shaded}
\begin{Highlighting}[]
\CommentTok{\# Editar archivo seleccionado}
\ExtensionTok{vim} \VariableTok{$(}\FunctionTok{find}\NormalTok{ . }\AttributeTok{{-}type}\NormalTok{ f }\KeywordTok{|} \ExtensionTok{fzf}\VariableTok{)}

\CommentTok{\# Abrir en VS Code}
\ExtensionTok{code} \VariableTok{$(}\FunctionTok{find}\NormalTok{ . }\AttributeTok{{-}name} \StringTok{"*.py"} \KeywordTok{|} \ExtensionTok{fzf}\VariableTok{)}

\CommentTok{\# Ver contenido de archivo seleccionado}
\FunctionTok{cat} \VariableTok{$(}\FunctionTok{find}\NormalTok{ . }\AttributeTok{{-}name} \StringTok{"*.md"} \KeywordTok{|} \ExtensionTok{fzf}\VariableTok{)}

\CommentTok{\# Ir a directorio seleccionado}
\BuiltInTok{cd} \VariableTok{$(}\FunctionTok{find}\NormalTok{ . }\AttributeTok{{-}type}\NormalTok{ d }\KeywordTok{|} \ExtensionTok{fzf}\VariableTok{)}
\end{Highlighting}
\end{Shaded}

\subsubsection{Git workflow}\label{git-workflow}

\begin{Shaded}
\begin{Highlighting}[]
\CommentTok{\# Checkout a branch}
\FunctionTok{git}\NormalTok{ checkout }\VariableTok{$(}\FunctionTok{git}\NormalTok{ branch }\KeywordTok{|} \ExtensionTok{fzf} \KeywordTok{|} \FunctionTok{tr} \AttributeTok{{-}d} \StringTok{\textquotesingle{} \textquotesingle{}}\VariableTok{)}

\CommentTok{\# Ver commit específico}
\FunctionTok{git}\NormalTok{ show }\VariableTok{$(}\FunctionTok{git}\NormalTok{ log }\AttributeTok{{-}{-}oneline} \KeywordTok{|} \ExtensionTok{fzf} \KeywordTok{|} \FunctionTok{cut} \AttributeTok{{-}d}\StringTok{\textquotesingle{} \textquotesingle{}} \AttributeTok{{-}f1}\VariableTok{)}

\CommentTok{\# Agregar archivos selectivamente}
\FunctionTok{git}\NormalTok{ add }\VariableTok{$(}\FunctionTok{git}\NormalTok{ status }\AttributeTok{{-}{-}porcelain} \KeywordTok{|} \ExtensionTok{fzf} \AttributeTok{{-}m} \KeywordTok{|} \FunctionTok{cut} \AttributeTok{{-}c4{-}}\VariableTok{)}

\CommentTok{\# Ver diff de archivo}
\FunctionTok{git}\NormalTok{ diff }\VariableTok{$(}\FunctionTok{git}\NormalTok{ status }\AttributeTok{{-}{-}porcelain} \KeywordTok{|} \ExtensionTok{fzf} \KeywordTok{|} \FunctionTok{cut} \AttributeTok{{-}c4{-}}\VariableTok{)}
\end{Highlighting}
\end{Shaded}

\subsubsection{Gestión de procesos}\label{gestiuxf3n-de-procesos}

\begin{Shaded}
\begin{Highlighting}[]
\CommentTok{\# Matar proceso seleccionado}
\BuiltInTok{kill} \VariableTok{$(}\FunctionTok{ps}\NormalTok{ aux }\KeywordTok{|} \ExtensionTok{fzf} \KeywordTok{|} \FunctionTok{awk} \StringTok{\textquotesingle{}\{print $2\}\textquotesingle{}}\VariableTok{)}

\CommentTok{\# Ver logs de servicio systemd}
\ExtensionTok{journalctl} \AttributeTok{{-}u} \VariableTok{$(}\ExtensionTok{systemctl}\NormalTok{ list{-}units }\AttributeTok{{-}{-}type}\OperatorTok{=}\NormalTok{service }\KeywordTok{|} \ExtensionTok{fzf} \KeywordTok{|} \FunctionTok{awk} \StringTok{\textquotesingle{}\{print $1\}\textquotesingle{}}\VariableTok{)}

\CommentTok{\# Conectar por SSH}
\FunctionTok{ssh} \VariableTok{$(}\FunctionTok{grep} \StringTok{"\^{}Host "}\NormalTok{ \textasciitilde{}/.ssh/config }\KeywordTok{|} \FunctionTok{cut} \AttributeTok{{-}d}\StringTok{\textquotesingle{} \textquotesingle{}} \AttributeTok{{-}f2} \KeywordTok{|} \ExtensionTok{fzf}\VariableTok{)}
\end{Highlighting}
\end{Shaded}

\subsection{Integración avanzada}\label{integraciuxf3n-avanzada}

\subsubsection{Funciones de shell
personalizadas}\label{funciones-de-shell-personalizadas}

\begin{Shaded}
\begin{Highlighting}[]
\CommentTok{\# Función para buscar y editar archivos}
\FunctionTok{fe()} \KeywordTok{\{}
    \BuiltInTok{local} \VariableTok{files}
    \VariableTok{files}\OperatorTok{=}\VariableTok{$(}\FunctionTok{find}\NormalTok{ . }\AttributeTok{{-}type}\NormalTok{ f }\KeywordTok{|} \ExtensionTok{fzf} \AttributeTok{{-}{-}multi} \AttributeTok{{-}{-}preview} \StringTok{\textquotesingle{}bat {-}{-}color=always \{\}\textquotesingle{}}\VariableTok{)}
    \KeywordTok{[[} \OtherTok{{-}n} \StringTok{"}\VariableTok{$files}\StringTok{"} \KeywordTok{]]} \KeywordTok{\&\&} \VariableTok{$\{EDITOR}\OperatorTok{:{-}}\NormalTok{vim}\VariableTok{\}} \StringTok{"}\VariableTok{$\{files}\OperatorTok{[@]}\VariableTok{\}}\StringTok{"}
\KeywordTok{\}}

\CommentTok{\# Función para navegar a directorio frecuente}
\FunctionTok{fd()} \KeywordTok{\{}
    \BuiltInTok{local} \VariableTok{dir}
    \VariableTok{dir}\OperatorTok{=}\VariableTok{$(}\FunctionTok{find} \VariableTok{$\{1}\OperatorTok{:{-}}\NormalTok{.}\VariableTok{\}} \AttributeTok{{-}type}\NormalTok{ d }\DecValTok{2}\OperatorTok{\textgreater{}}\NormalTok{/dev/null }\KeywordTok{|} \ExtensionTok{fzf}\NormalTok{ +m}\VariableTok{)} \KeywordTok{\&\&} \BuiltInTok{cd} \StringTok{"}\VariableTok{$dir}\StringTok{"}
\KeywordTok{\}}

\CommentTok{\# Función para ver historia de comandos y ejecutar}
\FunctionTok{fh()} \KeywordTok{\{}
    \BuiltInTok{eval} \VariableTok{$(} \KeywordTok{(}\BuiltInTok{[} \OtherTok{{-}n} \StringTok{"}\VariableTok{$ZSH\_NAME}\StringTok{"} \BuiltInTok{]} \KeywordTok{\&\&} \BuiltInTok{fc} \AttributeTok{{-}l}\NormalTok{ 1 }\KeywordTok{||} \BuiltInTok{history}\KeywordTok{)} \KeywordTok{|} \ExtensionTok{fzf}\NormalTok{ +s }\AttributeTok{{-}{-}tac} \KeywordTok{|} \FunctionTok{sed} \StringTok{\textquotesingle{}s/ *[0{-}9]* *//\textquotesingle{}}\VariableTok{)}
\KeywordTok{\}}

\CommentTok{\# Función para matar procesos interactivamente}
\FunctionTok{fkill()} \KeywordTok{\{}
    \BuiltInTok{local} \VariableTok{pid}
    \VariableTok{pid}\OperatorTok{=}\VariableTok{$(}\FunctionTok{ps} \AttributeTok{{-}ef} \KeywordTok{|} \FunctionTok{sed}\NormalTok{ 1d }\KeywordTok{|} \ExtensionTok{fzf} \AttributeTok{{-}m} \KeywordTok{|} \FunctionTok{awk} \StringTok{\textquotesingle{}\{print $2\}\textquotesingle{}}\VariableTok{)}
    \ControlFlowTok{if} \BuiltInTok{[} \StringTok{"x}\VariableTok{$pid}\StringTok{"} \OtherTok{!=} \StringTok{"x"} \BuiltInTok{]}\KeywordTok{;} \ControlFlowTok{then}
        \BuiltInTok{echo} \VariableTok{$pid} \KeywordTok{|} \FunctionTok{xargs}\NormalTok{ kill }\AttributeTok{{-}}\VariableTok{$\{1}\OperatorTok{:{-}}\NormalTok{9}\VariableTok{\}}
    \ControlFlowTok{fi}
\KeywordTok{\}}
\end{Highlighting}
\end{Shaded}

\subsubsection{Vista previa avanzada}\label{vista-previa-avanzada}

\begin{Shaded}
\begin{Highlighting}[]
\CommentTok{\# Buscar archivos con vista previa}
\ExtensionTok{fzf} \AttributeTok{{-}{-}preview} \StringTok{\textquotesingle{}bat {-}{-}style=numbers {-}{-}color=always {-}{-}line{-}range :500 \{\}\textquotesingle{}}

\CommentTok{\# Buscar en git commits con vista previa}
\FunctionTok{git}\NormalTok{ log }\AttributeTok{{-}{-}oneline} \KeywordTok{|} \ExtensionTok{fzf} \AttributeTok{{-}{-}preview} \StringTok{\textquotesingle{}git show {-}{-}color=always \{1\}\textquotesingle{}}

\CommentTok{\# Buscar archivos de código con vista previa y números de línea}
\ExtensionTok{rg} \AttributeTok{{-}{-}files} \KeywordTok{|} \ExtensionTok{fzf} \AttributeTok{{-}{-}preview} \StringTok{\textquotesingle{}bat {-}{-}style=numbers {-}{-}color=always {-}{-}highlight{-}line \{2\} \{1\}\textquotesingle{}} \AttributeTok{{-}{-}delimiter} \StringTok{\textquotesingle{}:\textquotesingle{}}
\end{Highlighting}
\end{Shaded}

\begin{center}\rule{0.5\linewidth}{0.5pt}\end{center}

\section{jq - Procesador JSON}\label{sec-jq}

\texttt{jq} es como \texttt{sed} pero para datos JSON, permitiendo
consultar, filtrar y transformar datos JSON de manera elegante.

\subsection{Sintaxis básica}\label{sintaxis-buxe1sica-1}

\begin{Shaded}
\begin{Highlighting}[]
\CommentTok{\# Pretty print JSON}
\ExtensionTok{curl} \AttributeTok{{-}s}\NormalTok{ https://api.github.com/users/octocat }\KeywordTok{|} \ExtensionTok{jq}\NormalTok{ .}

\CommentTok{\# Extraer campo específico}
\BuiltInTok{echo} \StringTok{\textquotesingle{}\{"name":"Juan","age":30\}\textquotesingle{}} \KeywordTok{|} \ExtensionTok{jq}\NormalTok{ .name}

\CommentTok{\# Extraer campo anidado}
\BuiltInTok{echo} \StringTok{\textquotesingle{}\{"user":\{"name":"Juan"\}\}\textquotesingle{}} \KeywordTok{|} \ExtensionTok{jq}\NormalTok{ .user.name}
\end{Highlighting}
\end{Shaded}

\subsection{Filtros básicos}\label{filtros-buxe1sicos}

\subsubsection{Acceso a datos}\label{acceso-a-datos}

\begin{Shaded}
\begin{Highlighting}[]
\CommentTok{\# Array de objetos {-} extraer campo de todos}
\ExtensionTok{jq} \StringTok{\textquotesingle{}.[].name\textquotesingle{}}\NormalTok{ users.json}

\CommentTok{\# Filtrar por condición}
\ExtensionTok{jq} \StringTok{\textquotesingle{}.[] | select(.age \textgreater{} 25)\textquotesingle{}}\NormalTok{ users.json}

\CommentTok{\# Extraer múltiples campos}
\ExtensionTok{jq} \StringTok{\textquotesingle{}.[] | \{name: .name, email: .email\}\textquotesingle{}}\NormalTok{ users.json}

\CommentTok{\# Contar elementos}
\ExtensionTok{jq} \StringTok{\textquotesingle{}. | length\textquotesingle{}}\NormalTok{ array.json}
\end{Highlighting}
\end{Shaded}

\subsubsection{Transformaciones}\label{transformaciones}

\begin{Shaded}
\begin{Highlighting}[]
\CommentTok{\# Crear nuevo objeto}
\ExtensionTok{jq} \StringTok{\textquotesingle{}\{full\_name: (.first\_name + " " + .last\_name), age: .age\}\textquotesingle{}}\NormalTok{ person.json}

\CommentTok{\# Agregar campo}
\ExtensionTok{jq} \StringTok{\textquotesingle{}. + \{timestamp: now\}\textquotesingle{}}\NormalTok{ data.json}

\CommentTok{\# Eliminar campo}
\ExtensionTok{jq} \StringTok{\textquotesingle{}del(.password)\textquotesingle{}}\NormalTok{ user.json}

\CommentTok{\# Renombrar campo}
\ExtensionTok{jq} \StringTok{\textquotesingle{}\{username: .name, years: .age\}\textquotesingle{}}\NormalTok{ user.json}
\end{Highlighting}
\end{Shaded}

\subsection{Casos de uso con APIs}\label{casos-de-uso-con-apis}

\subsubsection{GitHub API}\label{github-api}

\begin{Shaded}
\begin{Highlighting}[]
\CommentTok{\# Listar repositorios públicos de un usuario}
\ExtensionTok{curl} \AttributeTok{{-}s}\NormalTok{ https://api.github.com/users/octocat/repos }\KeywordTok{|} \DataTypeTok{\textbackslash{}}
\ExtensionTok{jq} \StringTok{\textquotesingle{}.[] | \{name: .name, stars: .stargazers\_count, language: .language\}\textquotesingle{}}

\CommentTok{\# Top 5 repos por estrellas}
\ExtensionTok{curl} \AttributeTok{{-}s}\NormalTok{ https://api.github.com/users/octocat/repos }\KeywordTok{|} \DataTypeTok{\textbackslash{}}
\ExtensionTok{jq} \StringTok{\textquotesingle{}sort\_by(.stargazers\_count) | reverse | .[0:5] | .[] | .name\textquotesingle{}}

\CommentTok{\# Repos por lenguaje}
\ExtensionTok{curl} \AttributeTok{{-}s}\NormalTok{ https://api.github.com/users/octocat/repos }\KeywordTok{|} \DataTypeTok{\textbackslash{}}
\ExtensionTok{jq} \StringTok{\textquotesingle{}group\_by(.language) | .[] | \{language: .[0].language, count: length\}\textquotesingle{}}
\end{Highlighting}
\end{Shaded}

\subsubsection{Análisis de logs JSON}\label{anuxe1lisis-de-logs-json}

\begin{Shaded}
\begin{Highlighting}[]
\CommentTok{\# Logs de aplicación en formato JSON}
\FunctionTok{cat}\NormalTok{ app.log }\KeywordTok{|} \ExtensionTok{jq} \StringTok{\textquotesingle{}select(.level == "ERROR")\textquotesingle{}} 

\CommentTok{\# Agrupar errores por tipo}
\FunctionTok{cat}\NormalTok{ app.log }\KeywordTok{|} \ExtensionTok{jq} \StringTok{\textquotesingle{}select(.level == "ERROR") | .error\_type\textquotesingle{}} \KeywordTok{|} \FunctionTok{sort} \KeywordTok{|} \FunctionTok{uniq} \AttributeTok{{-}c}

\CommentTok{\# Estadísticas por hora}
\FunctionTok{cat}\NormalTok{ app.log }\KeywordTok{|} \ExtensionTok{jq} \AttributeTok{{-}r} \StringTok{\textquotesingle{}.timestamp[0:13]\textquotesingle{}} \KeywordTok{|} \FunctionTok{sort} \KeywordTok{|} \FunctionTok{uniq} \AttributeTok{{-}c}
\end{Highlighting}
\end{Shaded}

\subsection{Operaciones avanzadas}\label{operaciones-avanzadas}

\subsubsection{Agregaciones y
estadísticas}\label{agregaciones-y-estaduxedsticas}

\begin{Shaded}
\begin{Highlighting}[]
\CommentTok{\# Promedio de edades}
\ExtensionTok{jq} \StringTok{\textquotesingle{}[.[] | .age] | add / length\textquotesingle{}}\NormalTok{ users.json}

\CommentTok{\# Valor máximo}
\ExtensionTok{jq} \StringTok{\textquotesingle{}[.[] | .score] | max\textquotesingle{}}\NormalTok{ scores.json}

\CommentTok{\# Agrupar y contar}
\ExtensionTok{jq} \StringTok{\textquotesingle{}group\_by(.category) | .[] | \{category: .[0].category, count: length\}\textquotesingle{}}\NormalTok{ items.json}

\CommentTok{\# Crear histograma de frecuencias}
\ExtensionTok{jq} \StringTok{\textquotesingle{}[.[] | .status] | group\_by(.) | .[] | \{status: .[0], count: length\}\textquotesingle{}}\NormalTok{ requests.json}
\end{Highlighting}
\end{Shaded}

\subsubsection{Transformaciones
complejas}\label{transformaciones-complejas}

\begin{Shaded}
\begin{Highlighting}[]
\CommentTok{\# Convertir array de objetos a objeto indexado}
\ExtensionTok{jq} \StringTok{\textquotesingle{}map(\{(.id|tostring): .\}) | add\textquotesingle{}}\NormalTok{ items.json}

\CommentTok{\# Pivot de datos}
\ExtensionTok{jq} \StringTok{\textquotesingle{}group\_by(.category) | map(\{category: .[0].category, items: map(.name)\})\textquotesingle{}}\NormalTok{ products.json}

\CommentTok{\# Merge de múltiples objetos}
\ExtensionTok{jq} \StringTok{\textquotesingle{}. as $root | .users[] | . + \{company: $root.company\}\textquotesingle{}}\NormalTok{ company.json}
\end{Highlighting}
\end{Shaded}

\subsubsection{Validación y filtrado}\label{validaciuxf3n-y-filtrado}

\begin{Shaded}
\begin{Highlighting}[]
\CommentTok{\# Verificar estructura requerida}
\ExtensionTok{jq} \StringTok{\textquotesingle{}if has("name") and has("email") then . else empty end\textquotesingle{}}\NormalTok{ user.json}

\CommentTok{\# Filtros complejos con múltiples condiciones}
\ExtensionTok{jq} \StringTok{\textquotesingle{}.[] | select(.age \textgreater{} 18 and .status == "active" and (.roles | contains(["admin"])))\textquotesingle{}}\NormalTok{ users.json}

\CommentTok{\# Validar emails}
\ExtensionTok{jq} \StringTok{\textquotesingle{}.[] | select(.email | test("\^{}[\^{}@]+@[\^{}@]+\textbackslash{}\textbackslash{}.[\^{}@]+$"))\textquotesingle{}}\NormalTok{ contacts.json}
\end{Highlighting}
\end{Shaded}

\subsection{Integración con otras
herramientas}\label{integraciuxf3n-con-otras-herramientas}

\subsubsection{Con curl para APIs}\label{con-curl-para-apis}

\begin{Shaded}
\begin{Highlighting}[]
\CommentTok{\# Script para monitorear API}
\CommentTok{\#!/bin/bash}
\VariableTok{API\_URL}\OperatorTok{=}\StringTok{"https://api.example.com/status"}
\ControlFlowTok{while} \FunctionTok{true}\KeywordTok{;} \ControlFlowTok{do}
    \VariableTok{status}\OperatorTok{=}\VariableTok{$(}\ExtensionTok{curl} \AttributeTok{{-}s} \StringTok{"}\VariableTok{$API\_URL}\StringTok{"} \KeywordTok{|} \ExtensionTok{jq} \AttributeTok{{-}r} \StringTok{\textquotesingle{}.status\textquotesingle{}}\VariableTok{)}
    \ControlFlowTok{if} \BuiltInTok{[} \StringTok{"}\VariableTok{$status}\StringTok{"} \OtherTok{!=} \StringTok{"ok"} \BuiltInTok{]}\KeywordTok{;} \ControlFlowTok{then}
        \BuiltInTok{echo} \StringTok{"ALERT: API status is }\VariableTok{$status}\StringTok{"}
        \CommentTok{\# Enviar notificación}
    \ControlFlowTok{fi}
    \FunctionTok{sleep}\NormalTok{ 60}
\ControlFlowTok{done}
\end{Highlighting}
\end{Shaded}

\subsubsection{Con fzf para selección
interactiva}\label{con-fzf-para-selecciuxf3n-interactiva}

\begin{Shaded}
\begin{Highlighting}[]
\CommentTok{\# Seleccionar usuario de API y ver detalles}
\VariableTok{user\_id}\OperatorTok{=}\VariableTok{$(}\ExtensionTok{curl} \AttributeTok{{-}s}\NormalTok{ https://api.example.com/users }\KeywordTok{|} \DataTypeTok{\textbackslash{}}
          \ExtensionTok{jq} \AttributeTok{{-}r} \StringTok{\textquotesingle{}.[] | "\textbackslash{}(.id): \textbackslash{}(.name)"\textquotesingle{}} \KeywordTok{|} \DataTypeTok{\textbackslash{}}
          \ExtensionTok{fzf} \KeywordTok{|} \FunctionTok{cut} \AttributeTok{{-}d:} \AttributeTok{{-}f1}\VariableTok{)}

\ExtensionTok{curl} \AttributeTok{{-}s} \StringTok{"https://api.example.com/users/}\VariableTok{$user\_id}\StringTok{"} \KeywordTok{|} \ExtensionTok{jq}\NormalTok{ .}
\end{Highlighting}
\end{Shaded}

\subsubsection{Conversion de formatos}\label{conversion-de-formatos}

\begin{Shaded}
\begin{Highlighting}[]
\CommentTok{\# JSON a CSV}
\ExtensionTok{jq} \AttributeTok{{-}r} \StringTok{\textquotesingle{}.[] | [.name, .age, .email] | @csv\textquotesingle{}}\NormalTok{ users.json}

\CommentTok{\# JSON a tabla HTML}
\ExtensionTok{jq} \AttributeTok{{-}r} \StringTok{\textquotesingle{}.[] | "\textless{}tr\textgreater{}\textless{}td\textgreater{}\textbackslash{}(.name)\textless{}/td\textgreater{}\textless{}td\textgreater{}\textbackslash{}(.age)\textless{}/td\textgreater{}\textless{}/tr\textgreater{}"\textquotesingle{}}\NormalTok{ users.json}

\CommentTok{\# JSON a variables de entorno}
\ExtensionTok{jq} \AttributeTok{{-}r} \StringTok{\textquotesingle{}to\_entries[] | "export \textbackslash{}(.key)=\textbackslash{}(.value)"\textquotesingle{}}\NormalTok{ config.json}
\end{Highlighting}
\end{Shaded}

\begin{center}\rule{0.5\linewidth}{0.5pt}\end{center}

\section{Workflows complejos combinando
herramientas}\label{workflows-complejos-combinando-herramientas}

\subsection{Análisis completo de código
base}\label{anuxe1lisis-completo-de-cuxf3digo-base}

\begin{Shaded}
\begin{Highlighting}[]
\CommentTok{\#!/bin/bash}
\CommentTok{\# Script para analizar un proyecto completo}

\BuiltInTok{echo} \StringTok{"=== ANÁLISIS DEL PROYECTO ==="}

\CommentTok{\# 1. Estadísticas generales}
\BuiltInTok{echo} \StringTok{"Archivos por tipo:"}
\FunctionTok{find}\NormalTok{ . }\AttributeTok{{-}type}\NormalTok{ f }\KeywordTok{|} \FunctionTok{grep} \AttributeTok{{-}E} \StringTok{\textquotesingle{}\textbackslash{}.[a{-}z]+$\textquotesingle{}} \KeywordTok{|} \FunctionTok{sed} \StringTok{\textquotesingle{}s/.*\textbackslash{}.//\textquotesingle{}} \KeywordTok{|} \FunctionTok{sort} \KeywordTok{|} \FunctionTok{uniq} \AttributeTok{{-}c} \KeywordTok{|} \FunctionTok{sort} \AttributeTok{{-}nr}

\CommentTok{\# 2. Funciones más utilizadas}
\BuiltInTok{echo} \AttributeTok{{-}e} \StringTok{"\textbackslash{}nFunciones más referenciadas:"}
\ExtensionTok{rg} \AttributeTok{{-}o} \StringTok{\textquotesingle{}\textbackslash{}w+\textbackslash{}(\textquotesingle{}} \AttributeTok{{-}{-}type}\NormalTok{ js }\KeywordTok{|} \FunctionTok{sed} \StringTok{\textquotesingle{}s/(//\textquotesingle{}} \KeywordTok{|} \FunctionTok{sort} \KeywordTok{|} \FunctionTok{uniq} \AttributeTok{{-}c} \KeywordTok{|} \FunctionTok{sort} \AttributeTok{{-}nr} \KeywordTok{|} \FunctionTok{head} \AttributeTok{{-}10}

\CommentTok{\# 3. TODOs y FIXMEs}
\BuiltInTok{echo} \AttributeTok{{-}e} \StringTok{"\textbackslash{}nTareas pendientes:"}
\ExtensionTok{rg} \AttributeTok{{-}C}\NormalTok{ 1 }\StringTok{"(TODO|FIXME|HACK)"} \KeywordTok{|} \FunctionTok{head} \AttributeTok{{-}20}

\CommentTok{\# 4. Archivos grandes}
\BuiltInTok{echo} \AttributeTok{{-}e} \StringTok{"\textbackslash{}nArchivos más grandes:"}
\FunctionTok{find}\NormalTok{ . }\AttributeTok{{-}type}\NormalTok{ f }\AttributeTok{{-}exec}\NormalTok{ du }\AttributeTok{{-}h}\NormalTok{ \{\} + }\KeywordTok{|} \FunctionTok{sort} \AttributeTok{{-}rh} \KeywordTok{|} \FunctionTok{head} \AttributeTok{{-}10}

\CommentTok{\# 5. Dependencias externas (JS/Python)}
\BuiltInTok{echo} \AttributeTok{{-}e} \StringTok{"\textbackslash{}nImports externos más comunes:"}
\ExtensionTok{rg} \StringTok{"\^{}(import|from).*[\textquotesingle{}}\DataTypeTok{\textbackslash{}"}\StringTok{]([\^{}./].*)[\textquotesingle{}}\DataTypeTok{\textbackslash{}"}\StringTok{]"} \AttributeTok{{-}o} \AttributeTok{{-}{-}type}\NormalTok{ py }\AttributeTok{{-}{-}type}\NormalTok{ js }\KeywordTok{|} \DataTypeTok{\textbackslash{}}
\FunctionTok{cut} \AttributeTok{{-}d}\StringTok{\textquotesingle{}"\textquotesingle{}} \AttributeTok{{-}f2} \KeywordTok{|} \FunctionTok{cut} \AttributeTok{{-}d}\StringTok{"\textquotesingle{}"} \AttributeTok{{-}f2} \KeywordTok{|} \FunctionTok{sort} \KeywordTok{|} \FunctionTok{uniq} \AttributeTok{{-}c} \KeywordTok{|} \FunctionTok{sort} \AttributeTok{{-}nr} \KeywordTok{|} \FunctionTok{head} \AttributeTok{{-}10}
\end{Highlighting}
\end{Shaded}

\subsection{Pipeline de procesamiento de
datos}\label{pipeline-de-procesamiento-de-datos}

\begin{Shaded}
\begin{Highlighting}[]
\CommentTok{\#!/bin/bash}
\CommentTok{\# Procesar logs de servidor web y generar reporte}

\VariableTok{LOG\_FILE}\OperatorTok{=}\StringTok{"/var/log/nginx/access.log"}

\BuiltInTok{echo} \StringTok{"=== REPORTE DE ACCESO WEB ==="}

\CommentTok{\# 1. Top 10 IPs}
\BuiltInTok{echo} \StringTok{"Top 10 IPs por número de requests:"}
\ExtensionTok{rg} \AttributeTok{{-}o} \StringTok{\textquotesingle{}\^{}\textbackslash{}S+\textquotesingle{}} \StringTok{"}\VariableTok{$LOG\_FILE}\StringTok{"} \KeywordTok{|} \FunctionTok{sort} \KeywordTok{|} \FunctionTok{uniq} \AttributeTok{{-}c} \KeywordTok{|} \FunctionTok{sort} \AttributeTok{{-}nr} \KeywordTok{|} \FunctionTok{head} \AttributeTok{{-}10}

\CommentTok{\# 2. Páginas más visitadas}
\BuiltInTok{echo} \AttributeTok{{-}e} \StringTok{"\textbackslash{}nPáginas más visitadas:"}
\ExtensionTok{rg} \AttributeTok{{-}o} \StringTok{\textquotesingle{}"GET ([\^{}"]*)"\textquotesingle{}} \StringTok{"}\VariableTok{$LOG\_FILE}\StringTok{"} \KeywordTok{|} \FunctionTok{cut} \AttributeTok{{-}d}\StringTok{\textquotesingle{} \textquotesingle{}} \AttributeTok{{-}f2} \KeywordTok{|} \FunctionTok{sort} \KeywordTok{|} \FunctionTok{uniq} \AttributeTok{{-}c} \KeywordTok{|} \FunctionTok{sort} \AttributeTok{{-}nr} \KeywordTok{|} \FunctionTok{head} \AttributeTok{{-}10}

\CommentTok{\# 3. Códigos de error}
\BuiltInTok{echo} \AttributeTok{{-}e} \StringTok{"\textbackslash{}nCódigos de respuesta:"}
\ExtensionTok{rg} \AttributeTok{{-}o} \StringTok{\textquotesingle{}" \textbackslash{}d\{3\} \textquotesingle{}} \StringTok{"}\VariableTok{$LOG\_FILE}\StringTok{"} \KeywordTok{|} \FunctionTok{tr} \AttributeTok{{-}d} \StringTok{\textquotesingle{} "\textquotesingle{}} \KeywordTok{|} \FunctionTok{sort} \KeywordTok{|} \FunctionTok{uniq} \AttributeTok{{-}c} \KeywordTok{|} \FunctionTok{sort} \AttributeTok{{-}nr}

\CommentTok{\# 4. User agents más comunes}
\BuiltInTok{echo} \AttributeTok{{-}e} \StringTok{"\textbackslash{}nNavigadores/bots más comunes:"}
\ExtensionTok{rg} \AttributeTok{{-}o} \StringTok{\textquotesingle{}"[\^{}"]*" "[\^{}"]*" "([\^{}"]*)"$\textquotesingle{}} \StringTok{"}\VariableTok{$LOG\_FILE}\StringTok{"} \KeywordTok{|} \DataTypeTok{\textbackslash{}}
\FunctionTok{cut} \AttributeTok{{-}d}\StringTok{\textquotesingle{}"\textquotesingle{}} \AttributeTok{{-}f6} \KeywordTok{|} \FunctionTok{sort} \KeywordTok{|} \FunctionTok{uniq} \AttributeTok{{-}c} \KeywordTok{|} \FunctionTok{sort} \AttributeTok{{-}nr} \KeywordTok{|} \FunctionTok{head} \AttributeTok{{-}10}

\CommentTok{\# 5. Generar JSON para dashboard}
\BuiltInTok{echo} \AttributeTok{{-}e} \StringTok{"\textbackslash{}nGenerando reporte JSON..."}
\KeywordTok{\{}
    \BuiltInTok{echo} \StringTok{\textquotesingle{}\{\textquotesingle{}}
    \BuiltInTok{echo} \StringTok{\textquotesingle{}  "timestamp": "\textquotesingle{}}\VariableTok{$(}\FunctionTok{date} \AttributeTok{{-}Iseconds}\VariableTok{)}\StringTok{\textquotesingle{}",\textquotesingle{}}
    \BuiltInTok{echo} \StringTok{\textquotesingle{}  "total\_requests": \textquotesingle{}}\VariableTok{$(}\FunctionTok{wc} \AttributeTok{{-}l} \OperatorTok{\textless{}} \StringTok{"}\VariableTok{$LOG\_FILE}\StringTok{"}\VariableTok{)}\StringTok{\textquotesingle{},\textquotesingle{}}
    \BuiltInTok{echo} \StringTok{\textquotesingle{}  "unique\_ips": \textquotesingle{}}\VariableTok{$(}\ExtensionTok{rg} \AttributeTok{{-}o} \StringTok{\textquotesingle{}\^{}\textbackslash{}S+\textquotesingle{}} \StringTok{"}\VariableTok{$LOG\_FILE}\StringTok{"} \KeywordTok{|} \FunctionTok{sort} \AttributeTok{{-}u} \KeywordTok{|} \FunctionTok{wc} \AttributeTok{{-}l}\VariableTok{)}\StringTok{\textquotesingle{},\textquotesingle{}}
    \BuiltInTok{echo} \StringTok{\textquotesingle{}  "top\_pages": [\textquotesingle{}}
    \ExtensionTok{rg} \AttributeTok{{-}o} \StringTok{\textquotesingle{}"GET ([\^{}"]*)"\textquotesingle{}} \StringTok{"}\VariableTok{$LOG\_FILE}\StringTok{"} \KeywordTok{|} \FunctionTok{cut} \AttributeTok{{-}d}\StringTok{\textquotesingle{} \textquotesingle{}} \AttributeTok{{-}f2} \KeywordTok{|} \FunctionTok{sort} \KeywordTok{|} \FunctionTok{uniq} \AttributeTok{{-}c} \KeywordTok{|} \FunctionTok{sort} \AttributeTok{{-}nr} \KeywordTok{|} \FunctionTok{head} \AttributeTok{{-}5} \KeywordTok{|} \DataTypeTok{\textbackslash{}}
    \ExtensionTok{jq} \AttributeTok{{-}R} \StringTok{\textquotesingle{}. | split(" ") | \{count: .[0]|tonumber, page: .[1]\}\textquotesingle{}} \KeywordTok{|} \ExtensionTok{jq} \AttributeTok{{-}s}\NormalTok{ .}
    \BuiltInTok{echo} \StringTok{\textquotesingle{}  ]\textquotesingle{}}
    \BuiltInTok{echo} \StringTok{\textquotesingle{}\}\textquotesingle{}}
\KeywordTok{\}} \OperatorTok{\textgreater{}}\NormalTok{ web\_report.json}

\BuiltInTok{echo} \StringTok{"Reporte guardado en web\_report.json"}
\end{Highlighting}
\end{Shaded}

\subsection{Búsqueda interactiva
universal}\label{buxfasqueda-interactiva-universal}

\begin{Shaded}
\begin{Highlighting}[]
\CommentTok{\#!/bin/bash}
\CommentTok{\# Función para búsqueda universal con vista previa}

\FunctionTok{universal\_search()} \KeywordTok{\{}
    \BuiltInTok{local} \VariableTok{search\_term}\OperatorTok{=}\StringTok{"}\VariableTok{$1}\StringTok{"}
    
    \ControlFlowTok{if} \BuiltInTok{[} \OtherTok{{-}z} \StringTok{"}\VariableTok{$search\_term}\StringTok{"} \BuiltInTok{]}\KeywordTok{;} \ControlFlowTok{then}
        \BuiltInTok{echo} \StringTok{"Uso: universal\_search \textless{}término\textgreater{}"}
        \ControlFlowTok{return} \DecValTok{1}
    \ControlFlowTok{fi}
    
    \CommentTok{\# Buscar en archivos de código}
    \BuiltInTok{echo} \StringTok{"=== Archivos de código ==="}
    \ExtensionTok{rg} \AttributeTok{{-}l} \StringTok{"}\VariableTok{$search\_term}\StringTok{"} \AttributeTok{{-}{-}type{-}add} \StringTok{\textquotesingle{}code:*.\{py,js,ts,go,rs,java,cpp,c,h\}\textquotesingle{}} \AttributeTok{{-}t}\NormalTok{ code }\KeywordTok{|} \DataTypeTok{\textbackslash{}}
    \ExtensionTok{fzf} \AttributeTok{{-}{-}preview} \StringTok{"rg {-}{-}color=always \textquotesingle{}}\VariableTok{$search\_term}\StringTok{\textquotesingle{} \{\}"} \DataTypeTok{\textbackslash{}}
        \AttributeTok{{-}{-}preview{-}window}\OperatorTok{=}\NormalTok{right:60\% }\DataTypeTok{\textbackslash{}}
        \AttributeTok{{-}{-}header}\OperatorTok{=}\StringTok{"Archivos de código que contienen \textquotesingle{}}\VariableTok{$search\_term}\StringTok{\textquotesingle{}"}
    
    \CommentTok{\# Buscar en documentación}
    \BuiltInTok{echo} \AttributeTok{{-}e} \StringTok{"\textbackslash{}n=== Documentación ==="}
    \ExtensionTok{rg} \AttributeTok{{-}l} \StringTok{"}\VariableTok{$search\_term}\StringTok{"} \AttributeTok{{-}{-}type{-}add} \StringTok{\textquotesingle{}docs:*.\{md,txt,rst,org\}\textquotesingle{}} \AttributeTok{{-}t}\NormalTok{ docs }\KeywordTok{|} \DataTypeTok{\textbackslash{}}
    \ExtensionTok{fzf} \AttributeTok{{-}{-}preview} \StringTok{"bat {-}{-}color=always \{\}"} \DataTypeTok{\textbackslash{}}
        \AttributeTok{{-}{-}preview{-}window}\OperatorTok{=}\NormalTok{right:60\% }\DataTypeTok{\textbackslash{}}
        \AttributeTok{{-}{-}header}\OperatorTok{=}\StringTok{"Documentación que contiene \textquotesingle{}}\VariableTok{$search\_term}\StringTok{\textquotesingle{}"}
    
    \CommentTok{\# Buscar en configuración}
    \BuiltInTok{echo} \AttributeTok{{-}e} \StringTok{"\textbackslash{}n=== Archivos de configuración ==="}
    \ExtensionTok{rg} \AttributeTok{{-}l} \StringTok{"}\VariableTok{$search\_term}\StringTok{"} \AttributeTok{{-}{-}type{-}add} \StringTok{\textquotesingle{}config:*.\{yaml,yml,json,toml,ini,conf\}\textquotesingle{}} \AttributeTok{{-}t}\NormalTok{ config }\KeywordTok{|} \DataTypeTok{\textbackslash{}}
    \ExtensionTok{fzf} \AttributeTok{{-}{-}preview} \StringTok{"bat {-}{-}color=always \{\}"} \DataTypeTok{\textbackslash{}}
        \AttributeTok{{-}{-}preview{-}window}\OperatorTok{=}\NormalTok{right:60\% }\DataTypeTok{\textbackslash{}}
        \AttributeTok{{-}{-}header}\OperatorTok{=}\StringTok{"Configuración que contiene \textquotesingle{}}\VariableTok{$search\_term}\StringTok{\textquotesingle{}"}
\KeywordTok{\}}

\CommentTok{\# Uso: universal\_search "database"}
\end{Highlighting}
\end{Shaded}

\begin{tcolorbox}[enhanced jigsaw, toprule=.15mm, bottomrule=.15mm, opacityback=0, coltitle=black, rightrule=.15mm, colframe=quarto-callout-tip-color-frame, titlerule=0mm, opacitybacktitle=0.6, left=2mm, colback=white, bottomtitle=1mm, arc=.35mm, leftrule=.75mm, title=\textcolor{quarto-callout-tip-color}{\faLightbulb}\hspace{0.5em}{Tips para búsqueda eficiente}, colbacktitle=quarto-callout-tip-color!10!white, breakable, toptitle=1mm]

\begin{itemize}
\tightlist
\item
  Combina \texttt{rg} con \texttt{fzf} para búsqueda interactiva con
  vista previa
\item
  Usa \texttt{jq} con \texttt{-r} para output raw (sin comillas) cuando
  necesites procesar más
\item
  Configura aliases para combinaciones que uses frecuentemente
\item
  \texttt{ripgrep} respeta \texttt{.gitignore} automáticamente, úsalo en
  proyectos de código
\end{itemize}

\end{tcolorbox}

\begin{tcolorbox}[enhanced jigsaw, toprule=.15mm, bottomrule=.15mm, opacityback=0, coltitle=black, rightrule=.15mm, colframe=quarto-callout-important-color-frame, titlerule=0mm, opacitybacktitle=0.6, left=2mm, colback=white, bottomtitle=1mm, arc=.35mm, leftrule=.75mm, title=\textcolor{quarto-callout-important-color}{\faExclamation}\hspace{0.5em}{Rendimiento}, colbacktitle=quarto-callout-important-color!10!white, breakable, toptitle=1mm]

\begin{itemize}
\tightlist
\item
  \texttt{ripgrep} es extremadamente rápido, pero en directorios muy
  grandes considera usar \texttt{-\/-max-depth}
\item
  \texttt{jq} puede consumir mucha memoria con archivos JSON grandes,
  usa streaming con \texttt{-\/-stream} si es necesario
\item
  \texttt{fzf} es rápido, pero la vista previa puede ser lenta con
  archivos grandes
\end{itemize}

\end{tcolorbox}

En el próximo capítulo exploraremos herramientas esenciales para
desarrollo y control de versiones con Git.

\part{Desarrollo y Git}

\chapter{Desarrollo y Git}\label{desarrollo-y-git-2}

El desarrollo moderno requiere herramientas potentes para control de
versiones, gestión de repositorios y ejecución de código. Esta sección
cubre las herramientas esenciales para cualquier desarrollador
profesional.

\section{git - Control de versiones distribuido}\label{sec-git}

Git es el sistema de control de versiones más usado del mundo,
fundamental para cualquier proyecto de desarrollo moderno.

\subsection{Configuración inicial}\label{configuraciuxf3n-inicial-2}

\begin{Shaded}
\begin{Highlighting}[]
\CommentTok{\# Configuración básica}
\FunctionTok{git}\NormalTok{ config }\AttributeTok{{-}{-}global}\NormalTok{ user.name }\StringTok{"Tu Nombre"}
\FunctionTok{git}\NormalTok{ config }\AttributeTok{{-}{-}global}\NormalTok{ user.email }\StringTok{"tu@email.com"}

\CommentTok{\# Editor por defecto}
\FunctionTok{git}\NormalTok{ config }\AttributeTok{{-}{-}global}\NormalTok{ core.editor }\StringTok{"code {-}{-}wait"}

\CommentTok{\# Configuraciones útiles}
\FunctionTok{git}\NormalTok{ config }\AttributeTok{{-}{-}global}\NormalTok{ init.defaultBranch main}
\FunctionTok{git}\NormalTok{ config }\AttributeTok{{-}{-}global}\NormalTok{ pull.rebase false}
\FunctionTok{git}\NormalTok{ config }\AttributeTok{{-}{-}global}\NormalTok{ push.default simple}
\end{Highlighting}
\end{Shaded}

\subsection{Workflow básico}\label{workflow-buxe1sico}

\subsubsection{Inicialización y
clonado}\label{inicializaciuxf3n-y-clonado}

\begin{Shaded}
\begin{Highlighting}[]
\CommentTok{\# Inicializar nuevo repositorio}
\FunctionTok{git}\NormalTok{ init}
\FunctionTok{git}\NormalTok{ init mi{-}proyecto}

\CommentTok{\# Clonar repositorio existente}
\FunctionTok{git}\NormalTok{ clone https://github.com/usuario/repo.git}
\FunctionTok{git}\NormalTok{ clone git@github.com:usuario/repo.git}

\CommentTok{\# Clonar solo historia reciente (más rápido)}
\FunctionTok{git}\NormalTok{ clone }\AttributeTok{{-}{-}depth}\NormalTok{ 1 https://github.com/usuario/repo.git}
\end{Highlighting}
\end{Shaded}

\subsubsection{Área de staging y
commits}\label{uxe1rea-de-staging-y-commits}

\begin{Shaded}
\begin{Highlighting}[]
\CommentTok{\# Ver estado del repositorio}
\FunctionTok{git}\NormalTok{ status}
\FunctionTok{git}\NormalTok{ status }\AttributeTok{{-}s}  \CommentTok{\# Formato corto}

\CommentTok{\# Agregar archivos al staging}
\FunctionTok{git}\NormalTok{ add archivo.txt}
\FunctionTok{git}\NormalTok{ add .  }\CommentTok{\# Todos los archivos}
\FunctionTok{git}\NormalTok{ add }\PreprocessorTok{*}\NormalTok{.py  }\CommentTok{\# Por patrón}
\FunctionTok{git}\NormalTok{ add }\AttributeTok{{-}A}  \CommentTok{\# Incluir eliminados}

\CommentTok{\# Commits}
\FunctionTok{git}\NormalTok{ commit }\AttributeTok{{-}m} \StringTok{"Mensaje descriptivo"}
\FunctionTok{git}\NormalTok{ commit }\AttributeTok{{-}am} \StringTok{"Agregar y commitear archivos modificados"}
\FunctionTok{git}\NormalTok{ commit }\AttributeTok{{-}{-}amend}  \CommentTok{\# Modificar último commit}
\end{Highlighting}
\end{Shaded}

\subsection{Branching y merging}\label{branching-y-merging}

\subsubsection{Gestión de branches}\label{gestiuxf3n-de-branches}

\begin{Shaded}
\begin{Highlighting}[]
\CommentTok{\# Listar branches}
\FunctionTok{git}\NormalTok{ branch  }\CommentTok{\# Locales}
\FunctionTok{git}\NormalTok{ branch }\AttributeTok{{-}r}  \CommentTok{\# Remotos}
\FunctionTok{git}\NormalTok{ branch }\AttributeTok{{-}a}  \CommentTok{\# Todos}

\CommentTok{\# Crear y cambiar de branch}
\FunctionTok{git}\NormalTok{ branch nueva{-}feature}
\FunctionTok{git}\NormalTok{ checkout nueva{-}feature}
\CommentTok{\# O en un comando:}
\FunctionTok{git}\NormalTok{ checkout }\AttributeTok{{-}b}\NormalTok{ nueva{-}feature}

\CommentTok{\# Cambiar entre branches}
\FunctionTok{git}\NormalTok{ checkout main}
\FunctionTok{git}\NormalTok{ checkout }\AttributeTok{{-}}  \CommentTok{\# Branch anterior}

\CommentTok{\# Eliminar branch}
\FunctionTok{git}\NormalTok{ branch }\AttributeTok{{-}d}\NormalTok{ feature{-}completada}
\FunctionTok{git}\NormalTok{ branch }\AttributeTok{{-}D}\NormalTok{ feature{-}forzar{-}eliminar  }\CommentTok{\# Forzar}
\end{Highlighting}
\end{Shaded}

\subsubsection{Merge strategies}\label{merge-strategies}

\begin{Shaded}
\begin{Highlighting}[]
\CommentTok{\# Merge básico}
\FunctionTok{git}\NormalTok{ checkout main}
\FunctionTok{git}\NormalTok{ merge feature{-}branch}

\CommentTok{\# Merge sin fast{-}forward (siempre crea commit de merge)}
\FunctionTok{git}\NormalTok{ merge }\AttributeTok{{-}{-}no{-}ff}\NormalTok{ feature{-}branch}

\CommentTok{\# Squash merge (combina todos los commits en uno)}
\FunctionTok{git}\NormalTok{ merge }\AttributeTok{{-}{-}squash}\NormalTok{ feature{-}branch}
\FunctionTok{git}\NormalTok{ commit }\AttributeTok{{-}m} \StringTok{"Feature: descripción completa"}

\CommentTok{\# Rebase (historia lineal)}
\FunctionTok{git}\NormalTok{ checkout feature{-}branch}
\FunctionTok{git}\NormalTok{ rebase main}
\FunctionTok{git}\NormalTok{ checkout main}
\FunctionTok{git}\NormalTok{ merge feature{-}branch  }\CommentTok{\# Fast{-}forward}
\end{Highlighting}
\end{Shaded}

\subsection{Historial y navegación}\label{historial-y-navegaciuxf3n}

\subsubsection{Visualización de
historia}\label{visualizaciuxf3n-de-historia}

\begin{Shaded}
\begin{Highlighting}[]
\CommentTok{\# Log básico}
\FunctionTok{git}\NormalTok{ log}
\FunctionTok{git}\NormalTok{ log }\AttributeTok{{-}{-}oneline}  \CommentTok{\# Compacto}
\FunctionTok{git}\NormalTok{ log }\AttributeTok{{-}{-}graph} \AttributeTok{{-}{-}oneline} \AttributeTok{{-}{-}all}  \CommentTok{\# Gráfico}

\CommentTok{\# Historial de un archivo}
\FunctionTok{git}\NormalTok{ log }\AttributeTok{{-}{-}follow}\NormalTok{ archivo.txt}
\FunctionTok{git}\NormalTok{ log }\AttributeTok{{-}p}\NormalTok{ archivo.txt  }\CommentTok{\# Con diferencias}

\CommentTok{\# Buscar en commits}
\FunctionTok{git}\NormalTok{ log }\AttributeTok{{-}{-}grep}\OperatorTok{=}\StringTok{"fix"}  \CommentTok{\# Por mensaje}
\FunctionTok{git}\NormalTok{ log }\AttributeTok{{-}{-}author}\OperatorTok{=}\StringTok{"Juan"}  \CommentTok{\# Por autor}
\FunctionTok{git}\NormalTok{ log }\AttributeTok{{-}{-}since}\OperatorTok{=}\StringTok{"2 weeks ago"}  \CommentTok{\# Por fecha}
\end{Highlighting}
\end{Shaded}

\subsubsection{Diferencias y
comparaciones}\label{diferencias-y-comparaciones}

\begin{Shaded}
\begin{Highlighting}[]
\CommentTok{\# Ver cambios no commiteados}
\FunctionTok{git}\NormalTok{ diff}
\FunctionTok{git}\NormalTok{ diff }\AttributeTok{{-}{-}staged}  \CommentTok{\# En staging area}

\CommentTok{\# Comparar branches}
\FunctionTok{git}\NormalTok{ diff main..feature{-}branch}
\FunctionTok{git}\NormalTok{ diff main...feature{-}branch  }\CommentTok{\# Desde punto común}

\CommentTok{\# Comparar commits específicos}
\FunctionTok{git}\NormalTok{ diff abc123 def456}
\FunctionTok{git}\NormalTok{ diff HEAD\textasciitilde{}1 HEAD  }\CommentTok{\# Último commit vs anterior}
\end{Highlighting}
\end{Shaded}

\subsection{Casos de uso avanzados}\label{casos-de-uso-avanzados-5}

\subsubsection{Trabajo con remotes}\label{trabajo-con-remotes}

\begin{Shaded}
\begin{Highlighting}[]
\CommentTok{\# Configurar remotes}
\FunctionTok{git}\NormalTok{ remote add origin https://github.com/usuario/repo.git}
\FunctionTok{git}\NormalTok{ remote }\AttributeTok{{-}v}  \CommentTok{\# Ver remotes configurados}

\CommentTok{\# Fetch y pull}
\FunctionTok{git}\NormalTok{ fetch origin  }\CommentTok{\# Descargar cambios sin merge}
\FunctionTok{git}\NormalTok{ pull origin main  }\CommentTok{\# Fetch + merge}
\FunctionTok{git}\NormalTok{ pull }\AttributeTok{{-}{-}rebase}\NormalTok{ origin main  }\CommentTok{\# Fetch + rebase}

\CommentTok{\# Push}
\FunctionTok{git}\NormalTok{ push origin main}
\FunctionTok{git}\NormalTok{ push }\AttributeTok{{-}u}\NormalTok{ origin feature{-}branch  }\CommentTok{\# Primera vez}
\FunctionTok{git}\NormalTok{ push }\AttributeTok{{-}{-}force{-}with{-}lease}\NormalTok{ origin main  }\CommentTok{\# Force seguro}
\end{Highlighting}
\end{Shaded}

\subsubsection{Stashing (guardado
temporal)}\label{stashing-guardado-temporal}

\begin{Shaded}
\begin{Highlighting}[]
\CommentTok{\# Guardar cambios temporalmente}
\FunctionTok{git}\NormalTok{ stash}
\FunctionTok{git}\NormalTok{ stash push }\AttributeTok{{-}m} \StringTok{"Trabajo en progreso"}

\CommentTok{\# Ver stashes}
\FunctionTok{git}\NormalTok{ stash list}

\CommentTok{\# Aplicar stash}
\FunctionTok{git}\NormalTok{ stash pop  }\CommentTok{\# Aplicar y eliminar}
\FunctionTok{git}\NormalTok{ stash apply  }\CommentTok{\# Aplicar sin eliminar}
\FunctionTok{git}\NormalTok{ stash apply stash@\{1\}  }\CommentTok{\# Stash específico}

\CommentTok{\# Eliminar stashes}
\FunctionTok{git}\NormalTok{ stash drop stash@\{0\}}
\FunctionTok{git}\NormalTok{ stash clear  }\CommentTok{\# Todos}
\end{Highlighting}
\end{Shaded}

\subsubsection{Revertir cambios}\label{revertir-cambios}

\begin{Shaded}
\begin{Highlighting}[]
\CommentTok{\# Descartar cambios no commiteados}
\FunctionTok{git}\NormalTok{ checkout }\AttributeTok{{-}{-}}\NormalTok{ archivo.txt}
\FunctionTok{git}\NormalTok{ restore archivo.txt  }\CommentTok{\# Git 2.23+}

\CommentTok{\# Quitar de staging area}
\FunctionTok{git}\NormalTok{ reset HEAD archivo.txt}
\FunctionTok{git}\NormalTok{ restore }\AttributeTok{{-}{-}staged}\NormalTok{ archivo.txt  }\CommentTok{\# Git 2.23+}

\CommentTok{\# Revertir commits}
\FunctionTok{git}\NormalTok{ revert abc123  }\CommentTok{\# Crear commit que deshace otro}
\FunctionTok{git}\NormalTok{ reset }\AttributeTok{{-}{-}soft}\NormalTok{ HEAD\textasciitilde{}1  }\CommentTok{\# Deshacer commit, mantener cambios}
\FunctionTok{git}\NormalTok{ reset }\AttributeTok{{-}{-}hard}\NormalTok{ HEAD\textasciitilde{}1  }\CommentTok{\# Deshacer commit y cambios (peligroso)}
\end{Highlighting}
\end{Shaded}

\subsection{Workflows de equipo}\label{workflows-de-equipo}

\subsubsection{Feature branch workflow}\label{feature-branch-workflow}

\begin{Shaded}
\begin{Highlighting}[]
\CommentTok{\# 1. Crear feature branch desde main actualizado}
\FunctionTok{git}\NormalTok{ checkout main}
\FunctionTok{git}\NormalTok{ pull origin main}
\FunctionTok{git}\NormalTok{ checkout }\AttributeTok{{-}b}\NormalTok{ feature/nueva{-}funcionalidad}

\CommentTok{\# 2. Desarrollar y commitear}
\FunctionTok{git}\NormalTok{ add .}
\FunctionTok{git}\NormalTok{ commit }\AttributeTok{{-}m} \StringTok{"feat: implementar nueva funcionalidad"}

\CommentTok{\# 3. Actualizar con cambios de main}
\FunctionTok{git}\NormalTok{ fetch origin}
\FunctionTok{git}\NormalTok{ rebase origin/main}

\CommentTok{\# 4. Push de feature branch}
\FunctionTok{git}\NormalTok{ push }\AttributeTok{{-}u}\NormalTok{ origin feature/nueva{-}funcionalidad}

\CommentTok{\# 5. Crear Pull Request en GitHub/GitLab}
\CommentTok{\# 6. Después del merge, limpiar}
\FunctionTok{git}\NormalTok{ checkout main}
\FunctionTok{git}\NormalTok{ pull origin main}
\FunctionTok{git}\NormalTok{ branch }\AttributeTok{{-}d}\NormalTok{ feature/nueva{-}funcionalidad}
\end{Highlighting}
\end{Shaded}

\subsubsection{Git flow}\label{git-flow}

\begin{Shaded}
\begin{Highlighting}[]
\CommentTok{\# Branches principales: main, develop}
\CommentTok{\# Branches de soporte: feature/, release/, hotfix/}

\CommentTok{\# Iniciar feature}
\FunctionTok{git}\NormalTok{ checkout develop}
\FunctionTok{git}\NormalTok{ checkout }\AttributeTok{{-}b}\NormalTok{ feature/nueva{-}caracteristica}

\CommentTok{\# Finalizar feature}
\FunctionTok{git}\NormalTok{ checkout develop}
\FunctionTok{git}\NormalTok{ merge }\AttributeTok{{-}{-}no{-}ff}\NormalTok{ feature/nueva{-}caracteristica}
\FunctionTok{git}\NormalTok{ branch }\AttributeTok{{-}d}\NormalTok{ feature/nueva{-}caracteristica}

\CommentTok{\# Release branch}
\FunctionTok{git}\NormalTok{ checkout }\AttributeTok{{-}b}\NormalTok{ release/1.2.0 develop}
\CommentTok{\# Fix bugs, actualizar versión}
\FunctionTok{git}\NormalTok{ checkout main}
\FunctionTok{git}\NormalTok{ merge }\AttributeTok{{-}{-}no{-}ff}\NormalTok{ release/1.2.0}
\FunctionTok{git}\NormalTok{ tag }\AttributeTok{{-}a}\NormalTok{ v1.2.0 }\AttributeTok{{-}m} \StringTok{"Version 1.2.0"}
\FunctionTok{git}\NormalTok{ checkout develop}
\FunctionTok{git}\NormalTok{ merge }\AttributeTok{{-}{-}no{-}ff}\NormalTok{ release/1.2.0}
\end{Highlighting}
\end{Shaded}

\begin{center}\rule{0.5\linewidth}{0.5pt}\end{center}

\section{gh - GitHub CLI}\label{sec-gh}

La CLI oficial de GitHub permite gestionar repositorios, pull requests,
issues y más desde terminal.

\subsection{Autenticación}\label{autenticaciuxf3n}

\begin{Shaded}
\begin{Highlighting}[]
\CommentTok{\# Login interactivo}
\ExtensionTok{gh}\NormalTok{ auth login}

\CommentTok{\# Verificar autenticación}
\ExtensionTok{gh}\NormalTok{ auth status}

\CommentTok{\# Login con token}
\BuiltInTok{echo} \StringTok{"ghp\_xxxxxxxxxxxx"} \KeywordTok{|} \ExtensionTok{gh}\NormalTok{ auth login }\AttributeTok{{-}{-}with{-}token}
\end{Highlighting}
\end{Shaded}

\subsection{Gestión de repositorios}\label{gestiuxf3n-de-repositorios}

\subsubsection{Crear y clonar}\label{crear-y-clonar}

\begin{Shaded}
\begin{Highlighting}[]
\CommentTok{\# Crear repositorio nuevo}
\ExtensionTok{gh}\NormalTok{ repo create mi{-}proyecto}
\ExtensionTok{gh}\NormalTok{ repo create mi{-}proyecto }\AttributeTok{{-}{-}public}
\ExtensionTok{gh}\NormalTok{ repo create mi{-}proyecto }\AttributeTok{{-}{-}private} \AttributeTok{{-}{-}clone}

\CommentTok{\# Clonar repositorio}
\ExtensionTok{gh}\NormalTok{ repo clone usuario/repositorio}
\ExtensionTok{gh}\NormalTok{ repo clone https://github.com/usuario/repo}

\CommentTok{\# Ver información del repo}
\ExtensionTok{gh}\NormalTok{ repo view}
\ExtensionTok{gh}\NormalTok{ repo view usuario/repositorio}
\end{Highlighting}
\end{Shaded}

\subsubsection{Gestión básica}\label{gestiuxf3n-buxe1sica}

\begin{Shaded}
\begin{Highlighting}[]
\CommentTok{\# Listar repositorios}
\ExtensionTok{gh}\NormalTok{ repo list  }\CommentTok{\# Propios}
\ExtensionTok{gh}\NormalTok{ repo list usuario  }\CommentTok{\# De un usuario}
\ExtensionTok{gh}\NormalTok{ repo list }\AttributeTok{{-}{-}limit}\NormalTok{ 50}

\CommentTok{\# Fork de repositorio}
\ExtensionTok{gh}\NormalTok{ repo fork usuario/repositorio}
\ExtensionTok{gh}\NormalTok{ repo fork usuario/repositorio }\AttributeTok{{-}{-}clone}

\CommentTok{\# Sincronizar fork}
\ExtensionTok{gh}\NormalTok{ repo sync usuario/mi{-}fork}
\end{Highlighting}
\end{Shaded}

\subsection{Pull Requests}\label{pull-requests}

\subsubsection{Crear y gestionar PRs}\label{crear-y-gestionar-prs}

\begin{Shaded}
\begin{Highlighting}[]
\CommentTok{\# Crear PR desde branch actual}
\ExtensionTok{gh}\NormalTok{ pr create}
\ExtensionTok{gh}\NormalTok{ pr create }\AttributeTok{{-}{-}title} \StringTok{"Nueva feature"} \AttributeTok{{-}{-}body} \StringTok{"Descripción detallada"}

\CommentTok{\# Crear PR con template}
\ExtensionTok{gh}\NormalTok{ pr create }\AttributeTok{{-}{-}template}\OperatorTok{=}\NormalTok{.github/pull\_request\_template.md}

\CommentTok{\# Listar PRs}
\ExtensionTok{gh}\NormalTok{ pr list}
\ExtensionTok{gh}\NormalTok{ pr list }\AttributeTok{{-}{-}state}\OperatorTok{=}\NormalTok{open}
\ExtensionTok{gh}\NormalTok{ pr list }\AttributeTok{{-}{-}author}\OperatorTok{=}\NormalTok{@me}

\CommentTok{\# Ver detalles de PR}
\ExtensionTok{gh}\NormalTok{ pr view 123}
\ExtensionTok{gh}\NormalTok{ pr view }\AttributeTok{{-}{-}web}\NormalTok{ 123  }\CommentTok{\# Abrir en navegador}
\end{Highlighting}
\end{Shaded}

\subsubsection{Review y merge}\label{review-y-merge}

\begin{Shaded}
\begin{Highlighting}[]
\CommentTok{\# Checkout PR localmente}
\ExtensionTok{gh}\NormalTok{ pr checkout 123}

\CommentTok{\# Aprobar PR}
\ExtensionTok{gh}\NormalTok{ pr review 123 }\AttributeTok{{-}{-}approve}
\ExtensionTok{gh}\NormalTok{ pr review 123 }\AttributeTok{{-}{-}approve} \AttributeTok{{-}{-}body} \StringTok{"LGTM! 🚀"}

\CommentTok{\# Solicitar cambios}
\ExtensionTok{gh}\NormalTok{ pr review 123 }\AttributeTok{{-}{-}request{-}changes} \AttributeTok{{-}{-}body} \StringTok{"Necesita documentación"}

\CommentTok{\# Merge PR}
\ExtensionTok{gh}\NormalTok{ pr merge 123}
\ExtensionTok{gh}\NormalTok{ pr merge 123 }\AttributeTok{{-}{-}merge}  \CommentTok{\# Merge commit}
\ExtensionTok{gh}\NormalTok{ pr merge 123 }\AttributeTok{{-}{-}squash}  \CommentTok{\# Squash merge}
\ExtensionTok{gh}\NormalTok{ pr merge 123 }\AttributeTok{{-}{-}rebase}  \CommentTok{\# Rebase merge}
\end{Highlighting}
\end{Shaded}

\subsection{Issues}\label{issues}

\subsubsection{Gestión de issues}\label{gestiuxf3n-de-issues}

\begin{Shaded}
\begin{Highlighting}[]
\CommentTok{\# Crear issue}
\ExtensionTok{gh}\NormalTok{ issue create}
\ExtensionTok{gh}\NormalTok{ issue create }\AttributeTok{{-}{-}title} \StringTok{"Bug: error en login"} \AttributeTok{{-}{-}body} \StringTok{"Descripción del bug"}

\CommentTok{\# Listar issues}
\ExtensionTok{gh}\NormalTok{ issue list}
\ExtensionTok{gh}\NormalTok{ issue list }\AttributeTok{{-}{-}state}\OperatorTok{=}\NormalTok{open }\AttributeTok{{-}{-}assignee}\OperatorTok{=}\NormalTok{@me}
\ExtensionTok{gh}\NormalTok{ issue list }\AttributeTok{{-}{-}label}\OperatorTok{=}\NormalTok{bug}

\CommentTok{\# Ver y editar issue}
\ExtensionTok{gh}\NormalTok{ issue view 456}
\ExtensionTok{gh}\NormalTok{ issue edit 456 }\AttributeTok{{-}{-}add{-}label}\OperatorTok{=}\NormalTok{priority{-}high}
\ExtensionTok{gh}\NormalTok{ issue close 456}
\end{Highlighting}
\end{Shaded}

\subsubsection{Plantillas y automation}\label{plantillas-y-automation}

\begin{Shaded}
\begin{Highlighting}[]
\CommentTok{\# Crear issue con template}
\ExtensionTok{gh}\NormalTok{ issue create }\AttributeTok{{-}{-}template}\OperatorTok{=}\NormalTok{bug\_report.md}

\CommentTok{\# Asignar issue}
\ExtensionTok{gh}\NormalTok{ issue edit 456 }\AttributeTok{{-}{-}assignee}\OperatorTok{=}\NormalTok{usuario}

\CommentTok{\# Comentar en issue}
\ExtensionTok{gh}\NormalTok{ issue comment 456 }\AttributeTok{{-}{-}body} \StringTok{"Trabajando en esto"}
\end{Highlighting}
\end{Shaded}

\subsection{Actions y CI/CD}\label{actions-y-cicd}

\begin{Shaded}
\begin{Highlighting}[]
\CommentTok{\# Ver workflows}
\ExtensionTok{gh}\NormalTok{ workflow list}
\ExtensionTok{gh}\NormalTok{ workflow view ci.yml}

\CommentTok{\# Ver runs}
\ExtensionTok{gh}\NormalTok{ run list}
\ExtensionTok{gh}\NormalTok{ run view 123456}

\CommentTok{\# Cancelar run}
\ExtensionTok{gh}\NormalTok{ run cancel 123456}

\CommentTok{\# Re{-}ejecutar workflow}
\ExtensionTok{gh}\NormalTok{ run rerun 123456}
\end{Highlighting}
\end{Shaded}

\subsection{Casos de uso avanzados}\label{casos-de-uso-avanzados-6}

\subsubsection{Automatización con
scripts}\label{automatizaciuxf3n-con-scripts}

\begin{Shaded}
\begin{Highlighting}[]
\CommentTok{\#!/bin/bash}
\CommentTok{\# Script para crear PR automaticamente}

\CommentTok{\# Verificar que estamos en un branch que no sea main}
\VariableTok{current\_branch}\OperatorTok{=}\VariableTok{$(}\FunctionTok{git}\NormalTok{ branch }\AttributeTok{{-}{-}show{-}current}\VariableTok{)}
\ControlFlowTok{if} \BuiltInTok{[} \StringTok{"}\VariableTok{$current\_branch}\StringTok{"} \OtherTok{=} \StringTok{"main"} \BuiltInTok{]}\KeywordTok{;} \ControlFlowTok{then}
    \BuiltInTok{echo} \StringTok{"No puedes crear PR desde main"}
    \BuiltInTok{exit}\NormalTok{ 1}
\ControlFlowTok{fi}

\CommentTok{\# Push del branch actual}
\FunctionTok{git}\NormalTok{ push }\AttributeTok{{-}u}\NormalTok{ origin }\StringTok{"}\VariableTok{$current\_branch}\StringTok{"}

\CommentTok{\# Crear PR con información del branch}
\VariableTok{title}\OperatorTok{=}\VariableTok{$(}\BuiltInTok{echo} \StringTok{"}\VariableTok{$current\_branch}\StringTok{"} \KeywordTok{|} \FunctionTok{sed} \StringTok{\textquotesingle{}s/{-}/ /g\textquotesingle{}} \KeywordTok{|} \FunctionTok{sed} \StringTok{\textquotesingle{}s/feature\textbackslash{}///g\textquotesingle{}}\VariableTok{)}
\ExtensionTok{gh}\NormalTok{ pr create }\AttributeTok{{-}{-}title} \StringTok{"}\VariableTok{$title}\StringTok{"} \AttributeTok{{-}{-}body} \StringTok{"Resolves \#issue\_number"}

\BuiltInTok{echo} \StringTok{"PR creado: }\VariableTok{$(}\ExtensionTok{gh}\NormalTok{ pr view }\AttributeTok{{-}{-}json}\NormalTok{ url }\AttributeTok{{-}{-}jq}\NormalTok{ .url}\VariableTok{)}\StringTok{"}
\end{Highlighting}
\end{Shaded}

\subsubsection{Release automation}\label{release-automation}

\begin{Shaded}
\begin{Highlighting}[]
\CommentTok{\#!/bin/bash}
\CommentTok{\# Script para crear release automáticamente}

\VariableTok{version}\OperatorTok{=}\StringTok{"}\VariableTok{$1}\StringTok{"}
\ControlFlowTok{if} \BuiltInTok{[} \OtherTok{{-}z} \StringTok{"}\VariableTok{$version}\StringTok{"} \BuiltInTok{]}\KeywordTok{;} \ControlFlowTok{then}
    \BuiltInTok{echo} \StringTok{"Uso: }\VariableTok{$0}\StringTok{ \textless{}version\textgreater{}"}
    \BuiltInTok{exit}\NormalTok{ 1}
\ControlFlowTok{fi}

\CommentTok{\# Crear tag}
\FunctionTok{git}\NormalTok{ tag }\AttributeTok{{-}a} \StringTok{"v}\VariableTok{$version}\StringTok{"} \AttributeTok{{-}m} \StringTok{"Version }\VariableTok{$version}\StringTok{"}
\FunctionTok{git}\NormalTok{ push origin }\StringTok{"v}\VariableTok{$version}\StringTok{"}

\CommentTok{\# Generar changelog automático}
\VariableTok{changelog}\OperatorTok{=}\VariableTok{$(}\ExtensionTok{gh}\NormalTok{ api repos/:owner/:repo/releases/generate{-}notes }\DataTypeTok{\textbackslash{}}
    \AttributeTok{{-}f}\NormalTok{ tag\_name=}\StringTok{"v}\VariableTok{$version}\StringTok{"} \DataTypeTok{\textbackslash{}}
    \AttributeTok{{-}{-}jq}\NormalTok{ .body}\VariableTok{)}

\CommentTok{\# Crear release}
\ExtensionTok{gh}\NormalTok{ release create }\StringTok{"v}\VariableTok{$version}\StringTok{"} \DataTypeTok{\textbackslash{}}
    \AttributeTok{{-}{-}title} \StringTok{"Release v}\VariableTok{$version}\StringTok{"} \DataTypeTok{\textbackslash{}}
    \AttributeTok{{-}{-}notes} \StringTok{"}\VariableTok{$changelog}\StringTok{"} \DataTypeTok{\textbackslash{}}
    \AttributeTok{{-}{-}draft}\OperatorTok{=}\NormalTok{false }\DataTypeTok{\textbackslash{}}
    \AttributeTok{{-}{-}prerelease}\OperatorTok{=}\NormalTok{false}

\BuiltInTok{echo} \StringTok{"Release v}\VariableTok{$version}\StringTok{ creado exitosamente!"}
\end{Highlighting}
\end{Shaded}

\begin{center}\rule{0.5\linewidth}{0.5pt}\end{center}

\section{node - JavaScript Runtime}\label{sec-node}

Node.js permite ejecutar JavaScript fuera del navegador, siendo
fundamental para desarrollo web moderno.

\subsection{Gestión de versiones con
nvm}\label{gestiuxf3n-de-versiones-con-nvm}

\begin{Shaded}
\begin{Highlighting}[]
\CommentTok{\# Instalar versión específica}
\ExtensionTok{nvm}\NormalTok{ install 18.17.0}
\ExtensionTok{nvm}\NormalTok{ install }\AttributeTok{{-}{-}lts}  \CommentTok{\# Última LTS}

\CommentTok{\# Usar versión específica}
\ExtensionTok{nvm}\NormalTok{ use 18.17.0}
\ExtensionTok{nvm}\NormalTok{ use }\AttributeTok{{-}{-}lts}

\CommentTok{\# Ver versiones instaladas}
\ExtensionTok{nvm}\NormalTok{ list}
\ExtensionTok{nvm}\NormalTok{ list{-}remote  }\CommentTok{\# Disponibles para instalar}

\CommentTok{\# Establecer versión por defecto}
\ExtensionTok{nvm}\NormalTok{ alias default 18.17.0}
\end{Highlighting}
\end{Shaded}

\subsection{npm - Node Package
Manager}\label{npm---node-package-manager}

\subsubsection{Gestión de paquetes}\label{gestiuxf3n-de-paquetes}

\begin{Shaded}
\begin{Highlighting}[]
\CommentTok{\# Inicializar proyecto}
\ExtensionTok{npm}\NormalTok{ init}
\ExtensionTok{npm}\NormalTok{ init }\AttributeTok{{-}y}  \CommentTok{\# Con valores por defecto}

\CommentTok{\# Instalar paquetes}
\ExtensionTok{npm}\NormalTok{ install express}
\ExtensionTok{npm}\NormalTok{ install }\AttributeTok{{-}D}\NormalTok{ jest  }\CommentTok{\# Como devDependency}
\ExtensionTok{npm}\NormalTok{ install }\AttributeTok{{-}g}\NormalTok{ nodemon  }\CommentTok{\# Global}

\CommentTok{\# Instalar desde package.json}
\ExtensionTok{npm}\NormalTok{ install}
\ExtensionTok{npm}\NormalTok{ ci  }\CommentTok{\# Para CI/CD (más rápido, determinístico)}
\end{Highlighting}
\end{Shaded}

\subsubsection{Scripts y automation}\label{scripts-y-automation}

\begin{Shaded}
\begin{Highlighting}[]
\CommentTok{\# En package.json}
\KeywordTok{\{}
  \StringTok{"scripts"}\ExtensionTok{:}\NormalTok{ \{}
    \StringTok{"start"}\ExtensionTok{:} \StringTok{"node server.js"}\NormalTok{,}
    \StringTok{"dev"}\ExtensionTok{:} \StringTok{"nodemon server.js"}\NormalTok{,}
    \StringTok{"test"}\ExtensionTok{:} \StringTok{"jest"}\NormalTok{,}
    \StringTok{"build"}\ExtensionTok{:} \StringTok{"webpack {-}{-}mode=production"}\NormalTok{,}
    \StringTok{"lint"}\ExtensionTok{:} \StringTok{"eslint src/"}\NormalTok{,}
    \StringTok{"format"}\ExtensionTok{:} \StringTok{"prettier {-}{-}write src/"}
  \KeywordTok{\}}
\ErrorTok{\}}

\CommentTok{\# Ejecutar scripts}
\ExtensionTok{npm}\NormalTok{ start}
\ExtensionTok{npm}\NormalTok{ run dev}
\ExtensionTok{npm}\NormalTok{ test}
\ExtensionTok{npm}\NormalTok{ run build}
\end{Highlighting}
\end{Shaded}

\subsection{Desarrollo con Node.js}\label{desarrollo-con-node.js}

\subsubsection{REPL y debugging}\label{repl-y-debugging}

\begin{Shaded}
\begin{Highlighting}[]
\CommentTok{\# REPL interactivo}
\ExtensionTok{node}

\CommentTok{\# Ejecutar archivo}
\ExtensionTok{node}\NormalTok{ app.js}

\CommentTok{\# Debugging}
\ExtensionTok{node} \AttributeTok{{-}{-}inspect}\NormalTok{ app.js}
\ExtensionTok{node} \AttributeTok{{-}{-}inspect{-}brk}\NormalTok{ app.js  }\CommentTok{\# Pausar en primera línea}

\CommentTok{\# Profiling}
\ExtensionTok{node} \AttributeTok{{-}{-}prof}\NormalTok{ app.js}
\ExtensionTok{node} \AttributeTok{{-}{-}prof{-}process}\NormalTok{ isolate{-}}\PreprocessorTok{*}\NormalTok{.log}
\end{Highlighting}
\end{Shaded}

\subsubsection{Módulos y require}\label{muxf3dulos-y-require}

\begin{Shaded}
\begin{Highlighting}[]
\CommentTok{// CommonJS (Node.js tradicional)}
\KeywordTok{const}\NormalTok{ fs }\OperatorTok{=} \PreprocessorTok{require}\NormalTok{(}\StringTok{\textquotesingle{}fs\textquotesingle{}}\NormalTok{)}\OperatorTok{;}
\KeywordTok{const}\NormalTok{ path }\OperatorTok{=} \PreprocessorTok{require}\NormalTok{(}\StringTok{\textquotesingle{}path\textquotesingle{}}\NormalTok{)}\OperatorTok{;}
\KeywordTok{const}\NormalTok{ \{ promisify \} }\OperatorTok{=} \PreprocessorTok{require}\NormalTok{(}\StringTok{\textquotesingle{}util\textquotesingle{}}\NormalTok{)}\OperatorTok{;}

\CommentTok{// ES Modules (Node.js moderno)}
\ImportTok{import}\NormalTok{ fs }\ImportTok{from} \StringTok{\textquotesingle{}fs/promises\textquotesingle{}}\OperatorTok{;}
\ImportTok{import}\NormalTok{ path }\ImportTok{from} \StringTok{\textquotesingle{}path\textquotesingle{}}\OperatorTok{;}
\ImportTok{import}\NormalTok{ \{ fileURLToPath \} }\ImportTok{from} \StringTok{\textquotesingle{}url\textquotesingle{}}\OperatorTok{;}
\end{Highlighting}
\end{Shaded}

\subsection{Casos de uso prácticos}\label{casos-de-uso-pruxe1cticos-1}

\subsubsection{Script de
automatización}\label{script-de-automatizaciuxf3n}

\begin{Shaded}
\begin{Highlighting}[]
\CommentTok{\#!/usr/bin/env node}
\CommentTok{// build{-}tool.js}
\KeywordTok{const}\NormalTok{ fs }\OperatorTok{=} \PreprocessorTok{require}\NormalTok{(}\StringTok{\textquotesingle{}fs/promises\textquotesingle{}}\NormalTok{)}\OperatorTok{;}
\KeywordTok{const}\NormalTok{ path }\OperatorTok{=} \PreprocessorTok{require}\NormalTok{(}\StringTok{\textquotesingle{}path\textquotesingle{}}\NormalTok{)}\OperatorTok{;}
\KeywordTok{const}\NormalTok{ \{ execSync \} }\OperatorTok{=} \PreprocessorTok{require}\NormalTok{(}\StringTok{\textquotesingle{}child\_process\textquotesingle{}}\NormalTok{)}\OperatorTok{;}

\KeywordTok{async} \KeywordTok{function} \FunctionTok{buildProject}\NormalTok{() \{}
    \BuiltInTok{console}\OperatorTok{.}\FunctionTok{log}\NormalTok{(}\StringTok{\textquotesingle{}🚀 Iniciando build...\textquotesingle{}}\NormalTok{)}\OperatorTok{;}
    
    \CommentTok{// Limpiar directorio dist}
    \ControlFlowTok{await}\NormalTok{ fs}\OperatorTok{.}\FunctionTok{rmdir}\NormalTok{(}\StringTok{\textquotesingle{}./dist\textquotesingle{}}\OperatorTok{,}\NormalTok{ \{ }\DataTypeTok{recursive}\OperatorTok{:} \KeywordTok{true}\NormalTok{ \})}\OperatorTok{.}\FunctionTok{catch}\NormalTok{(() }\KeywordTok{=\textgreater{}}\NormalTok{ \{\})}\OperatorTok{;}
    \ControlFlowTok{await}\NormalTok{ fs}\OperatorTok{.}\FunctionTok{mkdir}\NormalTok{(}\StringTok{\textquotesingle{}./dist\textquotesingle{}}\OperatorTok{,}\NormalTok{ \{ }\DataTypeTok{recursive}\OperatorTok{:} \KeywordTok{true}\NormalTok{ \})}\OperatorTok{;}
    
    \CommentTok{// Compilar TypeScript}
    \BuiltInTok{console}\OperatorTok{.}\FunctionTok{log}\NormalTok{(}\StringTok{\textquotesingle{}📝 Compilando TypeScript...\textquotesingle{}}\NormalTok{)}\OperatorTok{;}
    \FunctionTok{execSync}\NormalTok{(}\StringTok{\textquotesingle{}tsc {-}{-}outDir dist\textquotesingle{}}\NormalTok{)}\OperatorTok{;}
    
    \CommentTok{// Copiar assets}
    \BuiltInTok{console}\OperatorTok{.}\FunctionTok{log}\NormalTok{(}\StringTok{\textquotesingle{}📁 Copiando assets...\textquotesingle{}}\NormalTok{)}\OperatorTok{;}
    \ControlFlowTok{await}\NormalTok{ fs}\OperatorTok{.}\FunctionTok{cp}\NormalTok{(}\StringTok{\textquotesingle{}./src/assets\textquotesingle{}}\OperatorTok{,} \StringTok{\textquotesingle{}./dist/assets\textquotesingle{}}\OperatorTok{,}\NormalTok{ \{ }\DataTypeTok{recursive}\OperatorTok{:} \KeywordTok{true}\NormalTok{ \})}\OperatorTok{;}
    
    \CommentTok{// Generar package.json para producción}
    \KeywordTok{const}\NormalTok{ pkg }\OperatorTok{=} \BuiltInTok{JSON}\OperatorTok{.}\FunctionTok{parse}\NormalTok{(}\ControlFlowTok{await}\NormalTok{ fs}\OperatorTok{.}\FunctionTok{readFile}\NormalTok{(}\StringTok{\textquotesingle{}./package.json\textquotesingle{}}\OperatorTok{,} \StringTok{\textquotesingle{}utf8\textquotesingle{}}\NormalTok{))}\OperatorTok{;}
    \KeywordTok{const}\NormalTok{ prodPkg }\OperatorTok{=}\NormalTok{ \{}
        \DataTypeTok{name}\OperatorTok{:}\NormalTok{ pkg}\OperatorTok{.}\AttributeTok{name}\OperatorTok{,}
        \DataTypeTok{version}\OperatorTok{:}\NormalTok{ pkg}\OperatorTok{.}\AttributeTok{version}\OperatorTok{,}
        \DataTypeTok{dependencies}\OperatorTok{:}\NormalTok{ pkg}\OperatorTok{.}\AttributeTok{dependencies}\OperatorTok{,}
        \DataTypeTok{main}\OperatorTok{:} \StringTok{\textquotesingle{}index.js\textquotesingle{}}
\NormalTok{    \}}\OperatorTok{;}
    \ControlFlowTok{await}\NormalTok{ fs}\OperatorTok{.}\FunctionTok{writeFile}\NormalTok{(}\StringTok{\textquotesingle{}./dist/package.json\textquotesingle{}}\OperatorTok{,} \BuiltInTok{JSON}\OperatorTok{.}\FunctionTok{stringify}\NormalTok{(prodPkg}\OperatorTok{,} \KeywordTok{null}\OperatorTok{,} \DecValTok{2}\NormalTok{))}\OperatorTok{;}
    
    \BuiltInTok{console}\OperatorTok{.}\FunctionTok{log}\NormalTok{(}\StringTok{\textquotesingle{}✅ Build completado!\textquotesingle{}}\NormalTok{)}\OperatorTok{;}
\NormalTok{\}}

\FunctionTok{buildProject}\NormalTok{()}\OperatorTok{.}\FunctionTok{catch}\NormalTok{(}\BuiltInTok{console}\OperatorTok{.}\FunctionTok{error}\NormalTok{)}\OperatorTok{;}
\end{Highlighting}
\end{Shaded}

\subsubsection{API simple con Express}\label{api-simple-con-express}

\begin{Shaded}
\begin{Highlighting}[]
\CommentTok{// server.js}
\KeywordTok{const}\NormalTok{ express }\OperatorTok{=} \PreprocessorTok{require}\NormalTok{(}\StringTok{\textquotesingle{}express\textquotesingle{}}\NormalTok{)}\OperatorTok{;}
\KeywordTok{const}\NormalTok{ cors }\OperatorTok{=} \PreprocessorTok{require}\NormalTok{(}\StringTok{\textquotesingle{}cors\textquotesingle{}}\NormalTok{)}\OperatorTok{;}
\KeywordTok{const}\NormalTok{ helmet }\OperatorTok{=} \PreprocessorTok{require}\NormalTok{(}\StringTok{\textquotesingle{}helmet\textquotesingle{}}\NormalTok{)}\OperatorTok{;}

\KeywordTok{const}\NormalTok{ app }\OperatorTok{=} \FunctionTok{express}\NormalTok{()}\OperatorTok{;}
\KeywordTok{const}\NormalTok{ PORT }\OperatorTok{=} \BuiltInTok{process}\OperatorTok{.}\AttributeTok{env}\OperatorTok{.}\AttributeTok{PORT} \OperatorTok{||} \DecValTok{3000}\OperatorTok{;}

\CommentTok{// Middleware}
\NormalTok{app}\OperatorTok{.}\FunctionTok{use}\NormalTok{(}\FunctionTok{helmet}\NormalTok{())}\OperatorTok{;}
\NormalTok{app}\OperatorTok{.}\FunctionTok{use}\NormalTok{(}\FunctionTok{cors}\NormalTok{())}\OperatorTok{;}
\NormalTok{app}\OperatorTok{.}\FunctionTok{use}\NormalTok{(express}\OperatorTok{.}\FunctionTok{json}\NormalTok{())}\OperatorTok{;}

\CommentTok{// Logging middleware}
\NormalTok{app}\OperatorTok{.}\FunctionTok{use}\NormalTok{((req}\OperatorTok{,}\NormalTok{ res}\OperatorTok{,}\NormalTok{ next) }\KeywordTok{=\textgreater{}}\NormalTok{ \{}
    \BuiltInTok{console}\OperatorTok{.}\FunctionTok{log}\NormalTok{(}\VerbatimStringTok{\textasciigrave{}}\SpecialCharTok{$\{}\KeywordTok{new} \BuiltInTok{Date}\NormalTok{()}\OperatorTok{.}\FunctionTok{toISOString}\NormalTok{()}\SpecialCharTok{\}}\VerbatimStringTok{ }\SpecialCharTok{$\{}\NormalTok{req}\OperatorTok{.}\AttributeTok{method}\SpecialCharTok{\}}\VerbatimStringTok{ }\SpecialCharTok{$\{}\NormalTok{req}\OperatorTok{.}\AttributeTok{path}\SpecialCharTok{\}}\VerbatimStringTok{\textasciigrave{}}\NormalTok{)}\OperatorTok{;}
    \FunctionTok{next}\NormalTok{()}\OperatorTok{;}
\NormalTok{\})}\OperatorTok{;}

\CommentTok{// Routes}
\NormalTok{app}\OperatorTok{.}\FunctionTok{get}\NormalTok{(}\StringTok{\textquotesingle{}/api/health\textquotesingle{}}\OperatorTok{,}\NormalTok{ (req}\OperatorTok{,}\NormalTok{ res) }\KeywordTok{=\textgreater{}}\NormalTok{ \{}
\NormalTok{    res}\OperatorTok{.}\FunctionTok{json}\NormalTok{(\{ }\DataTypeTok{status}\OperatorTok{:} \StringTok{\textquotesingle{}OK\textquotesingle{}}\OperatorTok{,} \DataTypeTok{timestamp}\OperatorTok{:} \KeywordTok{new} \BuiltInTok{Date}\NormalTok{()}\OperatorTok{.}\FunctionTok{toISOString}\NormalTok{() \})}\OperatorTok{;}
\NormalTok{\})}\OperatorTok{;}

\NormalTok{app}\OperatorTok{.}\FunctionTok{get}\NormalTok{(}\StringTok{\textquotesingle{}/api/users\textquotesingle{}}\OperatorTok{,}\NormalTok{ (req}\OperatorTok{,}\NormalTok{ res) }\KeywordTok{=\textgreater{}}\NormalTok{ \{}
    \CommentTok{// Simulación de datos}
\NormalTok{    res}\OperatorTok{.}\FunctionTok{json}\NormalTok{([}
\NormalTok{        \{ }\DataTypeTok{id}\OperatorTok{:} \DecValTok{1}\OperatorTok{,} \DataTypeTok{name}\OperatorTok{:} \StringTok{\textquotesingle{}Juan\textquotesingle{}}\OperatorTok{,} \DataTypeTok{email}\OperatorTok{:} \StringTok{\textquotesingle{}juan@example.com\textquotesingle{}}\NormalTok{ \}}\OperatorTok{,}
\NormalTok{        \{ }\DataTypeTok{id}\OperatorTok{:} \DecValTok{2}\OperatorTok{,} \DataTypeTok{name}\OperatorTok{:} \StringTok{\textquotesingle{}María\textquotesingle{}}\OperatorTok{,} \DataTypeTok{email}\OperatorTok{:} \StringTok{\textquotesingle{}maria@example.com\textquotesingle{}}\NormalTok{ \}}
\NormalTok{    ])}\OperatorTok{;}
\NormalTok{\})}\OperatorTok{;}

\CommentTok{// Error handling}
\NormalTok{app}\OperatorTok{.}\FunctionTok{use}\NormalTok{((err}\OperatorTok{,}\NormalTok{ req}\OperatorTok{,}\NormalTok{ res}\OperatorTok{,}\NormalTok{ next) }\KeywordTok{=\textgreater{}}\NormalTok{ \{}
    \BuiltInTok{console}\OperatorTok{.}\FunctionTok{error}\NormalTok{(err}\OperatorTok{.}\AttributeTok{stack}\NormalTok{)}\OperatorTok{;}
\NormalTok{    res}\OperatorTok{.}\FunctionTok{status}\NormalTok{(}\DecValTok{500}\NormalTok{)}\OperatorTok{.}\FunctionTok{json}\NormalTok{(\{ }\DataTypeTok{error}\OperatorTok{:} \StringTok{\textquotesingle{}Something went wrong!\textquotesingle{}}\NormalTok{ \})}\OperatorTok{;}
\NormalTok{\})}\OperatorTok{;}

\NormalTok{app}\OperatorTok{.}\FunctionTok{listen}\NormalTok{(PORT}\OperatorTok{,}\NormalTok{ () }\KeywordTok{=\textgreater{}}\NormalTok{ \{}
    \BuiltInTok{console}\OperatorTok{.}\FunctionTok{log}\NormalTok{(}\VerbatimStringTok{\textasciigrave{}🚀 Server running on http://localhost:}\SpecialCharTok{$\{}\NormalTok{PORT}\SpecialCharTok{\}}\VerbatimStringTok{\textasciigrave{}}\NormalTok{)}\OperatorTok{;}
\NormalTok{\})}\OperatorTok{;}
\end{Highlighting}
\end{Shaded}

\begin{center}\rule{0.5\linewidth}{0.5pt}\end{center}

\section{Workflows integrados}\label{workflows-integrados}

\subsection{Desarrollo Full-Stack}\label{desarrollo-full-stack}

\begin{Shaded}
\begin{Highlighting}[]
\CommentTok{\#!/bin/bash}
\CommentTok{\# Script de desarrollo completo}

\CommentTok{\# 1. Clonar y configurar proyecto}
\FunctionTok{clone\_and\_setup()} \KeywordTok{\{}
    \ExtensionTok{gh}\NormalTok{ repo clone mi{-}org/mi{-}proyecto}
    \BuiltInTok{cd}\NormalTok{ mi{-}proyecto}
    
    \CommentTok{\# Instalar dependencias}
    \ExtensionTok{npm}\NormalTok{ install}
    
    \CommentTok{\# Configurar git hooks}
    \ExtensionTok{npx}\NormalTok{ husky install}
    
    \CommentTok{\# Crear branch de desarrollo}
    \FunctionTok{git}\NormalTok{ checkout }\AttributeTok{{-}b}\NormalTok{ feature/nueva{-}funcionalidad}
\KeywordTok{\}}

\CommentTok{\# 2. Desarrollo con live reload}
\FunctionTok{dev\_server()} \KeywordTok{\{}
    \CommentTok{\# Terminal 1: Backend}
    \ExtensionTok{npm}\NormalTok{ run dev:server }\KeywordTok{\&}
    
    \CommentTok{\# Terminal 2: Frontend}
    \ExtensionTok{npm}\NormalTok{ run dev:client }\KeywordTok{\&}
    
    \CommentTok{\# Terminal 3: Tests en watch mode}
    \ExtensionTok{npm}\NormalTok{ run test:watch }\KeywordTok{\&}
    
    \BuiltInTok{wait}
\KeywordTok{\}}

\CommentTok{\# 3. Pre{-}commit workflow}
\FunctionTok{pre\_commit()} \KeywordTok{\{}
    \CommentTok{\# Linting}
    \ExtensionTok{npm}\NormalTok{ run lint:fix}
    
    \CommentTok{\# Formateo}
    \ExtensionTok{npm}\NormalTok{ run format}
    
    \CommentTok{\# Tests}
    \ExtensionTok{npm}\NormalTok{ test}
    
    \CommentTok{\# Type checking}
    \ExtensionTok{npm}\NormalTok{ run type{-}check}
    
    \ControlFlowTok{if} \BuiltInTok{[} \VariableTok{$?} \OtherTok{{-}eq}\NormalTok{ 0 }\BuiltInTok{]}\KeywordTok{;} \ControlFlowTok{then}
        \BuiltInTok{echo} \StringTok{"✅ Todo OK, procediendo con commit"}
        \FunctionTok{git}\NormalTok{ add .}
        \FunctionTok{git}\NormalTok{ commit }\AttributeTok{{-}m} \StringTok{"}\VariableTok{$1}\StringTok{"}
    \ControlFlowTok{else}
        \BuiltInTok{echo} \StringTok{"❌ Errores encontrados, revisar antes de commit"}
        \BuiltInTok{exit}\NormalTok{ 1}
    \ControlFlowTok{fi}
\KeywordTok{\}}

\CommentTok{\# 4. Deploy workflow}
\FunctionTok{deploy()} \KeywordTok{\{}
    \CommentTok{\# Build de producción}
    \ExtensionTok{npm}\NormalTok{ run build}
    
    \CommentTok{\# Push y crear PR}
    \FunctionTok{git}\NormalTok{ push }\AttributeTok{{-}u}\NormalTok{ origin }\StringTok{"}\VariableTok{$(}\FunctionTok{git}\NormalTok{ branch }\AttributeTok{{-}{-}show{-}current}\VariableTok{)}\StringTok{"}
    \ExtensionTok{gh}\NormalTok{ pr create }\AttributeTok{{-}{-}title} \StringTok{"Deploy: }\VariableTok{$(}\FunctionTok{date}\NormalTok{ +\%Y{-}\%m{-}\%d}\VariableTok{)}\StringTok{"} \AttributeTok{{-}{-}body} \StringTok{"Release automático"}
    
    \CommentTok{\# Esperar aprobación y merge}
    \VariableTok{pr\_number}\OperatorTok{=}\VariableTok{$(}\ExtensionTok{gh}\NormalTok{ pr view }\AttributeTok{{-}{-}json}\NormalTok{ number }\AttributeTok{{-}{-}jq}\NormalTok{ .number}\VariableTok{)}
    \BuiltInTok{echo} \StringTok{"PR \#}\VariableTok{$pr\_number}\StringTok{ creado. Esperando aprobación..."}
\KeywordTok{\}}
\end{Highlighting}
\end{Shaded}

\subsection{Git workflow avanzado}\label{git-workflow-avanzado}

\begin{Shaded}
\begin{Highlighting}[]
\CommentTok{\#!/bin/bash}
\CommentTok{\# Workflow completo de Git con integración GitHub}

\FunctionTok{git\_workflow()} \KeywordTok{\{}
    \BuiltInTok{local} \VariableTok{issue\_number}\OperatorTok{=}\StringTok{"}\VariableTok{$1}\StringTok{"}
    \BuiltInTok{local} \VariableTok{feature\_name}\OperatorTok{=}\StringTok{"}\VariableTok{$2}\StringTok{"}
    
    \ControlFlowTok{if} \BuiltInTok{[} \OtherTok{{-}z} \StringTok{"}\VariableTok{$issue\_number}\StringTok{"} \BuiltInTok{]} \KeywordTok{||} \BuiltInTok{[} \OtherTok{{-}z} \StringTok{"}\VariableTok{$feature\_name}\StringTok{"} \BuiltInTok{]}\KeywordTok{;} \ControlFlowTok{then}
        \BuiltInTok{echo} \StringTok{"Uso: git\_workflow \textless{}issue\_number\textgreater{} \textless{}feature\_name\textgreater{}"}
        \ControlFlowTok{return} \DecValTok{1}
    \ControlFlowTok{fi}
    
    \CommentTok{\# 1. Actualizar main}
    \FunctionTok{git}\NormalTok{ checkout main}
    \FunctionTok{git}\NormalTok{ pull origin main}
    
    \CommentTok{\# 2. Crear branch desde issue}
    \VariableTok{branch\_name}\OperatorTok{=}\StringTok{"feature/}\VariableTok{$\{issue\_number\}}\StringTok{{-}}\VariableTok{$\{feature\_name\}}\StringTok{"}
    \FunctionTok{git}\NormalTok{ checkout }\AttributeTok{{-}b} \StringTok{"}\VariableTok{$branch\_name}\StringTok{"}
    
    \CommentTok{\# 3. Configurar tracking}
    \FunctionTok{git}\NormalTok{ push }\AttributeTok{{-}u}\NormalTok{ origin }\StringTok{"}\VariableTok{$branch\_name}\StringTok{"}
    
    \CommentTok{\# 4. Asociar con issue}
    \ExtensionTok{gh}\NormalTok{ issue edit }\StringTok{"}\VariableTok{$issue\_number}\StringTok{"} \AttributeTok{{-}{-}add{-}assignee}\NormalTok{ @me}
    \ExtensionTok{gh}\NormalTok{ issue comment }\StringTok{"}\VariableTok{$issue\_number}\StringTok{"} \AttributeTok{{-}{-}body} \StringTok{"🚧 Trabajando en branch: }\VariableTok{$branch\_name}\StringTok{"}
    
    \BuiltInTok{echo} \StringTok{"✅ Branch }\VariableTok{$branch\_name}\StringTok{ creado y asociado con issue \#}\VariableTok{$issue\_number}\StringTok{"}
    \BuiltInTok{echo} \StringTok{"💡 Cuando termines, ejecuta: finish\_feature }\VariableTok{$issue\_number}\StringTok{"}
\KeywordTok{\}}

\FunctionTok{finish\_feature()} \KeywordTok{\{}
    \BuiltInTok{local} \VariableTok{issue\_number}\OperatorTok{=}\StringTok{"}\VariableTok{$1}\StringTok{"}
    
    \CommentTok{\# 1. Push final}
    \FunctionTok{git}\NormalTok{ push origin }\StringTok{"}\VariableTok{$(}\FunctionTok{git}\NormalTok{ branch }\AttributeTok{{-}{-}show{-}current}\VariableTok{)}\StringTok{"}
    
    \CommentTok{\# 2. Crear PR que cierre el issue}
    \ExtensionTok{gh}\NormalTok{ pr create }\DataTypeTok{\textbackslash{}}
        \AttributeTok{{-}{-}title} \StringTok{"Resolve \#}\VariableTok{$issue\_number}\StringTok{"} \DataTypeTok{\textbackslash{}}
        \AttributeTok{{-}{-}body} \StringTok{"Closes \#}\VariableTok{$issue\_number}\StringTok{"} \DataTypeTok{\textbackslash{}}
        \AttributeTok{{-}{-}reviewer}\NormalTok{ team{-}leads}
    
    \CommentTok{\# 3. Comentar en issue}
    \VariableTok{pr\_url}\OperatorTok{=}\VariableTok{$(}\ExtensionTok{gh}\NormalTok{ pr view }\AttributeTok{{-}{-}json}\NormalTok{ url }\AttributeTok{{-}{-}jq}\NormalTok{ .url}\VariableTok{)}
    \ExtensionTok{gh}\NormalTok{ issue comment }\StringTok{"}\VariableTok{$issue\_number}\StringTok{"} \AttributeTok{{-}{-}body} \StringTok{"🎉 PR creado: }\VariableTok{$pr\_url}\StringTok{"}
    
    \BuiltInTok{echo} \StringTok{"✅ PR creado. Revisa en: }\VariableTok{$pr\_url}\StringTok{"}
\KeywordTok{\}}

\CommentTok{\# Usar: git\_workflow 123 user{-}authentication}
\end{Highlighting}
\end{Shaded}

\begin{tcolorbox}[enhanced jigsaw, toprule=.15mm, bottomrule=.15mm, opacityback=0, coltitle=black, rightrule=.15mm, colframe=quarto-callout-tip-color-frame, titlerule=0mm, opacitybacktitle=0.6, left=2mm, colback=white, bottomtitle=1mm, arc=.35mm, leftrule=.75mm, title=\textcolor{quarto-callout-tip-color}{\faLightbulb}\hspace{0.5em}{Tips para desarrollo eficiente}, colbacktitle=quarto-callout-tip-color!10!white, breakable, toptitle=1mm]

\begin{itemize}
\tightlist
\item
  Configura aliases de Git para comandos frecuentes
\item
  Usa \texttt{gh} para automatizar workflows de GitHub
\item
  Combina \texttt{git\ hooks} con \texttt{husky} para validaciones
  automáticas
\item
  Mantén commits atómicos y mensajes descriptivos siguiendo Conventional
  Commits
\end{itemize}

\end{tcolorbox}

\begin{tcolorbox}[enhanced jigsaw, toprule=.15mm, bottomrule=.15mm, opacityback=0, coltitle=black, rightrule=.15mm, colframe=quarto-callout-important-color-frame, titlerule=0mm, opacitybacktitle=0.6, left=2mm, colback=white, bottomtitle=1mm, arc=.35mm, leftrule=.75mm, title=\textcolor{quarto-callout-important-color}{\faExclamation}\hspace{0.5em}{Mejores prácticas}, colbacktitle=quarto-callout-important-color!10!white, breakable, toptitle=1mm]

\begin{itemize}
\tightlist
\item
  Nunca hagas \texttt{push\ -\/-force} en branches compartidos
\item
  Usa \texttt{-\/-force-with-lease} en lugar de \texttt{-\/-force}
  cuando sea necesario
\item
  Mantén historial limpio con \texttt{rebase} interactivo antes de merge
\item
  Configura firma GPG para commits verificados
\end{itemize}

\end{tcolorbox}

En el próximo capítulo exploraremos herramientas de multimedia para
procesamiento de video, audio e imágenes.

\part{Multimedia}

\chapter{Multimedia}\label{multimedia-2}

El procesamiento de archivos multimedia desde la línea de comandos es
extremadamente potente y versátil. Esta sección cubre herramientas para
manipular video, audio e imágenes con precisión profesional.

\section{ffmpeg - Suite multimedia completa}\label{sec-ffmpeg}

FFmpeg es la herramienta más potente para procesamiento multimedia,
capaz de manejar prácticamente cualquier formato de video y audio.

\subsection{Conversión básica}\label{conversiuxf3n-buxe1sica}

\begin{Shaded}
\begin{Highlighting}[]
\CommentTok{\# Convertir formato de video}
\ExtensionTok{ffmpeg} \AttributeTok{{-}i}\NormalTok{ input.mov output.mp4}
\ExtensionTok{ffmpeg} \AttributeTok{{-}i}\NormalTok{ input.avi output.mkv}

\CommentTok{\# Conversión con codec específico}
\ExtensionTok{ffmpeg} \AttributeTok{{-}i}\NormalTok{ input.mp4 }\AttributeTok{{-}c:v}\NormalTok{ libx264 }\AttributeTok{{-}c:a}\NormalTok{ aac output.mp4}

\CommentTok{\# Conversión rápida (solo cambiar contenedor)}
\ExtensionTok{ffmpeg} \AttributeTok{{-}i}\NormalTok{ input.mkv }\AttributeTok{{-}c}\NormalTok{ copy output.mp4}
\end{Highlighting}
\end{Shaded}

\subsection{Manipulación de video}\label{manipulaciuxf3n-de-video}

\subsubsection{Redimensionamiento y
calidad}\label{redimensionamiento-y-calidad}

\begin{Shaded}
\begin{Highlighting}[]
\CommentTok{\# Cambiar resolución}
\ExtensionTok{ffmpeg} \AttributeTok{{-}i}\NormalTok{ input.mp4 }\AttributeTok{{-}vf}\NormalTok{ scale=1280:720 output.mp4}
\ExtensionTok{ffmpeg} \AttributeTok{{-}i}\NormalTok{ input.mp4 }\AttributeTok{{-}vf}\NormalTok{ scale=1920:{-}1 output.mp4  }\CommentTok{\# Mantener aspecto}

\CommentTok{\# Comprimir video}
\ExtensionTok{ffmpeg} \AttributeTok{{-}i}\NormalTok{ input.mp4 }\AttributeTok{{-}crf}\NormalTok{ 23 output.mp4  }\CommentTok{\# CRF 18{-}28 (menor = mejor calidad)}

\CommentTok{\# Cambiar bitrate}
\ExtensionTok{ffmpeg} \AttributeTok{{-}i}\NormalTok{ input.mp4 }\AttributeTok{{-}b:v}\NormalTok{ 1M }\AttributeTok{{-}b:a}\NormalTok{ 128k output.mp4}
\end{Highlighting}
\end{Shaded}

\subsubsection{Edición temporal}\label{ediciuxf3n-temporal}

\begin{Shaded}
\begin{Highlighting}[]
\CommentTok{\# Cortar video (desde segundo 30, duración 60s)}
\ExtensionTok{ffmpeg} \AttributeTok{{-}i}\NormalTok{ input.mp4 }\AttributeTok{{-}ss}\NormalTok{ 30 }\AttributeTok{{-}t}\NormalTok{ 60 output.mp4}

\CommentTok{\# Cortar desde inicio hasta minuto 2}
\ExtensionTok{ffmpeg} \AttributeTok{{-}i}\NormalTok{ input.mp4 }\AttributeTok{{-}t}\NormalTok{ 120 output.mp4}

\CommentTok{\# Cortar desde minuto 1 hasta el final}
\ExtensionTok{ffmpeg} \AttributeTok{{-}i}\NormalTok{ input.mp4 }\AttributeTok{{-}ss}\NormalTok{ 60 output.mp4}
\end{Highlighting}
\end{Shaded}

\subsubsection{Efectos y filtros}\label{efectos-y-filtros}

\begin{Shaded}
\begin{Highlighting}[]
\CommentTok{\# Rotar video}
\ExtensionTok{ffmpeg} \AttributeTok{{-}i}\NormalTok{ input.mp4 }\AttributeTok{{-}vf} \StringTok{"transpose=1"}\NormalTok{ output.mp4  }\CommentTok{\# 90° horario}

\CommentTok{\# Espejo horizontal}
\ExtensionTok{ffmpeg} \AttributeTok{{-}i}\NormalTok{ input.mp4 }\AttributeTok{{-}vf}\NormalTok{ hflip output.mp4}

\CommentTok{\# Acelerar/desacelerar}
\ExtensionTok{ffmpeg} \AttributeTok{{-}i}\NormalTok{ input.mp4 }\AttributeTok{{-}vf} \StringTok{"setpts=0.5*PTS"}\NormalTok{ output.mp4  }\CommentTok{\# 2x velocidad}
\ExtensionTok{ffmpeg} \AttributeTok{{-}i}\NormalTok{ input.mp4 }\AttributeTok{{-}vf} \StringTok{"setpts=2*PTS"}\NormalTok{ output.mp4   }\CommentTok{\# 0.5x velocidad}

\CommentTok{\# Agregar watermark}
\ExtensionTok{ffmpeg} \AttributeTok{{-}i}\NormalTok{ video.mp4 }\AttributeTok{{-}i}\NormalTok{ logo.png }\AttributeTok{{-}filter\_complex} \StringTok{"overlay=10:10"}\NormalTok{ output.mp4}
\end{Highlighting}
\end{Shaded}

\subsection{Audio processing}\label{audio-processing}

\begin{Shaded}
\begin{Highlighting}[]
\CommentTok{\# Extraer audio de video}
\ExtensionTok{ffmpeg} \AttributeTok{{-}i}\NormalTok{ video.mp4 }\AttributeTok{{-}vn} \AttributeTok{{-}acodec}\NormalTok{ copy audio.aac}
\ExtensionTok{ffmpeg} \AttributeTok{{-}i}\NormalTok{ video.mp4 }\AttributeTok{{-}vn}\NormalTok{ audio.mp3}

\CommentTok{\# Cambiar volumen}
\ExtensionTok{ffmpeg} \AttributeTok{{-}i}\NormalTok{ input.mp3 }\AttributeTok{{-}af} \StringTok{"volume=0.5"}\NormalTok{ output.mp3  }\CommentTok{\# 50\% volumen}
\ExtensionTok{ffmpeg} \AttributeTok{{-}i}\NormalTok{ input.mp3 }\AttributeTok{{-}af} \StringTok{"volume=10dB"}\NormalTok{ output.mp3  }\CommentTok{\# +10dB}

\CommentTok{\# Normalizar audio}
\ExtensionTok{ffmpeg} \AttributeTok{{-}i}\NormalTok{ input.mp3 }\AttributeTok{{-}af}\NormalTok{ loudnorm output.mp3}

\CommentTok{\# Combinar audio y video}
\ExtensionTok{ffmpeg} \AttributeTok{{-}i}\NormalTok{ video.mp4 }\AttributeTok{{-}i}\NormalTok{ audio.mp3 }\AttributeTok{{-}c:v}\NormalTok{ copy }\AttributeTok{{-}c:a}\NormalTok{ aac output.mp4}
\end{Highlighting}
\end{Shaded}

\subsection{Casos avanzados}\label{casos-avanzados}

\subsubsection{Crear GIF desde video}\label{crear-gif-desde-video}

\begin{Shaded}
\begin{Highlighting}[]
\CommentTok{\# GIF básico}
\ExtensionTok{ffmpeg} \AttributeTok{{-}i}\NormalTok{ input.mp4 }\AttributeTok{{-}vf} \StringTok{"fps=10,scale=320:{-}1:flags=lanczos"}\NormalTok{ output.gif}

\CommentTok{\# GIF optimizado con paleta}
\ExtensionTok{ffmpeg} \AttributeTok{{-}i}\NormalTok{ input.mp4 }\AttributeTok{{-}vf} \StringTok{"fps=15,scale=480:{-}1:flags=lanczos,palettegen"}\NormalTok{ palette.png}
\ExtensionTok{ffmpeg} \AttributeTok{{-}i}\NormalTok{ input.mp4 }\AttributeTok{{-}i}\NormalTok{ palette.png }\AttributeTok{{-}filter\_complex} \StringTok{"fps=15,scale=480:{-}1:flags=lanczos[x];[x][1:v]paletteuse"}\NormalTok{ output.gif}
\end{Highlighting}
\end{Shaded}

\subsubsection{Streaming y
transmisión}\label{streaming-y-transmisiuxf3n}

\begin{Shaded}
\begin{Highlighting}[]
\CommentTok{\# Stream a YouTube Live}
\ExtensionTok{ffmpeg} \AttributeTok{{-}re} \AttributeTok{{-}i}\NormalTok{ input.mp4 }\AttributeTok{{-}c:v}\NormalTok{ libx264 }\AttributeTok{{-}preset}\NormalTok{ fast }\AttributeTok{{-}maxrate}\NormalTok{ 3000k }\AttributeTok{{-}bufsize}\NormalTok{ 6000k }\DataTypeTok{\textbackslash{}}
\NormalTok{{-}pix\_fmt yuv420p }\AttributeTok{{-}g}\NormalTok{ 50 }\AttributeTok{{-}c:a}\NormalTok{ aac }\AttributeTok{{-}b:a}\NormalTok{ 160k }\AttributeTok{{-}ac}\NormalTok{ 2 }\AttributeTok{{-}ar}\NormalTok{ 44100 }\DataTypeTok{\textbackslash{}}
\NormalTok{{-}f flv rtmp://a.rtmp.youtube.com/live2/STREAM\_KEY}

\CommentTok{\# Crear HLS stream}
\ExtensionTok{ffmpeg} \AttributeTok{{-}i}\NormalTok{ input.mp4 }\AttributeTok{{-}profile:v}\NormalTok{ baseline }\AttributeTok{{-}level}\NormalTok{ 3.0 }\AttributeTok{{-}s}\NormalTok{ 640x360 }\AttributeTok{{-}start\_number}\NormalTok{ 0 }\DataTypeTok{\textbackslash{}}
\NormalTok{{-}hls\_time 10 }\AttributeTok{{-}hls\_list\_size}\NormalTok{ 0 }\AttributeTok{{-}f}\NormalTok{ hls output.m3u8}
\end{Highlighting}
\end{Shaded}

\begin{center}\rule{0.5\linewidth}{0.5pt}\end{center}

\section{yt-dlp - Descargador universal}\label{sec-yt-dlp}

Sucesor mejorado de youtube-dl, capaz de descargar de cientos de sitios.

\subsection{Descarga básica}\label{descarga-buxe1sica}

\begin{Shaded}
\begin{Highlighting}[]
\CommentTok{\# Descargar video en mejor calidad}
\ExtensionTok{yt{-}dlp} \StringTok{"https://youtube.com/watch?v=VIDEO\_ID"}

\CommentTok{\# Solo audio en MP3}
\ExtensionTok{yt{-}dlp} \AttributeTok{{-}x} \AttributeTok{{-}{-}audio{-}format}\NormalTok{ mp3 }\StringTok{"URL"}

\CommentTok{\# Calidad específica}
\ExtensionTok{yt{-}dlp} \AttributeTok{{-}f} \StringTok{"best[height\textless{}=720]"} \StringTok{"URL"}
\ExtensionTok{yt{-}dlp} \AttributeTok{{-}f} \StringTok{"worst"} \StringTok{"URL"}  \CommentTok{\# Menor calidad}
\end{Highlighting}
\end{Shaded}

\subsection{Opciones avanzadas}\label{opciones-avanzadas}

\subsubsection{Listas y canales}\label{listas-y-canales}

\begin{Shaded}
\begin{Highlighting}[]
\CommentTok{\# Playlist completa}
\ExtensionTok{yt{-}dlp} \StringTok{"https://youtube.com/playlist?list=PLAYLIST\_ID"}

\CommentTok{\# Solo videos nuevos de playlist}
\ExtensionTok{yt{-}dlp} \AttributeTok{{-}{-}download{-}archive}\NormalTok{ downloaded.txt }\StringTok{"PLAYLIST\_URL"}

\CommentTok{\# Canal completo}
\ExtensionTok{yt{-}dlp} \StringTok{"https://youtube.com/c/CHANNEL\_NAME/videos"}

\CommentTok{\# Últimos N videos de canal}
\ExtensionTok{yt{-}dlp} \AttributeTok{{-}{-}playlist{-}end}\NormalTok{ 10 }\StringTok{"CHANNEL\_URL"}
\end{Highlighting}
\end{Shaded}

\subsubsection{Filtros y selección}\label{filtros-y-selecciuxf3n}

\begin{Shaded}
\begin{Highlighting}[]
\CommentTok{\# Ver formatos disponibles}
\ExtensionTok{yt{-}dlp} \AttributeTok{{-}F} \StringTok{"URL"}

\CommentTok{\# Seleccionar formato específico}
\ExtensionTok{yt{-}dlp} \AttributeTok{{-}f}\NormalTok{ 137+140 }\StringTok{"URL"}  \CommentTok{\# Video 1080p + audio}

\CommentTok{\# Filtros complejos}
\ExtensionTok{yt{-}dlp} \AttributeTok{{-}f} \StringTok{"best[ext=mp4][height\textless{}=1080]"} \StringTok{"URL"}
\ExtensionTok{yt{-}dlp} \AttributeTok{{-}f} \StringTok{"bestvideo[height\textless{}=720]+bestaudio/best[height\textless{}=720]"} \StringTok{"URL"}
\end{Highlighting}
\end{Shaded}

\subsubsection{Metadata y
organización}\label{metadata-y-organizaciuxf3n}

\begin{Shaded}
\begin{Highlighting}[]
\CommentTok{\# Con subtítulos}
\ExtensionTok{yt{-}dlp} \AttributeTok{{-}{-}write{-}subs} \AttributeTok{{-}{-}sub{-}lang}\NormalTok{ es,en }\StringTok{"URL"}
\ExtensionTok{yt{-}dlp} \AttributeTok{{-}{-}write{-}auto{-}subs} \AttributeTok{{-}{-}sub{-}lang}\NormalTok{ es }\StringTok{"URL"}

\CommentTok{\# Con thumbnail}
\ExtensionTok{yt{-}dlp} \AttributeTok{{-}{-}write{-}thumbnail} \StringTok{"URL"}

\CommentTok{\# Template de nombres de archivo}
\ExtensionTok{yt{-}dlp} \AttributeTok{{-}o} \StringTok{"\%(uploader)s/\%(title)s.\%(ext)s"} \StringTok{"URL"}
\ExtensionTok{yt{-}dlp} \AttributeTok{{-}o} \StringTok{"\%(upload\_date)s{-}\%(title)s.\%(ext)s"} \StringTok{"URL"}
\end{Highlighting}
\end{Shaded}

\subsection{Scripts automatizados}\label{scripts-automatizados}

\subsubsection{Descargador de podcast}\label{descargador-de-podcast}

\begin{Shaded}
\begin{Highlighting}[]
\CommentTok{\#!/bin/bash}
\CommentTok{\# podcast{-}downloader.sh}

\VariableTok{PODCAST\_URL}\OperatorTok{=}\StringTok{"}\VariableTok{$1}\StringTok{"}
\VariableTok{DOWNLOAD\_DIR}\OperatorTok{=}\StringTok{"}\VariableTok{$HOME}\StringTok{/Podcasts"}

\ControlFlowTok{if} \BuiltInTok{[} \OtherTok{{-}z} \StringTok{"}\VariableTok{$PODCAST\_URL}\StringTok{"} \BuiltInTok{]}\KeywordTok{;} \ControlFlowTok{then}
    \BuiltInTok{echo} \StringTok{"Uso: }\VariableTok{$0}\StringTok{ \textless{}URL\_del\_podcast\textgreater{}"}
    \BuiltInTok{exit}\NormalTok{ 1}
\ControlFlowTok{fi}

\FunctionTok{mkdir} \AttributeTok{{-}p} \StringTok{"}\VariableTok{$DOWNLOAD\_DIR}\StringTok{"}
\BuiltInTok{cd} \StringTok{"}\VariableTok{$DOWNLOAD\_DIR}\StringTok{"}

\CommentTok{\# Descargar solo audio, organizados por canal}
\ExtensionTok{yt{-}dlp} \DataTypeTok{\textbackslash{}}
    \AttributeTok{{-}{-}extract{-}flat} \DataTypeTok{\textbackslash{}}
    \AttributeTok{{-}{-}download{-}archive}\NormalTok{ podcast{-}archive.txt }\DataTypeTok{\textbackslash{}}
    \AttributeTok{{-}x} \AttributeTok{{-}{-}audio{-}format}\NormalTok{ mp3 }\DataTypeTok{\textbackslash{}}
    \AttributeTok{{-}{-}audio{-}quality}\NormalTok{ 192K }\DataTypeTok{\textbackslash{}}
    \AttributeTok{{-}o} \StringTok{"\%(uploader)s/\%(upload\_date)s {-} \%(title)s.\%(ext)s"} \DataTypeTok{\textbackslash{}}
    \StringTok{"}\VariableTok{$PODCAST\_URL}\StringTok{"}

\BuiltInTok{echo} \StringTok{"Podcasts descargados en }\VariableTok{$DOWNLOAD\_DIR}\StringTok{"}
\end{Highlighting}
\end{Shaded}

\begin{center}\rule{0.5\linewidth}{0.5pt}\end{center}

\section{ImageMagick - Manipulación de imágenes}\label{sec-imagemagick}

Suite completa para edición de imágenes desde línea de comandos.

\subsection{Conversión y
redimensionamiento}\label{conversiuxf3n-y-redimensionamiento}

\begin{Shaded}
\begin{Highlighting}[]
\CommentTok{\# Convertir formato}
\ExtensionTok{convert}\NormalTok{ image.png image.jpg}
\ExtensionTok{convert}\NormalTok{ image.tiff image.webp}

\CommentTok{\# Redimensionar imagen}
\ExtensionTok{convert}\NormalTok{ image.jpg }\AttributeTok{{-}resize}\NormalTok{ 50\% small.jpg}
\ExtensionTok{convert}\NormalTok{ image.jpg }\AttributeTok{{-}resize}\NormalTok{ 800x600 resized.jpg}
\ExtensionTok{convert}\NormalTok{ image.jpg }\AttributeTok{{-}resize}\NormalTok{ 800x600! forced.jpg  }\CommentTok{\# Forzar dimensiones exactas}
\end{Highlighting}
\end{Shaded}

\subsection{Edición básica}\label{ediciuxf3n-buxe1sica}

\subsubsection{Rotación y volteo}\label{rotaciuxf3n-y-volteo}

\begin{Shaded}
\begin{Highlighting}[]
\CommentTok{\# Rotar imagen}
\ExtensionTok{convert}\NormalTok{ image.jpg }\AttributeTok{{-}rotate}\NormalTok{ 90 rotated.jpg}
\ExtensionTok{convert}\NormalTok{ image.jpg }\AttributeTok{{-}rotate} \AttributeTok{{-}45}\NormalTok{ rotated.jpg}

\CommentTok{\# Espejo}
\ExtensionTok{convert}\NormalTok{ image.jpg }\AttributeTok{{-}flop}\NormalTok{ horizontal.jpg  }\CommentTok{\# Horizontal}
\ExtensionTok{convert}\NormalTok{ image.jpg }\AttributeTok{{-}flip}\NormalTok{ vertical.jpg    }\CommentTok{\# Vertical}
\end{Highlighting}
\end{Shaded}

\subsubsection{Efectos y filtros}\label{efectos-y-filtros-1}

\begin{Shaded}
\begin{Highlighting}[]
\CommentTok{\# Blur}
\ExtensionTok{convert}\NormalTok{ image.jpg }\AttributeTok{{-}blur}\NormalTok{ 0x8 blurred.jpg}

\CommentTok{\# Sharpen}
\ExtensionTok{convert}\NormalTok{ image.jpg }\AttributeTok{{-}sharpen}\NormalTok{ 0x1 sharp.jpg}

\CommentTok{\# Sepia}
\ExtensionTok{convert}\NormalTok{ image.jpg }\AttributeTok{{-}sepia{-}tone}\NormalTok{ 80\% sepia.jpg}

\CommentTok{\# Blanco y negro}
\ExtensionTok{convert}\NormalTok{ image.jpg }\AttributeTok{{-}colorspace}\NormalTok{ gray bw.jpg}
\end{Highlighting}
\end{Shaded}

\subsection{Procesamiento por lotes}\label{procesamiento-por-lotes}

\subsubsection{mogrify para múltiples
archivos}\label{mogrify-para-muxfaltiples-archivos}

\begin{Shaded}
\begin{Highlighting}[]
\CommentTok{\# Redimensionar todas las imágenes JPG}
\ExtensionTok{mogrify} \AttributeTok{{-}resize}\NormalTok{ 50\% }\PreprocessorTok{*}\NormalTok{.jpg}

\CommentTok{\# Convertir PNG a JPG}
\ExtensionTok{mogrify} \AttributeTok{{-}format}\NormalTok{ jpg }\PreprocessorTok{*}\NormalTok{.png}

\CommentTok{\# Agregar borde a todas las imágenes}
\ExtensionTok{mogrify} \AttributeTok{{-}border}\NormalTok{ 10x10 }\AttributeTok{{-}bordercolor}\NormalTok{ black }\PreprocessorTok{*}\NormalTok{.jpg}
\end{Highlighting}
\end{Shaded}

\subsubsection{Scripts de procesamiento
masivo}\label{scripts-de-procesamiento-masivo}

\begin{Shaded}
\begin{Highlighting}[]
\CommentTok{\#!/bin/bash}
\CommentTok{\# optimize{-}images.sh {-} Optimizar imágenes para web}

\VariableTok{input\_dir}\OperatorTok{=}\StringTok{"}\VariableTok{$1}\StringTok{"}
\VariableTok{output\_dir}\OperatorTok{=}\StringTok{"}\VariableTok{$2}\StringTok{"}

\ControlFlowTok{if} \BuiltInTok{[} \OtherTok{!} \OtherTok{{-}d} \StringTok{"}\VariableTok{$input\_dir}\StringTok{"} \BuiltInTok{]}\KeywordTok{;} \ControlFlowTok{then}
    \BuiltInTok{echo} \StringTok{"Directorio de entrada no existe: }\VariableTok{$input\_dir}\StringTok{"}
    \BuiltInTok{exit}\NormalTok{ 1}
\ControlFlowTok{fi}

\FunctionTok{mkdir} \AttributeTok{{-}p} \StringTok{"}\VariableTok{$output\_dir}\StringTok{"}

\CommentTok{\# Procesar cada imagen}
\FunctionTok{find} \StringTok{"}\VariableTok{$input\_dir}\StringTok{"} \AttributeTok{{-}type}\NormalTok{ f }\DataTypeTok{\textbackslash{}(} \AttributeTok{{-}iname} \StringTok{"*.jpg"} \AttributeTok{{-}o} \AttributeTok{{-}iname} \StringTok{"*.png"} \DataTypeTok{\textbackslash{})} \KeywordTok{|} \ControlFlowTok{while} \BuiltInTok{read} \AttributeTok{{-}r} \VariableTok{img}\KeywordTok{;} \ControlFlowTok{do}
    \VariableTok{filename}\OperatorTok{=}\VariableTok{$(}\FunctionTok{basename} \StringTok{"}\VariableTok{$img}\StringTok{"}\VariableTok{)}
    \VariableTok{name}\OperatorTok{=}\StringTok{"}\VariableTok{$\{filename}\OperatorTok{\%}\NormalTok{.}\PreprocessorTok{*}\VariableTok{\}}\StringTok{"}
    
    \BuiltInTok{echo} \StringTok{"Procesando: }\VariableTok{$filename}\StringTok{"}
    
    \CommentTok{\# Crear versiones optimizadas}
    \ExtensionTok{convert} \StringTok{"}\VariableTok{$img}\StringTok{"} \AttributeTok{{-}resize}\NormalTok{ 1920x1920}\DataTypeTok{\textbackslash{}\textgreater{}} \AttributeTok{{-}quality}\NormalTok{ 85 }\StringTok{"}\VariableTok{$output\_dir}\StringTok{/}\VariableTok{$\{name\}}\StringTok{\_large.jpg"}
    \ExtensionTok{convert} \StringTok{"}\VariableTok{$img}\StringTok{"} \AttributeTok{{-}resize}\NormalTok{ 800x800}\DataTypeTok{\textbackslash{}\textgreater{}} \AttributeTok{{-}quality}\NormalTok{ 80 }\StringTok{"}\VariableTok{$output\_dir}\StringTok{/}\VariableTok{$\{name\}}\StringTok{\_medium.jpg"}
    \ExtensionTok{convert} \StringTok{"}\VariableTok{$img}\StringTok{"} \AttributeTok{{-}resize}\NormalTok{ 400x400}\DataTypeTok{\textbackslash{}\textgreater{}} \AttributeTok{{-}quality}\NormalTok{ 75 }\StringTok{"}\VariableTok{$output\_dir}\StringTok{/}\VariableTok{$\{name\}}\StringTok{\_small.jpg"}
    \ExtensionTok{convert} \StringTok{"}\VariableTok{$img}\StringTok{"} \AttributeTok{{-}resize}\NormalTok{ 150x150\^{} }\AttributeTok{{-}gravity}\NormalTok{ center }\AttributeTok{{-}extent}\NormalTok{ 150x150 }\AttributeTok{{-}quality}\NormalTok{ 70 }\StringTok{"}\VariableTok{$output\_dir}\StringTok{/}\VariableTok{$\{name\}}\StringTok{\_thumb.jpg"}
\ControlFlowTok{done}

\BuiltInTok{echo} \StringTok{"Optimización completada. Archivos en: }\VariableTok{$output\_dir}\StringTok{"}
\end{Highlighting}
\end{Shaded}

\subsection{Casos avanzados}\label{casos-avanzados-1}

\subsubsection{Crear montajes y
collages}\label{crear-montajes-y-collages}

\begin{Shaded}
\begin{Highlighting}[]
\CommentTok{\# Montage simple}
\ExtensionTok{montage} \PreprocessorTok{*}\NormalTok{.jpg }\AttributeTok{{-}tile}\NormalTok{ 3x3 }\AttributeTok{{-}geometry}\NormalTok{ +2+2 collage.jpg}

\CommentTok{\# Montage con labels}
\ExtensionTok{montage} \PreprocessorTok{*}\NormalTok{.jpg }\AttributeTok{{-}tile}\NormalTok{ 2x2 }\AttributeTok{{-}geometry}\NormalTok{ 300x300+10+10 }\AttributeTok{{-}pointsize}\NormalTok{ 20 }\AttributeTok{{-}label} \StringTok{\textquotesingle{}\%t\textquotesingle{}}\NormalTok{ labeled\_montage.jpg}
\end{Highlighting}
\end{Shaded}

\subsubsection{Procesamiento de imágenes con
texto}\label{procesamiento-de-imuxe1genes-con-texto}

\begin{Shaded}
\begin{Highlighting}[]
\CommentTok{\# Agregar texto}
\ExtensionTok{convert}\NormalTok{ image.jpg }\AttributeTok{{-}pointsize}\NormalTok{ 30 }\AttributeTok{{-}fill}\NormalTok{ white }\AttributeTok{{-}gravity}\NormalTok{ south }\DataTypeTok{\textbackslash{}}
\NormalTok{{-}annotate +0+10 }\StringTok{"Mi texto"}\NormalTok{ output.jpg}

\CommentTok{\# Crear imagen con solo texto}
\ExtensionTok{convert} \AttributeTok{{-}size}\NormalTok{ 800x600 xc:white }\AttributeTok{{-}pointsize}\NormalTok{ 72 }\AttributeTok{{-}fill}\NormalTok{ black }\DataTypeTok{\textbackslash{}}
\NormalTok{{-}gravity center }\AttributeTok{{-}annotate}\NormalTok{ +0+0 }\StringTok{"Hello World"}\NormalTok{ text\_image.jpg}
\end{Highlighting}
\end{Shaded}

\begin{center}\rule{0.5\linewidth}{0.5pt}\end{center}

\section{Workflows multimedia
complejos}\label{workflows-multimedia-complejos}

\subsection{Pipeline completo de
video}\label{pipeline-completo-de-video}

\begin{Shaded}
\begin{Highlighting}[]
\CommentTok{\#!/bin/bash}
\CommentTok{\# video{-}pipeline.sh {-} Procesamiento completo de video}

\VariableTok{input\_video}\OperatorTok{=}\StringTok{"}\VariableTok{$1}\StringTok{"}
\VariableTok{output\_dir}\OperatorTok{=}\StringTok{"./processed"}

\ControlFlowTok{if} \BuiltInTok{[} \OtherTok{!} \OtherTok{{-}f} \StringTok{"}\VariableTok{$input\_video}\StringTok{"} \BuiltInTok{]}\KeywordTok{;} \ControlFlowTok{then}
    \BuiltInTok{echo} \StringTok{"Archivo no encontrado: }\VariableTok{$input\_video}\StringTok{"}
    \BuiltInTok{exit}\NormalTok{ 1}
\ControlFlowTok{fi}

\FunctionTok{mkdir} \AttributeTok{{-}p} \StringTok{"}\VariableTok{$output\_dir}\StringTok{"}
\VariableTok{filename}\OperatorTok{=}\VariableTok{$(}\FunctionTok{basename} \StringTok{"}\VariableTok{$input\_video}\StringTok{"}\NormalTok{ .mp4}\VariableTok{)}

\BuiltInTok{echo} \StringTok{"🎬 Procesando: }\VariableTok{$input\_video}\StringTok{"}

\CommentTok{\# 1. Extraer frames para análisis}
\ExtensionTok{ffmpeg} \AttributeTok{{-}i} \StringTok{"}\VariableTok{$input\_video}\StringTok{"} \AttributeTok{{-}vf}\NormalTok{ fps=1 }\StringTok{"}\VariableTok{$output\_dir}\StringTok{/frames/frame\_\%04d.png"}

\CommentTok{\# 2. Crear thumbnail}
\ExtensionTok{ffmpeg} \AttributeTok{{-}i} \StringTok{"}\VariableTok{$input\_video}\StringTok{"} \AttributeTok{{-}ss}\NormalTok{ 00:00:05 }\AttributeTok{{-}vframes}\NormalTok{ 1 }\AttributeTok{{-}vf}\NormalTok{ scale=320:180 }\StringTok{"}\VariableTok{$output\_dir}\StringTok{/}\VariableTok{$\{filename\}}\StringTok{\_thumb.jpg"}

\CommentTok{\# 3. Versiones en diferentes calidades}
\ExtensionTok{ffmpeg} \AttributeTok{{-}i} \StringTok{"}\VariableTok{$input\_video}\StringTok{"} \AttributeTok{{-}c:v}\NormalTok{ libx264 }\AttributeTok{{-}crf}\NormalTok{ 23 }\AttributeTok{{-}vf}\NormalTok{ scale=1920:1080 }\StringTok{"}\VariableTok{$output\_dir}\StringTok{/}\VariableTok{$\{filename\}}\StringTok{\_1080p.mp4"}
\ExtensionTok{ffmpeg} \AttributeTok{{-}i} \StringTok{"}\VariableTok{$input\_video}\StringTok{"} \AttributeTok{{-}c:v}\NormalTok{ libx264 }\AttributeTok{{-}crf}\NormalTok{ 28 }\AttributeTok{{-}vf}\NormalTok{ scale=1280:720 }\StringTok{"}\VariableTok{$output\_dir}\StringTok{/}\VariableTok{$\{filename\}}\StringTok{\_720p.mp4"}
\ExtensionTok{ffmpeg} \AttributeTok{{-}i} \StringTok{"}\VariableTok{$input\_video}\StringTok{"} \AttributeTok{{-}c:v}\NormalTok{ libx264 }\AttributeTok{{-}crf}\NormalTok{ 32 }\AttributeTok{{-}vf}\NormalTok{ scale=854:480 }\StringTok{"}\VariableTok{$output\_dir}\StringTok{/}\VariableTok{$\{filename\}}\StringTok{\_480p.mp4"}

\CommentTok{\# 4. Extraer audio}
\ExtensionTok{ffmpeg} \AttributeTok{{-}i} \StringTok{"}\VariableTok{$input\_video}\StringTok{"} \AttributeTok{{-}vn} \AttributeTok{{-}acodec}\NormalTok{ libmp3lame }\AttributeTok{{-}ab}\NormalTok{ 192k }\StringTok{"}\VariableTok{$output\_dir}\StringTok{/}\VariableTok{$\{filename\}}\StringTok{\_audio.mp3"}

\CommentTok{\# 5. Crear GIF preview}
\ExtensionTok{ffmpeg} \AttributeTok{{-}i} \StringTok{"}\VariableTok{$input\_video}\StringTok{"} \AttributeTok{{-}ss}\NormalTok{ 00:00:10 }\AttributeTok{{-}t}\NormalTok{ 5 }\AttributeTok{{-}vf} \StringTok{"fps=10,scale=480:{-}1:flags=lanczos"} \StringTok{"}\VariableTok{$output\_dir}\StringTok{/}\VariableTok{$\{filename\}}\StringTok{\_preview.gif"}

\BuiltInTok{echo} \StringTok{"✅ Procesamiento completado en: }\VariableTok{$output\_dir}\StringTok{"}
\end{Highlighting}
\end{Shaded}

\subsection{Optimización automática de
medios}\label{optimizaciuxf3n-automuxe1tica-de-medios}

\begin{Shaded}
\begin{Highlighting}[]
\CommentTok{\#!/bin/bash}
\CommentTok{\# media{-}optimizer.sh {-} Optimizar medios para web}

\FunctionTok{process\_images()} \KeywordTok{\{}
    \BuiltInTok{local} \VariableTok{dir}\OperatorTok{=}\StringTok{"}\VariableTok{$1}\StringTok{"}
    \BuiltInTok{echo} \StringTok{"📸 Optimizando imágenes en: }\VariableTok{$dir}\StringTok{"}
    
    \FunctionTok{find} \StringTok{"}\VariableTok{$dir}\StringTok{"} \AttributeTok{{-}type}\NormalTok{ f }\DataTypeTok{\textbackslash{}(} \AttributeTok{{-}iname} \StringTok{"*.jpg"} \AttributeTok{{-}o} \AttributeTok{{-}iname} \StringTok{"*.png"} \DataTypeTok{\textbackslash{})} \KeywordTok{|} \ControlFlowTok{while} \BuiltInTok{read} \AttributeTok{{-}r} \VariableTok{img}\KeywordTok{;} \ControlFlowTok{do}
        \CommentTok{\# Optimizar sin perder calidad visible}
        \ControlFlowTok{if} \KeywordTok{[[} \StringTok{"}\VariableTok{$img}\StringTok{"} \OtherTok{==} \PreprocessorTok{*}\NormalTok{.jpg }\KeywordTok{]];} \ControlFlowTok{then}
            \ExtensionTok{convert} \StringTok{"}\VariableTok{$img}\StringTok{"} \AttributeTok{{-}strip} \AttributeTok{{-}interlace}\NormalTok{ Plane }\AttributeTok{{-}gaussian{-}blur}\NormalTok{ 0.05 }\AttributeTok{{-}quality}\NormalTok{ 85\% }\StringTok{"}\VariableTok{$img}\StringTok{.optimized"}
            \FunctionTok{mv} \StringTok{"}\VariableTok{$img}\StringTok{.optimized"} \StringTok{"}\VariableTok{$img}\StringTok{"}
        \ControlFlowTok{elif} \KeywordTok{[[} \StringTok{"}\VariableTok{$img}\StringTok{"} \OtherTok{==} \PreprocessorTok{*}\NormalTok{.png }\KeywordTok{]];} \ControlFlowTok{then}
            \ExtensionTok{convert} \StringTok{"}\VariableTok{$img}\StringTok{"} \AttributeTok{{-}strip} \StringTok{"}\VariableTok{$img}\StringTok{.optimized"}
            \FunctionTok{mv} \StringTok{"}\VariableTok{$img}\StringTok{.optimized"} \StringTok{"}\VariableTok{$img}\StringTok{"}
        \ControlFlowTok{fi}
    \ControlFlowTok{done}
\KeywordTok{\}}

\FunctionTok{process\_videos()} \KeywordTok{\{}
    \BuiltInTok{local} \VariableTok{dir}\OperatorTok{=}\StringTok{"}\VariableTok{$1}\StringTok{"}
    \BuiltInTok{echo} \StringTok{"🎬 Optimizando videos en: }\VariableTok{$dir}\StringTok{"}
    
    \FunctionTok{find} \StringTok{"}\VariableTok{$dir}\StringTok{"} \AttributeTok{{-}type}\NormalTok{ f }\DataTypeTok{\textbackslash{}(} \AttributeTok{{-}iname} \StringTok{"*.mp4"} \AttributeTok{{-}o} \AttributeTok{{-}iname} \StringTok{"*.mov"} \DataTypeTok{\textbackslash{})} \KeywordTok{|} \ControlFlowTok{while} \BuiltInTok{read} \AttributeTok{{-}r} \VariableTok{video}\KeywordTok{;} \ControlFlowTok{do}
        \VariableTok{filename}\OperatorTok{=}\StringTok{"}\VariableTok{$\{video}\OperatorTok{\%}\NormalTok{.}\PreprocessorTok{*}\VariableTok{\}}\StringTok{"}
        \VariableTok{ext}\OperatorTok{=}\StringTok{"}\VariableTok{$\{video}\OperatorTok{\#\#}\PreprocessorTok{*}\NormalTok{.}\VariableTok{\}}\StringTok{"}
        
        \CommentTok{\# Crear versión optimizada}
        \ExtensionTok{ffmpeg} \AttributeTok{{-}i} \StringTok{"}\VariableTok{$video}\StringTok{"} \AttributeTok{{-}c:v}\NormalTok{ libx264 }\AttributeTok{{-}crf}\NormalTok{ 23 }\AttributeTok{{-}preset}\NormalTok{ slow }\AttributeTok{{-}c:a}\NormalTok{ aac }\AttributeTok{{-}b:a}\NormalTok{ 128k }\StringTok{"}\VariableTok{$\{filename\}}\StringTok{\_optimized.mp4"}
        
        \CommentTok{\# Comparar tamaños}
        \VariableTok{original\_size}\OperatorTok{=}\VariableTok{$(}\FunctionTok{stat} \AttributeTok{{-}f\%z} \StringTok{"}\VariableTok{$video}\StringTok{"}\VariableTok{)}
        \VariableTok{optimized\_size}\OperatorTok{=}\VariableTok{$(}\FunctionTok{stat} \AttributeTok{{-}f\%z} \StringTok{"}\VariableTok{$\{filename\}}\StringTok{\_optimized.mp4"}\VariableTok{)}
        
        \ControlFlowTok{if} \BuiltInTok{[} \StringTok{"}\VariableTok{$optimized\_size}\StringTok{"} \OtherTok{{-}lt} \StringTok{"}\VariableTok{$original\_size}\StringTok{"} \BuiltInTok{]}\KeywordTok{;} \ControlFlowTok{then}
            \FunctionTok{mv} \StringTok{"}\VariableTok{$\{filename\}}\StringTok{\_optimized.mp4"} \StringTok{"}\VariableTok{$video}\StringTok{"}
            \BuiltInTok{echo} \StringTok{"✅ }\VariableTok{$video}\StringTok{ optimizado (}\VariableTok{$(}\BuiltInTok{echo} \StringTok{"}\VariableTok{$original\_size}\StringTok{ {-} }\VariableTok{$optimized\_size}\StringTok{"} \KeywordTok{|} \FunctionTok{bc}\VariableTok{)}\StringTok{ bytes ahorrados)"}
        \ControlFlowTok{else}
            \FunctionTok{rm} \StringTok{"}\VariableTok{$\{filename\}}\StringTok{\_optimized.mp4"}
            \BuiltInTok{echo} \StringTok{"ℹ️ }\VariableTok{$video}\StringTok{ ya está optimizado"}
        \ControlFlowTok{fi}
    \ControlFlowTok{done}
\KeywordTok{\}}

\CommentTok{\# Procesar directorio especificado}
\VariableTok{target\_dir}\OperatorTok{=}\StringTok{"}\VariableTok{$\{1}\OperatorTok{:{-}}\NormalTok{.}\VariableTok{\}}\StringTok{"}
\ExtensionTok{process\_images} \StringTok{"}\VariableTok{$target\_dir}\StringTok{"}
\ExtensionTok{process\_videos} \StringTok{"}\VariableTok{$target\_dir}\StringTok{"}

\BuiltInTok{echo} \StringTok{"🎉 Optimización completada!"}
\end{Highlighting}
\end{Shaded}

\begin{tcolorbox}[enhanced jigsaw, toprule=.15mm, bottomrule=.15mm, opacityback=0, coltitle=black, rightrule=.15mm, colframe=quarto-callout-tip-color-frame, titlerule=0mm, opacitybacktitle=0.6, left=2mm, colback=white, bottomtitle=1mm, arc=.35mm, leftrule=.75mm, title=\textcolor{quarto-callout-tip-color}{\faLightbulb}\hspace{0.5em}{Tips para multimedia}, colbacktitle=quarto-callout-tip-color!10!white, breakable, toptitle=1mm]

\begin{itemize}
\tightlist
\item
  Usa \texttt{ffprobe} para analizar archivos multimedia antes de
  procesar
\item
  FFmpeg puede procesar múltiples archivos en paralelo con GNU parallel
\item
  ImageMagick's \texttt{identify} te da información detallada de
  imágenes
\item
  Mantén siempre respaldos antes de procesamiento destructivo
\end{itemize}

\end{tcolorbox}

\begin{tcolorbox}[enhanced jigsaw, toprule=.15mm, bottomrule=.15mm, opacityback=0, coltitle=black, rightrule=.15mm, colframe=quarto-callout-warning-color-frame, titlerule=0mm, opacitybacktitle=0.6, left=2mm, colback=white, bottomtitle=1mm, arc=.35mm, leftrule=.75mm, title=\textcolor{quarto-callout-warning-color}{\faExclamationTriangle}\hspace{0.5em}{Consideraciones de rendimiento}, colbacktitle=quarto-callout-warning-color!10!white, breakable, toptitle=1mm]

\begin{itemize}
\tightlist
\item
  FFmpeg es intensivo en CPU, especialmente con video HD/4K
\item
  ImageMagick puede usar mucha RAM con imágenes grandes
\item
  yt-dlp respeta rate limits, pero algunos sitios pueden bloquear por
  uso excesivo
\end{itemize}

\end{tcolorbox}

En el próximo capítulo exploraremos herramientas de red y descargas para
transferencia de datos e interacción con APIs.

\part{Red y Descargas}

\chapter{Red y Descargas}\label{red-y-descargas-2}

Las herramientas de red son esenciales para interactuar con APIs,
descargar archivos y transferir datos eficientemente. Esta sección cubre
las herramientas más potentes para comunicación HTTP y transferencia de
archivos.

\section{curl - Cliente HTTP versátil}\label{sec-curl}

curl es la herramienta universal para transferir datos con URLs,
soportando múltiples protocolos.

\subsection{GET básico}\label{get-buxe1sico}

\begin{Shaded}
\begin{Highlighting}[]
\CommentTok{\# Petición simple}
\ExtensionTok{curl}\NormalTok{ https://api.github.com/users/octocat}

\CommentTok{\# Guardar respuesta en archivo}
\ExtensionTok{curl} \AttributeTok{{-}o}\NormalTok{ response.json https://api.github.com/users/octocat}
\ExtensionTok{curl} \AttributeTok{{-}O}\NormalTok{ https://example.com/file.zip  }\CommentTok{\# Usar nombre del archivo}

\CommentTok{\# Seguir redirects}
\ExtensionTok{curl} \AttributeTok{{-}L}\NormalTok{ https://git.io/shortened{-}url}

\CommentTok{\# Mostrar headers de respuesta}
\ExtensionTok{curl} \AttributeTok{{-}I}\NormalTok{ https://google.com}
\ExtensionTok{curl} \AttributeTok{{-}i}\NormalTok{ https://api.github.com/users/octocat  }\CommentTok{\# Headers + body}
\end{Highlighting}
\end{Shaded}

\subsection{POST y datos}\label{post-y-datos}

\begin{Shaded}
\begin{Highlighting}[]
\CommentTok{\# POST con datos JSON}
\ExtensionTok{curl} \AttributeTok{{-}X}\NormalTok{ POST }\AttributeTok{{-}H} \StringTok{"Content{-}Type: application/json"} \DataTypeTok{\textbackslash{}}
     \AttributeTok{{-}d} \StringTok{\textquotesingle{}\{"name":"Juan","email":"juan@example.com"\}\textquotesingle{}} \DataTypeTok{\textbackslash{}}
\NormalTok{     https://api.example.com/users}

\CommentTok{\# POST con archivo}
\ExtensionTok{curl} \AttributeTok{{-}X}\NormalTok{ POST }\AttributeTok{{-}H} \StringTok{"Content{-}Type: application/json"} \DataTypeTok{\textbackslash{}}
     \AttributeTok{{-}d}\NormalTok{ @data.json https://api.example.com/users}

\CommentTok{\# Form data}
\ExtensionTok{curl} \AttributeTok{{-}X}\NormalTok{ POST }\AttributeTok{{-}d} \StringTok{"name=Juan\&email=juan@example.com"} \DataTypeTok{\textbackslash{}}
\NormalTok{     https://api.example.com/users}

\CommentTok{\# Multipart form (subir archivo)}
\ExtensionTok{curl} \AttributeTok{{-}X}\NormalTok{ POST }\AttributeTok{{-}F} \StringTok{"file=@documento.pdf"} \AttributeTok{{-}F} \StringTok{"description=Mi archivo"} \DataTypeTok{\textbackslash{}}
\NormalTok{     https://upload.example.com}
\end{Highlighting}
\end{Shaded}

\subsection{Autenticación}\label{autenticaciuxf3n-1}

\begin{Shaded}
\begin{Highlighting}[]
\CommentTok{\# Basic Auth}
\ExtensionTok{curl} \AttributeTok{{-}u}\NormalTok{ username:password https://api.example.com/protected}

\CommentTok{\# Bearer Token}
\ExtensionTok{curl} \AttributeTok{{-}H} \StringTok{"Authorization: Bearer your\_token\_here"} \DataTypeTok{\textbackslash{}}
\NormalTok{     https://api.example.com/protected}

\CommentTok{\# API Key en header}
\ExtensionTok{curl} \AttributeTok{{-}H} \StringTok{"X{-}API{-}Key: your\_api\_key"}\NormalTok{ https://api.example.com/data}

\CommentTok{\# OAuth2}
\ExtensionTok{curl} \AttributeTok{{-}H} \StringTok{"Authorization: Bearer }\VariableTok{$(}\FunctionTok{cat}\NormalTok{ token.txt}\VariableTok{)}\StringTok{"} \DataTypeTok{\textbackslash{}}
\NormalTok{     https://api.example.com/user}
\end{Highlighting}
\end{Shaded}

\subsection{Debugging y análisis}\label{debugging-y-anuxe1lisis}

\begin{Shaded}
\begin{Highlighting}[]
\CommentTok{\# Verbose output}
\ExtensionTok{curl} \AttributeTok{{-}v}\NormalTok{ https://api.github.com/users/octocat}

\CommentTok{\# Solo headers de request}
\ExtensionTok{curl} \AttributeTok{{-}D}\NormalTok{ headers.txt https://api.example.com}

\CommentTok{\# Tiempo de respuesta}
\ExtensionTok{curl} \AttributeTok{{-}w} \StringTok{"@curl{-}format.txt"} \AttributeTok{{-}o}\NormalTok{ /dev/null }\AttributeTok{{-}s}\NormalTok{ https://example.com}

\CommentTok{\# curl{-}format.txt:}
\CommentTok{\#      time\_namelookup:  \%\{time\_namelookup\}s\textbackslash{}n}
\CommentTok{\#         time\_connect:  \%\{time\_connect\}s\textbackslash{}n}
\CommentTok{\#      time\_appconnect:  \%\{time\_appconnect\}s\textbackslash{}n}
\CommentTok{\#     time\_pretransfer:  \%\{time\_pretransfer\}s\textbackslash{}n}
\CommentTok{\#        time\_redirect:  \%\{time\_redirect\}s\textbackslash{}n}
\CommentTok{\#   time\_starttransfer:  \%\{time\_starttransfer\}s\textbackslash{}n}
\CommentTok{\#                     {-}{-}{-}{-}{-}{-}{-}{-}{-}{-}\textbackslash{}n}
\CommentTok{\#           time\_total:  \%\{time\_total\}s\textbackslash{}n}
\end{Highlighting}
\end{Shaded}

\subsection{Casos avanzados}\label{casos-avanzados-2}

\subsubsection{Monitoreo de APIs}\label{monitoreo-de-apis}

\begin{Shaded}
\begin{Highlighting}[]
\CommentTok{\#!/bin/bash}
\CommentTok{\# api{-}monitor.sh}

\VariableTok{API\_URL}\OperatorTok{=}\StringTok{"https://api.example.com/health"}
\VariableTok{THRESHOLD}\OperatorTok{=}\NormalTok{2  }\CommentTok{\# segundos}

\ControlFlowTok{while} \FunctionTok{true}\KeywordTok{;} \ControlFlowTok{do}
    \VariableTok{response\_time}\OperatorTok{=}\VariableTok{$(}\ExtensionTok{curl} \AttributeTok{{-}w} \StringTok{"\%\{time\_total\}"} \AttributeTok{{-}o}\NormalTok{ /dev/null }\AttributeTok{{-}s} \StringTok{"}\VariableTok{$API\_URL}\StringTok{"}\VariableTok{)}
    \VariableTok{status\_code}\OperatorTok{=}\VariableTok{$(}\ExtensionTok{curl} \AttributeTok{{-}w} \StringTok{"\%\{http\_code\}"} \AttributeTok{{-}o}\NormalTok{ /dev/null }\AttributeTok{{-}s} \StringTok{"}\VariableTok{$API\_URL}\StringTok{"}\VariableTok{)}
    
    \ControlFlowTok{if} \KeywordTok{((} \VariableTok{$(}\BuiltInTok{echo} \StringTok{"}\VariableTok{$response\_time}\StringTok{ \textgreater{} }\VariableTok{$THRESHOLD}\StringTok{"} \KeywordTok{|} \FunctionTok{bc} \AttributeTok{{-}l}\VariableTok{)} \KeywordTok{));} \ControlFlowTok{then}
        \BuiltInTok{echo} \StringTok{"}\VariableTok{$(}\FunctionTok{date}\VariableTok{)}\StringTok{: SLOW {-} }\VariableTok{$response\_time}\StringTok{ s (Status: }\VariableTok{$status\_code}\StringTok{)"}
    \ControlFlowTok{elif} \BuiltInTok{[} \StringTok{"}\VariableTok{$status\_code}\StringTok{"} \OtherTok{!=} \StringTok{"200"} \BuiltInTok{]}\KeywordTok{;} \ControlFlowTok{then}
        \BuiltInTok{echo} \StringTok{"}\VariableTok{$(}\FunctionTok{date}\VariableTok{)}\StringTok{: ERROR {-} Status: }\VariableTok{$status\_code}\StringTok{"}
    \ControlFlowTok{else}
        \BuiltInTok{echo} \StringTok{"}\VariableTok{$(}\FunctionTok{date}\VariableTok{)}\StringTok{: OK {-} }\VariableTok{$response\_time}\StringTok{ s"}
    \ControlFlowTok{fi}
    
    \FunctionTok{sleep}\NormalTok{ 30}
\ControlFlowTok{done}
\end{Highlighting}
\end{Shaded}

\begin{center}\rule{0.5\linewidth}{0.5pt}\end{center}

\section{httpie - Cliente HTTP amigable}\label{sec-httpie}

HTTPie es una alternativa más amigable a curl con sintaxis intuitiva.

\subsection{Sintaxis básica}\label{sintaxis-buxe1sica-2}

\begin{Shaded}
\begin{Highlighting}[]
\CommentTok{\# GET simple}
\ExtensionTok{http}\NormalTok{ GET httpbin.org/json}

\CommentTok{\# POST con JSON (automático)}
\ExtensionTok{http}\NormalTok{ POST httpbin.org/post name=Juan age:=30 married:=true}

\CommentTok{\# Headers personalizados}
\ExtensionTok{http}\NormalTok{ GET example.com Authorization:}\StringTok{"Bearer token"}\NormalTok{ User{-}Agent:}\StringTok{"MyApp/1.0"}

\CommentTok{\# Query parameters}
\ExtensionTok{http}\NormalTok{ GET httpbin.org/get search==}\StringTok{"python tutorial"}\NormalTok{ limit:=10}
\end{Highlighting}
\end{Shaded}

\subsection{Tipos de datos}\label{tipos-de-datos}

\begin{Shaded}
\begin{Highlighting}[]
\CommentTok{\# String (por defecto)}
\ExtensionTok{http}\NormalTok{ POST httpbin.org/post name=Juan}

\CommentTok{\# Números y booleanos}
\ExtensionTok{http}\NormalTok{ POST httpbin.org/post age:=30 active:=true}

\CommentTok{\# Arrays}
\ExtensionTok{http}\NormalTok{ POST httpbin.org/post items:=}\StringTok{\textquotesingle{}["a","b","c"]\textquotesingle{}}

\CommentTok{\# Objetos anidados}
\ExtensionTok{http}\NormalTok{ POST httpbin.org/post user:=}\StringTok{\textquotesingle{}\{"name":"Juan","age":30\}\textquotesingle{}}

\CommentTok{\# Raw JSON}
\ExtensionTok{http}\NormalTok{ POST httpbin.org/post }\OperatorTok{\textless{}}\NormalTok{ data.json}
\end{Highlighting}
\end{Shaded}

\subsection{Archivos y forms}\label{archivos-y-forms}

\begin{Shaded}
\begin{Highlighting}[]
\CommentTok{\# Subir archivo}
\ExtensionTok{http} \AttributeTok{{-}{-}form}\NormalTok{ POST httpbin.org/post file@document.pdf description=}\StringTok{"Mi archivo"}

\CommentTok{\# Form data tradicional}
\ExtensionTok{http} \AttributeTok{{-}{-}form}\NormalTok{ POST httpbin.org/post name=Juan email=juan@example.com}

\CommentTok{\# Múltiples archivos}
\ExtensionTok{http} \AttributeTok{{-}{-}form}\NormalTok{ POST httpbin.org/post file1@file1.txt file2@file2.txt}
\end{Highlighting}
\end{Shaded}

\subsection{Casos prácticos}\label{casos-pruxe1cticos}

\subsubsection{Testing de APIs}\label{testing-de-apis}

\begin{Shaded}
\begin{Highlighting}[]
\CommentTok{\#!/bin/bash}
\CommentTok{\# test{-}api.sh}

\VariableTok{BASE\_URL}\OperatorTok{=}\StringTok{"https://api.example.com"}
\VariableTok{TOKEN}\OperatorTok{=}\VariableTok{$(}\ExtensionTok{http}\NormalTok{ POST }\StringTok{"}\VariableTok{$BASE\_URL}\StringTok{/auth/login"}\NormalTok{ username=admin password=secret }\KeywordTok{|} \ExtensionTok{jq} \AttributeTok{{-}r}\NormalTok{ .token}\VariableTok{)}

\CommentTok{\# Test endpoints}
\BuiltInTok{echo} \StringTok{"Testing GET /users"}
\ExtensionTok{http}\NormalTok{ GET }\StringTok{"}\VariableTok{$BASE\_URL}\StringTok{/users"}\NormalTok{ Authorization:}\StringTok{"Bearer }\VariableTok{$TOKEN}\StringTok{"}

\BuiltInTok{echo} \StringTok{"Testing POST /users"}
\ExtensionTok{http}\NormalTok{ POST }\StringTok{"}\VariableTok{$BASE\_URL}\StringTok{/users"}\NormalTok{ Authorization:}\StringTok{"Bearer }\VariableTok{$TOKEN}\StringTok{"} \DataTypeTok{\textbackslash{}}
\NormalTok{     name=}\StringTok{"Test User"}\NormalTok{ email=}\StringTok{"test@example.com"}

\BuiltInTok{echo} \StringTok{"Testing PUT /users/1"}
\ExtensionTok{http}\NormalTok{ PUT }\StringTok{"}\VariableTok{$BASE\_URL}\StringTok{/users/1"}\NormalTok{ Authorization:}\StringTok{"Bearer }\VariableTok{$TOKEN}\StringTok{"} \DataTypeTok{\textbackslash{}}
\NormalTok{     name=}\StringTok{"Updated User"}

\BuiltInTok{echo} \StringTok{"Testing DELETE /users/1"}
\ExtensionTok{http}\NormalTok{ DELETE }\StringTok{"}\VariableTok{$BASE\_URL}\StringTok{/users/1"}\NormalTok{ Authorization:}\StringTok{"Bearer }\VariableTok{$TOKEN}\StringTok{"}
\end{Highlighting}
\end{Shaded}

\begin{center}\rule{0.5\linewidth}{0.5pt}\end{center}

\section{wget - Descargador web}\label{sec-wget}

wget es un descargador no interactivo ideal para automatización y
descargas masivas.

\subsection{Descarga básica}\label{descarga-buxe1sica-1}

\begin{Shaded}
\begin{Highlighting}[]
\CommentTok{\# Descargar archivo}
\FunctionTok{wget}\NormalTok{ https://example.com/file.zip}

\CommentTok{\# Continuar descarga interrumpida}
\FunctionTok{wget} \AttributeTok{{-}c}\NormalTok{ https://example.com/large{-}file.zip}

\CommentTok{\# Descargar en background}
\FunctionTok{wget} \AttributeTok{{-}b}\NormalTok{ https://example.com/file.zip}

\CommentTok{\# Limitar velocidad}
\FunctionTok{wget} \AttributeTok{{-}{-}limit{-}rate}\OperatorTok{=}\NormalTok{200k https://example.com/file.zip}
\end{Highlighting}
\end{Shaded}

\subsection{Descargas recursivas}\label{descargas-recursivas}

\begin{Shaded}
\begin{Highlighting}[]
\CommentTok{\# Descargar sitio completo}
\FunctionTok{wget} \AttributeTok{{-}r} \AttributeTok{{-}p} \AttributeTok{{-}k}\NormalTok{ https://example.com}

\CommentTok{\# Opciones útiles para sitios:}
\CommentTok{\# {-}r: recursivo}
\CommentTok{\# {-}p: descargar imágenes, CSS, etc.}
\CommentTok{\# {-}k: convertir links a locales}
\CommentTok{\# {-}np: no subir a directorio padre}
\CommentTok{\# {-}l 2: máximo 2 niveles de profundidad}

\CommentTok{\# Espejo completo de sitio}
\FunctionTok{wget} \AttributeTok{{-}{-}mirror} \AttributeTok{{-}{-}convert{-}links} \AttributeTok{{-}{-}adjust{-}extension} \DataTypeTok{\textbackslash{}}
     \AttributeTok{{-}{-}page{-}requisites} \AttributeTok{{-}{-}no{-}parent}\NormalTok{ https://example.com}
\end{Highlighting}
\end{Shaded}

\subsection{Filtros y restricciones}\label{filtros-y-restricciones}

\begin{Shaded}
\begin{Highlighting}[]
\CommentTok{\# Solo ciertos tipos de archivo}
\FunctionTok{wget} \AttributeTok{{-}r} \AttributeTok{{-}A} \StringTok{"*.pdf,*.doc"}\NormalTok{ https://example.com/documents/}

\CommentTok{\# Excluir tipos}
\FunctionTok{wget} \AttributeTok{{-}r} \AttributeTok{{-}R} \StringTok{"*.gif,*.jpg"}\NormalTok{ https://example.com/}

\CommentTok{\# Limitar por tamaño}
\FunctionTok{wget} \AttributeTok{{-}{-}quota}\OperatorTok{=}\NormalTok{100m }\AttributeTok{{-}r}\NormalTok{ https://example.com/}

\CommentTok{\# User agent personalizado}
\FunctionTok{wget} \AttributeTok{{-}{-}user{-}agent}\OperatorTok{=}\StringTok{"Mozilla/5.0 (compatible; MyBot/1.0)"}\NormalTok{ https://example.com}
\end{Highlighting}
\end{Shaded}

\begin{center}\rule{0.5\linewidth}{0.5pt}\end{center}

\section{aria2 - Descargador avanzado}\label{sec-aria2}

aria2 es un descargador multihilo que soporta HTTP, FTP, BitTorrent y
más.

\subsection{Descarga acelerada}\label{descarga-acelerada}

\begin{Shaded}
\begin{Highlighting}[]
\CommentTok{\# Múltiples conexiones}
\ExtensionTok{aria2c} \AttributeTok{{-}x}\NormalTok{ 16 https://example.com/large{-}file.zip}

\CommentTok{\# Múltiples servidores}
\ExtensionTok{aria2c}\NormalTok{ https://mirror1.com/file.zip https://mirror2.com/file.zip}

\CommentTok{\# Continuar descargas}
\ExtensionTok{aria2c} \AttributeTok{{-}c}\NormalTok{ https://example.com/file.zip}

\CommentTok{\# Limitar velocidad}
\ExtensionTok{aria2c} \AttributeTok{{-}{-}max{-}download{-}limit}\OperatorTok{=}\NormalTok{1M https://example.com/file.zip}
\end{Highlighting}
\end{Shaded}

\subsection{Descargas por lotes}\label{descargas-por-lotes}

\begin{Shaded}
\begin{Highlighting}[]
\CommentTok{\# Desde archivo de URLs}
\BuiltInTok{echo} \StringTok{"https://example.com/file1.zip"} \OperatorTok{\textgreater{}}\NormalTok{ urls.txt}
\BuiltInTok{echo} \StringTok{"https://example.com/file2.zip"} \OperatorTok{\textgreater{}\textgreater{}}\NormalTok{ urls.txt}
\ExtensionTok{aria2c} \AttributeTok{{-}i}\NormalTok{ urls.txt}

\CommentTok{\# Con configuración avanzada}
\ExtensionTok{aria2c} \AttributeTok{{-}i}\NormalTok{ urls.txt }\AttributeTok{{-}x}\NormalTok{ 8 }\AttributeTok{{-}s}\NormalTok{ 8 }\AttributeTok{{-}{-}max{-}download{-}limit}\OperatorTok{=}\NormalTok{2M}
\end{Highlighting}
\end{Shaded}

\subsection{BitTorrent y magnet}\label{bittorrent-y-magnet}

\begin{Shaded}
\begin{Highlighting}[]
\CommentTok{\# Descargar torrent}
\ExtensionTok{aria2c}\NormalTok{ file.torrent}

\CommentTok{\# Magnet link}
\ExtensionTok{aria2c} \StringTok{"magnet:?xt=urn:btih:..."}

\CommentTok{\# Configuración para torrents}
\ExtensionTok{aria2c} \AttributeTok{{-}{-}seed{-}time}\OperatorTok{=}\NormalTok{60 }\AttributeTok{{-}{-}max{-}upload{-}limit}\OperatorTok{=}\NormalTok{100K file.torrent}
\end{Highlighting}
\end{Shaded}

\begin{center}\rule{0.5\linewidth}{0.5pt}\end{center}

\section{Workflows de red complejos}\label{workflows-de-red-complejos}

\subsection{Script de backup remoto}\label{script-de-backup-remoto}

\begin{Shaded}
\begin{Highlighting}[]
\CommentTok{\#!/bin/bash}
\CommentTok{\# remote{-}backup.sh}

\VariableTok{REMOTE\_SERVER}\OperatorTok{=}\StringTok{"backup.example.com"}
\VariableTok{BACKUP\_DIR}\OperatorTok{=}\StringTok{"/backups/}\VariableTok{$(}\FunctionTok{date}\NormalTok{ +\%Y\%m\%d}\VariableTok{)}\StringTok{"}
\VariableTok{LOCAL\_DIRS}\OperatorTok{=}\VariableTok{(}\StringTok{"}\VariableTok{$HOME}\StringTok{/Documents"} \StringTok{"}\VariableTok{$HOME}\StringTok{/Projects"}\VariableTok{)}

\CommentTok{\# Crear directorio de backup remoto}
\FunctionTok{ssh} \StringTok{"}\VariableTok{$REMOTE\_SERVER}\StringTok{"} \StringTok{"mkdir {-}p }\VariableTok{$BACKUP\_DIR}\StringTok{"}

\CommentTok{\# Backup de cada directorio}
\ControlFlowTok{for}\NormalTok{ dir }\KeywordTok{in} \StringTok{"}\VariableTok{$\{LOCAL\_DIRS}\OperatorTok{[@]}\VariableTok{\}}\StringTok{"}\KeywordTok{;} \ControlFlowTok{do}
    \ControlFlowTok{if} \BuiltInTok{[} \OtherTok{{-}d} \StringTok{"}\VariableTok{$dir}\StringTok{"} \BuiltInTok{]}\KeywordTok{;} \ControlFlowTok{then}
        \BuiltInTok{echo} \StringTok{"Backing up: }\VariableTok{$dir}\StringTok{"}
        \FunctionTok{tar}\NormalTok{ czf }\AttributeTok{{-}} \StringTok{"}\VariableTok{$dir}\StringTok{"} \KeywordTok{|} \DataTypeTok{\textbackslash{}}
        \ExtensionTok{curl} \AttributeTok{{-}X}\NormalTok{ POST }\DataTypeTok{\textbackslash{}}
             \AttributeTok{{-}H} \StringTok{"Content{-}Type: application/gzip"} \DataTypeTok{\textbackslash{}}
             \AttributeTok{{-}H} \StringTok{"X{-}Backup{-}Path: }\VariableTok{$(}\FunctionTok{basename} \StringTok{"}\VariableTok{$dir}\StringTok{"}\VariableTok{)}\StringTok{"} \DataTypeTok{\textbackslash{}}
             \AttributeTok{{-}{-}data{-}binary}\NormalTok{ @{-} }\DataTypeTok{\textbackslash{}}
             \StringTok{"https://}\VariableTok{$REMOTE\_SERVER}\StringTok{/api/backup/upload"}
    \ControlFlowTok{fi}
\ControlFlowTok{done}

\BuiltInTok{echo} \StringTok{"Backup completed to }\VariableTok{$REMOTE\_SERVER}\StringTok{:}\VariableTok{$BACKUP\_DIR}\StringTok{"}
\end{Highlighting}
\end{Shaded}

\subsection{Monitor de sitios web}\label{monitor-de-sitios-web}

\begin{Shaded}
\begin{Highlighting}[]
\CommentTok{\#!/bin/bash}
\CommentTok{\# website{-}monitor.sh}

\VariableTok{SITES}\OperatorTok{=}\VariableTok{(}
    \StringTok{"https://example.com"}
    \StringTok{"https://api.example.com/health"}
    \StringTok{"https://blog.example.com"}
\VariableTok{)}

\VariableTok{WEBHOOK\_URL}\OperatorTok{=}\StringTok{"https://hooks.slack.com/services/YOUR/SLACK/WEBHOOK"}

\FunctionTok{check\_site()} \KeywordTok{\{}
    \BuiltInTok{local} \VariableTok{url}\OperatorTok{=}\StringTok{"}\VariableTok{$1}\StringTok{"}
    \BuiltInTok{local} \VariableTok{response\_code}\OperatorTok{=}\VariableTok{$(}\ExtensionTok{curl} \AttributeTok{{-}o}\NormalTok{ /dev/null }\AttributeTok{{-}s} \AttributeTok{{-}w} \StringTok{"\%\{http\_code\}"} \StringTok{"}\VariableTok{$url}\StringTok{"}\VariableTok{)}
    \BuiltInTok{local} \VariableTok{response\_time}\OperatorTok{=}\VariableTok{$(}\ExtensionTok{curl} \AttributeTok{{-}o}\NormalTok{ /dev/null }\AttributeTok{{-}s} \AttributeTok{{-}w} \StringTok{"\%\{time\_total\}"} \StringTok{"}\VariableTok{$url}\StringTok{"}\VariableTok{)}
    
    \ControlFlowTok{if} \BuiltInTok{[} \StringTok{"}\VariableTok{$response\_code}\StringTok{"} \OtherTok{!=} \StringTok{"200"} \BuiltInTok{]}\KeywordTok{;} \ControlFlowTok{then}
        \CommentTok{\# Enviar alerta}
        \ExtensionTok{curl} \AttributeTok{{-}X}\NormalTok{ POST }\AttributeTok{{-}H} \StringTok{\textquotesingle{}Content{-}type: application/json\textquotesingle{}} \DataTypeTok{\textbackslash{}}
             \AttributeTok{{-}{-}data} \StringTok{"\{}\DataTypeTok{\textbackslash{}"}\StringTok{text}\DataTypeTok{\textbackslash{}"}\StringTok{:}\DataTypeTok{\textbackslash{}"}\StringTok{🚨 }\VariableTok{$url}\StringTok{ is DOWN (Status: }\VariableTok{$response\_code}\StringTok{)}\DataTypeTok{\textbackslash{}"}\StringTok{\}"} \DataTypeTok{\textbackslash{}}
             \StringTok{"}\VariableTok{$WEBHOOK\_URL}\StringTok{"}
        \ControlFlowTok{return} \DecValTok{1}
    \ControlFlowTok{elif} \KeywordTok{((} \VariableTok{$(}\BuiltInTok{echo} \StringTok{"}\VariableTok{$response\_time}\StringTok{ \textgreater{} 5"} \KeywordTok{|} \FunctionTok{bc} \AttributeTok{{-}l}\VariableTok{)} \KeywordTok{));} \ControlFlowTok{then}
        \ExtensionTok{curl} \AttributeTok{{-}X}\NormalTok{ POST }\AttributeTok{{-}H} \StringTok{\textquotesingle{}Content{-}type: application/json\textquotesingle{}} \DataTypeTok{\textbackslash{}}
             \AttributeTok{{-}{-}data} \StringTok{"\{}\DataTypeTok{\textbackslash{}"}\StringTok{text}\DataTypeTok{\textbackslash{}"}\StringTok{:}\DataTypeTok{\textbackslash{}"}\StringTok{⚠️ }\VariableTok{$url}\StringTok{ is SLOW (}\VariableTok{$\{response\_time\}}\StringTok{s)}\DataTypeTok{\textbackslash{}"}\StringTok{\}"} \DataTypeTok{\textbackslash{}}
             \StringTok{"}\VariableTok{$WEBHOOK\_URL}\StringTok{"}
    \ControlFlowTok{fi}
    
    \BuiltInTok{echo} \StringTok{"}\VariableTok{$url}\StringTok{: OK (}\VariableTok{$response\_time}\StringTok{ s)"}
\KeywordTok{\}}

\CommentTok{\# Verificar todos los sitios}
\ControlFlowTok{for}\NormalTok{ site }\KeywordTok{in} \StringTok{"}\VariableTok{$\{SITES}\OperatorTok{[@]}\VariableTok{\}}\StringTok{"}\KeywordTok{;} \ControlFlowTok{do}
    \ExtensionTok{check\_site} \StringTok{"}\VariableTok{$site}\StringTok{"}
\ControlFlowTok{done}
\end{Highlighting}
\end{Shaded}

\subsection{Sincronización de APIs}\label{sincronizaciuxf3n-de-apis}

\begin{Shaded}
\begin{Highlighting}[]
\CommentTok{\#!/bin/bash}
\CommentTok{\# api{-}sync.sh {-} Sincronizar datos entre APIs}

\VariableTok{SOURCE\_API}\OperatorTok{=}\StringTok{"https://source.example.com/api"}
\VariableTok{TARGET\_API}\OperatorTok{=}\StringTok{"https://target.example.com/api"}
\VariableTok{SOURCE\_TOKEN}\OperatorTok{=}\VariableTok{$(}\FunctionTok{cat}\NormalTok{ \textasciitilde{}/.tokens/source\_api}\VariableTok{)}
\VariableTok{TARGET\_TOKEN}\OperatorTok{=}\VariableTok{$(}\FunctionTok{cat}\NormalTok{ \textasciitilde{}/.tokens/target\_api}\VariableTok{)}

\CommentTok{\# Obtener datos de origen}
\BuiltInTok{echo} \StringTok{"Fetching data from source API..."}
\ExtensionTok{http}\NormalTok{ GET }\StringTok{"}\VariableTok{$SOURCE\_API}\StringTok{/users"}\NormalTok{ Authorization:}\StringTok{"Bearer }\VariableTok{$SOURCE\_TOKEN}\StringTok{"} \OperatorTok{\textgreater{}}\NormalTok{ source\_users.json}

\CommentTok{\# Procesar y transformar datos}
\BuiltInTok{echo} \StringTok{"Processing data..."}
\ExtensionTok{jq} \StringTok{\textquotesingle{}.users[] | \{}
\StringTok{    id: .id,}
\StringTok{    name: .full\_name,}
\StringTok{    email: .email\_address,}
\StringTok{    active: .status == "active"}
\StringTok{\}\textquotesingle{}}\NormalTok{ source\_users.json }\OperatorTok{\textgreater{}}\NormalTok{ transformed\_users.json}

\CommentTok{\# Enviar a API destino}
\BuiltInTok{echo} \StringTok{"Syncing to target API..."}
\ExtensionTok{jq} \AttributeTok{{-}c} \StringTok{\textquotesingle{}.[]\textquotesingle{}}\NormalTok{ transformed\_users.json }\KeywordTok{|} \ControlFlowTok{while} \BuiltInTok{read} \AttributeTok{{-}r} \VariableTok{user}\KeywordTok{;} \ControlFlowTok{do}
    \VariableTok{user\_id}\OperatorTok{=}\VariableTok{$(}\BuiltInTok{echo} \StringTok{"}\VariableTok{$user}\StringTok{"} \KeywordTok{|} \ExtensionTok{jq} \AttributeTok{{-}r} \StringTok{\textquotesingle{}.id\textquotesingle{}}\VariableTok{)}
    
    \CommentTok{\# Verificar si usuario existe}
    \VariableTok{existing}\OperatorTok{=}\VariableTok{$(}\ExtensionTok{http}\NormalTok{ GET }\StringTok{"}\VariableTok{$TARGET\_API}\StringTok{/users/}\VariableTok{$user\_id}\StringTok{"} \DataTypeTok{\textbackslash{}}
\NormalTok{               Authorization:}\StringTok{"Bearer }\VariableTok{$TARGET\_TOKEN}\StringTok{"} \DecValTok{2}\OperatorTok{\textgreater{}}\NormalTok{/dev/null}\VariableTok{)}
    
    \ControlFlowTok{if} \BuiltInTok{[} \VariableTok{$?} \OtherTok{{-}eq}\NormalTok{ 0 }\BuiltInTok{]}\KeywordTok{;} \ControlFlowTok{then}
        \CommentTok{\# Actualizar usuario existente}
        \BuiltInTok{echo} \StringTok{"}\VariableTok{$user}\StringTok{"} \KeywordTok{|} \ExtensionTok{http}\NormalTok{ PUT }\StringTok{"}\VariableTok{$TARGET\_API}\StringTok{/users/}\VariableTok{$user\_id}\StringTok{"} \DataTypeTok{\textbackslash{}}
\NormalTok{                       Authorization:}\StringTok{"Bearer }\VariableTok{$TARGET\_TOKEN}\StringTok{"}
    \ControlFlowTok{else}
        \CommentTok{\# Crear nuevo usuario}
        \BuiltInTok{echo} \StringTok{"}\VariableTok{$user}\StringTok{"} \KeywordTok{|} \ExtensionTok{http}\NormalTok{ POST }\StringTok{"}\VariableTok{$TARGET\_API}\StringTok{/users"} \DataTypeTok{\textbackslash{}}
\NormalTok{                       Authorization:}\StringTok{"Bearer }\VariableTok{$TARGET\_TOKEN}\StringTok{"}
    \ControlFlowTok{fi}
\ControlFlowTok{done}

\BuiltInTok{echo} \StringTok{"Sync completed!"}
\end{Highlighting}
\end{Shaded}

\begin{tcolorbox}[enhanced jigsaw, toprule=.15mm, bottomrule=.15mm, opacityback=0, coltitle=black, rightrule=.15mm, colframe=quarto-callout-tip-color-frame, titlerule=0mm, opacitybacktitle=0.6, left=2mm, colback=white, bottomtitle=1mm, arc=.35mm, leftrule=.75mm, title=\textcolor{quarto-callout-tip-color}{\faLightbulb}\hspace{0.5em}{Tips para herramientas de red}, colbacktitle=quarto-callout-tip-color!10!white, breakable, toptitle=1mm]

\begin{itemize}
\tightlist
\item
  Usa \texttt{jq} para procesar respuestas JSON de APIs
\item
  Combina \texttt{curl} con \texttt{fzf} para testing interactivo de
  endpoints
\item
  \texttt{httpie} es más legible para testing manual, \texttt{curl}
  mejor para scripts
\item
  Siempre maneja rate limiting en scripts automatizados
\end{itemize}

\end{tcolorbox}

\begin{tcolorbox}[enhanced jigsaw, toprule=.15mm, bottomrule=.15mm, opacityback=0, coltitle=black, rightrule=.15mm, colframe=quarto-callout-important-color-frame, titlerule=0mm, opacitybacktitle=0.6, left=2mm, colback=white, bottomtitle=1mm, arc=.35mm, leftrule=.75mm, title=\textcolor{quarto-callout-important-color}{\faExclamation}\hspace{0.5em}{Seguridad en red}, colbacktitle=quarto-callout-important-color!10!white, breakable, toptitle=1mm]

\begin{itemize}
\tightlist
\item
  Nunca hardcodees tokens o passwords en scripts
\item
  Usa variables de entorno o archivos de configuración seguros
\item
  Verifica certificados SSL con \texttt{curl\ -\/-cacert} en producción
\item
  Implementa timeouts apropiados para evitar scripts colgados
\end{itemize}

\end{tcolorbox}

En los próximos capítulos completaremos el libro con herramientas de
monitoreo, texto, utilidades y workflows avanzados.

\part{Monitoreo del Sistema}

\chapter{Monitoreo del Sistema}\label{monitoreo-del-sistema-2}

Las herramientas de monitoreo te permiten supervisar el rendimiento,
recursos y estado general del sistema en tiempo real con visualizaciones
modernas.

\section{htop - Monitor interactivo de procesos}\label{sec-htop}

htop es una versión mejorada de \texttt{top} con interfaz colorida e
interactiva.

\subsection{Navegación básica}\label{navegaciuxf3n-buxe1sica-1}

\begin{Shaded}
\begin{Highlighting}[]
\CommentTok{\# Ejecutar htop}
\ExtensionTok{htop}

\CommentTok{\# Comandos dentro de htop:}
\CommentTok{\# F1: Ayuda}
\CommentTok{\# F2: Configuración}
\CommentTok{\# F3: Buscar proceso}
\CommentTok{\# F4: Filtrar procesos}
\CommentTok{\# F5: Vista de árbol}
\CommentTok{\# F6: Ordenar por columna}
\CommentTok{\# F9: Matar proceso}
\CommentTok{\# F10: Salir}
\end{Highlighting}
\end{Shaded}

\subsection{Funciones útiles}\label{funciones-uxfatiles}

\subsubsection{Gestión de procesos}\label{gestiuxf3n-de-procesos-1}

\begin{itemize}
\tightlist
\item
  \textbf{Space}: Marcar proceso
\item
  \textbf{U}: Mostrar solo procesos de un usuario
\item
  \textbf{t}: Vista de árbol de procesos
\item
  \textbf{H}: Mostrar/ocultar hilos de usuario
\item
  \textbf{K}: Mostrar/ocultar hilos del kernel
\end{itemize}

\subsubsection{Filtros y búsqueda}\label{filtros-y-buxfasqueda}

\begin{Shaded}
\begin{Highlighting}[]
\CommentTok{\# Dentro de htop:}
\CommentTok{\# F4 + "python" {-} Filtrar solo procesos de Python}
\CommentTok{\# F3 + "nginx" {-} Buscar proceso nginx}
\CommentTok{\# \textbackslash{} {-} Filtro incremental}
\end{Highlighting}
\end{Shaded}

\subsection{Configuración
personalizada}\label{configuraciuxf3n-personalizada-1}

Archivo \texttt{\textasciitilde{}/.config/htop/htoprc}:

\begin{Shaded}
\begin{Highlighting}[]
\CommentTok{\# Campos a mostrar}
\DataTypeTok{fields}\OtherTok{=}\StringTok{0 48 17 18 38 39 40 2 46 47 49 1}
\CommentTok{\# Opciones}
\DataTypeTok{hide\_kernel\_threads}\OtherTok{=}\DecValTok{1}
\DataTypeTok{hide\_userland\_threads}\OtherTok{=}\DecValTok{1}
\DataTypeTok{show\_thread\_names}\OtherTok{=}\DecValTok{0}
\DataTypeTok{show\_program\_path}\OtherTok{=}\DecValTok{1}
\DataTypeTok{highlight\_base\_name}\OtherTok{=}\DecValTok{1}
\DataTypeTok{tree\_view}\OtherTok{=}\DecValTok{1}
\end{Highlighting}
\end{Shaded}

\begin{center}\rule{0.5\linewidth}{0.5pt}\end{center}

\section{fastfetch - Información del sistema
moderna}\label{sec-fastfetch}

\texttt{fastfetch} es el sucesor oficial y moderno de \texttt{neofetch},
ofreciendo mejor rendimiento y más opciones de personalización.

\begin{tcolorbox}[enhanced jigsaw, toprule=.15mm, bottomrule=.15mm, opacityback=0, coltitle=black, rightrule=.15mm, colframe=quarto-callout-warning-color-frame, titlerule=0mm, opacitybacktitle=0.6, left=2mm, colback=white, bottomtitle=1mm, arc=.35mm, leftrule=.75mm, title=\textcolor{quarto-callout-warning-color}{\faExclamationTriangle}\hspace{0.5em}{⚠️ Migración desde neofetch}, colbacktitle=quarto-callout-warning-color!10!white, breakable, toptitle=1mm]

\texttt{neofetch} ha sido marcado como \textbf{obsoleto} y ya no recibe
actualizaciones. \texttt{fastfetch} es el reemplazo oficial recomendado
por la comunidad.

\begin{Shaded}
\begin{Highlighting}[]
\CommentTok{\# Desinstalar neofetch si está instalado}
\ExtensionTok{brew}\NormalTok{ uninstall neofetch }\DecValTok{2}\OperatorTok{\textgreater{}}\NormalTok{/dev/null }\KeywordTok{||} \FunctionTok{true}

\CommentTok{\# Instalar fastfetch (el reemplazo moderno)}
\ExtensionTok{brew}\NormalTok{ install fastfetch}
\end{Highlighting}
\end{Shaded}

\end{tcolorbox}

\subsection{Uso básico}\label{uso-buxe1sico-4}

\begin{Shaded}
\begin{Highlighting}[]
\CommentTok{\# Información completa del sistema}
\ExtensionTok{fastfetch}

\CommentTok{\# Solo información específica}
\ExtensionTok{fastfetch} \AttributeTok{{-}{-}disable}\NormalTok{ packages shell resolution de wm theme icons cursor}

\CommentTok{\# Con logo personalizado}
\ExtensionTok{fastfetch} \AttributeTok{{-}{-}logo}\NormalTok{ /path/to/image.png}

\CommentTok{\# ASCII distro específico}
\ExtensionTok{fastfetch} \AttributeTok{{-}{-}logo}\NormalTok{ arch}
\end{Highlighting}
\end{Shaded}

\subsection{Configuración
personalizada}\label{configuraciuxf3n-personalizada-2}

Archivo \texttt{\textasciitilde{}/.config/fastfetch/config.jsonc}:

\begin{Shaded}
\begin{Highlighting}[]
\FunctionTok{\{}
    \DataTypeTok{"logo"}\FunctionTok{:} \FunctionTok{\{}
        \DataTypeTok{"source"}\FunctionTok{:} \StringTok{"macos"}\FunctionTok{,}
        \DataTypeTok{"padding"}\FunctionTok{:} \FunctionTok{\{}
            \DataTypeTok{"top"}\FunctionTok{:} \DecValTok{2}\FunctionTok{,}
            \DataTypeTok{"left"}\FunctionTok{:} \DecValTok{4}
        \FunctionTok{\}}
    \FunctionTok{\},}
    \DataTypeTok{"display"}\FunctionTok{:} \FunctionTok{\{}
        \DataTypeTok{"separator"}\FunctionTok{:} \StringTok{" {-}\textgreater{} "}\FunctionTok{,}
        \DataTypeTok{"key"}\FunctionTok{:} \FunctionTok{\{}
            \DataTypeTok{"width"}\FunctionTok{:} \DecValTok{10}
        \FunctionTok{\}}
    \FunctionTok{\},}
    \DataTypeTok{"modules"}\FunctionTok{:} \OtherTok{[}
        \StringTok{"title"}\OtherTok{,}
        \StringTok{"separator"}\OtherTok{,}
        \StringTok{"os"}\OtherTok{,}
        \StringTok{"host"}\OtherTok{,} 
        \StringTok{"kernel"}\OtherTok{,}
        \StringTok{"uptime"}\OtherTok{,}
        \StringTok{"packages"}\OtherTok{,}
        \StringTok{"shell"}\OtherTok{,}
        \StringTok{"terminal"}\OtherTok{,}
        \StringTok{"cpu"}\OtherTok{,}
        \StringTok{"gpu"}\OtherTok{,}
        \StringTok{"memory"}\OtherTok{,}
        \StringTok{"disk"}
    \OtherTok{]}
\FunctionTok{\}}
\end{Highlighting}
\end{Shaded}

\subsection{Comparación de
rendimiento}\label{comparaciuxf3n-de-rendimiento}

\begin{longtable}[]{@{}lllll@{}}
\toprule\noalign{}
Herramienta & Estado & Velocidad & Memoria & Mantenimiento \\
\midrule\noalign{}
\endhead
\bottomrule\noalign{}
\endlastfoot
\texttt{fastfetch} & ✅ Activo & \textasciitilde50ms & Bajo & Activo \\
\texttt{neofetch} & ❌ Obsoleto & \textasciitilde200ms & Alto &
Abandonado \\
\texttt{screenfetch} & ⚠️ Inactivo & \textasciitilde150ms & Medio &
Mínimo \\
\end{longtable}

\subsection{Scripts con fastfetch}\label{scripts-con-fastfetch}

\subsubsection{System info script
moderno}\label{system-info-script-moderno}

\begin{Shaded}
\begin{Highlighting}[]
\CommentTok{\#!/bin/bash}
\CommentTok{\# system{-}report.sh}

\BuiltInTok{echo} \StringTok{"=== SYSTEM REPORT }\VariableTok{$(}\FunctionTok{date}\VariableTok{)}\StringTok{ ==="}
\ExtensionTok{fastfetch} \AttributeTok{{-}{-}format}\NormalTok{ json }\OperatorTok{\textgreater{}}\NormalTok{ /tmp/system\_info.json}

\CommentTok{\# Extraer información específica con jq}
\VariableTok{OS}\OperatorTok{=}\VariableTok{$(}\ExtensionTok{jq} \AttributeTok{{-}r} \StringTok{\textquotesingle{}.os.name\textquotesingle{}}\NormalTok{ /tmp/system\_info.json}\VariableTok{)}
\VariableTok{KERNEL}\OperatorTok{=}\VariableTok{$(}\ExtensionTok{jq} \AttributeTok{{-}r} \StringTok{\textquotesingle{}.kernel.release\textquotesingle{}}\NormalTok{ /tmp/system\_info.json}\VariableTok{)}
\VariableTok{CPU}\OperatorTok{=}\VariableTok{$(}\ExtensionTok{jq} \AttributeTok{{-}r} \StringTok{\textquotesingle{}.cpu.name\textquotesingle{}}\NormalTok{ /tmp/system\_info.json}\VariableTok{)}
\VariableTok{MEMORY}\OperatorTok{=}\VariableTok{$(}\ExtensionTok{jq} \AttributeTok{{-}r} \StringTok{\textquotesingle{}.memory.used + " / " + .memory.total\textquotesingle{}}\NormalTok{ /tmp/system\_info.json}\VariableTok{)}

\BuiltInTok{echo} \StringTok{"OS: }\VariableTok{$OS}\StringTok{"}
\BuiltInTok{echo} \StringTok{"Kernel: }\VariableTok{$KERNEL}\StringTok{"} 
\BuiltInTok{echo} \StringTok{"CPU: }\VariableTok{$CPU}\StringTok{"}
\BuiltInTok{echo} \StringTok{"Memory: }\VariableTok{$MEMORY}\StringTok{"}

\BuiltInTok{echo} \AttributeTok{{-}e} \StringTok{"\textbackslash{}n=== DISK USAGE ==="}
\FunctionTok{df} \AttributeTok{{-}h} \KeywordTok{|} \FunctionTok{grep} \AttributeTok{{-}E} \StringTok{\textquotesingle{}\^{}(/dev/)\textquotesingle{}}

\BuiltInTok{echo} \AttributeTok{{-}e} \StringTok{"\textbackslash{}n=== TOP PROCESSES ==="}
\FunctionTok{ps}\NormalTok{ aux }\AttributeTok{{-}{-}sort}\OperatorTok{=}\NormalTok{{-}\%cpu }\KeywordTok{|} \FunctionTok{head} \AttributeTok{{-}6}

\BuiltInTok{echo} \AttributeTok{{-}e} \StringTok{"\textbackslash{}n=== NETWORK INTERFACES ==="}
\ExtensionTok{ip}\NormalTok{ addr show }\KeywordTok{|} \FunctionTok{grep} \AttributeTok{{-}E} \StringTok{\textquotesingle{}\^{}[0{-}9]+:|inet \textquotesingle{}}

\BuiltInTok{echo} \AttributeTok{{-}e} \StringTok{"\textbackslash{}n=== RECENT LOGINS ==="}
\FunctionTok{last} \AttributeTok{{-}n}\NormalTok{ 5}

\BuiltInTok{echo} \AttributeTok{{-}e} \StringTok{"\textbackslash{}n=== SYSTEM LOAD ==="}
\FunctionTok{uptime}
\end{Highlighting}
\end{Shaded}

\begin{center}\rule{0.5\linewidth}{0.5pt}\end{center}

\section{Herramientas
complementarias}\label{herramientas-complementarias-1}

\subsection{iotop - Monitor de I/O}\label{iotop---monitor-de-io}

\begin{Shaded}
\begin{Highlighting}[]
\CommentTok{\# Monitorear I/O por proceso}
\FunctionTok{sudo}\NormalTok{ iotop}

\CommentTok{\# Solo procesos activos}
\FunctionTok{sudo}\NormalTok{ iotop }\AttributeTok{{-}o}

\CommentTok{\# Acumulativo}
\FunctionTok{sudo}\NormalTok{ iotop }\AttributeTok{{-}a}
\end{Highlighting}
\end{Shaded}

\subsection{nethogs - Monitor de red por
proceso}\label{nethogs---monitor-de-red-por-proceso}

\begin{Shaded}
\begin{Highlighting}[]
\CommentTok{\# Monitor de ancho de banda}
\FunctionTok{sudo}\NormalTok{ nethogs}

\CommentTok{\# Interfaz específica}
\FunctionTok{sudo}\NormalTok{ nethogs eth0}
\end{Highlighting}
\end{Shaded}

\subsection{Monitoring script
completo}\label{monitoring-script-completo}

\begin{Shaded}
\begin{Highlighting}[]
\CommentTok{\#!/bin/bash}
\CommentTok{\# system{-}monitor.sh}

\VariableTok{LOG\_FILE}\OperatorTok{=}\StringTok{"/var/log/system{-}monitor.log"}
\VariableTok{ALERT\_THRESHOLD\_CPU}\OperatorTok{=}\NormalTok{80}
\VariableTok{ALERT\_THRESHOLD\_MEM}\OperatorTok{=}\NormalTok{85}
\VariableTok{ALERT\_THRESHOLD\_DISK}\OperatorTok{=}\NormalTok{90}

\FunctionTok{log\_message()} \KeywordTok{\{}
    \BuiltInTok{echo} \StringTok{"[}\VariableTok{$(}\FunctionTok{date} \StringTok{\textquotesingle{}+\%Y{-}\%m{-}\%d \%H:\%M:\%S\textquotesingle{}}\VariableTok{)}\StringTok{] }\VariableTok{$1}\StringTok{"} \KeywordTok{|} \FunctionTok{tee} \AttributeTok{{-}a} \StringTok{"}\VariableTok{$LOG\_FILE}\StringTok{"}
\KeywordTok{\}}

\FunctionTok{check\_cpu()} \KeywordTok{\{}
    \VariableTok{cpu\_usage}\OperatorTok{=}\VariableTok{$(}\ExtensionTok{top} \AttributeTok{{-}bn1} \KeywordTok{|} \FunctionTok{grep} \StringTok{"Cpu(s)"} \KeywordTok{|} \FunctionTok{awk} \StringTok{\textquotesingle{}\{print $2\}\textquotesingle{}} \KeywordTok{|} \FunctionTok{awk} \AttributeTok{{-}F}\StringTok{\textquotesingle{}\%\textquotesingle{}} \StringTok{\textquotesingle{}\{print $1\}\textquotesingle{}}\VariableTok{)}
    \VariableTok{cpu\_usage}\OperatorTok{=}\VariableTok{$\{cpu\_usage}\OperatorTok{\%}\NormalTok{.}\PreprocessorTok{*}\VariableTok{\}}  \CommentTok{\# Remove decimal}
    
    \ControlFlowTok{if} \BuiltInTok{[} \StringTok{"}\VariableTok{$cpu\_usage}\StringTok{"} \OtherTok{{-}gt} \StringTok{"}\VariableTok{$ALERT\_THRESHOLD\_CPU}\StringTok{"} \BuiltInTok{]}\KeywordTok{;} \ControlFlowTok{then}
        \ExtensionTok{log\_message} \StringTok{"ALERT: CPU usage is }\VariableTok{$\{cpu\_usage\}}\StringTok{\%"}
        \ControlFlowTok{return} \DecValTok{1}
    \ControlFlowTok{fi}
    \ControlFlowTok{return} \DecValTok{0}
\KeywordTok{\}}

\FunctionTok{check\_memory()} \KeywordTok{\{}
    \VariableTok{mem\_usage}\OperatorTok{=}\VariableTok{$(}\FunctionTok{free} \KeywordTok{|} \FunctionTok{grep}\NormalTok{ Mem }\KeywordTok{|} \FunctionTok{awk} \StringTok{\textquotesingle{}\{printf("\%.0f", $3/$2 * 100.0)\}\textquotesingle{}}\VariableTok{)}
    
    \ControlFlowTok{if} \BuiltInTok{[} \StringTok{"}\VariableTok{$mem\_usage}\StringTok{"} \OtherTok{{-}gt} \StringTok{"}\VariableTok{$ALERT\_THRESHOLD\_MEM}\StringTok{"} \BuiltInTok{]}\KeywordTok{;} \ControlFlowTok{then}
        \ExtensionTok{log\_message} \StringTok{"ALERT: Memory usage is }\VariableTok{$\{mem\_usage\}}\StringTok{\%"}
        \ControlFlowTok{return} \DecValTok{1}
    \ControlFlowTok{fi}
    \ControlFlowTok{return} \DecValTok{0}
\KeywordTok{\}}

\FunctionTok{check\_disk()} \KeywordTok{\{}
    \VariableTok{disk\_usage}\OperatorTok{=}\VariableTok{$(}\FunctionTok{df}\NormalTok{ / }\KeywordTok{|} \FunctionTok{tail} \AttributeTok{{-}1} \KeywordTok{|} \FunctionTok{awk} \StringTok{\textquotesingle{}\{print $5\}\textquotesingle{}} \KeywordTok{|} \FunctionTok{sed} \StringTok{\textquotesingle{}s/\%//\textquotesingle{}}\VariableTok{)}
    
    \ControlFlowTok{if} \BuiltInTok{[} \StringTok{"}\VariableTok{$disk\_usage}\StringTok{"} \OtherTok{{-}gt} \StringTok{"}\VariableTok{$ALERT\_THRESHOLD\_DISK}\StringTok{"} \BuiltInTok{]}\KeywordTok{;} \ControlFlowTok{then}
        \ExtensionTok{log\_message} \StringTok{"ALERT: Disk usage is }\VariableTok{$\{disk\_usage\}}\StringTok{\%"}
        \ControlFlowTok{return} \DecValTok{1}
    \ControlFlowTok{fi}
    \ControlFlowTok{return} \DecValTok{0}
\KeywordTok{\}}

\CommentTok{\# Ejecutar verificaciones}
\ExtensionTok{log\_message} \StringTok{"Starting system check..."}

\ExtensionTok{check\_cpu} \KeywordTok{\&\&} \ExtensionTok{log\_message} \StringTok{"CPU: OK"}
\ExtensionTok{check\_memory} \KeywordTok{\&\&} \ExtensionTok{log\_message} \StringTok{"Memory: OK"}
\ExtensionTok{check\_disk} \KeywordTok{\&\&} \ExtensionTok{log\_message} \StringTok{"Disk: OK"}

\ExtensionTok{log\_message} \StringTok{"System check completed"}
\end{Highlighting}
\end{Shaded}

\begin{tcolorbox}[enhanced jigsaw, toprule=.15mm, bottomrule=.15mm, opacityback=0, coltitle=black, rightrule=.15mm, colframe=quarto-callout-tip-color-frame, titlerule=0mm, opacitybacktitle=0.6, left=2mm, colback=white, bottomtitle=1mm, arc=.35mm, leftrule=.75mm, title=\textcolor{quarto-callout-tip-color}{\faLightbulb}\hspace{0.5em}{Tips de monitoreo}, colbacktitle=quarto-callout-tip-color!10!white, breakable, toptitle=1mm]

\begin{itemize}
\tightlist
\item
  Usa htop para debugging de rendimiento en tiempo real
\item
  fastfetch es útil para screenshots y reportes de sistema (reemplaza
  neofetch)
\item
  Combina estas herramientas con scripts para monitoreo automatizado
\end{itemize}

\end{tcolorbox}

En el próximo capítulo cubriremos herramientas de texto y documentos.

\part{Texto y Documentos}

\chapter{Texto y Documentos}\label{texto-y-documentos-3}

Las herramientas de texto modernas mejoran significativamente la
experiencia de lectura, edición y conversión de documentos desde la
terminal.

\section{bat - Visualizador con sintaxis}\label{sec-bat}

bat es un reemplazo de \texttt{cat} con resaltado de sintaxis y
características avanzadas.

\subsection{Uso básico}\label{uso-buxe1sico-5}

\begin{Shaded}
\begin{Highlighting}[]
\CommentTok{\# Ver archivo con sintaxis highlight}
\ExtensionTok{bat}\NormalTok{ archivo.py}
\ExtensionTok{bat}\NormalTok{ package.json}

\CommentTok{\# Múltiples archivos}
\ExtensionTok{bat} \PreprocessorTok{*}\NormalTok{.js}

\CommentTok{\# Con números de línea}
\ExtensionTok{bat} \AttributeTok{{-}n}\NormalTok{ archivo.txt}

\CommentTok{\# Rango específico de líneas}
\ExtensionTok{bat} \AttributeTok{{-}r}\NormalTok{ 10:20 archivo.py}
\end{Highlighting}
\end{Shaded}

\subsection{Configuración y temas}\label{configuraciuxf3n-y-temas}

\begin{Shaded}
\begin{Highlighting}[]
\CommentTok{\# Listar temas disponibles}
\ExtensionTok{bat} \AttributeTok{{-}{-}list{-}themes}

\CommentTok{\# Usar tema específico}
\ExtensionTok{bat} \AttributeTok{{-}{-}theme}\OperatorTok{=}\StringTok{"Monokai Extended"}\NormalTok{ archivo.py}

\CommentTok{\# Configurar tema por defecto}
\BuiltInTok{export} \VariableTok{BAT\_THEME}\OperatorTok{=}\StringTok{"gruvbox{-}dark"}

\CommentTok{\# Configurar en \textasciitilde{}/.config/bat/config}
\ExtensionTok{{-}{-}theme=}\StringTok{"gruvbox{-}dark"}
\ExtensionTok{{-}{-}style=}\StringTok{"numbers,changes,header"}
\ExtensionTok{{-}{-}paging=auto}
\end{Highlighting}
\end{Shaded}

\subsection{Integración con otras
herramientas}\label{integraciuxf3n-con-otras-herramientas-1}

\begin{Shaded}
\begin{Highlighting}[]
\CommentTok{\# Como pager para man}
\BuiltInTok{export} \VariableTok{MANPAGER}\OperatorTok{=}\StringTok{"sh {-}c \textquotesingle{}col {-}bx | bat {-}l man {-}p\textquotesingle{}"}

\CommentTok{\# Con find}
\FunctionTok{find}\NormalTok{ . }\AttributeTok{{-}name} \StringTok{"*.py"} \AttributeTok{{-}exec}\NormalTok{ bat \{\} +}

\CommentTok{\# Con fzf para vista previa}
\FunctionTok{find}\NormalTok{ . }\AttributeTok{{-}name} \StringTok{"*.md"} \KeywordTok{|} \ExtensionTok{fzf} \AttributeTok{{-}{-}preview} \StringTok{\textquotesingle{}bat {-}{-}color=always \{\}\textquotesingle{}}
\end{Highlighting}
\end{Shaded}

\begin{center}\rule{0.5\linewidth}{0.5pt}\end{center}

\section{pandoc - Conversor universal}\label{sec-pandoc}

pandoc convierte documentos entre múltiples formatos.

\subsection{Conversiones básicas}\label{conversiones-buxe1sicas}

\begin{Shaded}
\begin{Highlighting}[]
\CommentTok{\# Markdown a HTML}
\ExtensionTok{pandoc}\NormalTok{ documento.md }\AttributeTok{{-}o}\NormalTok{ documento.html}

\CommentTok{\# Markdown a PDF (requiere LaTeX)}
\ExtensionTok{pandoc}\NormalTok{ documento.md }\AttributeTok{{-}o}\NormalTok{ documento.pdf}

\CommentTok{\# HTML a Markdown}
\ExtensionTok{pandoc}\NormalTok{ pagina.html }\AttributeTok{{-}o}\NormalTok{ pagina.md}

\CommentTok{\# Multiple formatos}
\ExtensionTok{pandoc}\NormalTok{ input.md }\AttributeTok{{-}o}\NormalTok{ output.docx}
\ExtensionTok{pandoc}\NormalTok{ input.md }\AttributeTok{{-}o}\NormalTok{ output.epub}
\end{Highlighting}
\end{Shaded}

\subsection{Opciones avanzadas}\label{opciones-avanzadas-1}

\begin{Shaded}
\begin{Highlighting}[]
\CommentTok{\# Con tabla de contenidos}
\ExtensionTok{pandoc} \AttributeTok{{-}{-}toc}\NormalTok{ documento.md }\AttributeTok{{-}o}\NormalTok{ documento.html}

\CommentTok{\# CSS personalizado}
\ExtensionTok{pandoc} \AttributeTok{{-}c}\NormalTok{ styles.css documento.md }\AttributeTok{{-}o}\NormalTok{ documento.html}

\CommentTok{\# Plantilla personalizada}
\ExtensionTok{pandoc} \AttributeTok{{-}{-}template}\OperatorTok{=}\NormalTok{mi{-}plantilla.html documento.md }\AttributeTok{{-}o}\NormalTok{ documento.html}

\CommentTok{\# Metadatos}
\ExtensionTok{pandoc} \AttributeTok{{-}M}\NormalTok{ title=}\StringTok{"Mi Documento"} \AttributeTok{{-}M}\NormalTok{ author=}\StringTok{"Tu Nombre"}\NormalTok{ documento.md }\AttributeTok{{-}o}\NormalTok{ documento.pdf}
\end{Highlighting}
\end{Shaded}

\subsection{Script de conversión
masiva}\label{script-de-conversiuxf3n-masiva}

\begin{Shaded}
\begin{Highlighting}[]
\CommentTok{\#!/bin/bash}
\CommentTok{\# convert{-}docs.sh}

\VariableTok{input\_dir}\OperatorTok{=}\StringTok{"}\VariableTok{$1}\StringTok{"}
\VariableTok{output\_dir}\OperatorTok{=}\StringTok{"}\VariableTok{$2}\StringTok{"}
\VariableTok{format}\OperatorTok{=}\StringTok{"}\VariableTok{$3}\StringTok{"}

\ControlFlowTok{if} \BuiltInTok{[} \VariableTok{$\#} \OtherTok{{-}ne}\NormalTok{ 3 }\BuiltInTok{]}\KeywordTok{;} \ControlFlowTok{then}
    \BuiltInTok{echo} \StringTok{"Uso: }\VariableTok{$0}\StringTok{ \textless{}input\_dir\textgreater{} \textless{}output\_dir\textgreater{} \textless{}format\textgreater{}"}
    \BuiltInTok{echo} \StringTok{"Formatos: html, pdf, docx, epub"}
    \BuiltInTok{exit}\NormalTok{ 1}
\ControlFlowTok{fi}

\FunctionTok{mkdir} \AttributeTok{{-}p} \StringTok{"}\VariableTok{$output\_dir}\StringTok{"}

\FunctionTok{find} \StringTok{"}\VariableTok{$input\_dir}\StringTok{"} \AttributeTok{{-}name} \StringTok{"*.md"} \KeywordTok{|} \ControlFlowTok{while} \BuiltInTok{read} \AttributeTok{{-}r} \VariableTok{file}\KeywordTok{;} \ControlFlowTok{do}
    \VariableTok{filename}\OperatorTok{=}\VariableTok{$(}\FunctionTok{basename} \StringTok{"}\VariableTok{$file}\StringTok{"}\NormalTok{ .md}\VariableTok{)}
    \BuiltInTok{echo} \StringTok{"Converting: }\VariableTok{$file}\StringTok{"}
    
    \ControlFlowTok{case} \StringTok{"}\VariableTok{$format}\StringTok{"} \KeywordTok{in}
        \SpecialStringTok{html}\KeywordTok{)}
            \ExtensionTok{pandoc} \AttributeTok{{-}{-}toc} \AttributeTok{{-}c}\NormalTok{ github.css }\StringTok{"}\VariableTok{$file}\StringTok{"} \AttributeTok{{-}o} \StringTok{"}\VariableTok{$output\_dir}\StringTok{/}\VariableTok{$\{filename\}}\StringTok{.html"}
            \ControlFlowTok{;;}
        \SpecialStringTok{pdf}\KeywordTok{)}
            \ExtensionTok{pandoc} \AttributeTok{{-}{-}toc} \StringTok{"}\VariableTok{$file}\StringTok{"} \AttributeTok{{-}o} \StringTok{"}\VariableTok{$output\_dir}\StringTok{/}\VariableTok{$\{filename\}}\StringTok{.pdf"}
            \ControlFlowTok{;;}
        \SpecialStringTok{docx}\KeywordTok{)}
            \ExtensionTok{pandoc} \StringTok{"}\VariableTok{$file}\StringTok{"} \AttributeTok{{-}o} \StringTok{"}\VariableTok{$output\_dir}\StringTok{/}\VariableTok{$\{filename\}}\StringTok{.docx"}
            \ControlFlowTok{;;}
        \SpecialStringTok{epub}\KeywordTok{)}
            \ExtensionTok{pandoc} \AttributeTok{{-}{-}toc} \StringTok{"}\VariableTok{$file}\StringTok{"} \AttributeTok{{-}o} \StringTok{"}\VariableTok{$output\_dir}\StringTok{/}\VariableTok{$\{filename\}}\StringTok{.epub"}
            \ControlFlowTok{;;}
    \ControlFlowTok{esac}
\ControlFlowTok{done}
\end{Highlighting}
\end{Shaded}

\begin{center}\rule{0.5\linewidth}{0.5pt}\end{center}

\section{glow - Renderizador Markdown}\label{sec-glow}

glow renderiza archivos Markdown con estilo en la terminal.

\subsection{Uso básico}\label{uso-buxe1sico-6}

\begin{Shaded}
\begin{Highlighting}[]
\CommentTok{\# Ver archivo Markdown}
\ExtensionTok{glow}\NormalTok{ README.md}

\CommentTok{\# Buscar y ver archivos .md}
\ExtensionTok{glow}\NormalTok{ .}

\CommentTok{\# Modo pager}
\ExtensionTok{glow} \AttributeTok{{-}p}\NormalTok{ README.md}

\CommentTok{\# Ancho específico}
\ExtensionTok{glow} \AttributeTok{{-}w}\NormalTok{ 100 README.md}
\end{Highlighting}
\end{Shaded}

\subsection{Estilos y configuración}\label{estilos-y-configuraciuxf3n}

\begin{Shaded}
\begin{Highlighting}[]
\CommentTok{\# Diferentes estilos}
\ExtensionTok{glow} \AttributeTok{{-}s}\NormalTok{ dark README.md}
\ExtensionTok{glow} \AttributeTok{{-}s}\NormalTok{ light README.md}
\ExtensionTok{glow} \AttributeTok{{-}s}\NormalTok{ notty README.md  }\CommentTok{\# Sin colores}

\CommentTok{\# Configuración personalizada}
\ExtensionTok{glow}\NormalTok{ config set style dark}
\ExtensionTok{glow}\NormalTok{ config set width 120}
\ExtensionTok{glow}\NormalTok{ config set mouse true}
\end{Highlighting}
\end{Shaded}

\subsection{Workflow de
documentación}\label{workflow-de-documentaciuxf3n}

\begin{Shaded}
\begin{Highlighting}[]
\CommentTok{\#!/bin/bash}
\CommentTok{\# doc{-}viewer.sh {-} Navegador de documentación}

\VariableTok{DOC\_DIR}\OperatorTok{=}\StringTok{"}\VariableTok{$\{1}\OperatorTok{:{-}}\NormalTok{.}\VariableTok{\}}\StringTok{"}

\CommentTok{\# Función para seleccionar y ver documento}
\FunctionTok{view\_docs()} \KeywordTok{\{}
    \BuiltInTok{local} \VariableTok{doc\_file}
    \VariableTok{doc\_file}\OperatorTok{=}\VariableTok{$(}\FunctionTok{find} \StringTok{"}\VariableTok{$DOC\_DIR}\StringTok{"} \AttributeTok{{-}name} \StringTok{"*.md"} \KeywordTok{|} \ExtensionTok{fzf} \AttributeTok{{-}{-}preview} \StringTok{\textquotesingle{}glow \{\}\textquotesingle{}}\VariableTok{)}
    
    \ControlFlowTok{if} \BuiltInTok{[} \OtherTok{{-}n} \StringTok{"}\VariableTok{$doc\_file}\StringTok{"} \BuiltInTok{]}\KeywordTok{;} \ControlFlowTok{then}
        \ExtensionTok{glow} \StringTok{"}\VariableTok{$doc\_file}\StringTok{"}
        
        \CommentTok{\# Preguntar si editar}
        \BuiltInTok{echo} \AttributeTok{{-}e} \StringTok{"\textbackslash{}n¿Editar archivo? (y/n)"}
        \BuiltInTok{read} \AttributeTok{{-}r} \VariableTok{response}
        \ControlFlowTok{if} \BuiltInTok{[} \StringTok{"}\VariableTok{$response}\StringTok{"} \OtherTok{=} \StringTok{"y"} \BuiltInTok{]}\KeywordTok{;} \ControlFlowTok{then}
            \VariableTok{$\{EDITOR}\OperatorTok{:{-}}\NormalTok{vim}\VariableTok{\}} \StringTok{"}\VariableTok{$doc\_file}\StringTok{"}
        \ControlFlowTok{fi}
    \ControlFlowTok{fi}
\KeywordTok{\}}

\ExtensionTok{view\_docs}
\end{Highlighting}
\end{Shaded}

\begin{center}\rule{0.5\linewidth}{0.5pt}\end{center}

\section{Herramientas
complementarias}\label{herramientas-complementarias-2}

\subsection{vale - Linter de prosa}\label{vale---linter-de-prosa}

\begin{Shaded}
\begin{Highlighting}[]
\CommentTok{\# Instalar}
\ExtensionTok{brew}\NormalTok{ install vale}

\CommentTok{\# Verificar documento}
\ExtensionTok{vale}\NormalTok{ documento.md}

\CommentTok{\# Con configuración personalizada}
\ExtensionTok{vale} \AttributeTok{{-}{-}config}\OperatorTok{=}\NormalTok{.vale.ini documento.md}
\end{Highlighting}
\end{Shaded}

\subsection{mdless - Pager para
Markdown}\label{mdless---pager-para-markdown}

\begin{Shaded}
\begin{Highlighting}[]
\CommentTok{\# Ver Markdown paginado}
\ExtensionTok{mdless}\NormalTok{ README.md}

\CommentTok{\# Con números de línea}
\ExtensionTok{mdless} \AttributeTok{{-}l}\NormalTok{ README.md}
\end{Highlighting}
\end{Shaded}

\subsection{Workflow completo de
documentación}\label{workflow-completo-de-documentaciuxf3n}

\begin{Shaded}
\begin{Highlighting}[]
\CommentTok{\#!/bin/bash}
\CommentTok{\# doc{-}workflow.sh}

\VariableTok{PROJECT\_DIR}\OperatorTok{=}\StringTok{"}\VariableTok{$\{1}\OperatorTok{:{-}}\NormalTok{.}\VariableTok{\}}\StringTok{"}
\VariableTok{DOCS\_DIR}\OperatorTok{=}\StringTok{"}\VariableTok{$PROJECT\_DIR}\StringTok{/docs"}
\VariableTok{BUILD\_DIR}\OperatorTok{=}\StringTok{"}\VariableTok{$PROJECT\_DIR}\StringTok{/docs{-}build"}

\CommentTok{\# Función para crear estructura de documentos}
\FunctionTok{init\_docs()} \KeywordTok{\{}
    \FunctionTok{mkdir} \AttributeTok{{-}p} \StringTok{"}\VariableTok{$DOCS\_DIR}\StringTok{"}\NormalTok{/}\DataTypeTok{\{guides}\OperatorTok{,}\DataTypeTok{api}\OperatorTok{,}\DataTypeTok{tutorials\}}
    
    \FunctionTok{cat} \OperatorTok{\textgreater{}} \StringTok{"}\VariableTok{$DOCS\_DIR}\StringTok{/README.md"} \OperatorTok{\textless{}\textless{} \textquotesingle{}EOF\textquotesingle{}}
\StringTok{\# Project Documentation}

\StringTok{\#\# Structure}

\StringTok{{-} [Guides](guides/) {-} User guides and how{-}tos}
\StringTok{{-} [API](api/) {-} API documentation}
\StringTok{{-} [Tutorials](tutorials/) {-} Step{-}by{-}step tutorials}

\StringTok{\#\# Contributing}

\StringTok{Please follow the [documentation style guide](STYLE\_GUIDE.md).}
\OperatorTok{EOF}

    \BuiltInTok{echo} \StringTok{"📝 Documentation structure created in }\VariableTok{$DOCS\_DIR}\StringTok{"}
\KeywordTok{\}}

\CommentTok{\# Función para validar documentos}
\FunctionTok{validate\_docs()} \KeywordTok{\{}
    \BuiltInTok{echo} \StringTok{"🔍 Validating documentation..."}
    
    \CommentTok{\# Verificar links rotos}
    \FunctionTok{find} \StringTok{"}\VariableTok{$DOCS\_DIR}\StringTok{"} \AttributeTok{{-}name} \StringTok{"*.md"} \AttributeTok{{-}exec}\NormalTok{ markdown{-}link{-}check \{\} }\DataTypeTok{\textbackslash{};}
    
    \CommentTok{\# Linting de prosa (si vale está instalado)}
    \ControlFlowTok{if} \BuiltInTok{command} \AttributeTok{{-}v}\NormalTok{ vale }\OperatorTok{\&\textgreater{}}\NormalTok{ /dev/null}\KeywordTok{;} \ControlFlowTok{then}
        \ExtensionTok{vale} \StringTok{"}\VariableTok{$DOCS\_DIR}\StringTok{"}
    \ControlFlowTok{fi}
    
    \CommentTok{\# Verificar formato}
    \FunctionTok{find} \StringTok{"}\VariableTok{$DOCS\_DIR}\StringTok{"} \AttributeTok{{-}name} \StringTok{"*.md"} \KeywordTok{|} \ControlFlowTok{while} \BuiltInTok{read} \AttributeTok{{-}r} \VariableTok{file}\KeywordTok{;} \ControlFlowTok{do}
        \BuiltInTok{echo} \StringTok{"Checking: }\VariableTok{$file}\StringTok{"}
        \ExtensionTok{glow} \StringTok{"}\VariableTok{$file}\StringTok{"} \OperatorTok{\textgreater{}}\NormalTok{ /dev/null }\KeywordTok{||} \BuiltInTok{echo} \StringTok{"❌ Error rendering }\VariableTok{$file}\StringTok{"}
    \ControlFlowTok{done}
\KeywordTok{\}}

\CommentTok{\# Función para generar sitio estático}
\FunctionTok{build\_docs()} \KeywordTok{\{}
    \BuiltInTok{echo} \StringTok{"🏗️ Building documentation site..."}
    
    \FunctionTok{mkdir} \AttributeTok{{-}p} \StringTok{"}\VariableTok{$BUILD\_DIR}\StringTok{"}
    
    \CommentTok{\# Convertir todos los MD a HTML}
    \FunctionTok{find} \StringTok{"}\VariableTok{$DOCS\_DIR}\StringTok{"} \AttributeTok{{-}name} \StringTok{"*.md"} \KeywordTok{|} \ControlFlowTok{while} \BuiltInTok{read} \AttributeTok{{-}r} \VariableTok{file}\KeywordTok{;} \ControlFlowTok{do}
        \VariableTok{relative\_path}\OperatorTok{=}\VariableTok{$\{file}\OperatorTok{\#}\VariableTok{$DOCS\_DIR}\NormalTok{/}\VariableTok{\}}
        \VariableTok{output\_file}\OperatorTok{=}\StringTok{"}\VariableTok{$BUILD\_DIR}\StringTok{/}\VariableTok{$\{relative\_path}\OperatorTok{\%}\NormalTok{.md}\VariableTok{\}}\StringTok{.html"}
        \VariableTok{output\_dir}\OperatorTok{=}\VariableTok{$(}\FunctionTok{dirname} \StringTok{"}\VariableTok{$output\_file}\StringTok{"}\VariableTok{)}
        
        \FunctionTok{mkdir} \AttributeTok{{-}p} \StringTok{"}\VariableTok{$output\_dir}\StringTok{"}
        
        \ExtensionTok{pandoc} \AttributeTok{{-}{-}toc} \AttributeTok{{-}c} \StringTok{"../styles.css"} \DataTypeTok{\textbackslash{}}
               \AttributeTok{{-}{-}template}\OperatorTok{=}\StringTok{"templates/doc.html"} \DataTypeTok{\textbackslash{}}
               \StringTok{"}\VariableTok{$file}\StringTok{"} \AttributeTok{{-}o} \StringTok{"}\VariableTok{$output\_file}\StringTok{"}
    \ControlFlowTok{done}
    
    \CommentTok{\# Copiar assets}
    \FunctionTok{cp} \AttributeTok{{-}r} \StringTok{"}\VariableTok{$PROJECT\_DIR}\StringTok{/assets"}\NormalTok{/}\PreprocessorTok{*} \StringTok{"}\VariableTok{$BUILD\_DIR}\StringTok{/"} \DecValTok{2}\OperatorTok{\textgreater{}}\NormalTok{/dev/null }\KeywordTok{||} \FunctionTok{true}
    
    \BuiltInTok{echo} \StringTok{"✅ Documentation built in }\VariableTok{$BUILD\_DIR}\StringTok{"}
\KeywordTok{\}}

\CommentTok{\# Función para servir documentos localmente}
\FunctionTok{serve\_docs()} \KeywordTok{\{}
    \BuiltInTok{echo} \StringTok{"🌐 Serving documentation at http://localhost:8000"}
    \BuiltInTok{cd} \StringTok{"}\VariableTok{$BUILD\_DIR}\StringTok{"} \KeywordTok{\&\&} \ExtensionTok{python3} \AttributeTok{{-}m}\NormalTok{ http.server 8000}
\KeywordTok{\}}

\CommentTok{\# Menu principal}
\ControlFlowTok{case} \StringTok{"}\VariableTok{$\{2}\OperatorTok{:{-}}\NormalTok{help}\VariableTok{\}}\StringTok{"} \KeywordTok{in}
    \SpecialStringTok{init}\KeywordTok{)}
        \ExtensionTok{init\_docs}
        \ControlFlowTok{;;}
    \SpecialStringTok{validate}\KeywordTok{)}
        \ExtensionTok{validate\_docs}
        \ControlFlowTok{;;}
    \SpecialStringTok{build}\KeywordTok{)}
        \ExtensionTok{build\_docs}
        \ControlFlowTok{;;}
    \SpecialStringTok{serve}\KeywordTok{)}
        \ExtensionTok{serve\_docs}
        \ControlFlowTok{;;}
    \SpecialStringTok{all}\KeywordTok{)}
        \ExtensionTok{validate\_docs} \KeywordTok{\&\&} \ExtensionTok{build\_docs} \KeywordTok{\&\&} \ExtensionTok{serve\_docs}
        \ControlFlowTok{;;}
    \PreprocessorTok{*}\KeywordTok{)}
        \BuiltInTok{echo} \StringTok{"Uso: }\VariableTok{$0}\StringTok{ \textless{}project\_dir\textgreater{} \textless{}command\textgreater{}"}
        \BuiltInTok{echo} \StringTok{"Comandos:"}
        \BuiltInTok{echo} \StringTok{"  init     {-} Crear estructura de documentación"}
        \BuiltInTok{echo} \StringTok{"  validate {-} Validar documentos existentes"}
        \BuiltInTok{echo} \StringTok{"  build    {-} Generar sitio estático"}
        \BuiltInTok{echo} \StringTok{"  serve    {-} Servir documentación localmente"}
        \BuiltInTok{echo} \StringTok{"  all      {-} Validar, construir y servir"}
        \ControlFlowTok{;;}
\ControlFlowTok{esac}
\end{Highlighting}
\end{Shaded}

\begin{tcolorbox}[enhanced jigsaw, toprule=.15mm, bottomrule=.15mm, opacityback=0, coltitle=black, rightrule=.15mm, colframe=quarto-callout-tip-color-frame, titlerule=0mm, opacitybacktitle=0.6, left=2mm, colback=white, bottomtitle=1mm, arc=.35mm, leftrule=.75mm, title=\textcolor{quarto-callout-tip-color}{\faLightbulb}\hspace{0.5em}{Tips para documentación}, colbacktitle=quarto-callout-tip-color!10!white, breakable, toptitle=1mm]

\begin{itemize}
\tightlist
\item
  Usa \texttt{bat} como reemplazo de \texttt{cat} para mejor legibilidad
\item
  \texttt{glow} es perfecto para revisar documentación rápidamente
\item
  \texttt{pandoc} es extremadamente potente para conversiones complejas
\item
  Combina estas herramientas en scripts para workflows automatizados
\end{itemize}

\end{tcolorbox}

En el próximo capítulo cubriremos las utilidades diversas que mejoran la
experiencia general en terminal.

\part{Utilidades Diversas}

\chapter{Utilidades Diversas}\label{utilidades-diversas-3}

Esta sección cubre herramientas que mejoran significativamente la
experiencia general en terminal, desde corrección automática hasta
personalización avanzada del prompt.

\section{tealdeer - Ejemplos prácticos de comandos}\label{sec-tealdeer}

\texttt{tealdeer} es una implementación moderna y rápida de
\texttt{tldr}, compatible con las páginas tldr pero escrita en Rust.

\begin{tcolorbox}[enhanced jigsaw, toprule=.15mm, bottomrule=.15mm, opacityback=0, coltitle=black, rightrule=.15mm, colframe=quarto-callout-warning-color-frame, titlerule=0mm, opacitybacktitle=0.6, left=2mm, colback=white, bottomtitle=1mm, arc=.35mm, leftrule=.75mm, title=\textcolor{quarto-callout-warning-color}{\faExclamationTriangle}\hspace{0.5em}{⚠️ Migración desde tldr}, colbacktitle=quarto-callout-warning-color!10!white, breakable, toptitle=1mm]

La fórmula original de \texttt{tldr} en Homebrew ha sido marcada como
\textbf{obsoleta}. Las alternativas recomendadas son:

\textbf{Opción 1: tealdeer (recomendado para Homebrew)}

\begin{Shaded}
\begin{Highlighting}[]
\CommentTok{\# Desinstalar tldr si está instalado}
\ExtensionTok{brew}\NormalTok{ uninstall tldr }\DecValTok{2}\OperatorTok{\textgreater{}}\NormalTok{/dev/null }\KeywordTok{||} \FunctionTok{true}

\CommentTok{\# Instalar tealdeer (implementación en Rust)}
\ExtensionTok{brew}\NormalTok{ install tealdeer}
\end{Highlighting}
\end{Shaded}

\textbf{Opción 2: tldr vía npm (versión oficial mantenida)}

\begin{Shaded}
\begin{Highlighting}[]
\ExtensionTok{npm}\NormalTok{ install }\AttributeTok{{-}g}\NormalTok{ tldr}
\end{Highlighting}
\end{Shaded}

\end{tcolorbox}

\subsection{Uso básico}\label{uso-buxe1sico-7}

\begin{Shaded}
\begin{Highlighting}[]
\CommentTok{\# Ver ejemplos de un comando (usa comando \textquotesingle{}tldr\textquotesingle{} igual que antes)}
\ExtensionTok{tldr}\NormalTok{ git}
\ExtensionTok{tldr}\NormalTok{ curl}
\ExtensionTok{tldr}\NormalTok{ find}

\CommentTok{\# Actualizar base de datos}
\ExtensionTok{tldr} \AttributeTok{{-}{-}update}

\CommentTok{\# Buscar comando por funcionalidad}
\ExtensionTok{tldr} \AttributeTok{{-}{-}search} \StringTok{"compress"}
\ExtensionTok{tldr} \AttributeTok{{-}{-}search} \StringTok{"network"}

\CommentTok{\# Plataforma específica}
\ExtensionTok{tldr} \AttributeTok{{-}p}\NormalTok{ linux tar}
\ExtensionTok{tldr} \AttributeTok{{-}p}\NormalTok{ macos ls}
\end{Highlighting}
\end{Shaded}

\subsection{Configuración avanzada}\label{configuraciuxf3n-avanzada}

\begin{Shaded}
\begin{Highlighting}[]
\CommentTok{\# Configurar tema}
\ExtensionTok{tldr} \AttributeTok{{-}{-}theme}\NormalTok{ base16}

\CommentTok{\# Idioma específico (si está disponible)  }
\ExtensionTok{tldr} \AttributeTok{{-}L}\NormalTok{ es git}

\CommentTok{\# Configuración en \textasciitilde{}/.config/tealdeer/config.toml}
\ExtensionTok{[display]}
\ExtensionTok{compact}\NormalTok{ = false}
\ExtensionTok{use\_pager}\NormalTok{ = false}

\ExtensionTok{[updates]}
\ExtensionTok{auto\_update}\NormalTok{ = true}
\ExtensionTok{auto\_update\_interval\_hours}\NormalTok{ = 24}

\ExtensionTok{[style]}
\ExtensionTok{description}\NormalTok{ = }\StringTok{"white"}
\ExtensionTok{code}\NormalTok{ = }\StringTok{"green"}
\ExtensionTok{parameter}\NormalTok{ = }\StringTok{"cyan"}
\end{Highlighting}
\end{Shaded}

\subsection{Comparación de
alternativas}\label{comparaciuxf3n-de-alternativas}

\begin{longtable}[]{@{}llll@{}}
\toprule\noalign{}
Implementación & Instalación & Velocidad & Estado \\
\midrule\noalign{}
\endhead
\bottomrule\noalign{}
\endlastfoot
\texttt{tealdeer} & \texttt{brew\ install} & ⚡ Muy rápida & ✅
Mantenido \\
\texttt{tldr} (npm) & \texttt{npm\ install\ -g} & 🚀 Rápida & ✅
Oficial \\
\texttt{tldr} (brew) & ❌ Obsoleto & 🐌 Regular & ❌ Deprecated \\
\end{longtable}

\begin{center}\rule{0.5\linewidth}{0.5pt}\end{center}

\section{thefuck - Corrector automático}\label{sec-thefuck}

thefuck corrige automáticamente comandos mal escritos.

\subsection{Configuración inicial}\label{configuraciuxf3n-inicial-3}

\begin{Shaded}
\begin{Highlighting}[]
\CommentTok{\# Agregar a \textasciitilde{}/.zshrc o \textasciitilde{}/.bashrc}
\BuiltInTok{eval} \VariableTok{$(}\ExtensionTok{thefuck} \AttributeTok{{-}{-}alias}\VariableTok{)}
\CommentTok{\# O crear alias personalizado}
\BuiltInTok{eval} \VariableTok{$(}\ExtensionTok{thefuck} \AttributeTok{{-}{-}alias}\NormalTok{ fix}\VariableTok{)}
\end{Highlighting}
\end{Shaded}

\subsection{Ejemplos de uso}\label{ejemplos-de-uso}

\begin{Shaded}
\begin{Highlighting}[]
\CommentTok{\# Comando mal escrito}
\ExtensionTok{$}\NormalTok{ gut push}
\ExtensionTok{git:} \StringTok{\textquotesingle{}gut\textquotesingle{}}\NormalTok{ is not a git command}

\ExtensionTok{$}\NormalTok{ fuck}
\FunctionTok{git}\NormalTok{ push }\PreprocessorTok{[}\SpecialStringTok{enter/↑/↓/ctrl+c}\PreprocessorTok{]}

\CommentTok{\# Puerto ocupado}
\ExtensionTok{$}\NormalTok{ python }\AttributeTok{{-}m}\NormalTok{ http.server 8000}
\ExtensionTok{OSError:}\NormalTok{ [Errno 48] Address already in use}

\ExtensionTok{$}\NormalTok{ fuck}
\ExtensionTok{python} \AttributeTok{{-}m}\NormalTok{ http.server 8001 }\PreprocessorTok{[}\SpecialStringTok{enter/↑/↓/ctrl+c}\PreprocessorTok{]}
\end{Highlighting}
\end{Shaded}

\subsection{Configuración avanzada}\label{configuraciuxf3n-avanzada-1}

Archivo \texttt{\textasciitilde{}/.config/thefuck/settings.py}:

\begin{Shaded}
\begin{Highlighting}[]
\CommentTok{\# Configuración personalizada}
\NormalTok{rules }\OperatorTok{=}\NormalTok{ [}\StringTok{\textquotesingle{}git\_push\textquotesingle{}}\NormalTok{, }\StringTok{\textquotesingle{}python\_command\textquotesingle{}}\NormalTok{, }\StringTok{\textquotesingle{}ls\_lah\textquotesingle{}}\NormalTok{, }\StringTok{\textquotesingle{}cd\_mkdir\textquotesingle{}}\NormalTok{]}
\NormalTok{exclude\_rules }\OperatorTok{=}\NormalTok{ [}\StringTok{\textquotesingle{}rm\_root\textquotesingle{}}\NormalTok{]}
\NormalTok{require\_confirmation }\OperatorTok{=} \VariableTok{True}
\NormalTok{wait\_command }\OperatorTok{=} \DecValTok{3}
\NormalTok{history\_limit }\OperatorTok{=} \DecValTok{1000}

\CommentTok{\# Alias personalizado}
\NormalTok{alias }\OperatorTok{=} \StringTok{\textquotesingle{}fix\textquotesingle{}}
\end{Highlighting}
\end{Shaded}

\begin{center}\rule{0.5\linewidth}{0.5pt}\end{center}

\section{cowsay - Arte ASCII divertido}\label{sec-cowsay}

cowsay genera arte ASCII con mensajes personalizados.

\subsection{Uso básico}\label{uso-buxe1sico-8}

\begin{Shaded}
\begin{Highlighting}[]
\CommentTok{\# Mensaje básico}
\ExtensionTok{cowsay} \StringTok{"Hola mundo"}

\CommentTok{\# Diferentes animales}
\ExtensionTok{cowsay} \AttributeTok{{-}f}\NormalTok{ dragon }\StringTok{"Soy un dragón"}
\ExtensionTok{cowsay} \AttributeTok{{-}f}\NormalTok{ tux }\StringTok{"¡Linux!"}

\CommentTok{\# Listar animales disponibles}
\ExtensionTok{cowsay} \AttributeTok{{-}l}
\end{Highlighting}
\end{Shaded}

\subsection{Expresiones de la vaca}\label{expresiones-de-la-vaca}

\begin{Shaded}
\begin{Highlighting}[]
\CommentTok{\# Diferentes estados de ánimo}
\ExtensionTok{cowsay} \AttributeTok{{-}b} \StringTok{"Ojos de Borg"}      \CommentTok{\# Borg}
\ExtensionTok{cowsay} \AttributeTok{{-}d} \StringTok{"Muerto"}            \CommentTok{\# Dead}
\ExtensionTok{cowsay} \AttributeTok{{-}g} \StringTok{"Codicioso"}         \CommentTok{\# Greedy}
\ExtensionTok{cowsay} \AttributeTok{{-}p} \StringTok{"Paranoico"}         \CommentTok{\# Paranoid}
\ExtensionTok{cowsay} \AttributeTok{{-}s} \StringTok{"Stoned"}            \CommentTok{\# Stoned}
\ExtensionTok{cowsay} \AttributeTok{{-}t} \StringTok{"Cansado"}           \CommentTok{\# Tired}
\ExtensionTok{cowsay} \AttributeTok{{-}w} \StringTok{"Cableado"}          \CommentTok{\# Wired}
\ExtensionTok{cowsay} \AttributeTok{{-}y} \StringTok{"Joven"}             \CommentTok{\# Young}
\end{Highlighting}
\end{Shaded}

\subsection{Integración con otras
herramientas}\label{integraciuxf3n-con-otras-herramientas-2}

\begin{Shaded}
\begin{Highlighting}[]
\CommentTok{\# Con fortune (si está instalado)}
\ExtensionTok{fortune} \KeywordTok{|} \ExtensionTok{cowsay}

\CommentTok{\# Con fecha}
\FunctionTok{date} \KeywordTok{|} \ExtensionTok{cowsay} \AttributeTok{{-}f}\NormalTok{ tux}

\CommentTok{\# En MOTD del sistema}
\BuiltInTok{echo} \StringTok{"}\VariableTok{$(}\FunctionTok{whoami}\VariableTok{)}\StringTok{, bienvenido a }\VariableTok{$(}\FunctionTok{hostname}\VariableTok{)}\StringTok{"} \KeywordTok{|} \ExtensionTok{cowsay} \AttributeTok{{-}f}\NormalTok{ dragon}

\CommentTok{\# Pipeline divertido}
\ExtensionTok{curl} \AttributeTok{{-}s} \StringTok{"https://api.quotegarden.com/api/v3/quotes/random"} \KeywordTok{|} \DataTypeTok{\textbackslash{}}
\ExtensionTok{jq} \AttributeTok{{-}r} \StringTok{\textquotesingle{}.data.quoteText\textquotesingle{}} \KeywordTok{|} \ExtensionTok{cowsay} \AttributeTok{{-}f}\NormalTok{ elephant}
\end{Highlighting}
\end{Shaded}

\begin{center}\rule{0.5\linewidth}{0.5pt}\end{center}

\section{direnv - Variables de entorno por directorio}\label{sec-direnv}

direnv carga automáticamente variables de entorno al entrar a
directorios.

\subsection{Configuración inicial}\label{configuraciuxf3n-inicial-4}

\begin{Shaded}
\begin{Highlighting}[]
\CommentTok{\# Agregar a \textasciitilde{}/.zshrc}
\BuiltInTok{eval} \StringTok{"}\VariableTok{$(}\ExtensionTok{direnv}\NormalTok{ hook zsh}\VariableTok{)}\StringTok{"}

\CommentTok{\# Para bash}
\BuiltInTok{eval} \StringTok{"}\VariableTok{$(}\ExtensionTok{direnv}\NormalTok{ hook bash}\VariableTok{)}\StringTok{"}
\end{Highlighting}
\end{Shaded}

\subsection{Uso básico}\label{uso-buxe1sico-9}

\begin{Shaded}
\begin{Highlighting}[]
\CommentTok{\# Crear archivo .envrc en tu proyecto}
\BuiltInTok{echo} \StringTok{\textquotesingle{}export DATABASE\_URL="postgres://localhost/mydb"\textquotesingle{}} \OperatorTok{\textgreater{}}\NormalTok{ .envrc}
\BuiltInTok{echo} \StringTok{\textquotesingle{}export DEBUG=true\textquotesingle{}} \OperatorTok{\textgreater{}\textgreater{}}\NormalTok{ .envrc}

\CommentTok{\# Permitir el archivo}
\ExtensionTok{direnv}\NormalTok{ allow}

\CommentTok{\# Al entrar al directorio, las variables se cargan automáticamente}
\CommentTok{\# Al salir, se descargan}
\end{Highlighting}
\end{Shaded}

\subsection{Ejemplos avanzados}\label{ejemplos-avanzados-2}

\subsubsection{Proyecto Python}\label{proyecto-python}

\begin{Shaded}
\begin{Highlighting}[]
\CommentTok{\# .envrc para proyecto Python}
\BuiltInTok{export} \VariableTok{VIRTUAL\_ENV}\OperatorTok{=}\StringTok{"}\VariableTok{$PWD}\StringTok{/.venv"}
\BuiltInTok{export} \VariableTok{PATH}\OperatorTok{=}\StringTok{"}\VariableTok{$VIRTUAL\_ENV}\StringTok{/bin:}\VariableTok{$PATH}\StringTok{"}
\BuiltInTok{export} \VariableTok{PYTHONPATH}\OperatorTok{=}\StringTok{"}\VariableTok{$PWD}\StringTok{/src:}\VariableTok{$PYTHONPATH}\StringTok{"}

\CommentTok{\# Variables de desarrollo}
\BuiltInTok{export} \VariableTok{FLASK\_ENV}\OperatorTok{=}\NormalTok{development}
\BuiltInTok{export} \VariableTok{DATABASE\_URL}\OperatorTok{=}\StringTok{"sqlite:///dev.db"}

\CommentTok{\# Activar virtualenv si existe}
\ControlFlowTok{if} \BuiltInTok{[} \OtherTok{{-}d} \StringTok{"}\VariableTok{$VIRTUAL\_ENV}\StringTok{"} \BuiltInTok{]}\KeywordTok{;} \ControlFlowTok{then}
    \BuiltInTok{source} \StringTok{"}\VariableTok{$VIRTUAL\_ENV}\StringTok{/bin/activate"}
\ControlFlowTok{fi}
\end{Highlighting}
\end{Shaded}

\subsubsection{Proyecto Node.js}\label{proyecto-node.js}

\begin{Shaded}
\begin{Highlighting}[]
\CommentTok{\# .envrc para proyecto Node.js}
\BuiltInTok{export} \VariableTok{NODE\_ENV}\OperatorTok{=}\NormalTok{development}
\BuiltInTok{export} \VariableTok{PORT}\OperatorTok{=}\NormalTok{3000}

\CommentTok{\# Usar versión específica de Node con nvm}
\ExtensionTok{use}\NormalTok{ node 18.17.0}

\CommentTok{\# Variables de API}
\BuiltInTok{export} \VariableTok{API\_KEY}\OperatorTok{=}\StringTok{"dev{-}api{-}key"}
\BuiltInTok{export} \VariableTok{REDIS\_URL}\OperatorTok{=}\StringTok{"redis://localhost:6379"}
\end{Highlighting}
\end{Shaded}

\subsubsection{Proyecto con Docker}\label{proyecto-con-docker}

\begin{Shaded}
\begin{Highlighting}[]
\CommentTok{\# .envrc para proyecto Docker}
\BuiltInTok{export} \VariableTok{COMPOSE\_PROJECT\_NAME}\OperatorTok{=}\StringTok{"myapp"}
\BuiltInTok{export} \VariableTok{DOCKER\_BUILDKIT}\OperatorTok{=}\NormalTok{1}

\CommentTok{\# Variables para docker{-}compose}
\BuiltInTok{export} \VariableTok{POSTGRES\_DB}\OperatorTok{=}\NormalTok{myapp\_dev}
\BuiltInTok{export} \VariableTok{POSTGRES\_USER}\OperatorTok{=}\NormalTok{developer}
\BuiltInTok{export} \VariableTok{POSTGRES\_PASSWORD}\OperatorTok{=}\NormalTok{devpass}

\CommentTok{\# Path para herramientas locales}
\ExtensionTok{PATH\_add}\NormalTok{ ./scripts}
\ExtensionTok{PATH\_add}\NormalTok{ ./node\_modules/.bin}
\end{Highlighting}
\end{Shaded}

\begin{center}\rule{0.5\linewidth}{0.5pt}\end{center}

\section{starship - Prompt personalizable}\label{sec-starship}

starship es un prompt de terminal minimalista y rápido.

\subsection{Configuración inicial}\label{configuraciuxf3n-inicial-5}

\begin{Shaded}
\begin{Highlighting}[]
\CommentTok{\# Agregar a \textasciitilde{}/.zshrc}
\BuiltInTok{eval} \StringTok{"}\VariableTok{$(}\ExtensionTok{starship}\NormalTok{ init zsh}\VariableTok{)}\StringTok{"}

\CommentTok{\# Para bash}
\BuiltInTok{eval} \StringTok{"}\VariableTok{$(}\ExtensionTok{starship}\NormalTok{ init bash}\VariableTok{)}\StringTok{"}
\end{Highlighting}
\end{Shaded}

\subsection{Configuración básica}\label{configuraciuxf3n-buxe1sica}

Archivo \texttt{\textasciitilde{}/.config/starship.toml}:

\begin{Shaded}
\begin{Highlighting}[]
\KeywordTok{[character]}
\DataTypeTok{success\_symbol} \OperatorTok{=} \StringTok{"[➜](bold green)"}
\DataTypeTok{error\_symbol} \OperatorTok{=} \StringTok{"[➜](bold red)"}

\KeywordTok{[directory]}
\DataTypeTok{truncation\_length} \OperatorTok{=} \DecValTok{3}
\DataTypeTok{truncation\_symbol} \OperatorTok{=} \StringTok{"…/"}

\KeywordTok{[git\_branch]}
\DataTypeTok{symbol} \OperatorTok{=} \StringTok{"🌱 "}
\DataTypeTok{format} \OperatorTok{=} \StringTok{"on [$symbol$branch]($style) "}

\KeywordTok{[git\_status]}
\DataTypeTok{conflicted} \OperatorTok{=} \StringTok{"🏳"}
\DataTypeTok{ahead} \OperatorTok{=} \StringTok{"🏎💨"}
\DataTypeTok{behind} \OperatorTok{=} \StringTok{"😰"}
\DataTypeTok{diverged} \OperatorTok{=} \StringTok{"😵"}
\DataTypeTok{untracked} \OperatorTok{=} \StringTok{"🤷‍"}
\DataTypeTok{stashed} \OperatorTok{=} \StringTok{"📦"}
\DataTypeTok{modified} \OperatorTok{=} \StringTok{"📝"}
\DataTypeTok{staged} \OperatorTok{=} \StringTok{\textquotesingle{}}\VerbatimStringTok{[++\textbackslash{}($count\textbackslash{})](green)}\StringTok{\textquotesingle{}}
\DataTypeTok{renamed} \OperatorTok{=} \StringTok{"👅"}
\DataTypeTok{deleted} \OperatorTok{=} \StringTok{"🗑"}

\KeywordTok{[nodejs]}
\DataTypeTok{symbol} \OperatorTok{=} \StringTok{"⬢ "}
\DataTypeTok{format} \OperatorTok{=} \StringTok{"via [$symbol($version )]($style)"}

\KeywordTok{[python]}
\DataTypeTok{symbol} \OperatorTok{=} \StringTok{"🐍 "}
\DataTypeTok{format} \OperatorTok{=} \StringTok{\textquotesingle{}}\VerbatimStringTok{via [$\{symbol\}$\{pyenv\_prefix\}($\{version\} )(\textbackslash{}($virtualenv\textbackslash{}) )]($style)}\StringTok{\textquotesingle{}}

\KeywordTok{[rust]}
\DataTypeTok{symbol} \OperatorTok{=} \StringTok{"🦀 "}
\DataTypeTok{format} \OperatorTok{=} \StringTok{"via [$symbol($version )]($style)"}

\KeywordTok{[docker\_context]}
\DataTypeTok{symbol} \OperatorTok{=} \StringTok{"🐳 "}
\DataTypeTok{format} \OperatorTok{=} \StringTok{"via [$symbol$context]($style) "}
\end{Highlighting}
\end{Shaded}

\subsection{Presets disponibles}\label{presets-disponibles}

\begin{Shaded}
\begin{Highlighting}[]
\CommentTok{\# Ver presets disponibles}
\ExtensionTok{starship}\NormalTok{ preset }\AttributeTok{{-}{-}list}

\CommentTok{\# Aplicar preset}
\ExtensionTok{starship}\NormalTok{ preset nerd{-}font{-}symbols }\AttributeTok{{-}o}\NormalTok{ \textasciitilde{}/.config/starship.toml}
\ExtensionTok{starship}\NormalTok{ preset tokyo{-}night }\AttributeTok{{-}o}\NormalTok{ \textasciitilde{}/.config/starship.toml}
\end{Highlighting}
\end{Shaded}

\begin{center}\rule{0.5\linewidth}{0.5pt}\end{center}

\section{Workflows con utilidades}\label{workflows-con-utilidades}

\subsection{Script de setup de
entorno}\label{script-de-setup-de-entorno}

\begin{Shaded}
\begin{Highlighting}[]
\CommentTok{\#!/bin/bash}
\CommentTok{\# dev{-}setup.sh {-} Configurar entorno de desarrollo}

\FunctionTok{setup\_git()} \KeywordTok{\{}
    \BuiltInTok{echo} \StringTok{"🔧 Configurando Git..."}
    
    \CommentTok{\# Configuración básica si no existe}
    \ControlFlowTok{if} \OtherTok{! }\FunctionTok{git}\NormalTok{ config }\AttributeTok{{-}{-}global}\NormalTok{ user.name }\OperatorTok{\textgreater{}}\NormalTok{/dev/null }\DecValTok{2}\OperatorTok{\textgreater{}\&}\DecValTok{1}\KeywordTok{;} \ControlFlowTok{then}
        \BuiltInTok{read} \AttributeTok{{-}p} \StringTok{"Nombre para Git: "} \VariableTok{git\_name}
        \FunctionTok{git}\NormalTok{ config }\AttributeTok{{-}{-}global}\NormalTok{ user.name }\StringTok{"}\VariableTok{$git\_name}\StringTok{"}
    \ControlFlowTok{fi}
    
    \ControlFlowTok{if} \OtherTok{! }\FunctionTok{git}\NormalTok{ config }\AttributeTok{{-}{-}global}\NormalTok{ user.email }\OperatorTok{\textgreater{}}\NormalTok{/dev/null }\DecValTok{2}\OperatorTok{\textgreater{}\&}\DecValTok{1}\KeywordTok{;} \ControlFlowTok{then}
        \BuiltInTok{read} \AttributeTok{{-}p} \StringTok{"Email para Git: "} \VariableTok{git\_email}
        \FunctionTok{git}\NormalTok{ config }\AttributeTok{{-}{-}global}\NormalTok{ user.email }\StringTok{"}\VariableTok{$git\_email}\StringTok{"}
    \ControlFlowTok{fi}
    
    \CommentTok{\# Aliases útiles}
    \FunctionTok{git}\NormalTok{ config }\AttributeTok{{-}{-}global}\NormalTok{ alias.st status}
    \FunctionTok{git}\NormalTok{ config }\AttributeTok{{-}{-}global}\NormalTok{ alias.co checkout}
    \FunctionTok{git}\NormalTok{ config }\AttributeTok{{-}{-}global}\NormalTok{ alias.br branch}
    \FunctionTok{git}\NormalTok{ config }\AttributeTok{{-}{-}global}\NormalTok{ alias.ci commit}
    \FunctionTok{git}\NormalTok{ config }\AttributeTok{{-}{-}global}\NormalTok{ alias.unstage }\StringTok{\textquotesingle{}reset HEAD {-}{-}\textquotesingle{}}
\KeywordTok{\}}

\FunctionTok{setup\_shell()} \KeywordTok{\{}
    \BuiltInTok{echo} \StringTok{"🐚 Configurando shell..."}
    
    \CommentTok{\# Backup de configuración existente}
    \BuiltInTok{[} \OtherTok{{-}f}\NormalTok{ \textasciitilde{}/.zshrc }\BuiltInTok{]} \KeywordTok{\&\&} \FunctionTok{cp}\NormalTok{ \textasciitilde{}/.zshrc \textasciitilde{}/.zshrc.backup}
    
    \CommentTok{\# Configurar starship}
    \BuiltInTok{echo} \StringTok{\textquotesingle{}eval "$(starship init zsh)"\textquotesingle{}} \OperatorTok{\textgreater{}\textgreater{}}\NormalTok{ \textasciitilde{}/.zshrc}
    
    \CommentTok{\# Configurar direnv}
    \BuiltInTok{echo} \StringTok{\textquotesingle{}eval "$(direnv hook zsh)"\textquotesingle{}} \OperatorTok{\textgreater{}\textgreater{}}\NormalTok{ \textasciitilde{}/.zshrc}
    
    \CommentTok{\# Configurar thefuck}
    \BuiltInTok{echo} \StringTok{\textquotesingle{}eval $(thefuck {-}{-}alias)\textquotesingle{}} \OperatorTok{\textgreater{}\textgreater{}}\NormalTok{ \textasciitilde{}/.zshrc}
    
    \CommentTok{\# Aliases útiles}
    \FunctionTok{cat} \OperatorTok{\textgreater{}\textgreater{}}\NormalTok{ \textasciitilde{}/.zshrc }\OperatorTok{\textless{}\textless{} \textquotesingle{}EOF\textquotesingle{}}
\StringTok{\# Aliases útiles}
\StringTok{alias ll=\textquotesingle{}eza {-}la {-}{-}git\textquotesingle{}}
\StringTok{alias cat=\textquotesingle{}bat\textquotesingle{}}
\StringTok{alias find=\textquotesingle{}fd\textquotesingle{}}
\StringTok{alias grep=\textquotesingle{}rg\textquotesingle{}}

\StringTok{\# Funciones útiles}
\StringTok{help() \{}
\StringTok{    tldr "$1" 2\textgreater{}/dev/null || man "$1"  \# tldr funciona igual con tealdeer}
\StringTok{\}}
\OperatorTok{EOF}
\KeywordTok{\}}

\FunctionTok{setup\_project\_templates()} \KeywordTok{\{}
    \BuiltInTok{echo} \StringTok{"📁 Creando templates de proyecto..."}
    
    \VariableTok{TEMPLATES\_DIR}\OperatorTok{=}\StringTok{"}\VariableTok{$HOME}\StringTok{/.project{-}templates"}
    \FunctionTok{mkdir} \AttributeTok{{-}p} \StringTok{"}\VariableTok{$TEMPLATES\_DIR}\StringTok{"}
    
    \CommentTok{\# Template para Python}
    \FunctionTok{mkdir} \AttributeTok{{-}p} \StringTok{"}\VariableTok{$TEMPLATES\_DIR}\StringTok{/python"}
    \FunctionTok{cat} \OperatorTok{\textgreater{}} \StringTok{"}\VariableTok{$TEMPLATES\_DIR}\StringTok{/python/.envrc"} \OperatorTok{\textless{}\textless{} \textquotesingle{}EOF\textquotesingle{}}
\StringTok{export PYTHONPATH="$PWD/src:$PYTHONPATH"}
\StringTok{export VIRTUAL\_ENV="$PWD/.venv"}
\StringTok{export PATH="$VIRTUAL\_ENV/bin:$PATH"}

\StringTok{if [ {-}d "$VIRTUAL\_ENV" ]; then}
\StringTok{    source "$VIRTUAL\_ENV/bin/activate"}
\StringTok{fi}
\OperatorTok{EOF}
    
    \CommentTok{\# Template para Node.js}
    \FunctionTok{mkdir} \AttributeTok{{-}p} \StringTok{"}\VariableTok{$TEMPLATES\_DIR}\StringTok{/nodejs"}
    \FunctionTok{cat} \OperatorTok{\textgreater{}} \StringTok{"}\VariableTok{$TEMPLATES\_DIR}\StringTok{/nodejs/.envrc"} \OperatorTok{\textless{}\textless{} \textquotesingle{}EOF\textquotesingle{}}
\StringTok{export NODE\_ENV=development}
\StringTok{export PATH="$PWD/node\_modules/.bin:$PATH"}

\StringTok{use node 18}
\OperatorTok{EOF}
    
    \BuiltInTok{echo} \StringTok{"Templates creados en }\VariableTok{$TEMPLATES\_DIR}\StringTok{"}
\KeywordTok{\}}

\CommentTok{\# Función principal}
\FunctionTok{main()} \KeywordTok{\{}
    \BuiltInTok{echo} \StringTok{"🚀 Configurando entorno de desarrollo..."}
    
    \ExtensionTok{setup\_git}
    \ExtensionTok{setup\_shell}
    \ExtensionTok{setup\_project\_templates}
    
    \BuiltInTok{echo} \StringTok{"✅ Configuración completada!"}
    \BuiltInTok{echo} \StringTok{"💡 Reinicia tu terminal para aplicar los cambios"}
    
    \CommentTok{\# Mensaje motivacional}
    \BuiltInTok{echo} \StringTok{"¡Listo para programar!"} \KeywordTok{|} \ExtensionTok{cowsay} \AttributeTok{{-}f}\NormalTok{ tux}
\KeywordTok{\}}

\ExtensionTok{main} \StringTok{"}\VariableTok{$@}\StringTok{"}
\end{Highlighting}
\end{Shaded}

\subsection{Sistema de notificaciones
inteligente}\label{sistema-de-notificaciones-inteligente}

\begin{Shaded}
\begin{Highlighting}[]
\CommentTok{\#!/bin/bash}
\CommentTok{\# smart{-}notify.sh {-} Sistema de notificaciones basado en contexto}

\VariableTok{NOTIFY\_FILE}\OperatorTok{=}\StringTok{"}\VariableTok{$HOME}\StringTok{/.local/notifications"}
\VariableTok{LAST\_COMMAND\_FILE}\OperatorTok{=}\StringTok{"}\VariableTok{$HOME}\StringTok{/.last\_command"}

\CommentTok{\# Función para enviar notificación}
\FunctionTok{send\_notification()} \KeywordTok{\{}
    \BuiltInTok{local} \VariableTok{title}\OperatorTok{=}\StringTok{"}\VariableTok{$1}\StringTok{"}
    \BuiltInTok{local} \VariableTok{message}\OperatorTok{=}\StringTok{"}\VariableTok{$2}\StringTok{"}
    \BuiltInTok{local} \VariableTok{urgency}\OperatorTok{=}\StringTok{"}\VariableTok{$\{3}\OperatorTok{:{-}}\NormalTok{normal}\VariableTok{\}}\StringTok{"}
    
    \CommentTok{\# macOS}
    \ControlFlowTok{if} \BuiltInTok{command} \AttributeTok{{-}v}\NormalTok{ osascript }\OperatorTok{\textgreater{}}\NormalTok{/dev/null}\KeywordTok{;} \ControlFlowTok{then}
        \ExtensionTok{osascript} \AttributeTok{{-}e} \StringTok{"display notification }\DataTypeTok{\textbackslash{}"}\VariableTok{$message}\DataTypeTok{\textbackslash{}"}\StringTok{ with title }\DataTypeTok{\textbackslash{}"}\VariableTok{$title}\DataTypeTok{\textbackslash{}"}\StringTok{"}
    \CommentTok{\# Linux}
    \ControlFlowTok{elif} \BuiltInTok{command} \AttributeTok{{-}v}\NormalTok{ notify{-}send }\OperatorTok{\textgreater{}}\NormalTok{/dev/null}\KeywordTok{;} \ControlFlowTok{then}
        \ExtensionTok{notify{-}send} \AttributeTok{{-}u} \StringTok{"}\VariableTok{$urgency}\StringTok{"} \StringTok{"}\VariableTok{$title}\StringTok{"} \StringTok{"}\VariableTok{$message}\StringTok{"}
    \ControlFlowTok{fi}
    
    \CommentTok{\# Log para debug}
    \BuiltInTok{echo} \StringTok{"[}\VariableTok{$(}\FunctionTok{date}\VariableTok{)}\StringTok{] }\VariableTok{$title}\StringTok{: }\VariableTok{$message}\StringTok{"} \OperatorTok{\textgreater{}\textgreater{}} \StringTok{"}\VariableTok{$NOTIFY\_FILE}\StringTok{"}
\KeywordTok{\}}

\CommentTok{\# Notificar cuando comando tarda mucho}
\FunctionTok{notify\_long\_command()} \KeywordTok{\{}
    \BuiltInTok{local} \VariableTok{cmd}\OperatorTok{=}\StringTok{"}\VariableTok{$1}\StringTok{"}
    \BuiltInTok{local} \VariableTok{duration}\OperatorTok{=}\StringTok{"}\VariableTok{$2}\StringTok{"}
    
    \ControlFlowTok{if} \BuiltInTok{[} \StringTok{"}\VariableTok{$duration}\StringTok{"} \OtherTok{{-}gt}\NormalTok{ 30 }\BuiltInTok{]}\KeywordTok{;} \ControlFlowTok{then}
        \ExtensionTok{send\_notification} \StringTok{"Comando completado"} \StringTok{"}\DataTypeTok{\textbackslash{}"}\VariableTok{$cmd}\DataTypeTok{\textbackslash{}"}\StringTok{ terminó en }\VariableTok{$\{duration\}}\StringTok{s"}
    \ControlFlowTok{fi}
\KeywordTok{\}}

\CommentTok{\# Hook para .zshrc}
\FunctionTok{precmd()} \KeywordTok{\{}
    \BuiltInTok{local} \VariableTok{exit\_code}\OperatorTok{=}\VariableTok{$?}
    \BuiltInTok{local} \VariableTok{end\_time}\OperatorTok{=}\VariableTok{$(}\FunctionTok{date}\NormalTok{ +\%s}\VariableTok{)}
    
    \ControlFlowTok{if} \BuiltInTok{[} \OtherTok{{-}f} \StringTok{"}\VariableTok{$LAST\_COMMAND\_FILE}\StringTok{"} \BuiltInTok{]}\KeywordTok{;} \ControlFlowTok{then}
        \BuiltInTok{local} \VariableTok{start\_time}\OperatorTok{=}\VariableTok{$(}\FunctionTok{cat} \StringTok{"}\VariableTok{$LAST\_COMMAND\_FILE}\StringTok{"}\VariableTok{)}
        \BuiltInTok{local} \VariableTok{duration}\OperatorTok{=}\VariableTok{$((end\_time} \OperatorTok{{-}} \VariableTok{start\_time))}
        
        \ControlFlowTok{if} \BuiltInTok{[} \VariableTok{$exit\_code} \OtherTok{{-}ne}\NormalTok{ 0 }\BuiltInTok{]}\KeywordTok{;} \ControlFlowTok{then}
            \ExtensionTok{send\_notification} \StringTok{"Error en comando"} \StringTok{"Comando falló con código }\VariableTok{$exit\_code}\StringTok{"} \StringTok{"critical"}
        \ControlFlowTok{else}
            \ExtensionTok{notify\_long\_command} \StringTok{"}\VariableTok{$history}\StringTok{[1]"} \StringTok{"}\VariableTok{$duration}\StringTok{"}
        \ControlFlowTok{fi}
        
        \FunctionTok{rm} \StringTok{"}\VariableTok{$LAST\_COMMAND\_FILE}\StringTok{"}
    \ControlFlowTok{fi}
\KeywordTok{\}}

\FunctionTok{preexec()} \KeywordTok{\{}
    \BuiltInTok{echo} \VariableTok{$(}\FunctionTok{date}\NormalTok{ +\%s}\VariableTok{)} \OperatorTok{\textgreater{}} \StringTok{"}\VariableTok{$LAST\_COMMAND\_FILE}\StringTok{"}
\KeywordTok{\}}
\end{Highlighting}
\end{Shaded}

\begin{tcolorbox}[enhanced jigsaw, toprule=.15mm, bottomrule=.15mm, opacityback=0, coltitle=black, rightrule=.15mm, colframe=quarto-callout-tip-color-frame, titlerule=0mm, opacitybacktitle=0.6, left=2mm, colback=white, bottomtitle=1mm, arc=.35mm, leftrule=.75mm, title=\textcolor{quarto-callout-tip-color}{\faLightbulb}\hspace{0.5em}{Tips para utilidades}, colbacktitle=quarto-callout-tip-color!10!white, breakable, toptitle=1mm]

\begin{itemize}
\tightlist
\item
  Combina \texttt{tealdeer} (comando \texttt{tldr}) con \texttt{man}
  para documentación completa
\item
  \texttt{direnv} es perfecto para proyectos con configuraciones
  específicas
\item
  \texttt{starship} puede mostrar información contextual del proyecto
  actual
\item
  \texttt{thefuck} aprende de tus errores comunes
\end{itemize}

\end{tcolorbox}

En el próximo capítulo exploraremos combinaciones avanzadas y workflows
complejos.

\part{Recursos Adicionales}

\chapter{Combinaciones Útiles}\label{combinaciones-uxfatiles}

Este capítulo explora workflows avanzados que combinan múltiples
herramientas CLI para crear soluciones potentes, automatizadas y
altamente eficientes.

\section{Workflows de desarrollo}\label{workflows-de-desarrollo}

\subsection{Pipeline completo de
código}\label{pipeline-completo-de-cuxf3digo}

\begin{Shaded}
\begin{Highlighting}[]
\CommentTok{\#!/bin/bash}
\CommentTok{\# dev{-}pipeline.sh {-} Pipeline completo de desarrollo}

\FunctionTok{analyze\_project()} \KeywordTok{\{}
    \BuiltInTok{echo} \StringTok{"🔍 Analizando proyecto..."}
    
    \CommentTok{\# Estructura del proyecto}
    \BuiltInTok{echo} \StringTok{"=== ESTRUCTURA ==="}
    \ExtensionTok{tree} \AttributeTok{{-}L}\NormalTok{ 3 }\AttributeTok{{-}I} \StringTok{\textquotesingle{}node\_modules|.git|dist|build\textquotesingle{}}
    
    \CommentTok{\# Estadísticas de código}
    \BuiltInTok{echo} \AttributeTok{{-}e} \StringTok{"\textbackslash{}n=== ESTADÍSTICAS ==="}
    \BuiltInTok{echo} \StringTok{"Archivos por tipo:"}
    \FunctionTok{find}\NormalTok{ . }\AttributeTok{{-}type}\NormalTok{ f }\KeywordTok{|} \FunctionTok{grep} \AttributeTok{{-}E} \StringTok{\textquotesingle{}\textbackslash{}.[a{-}z]+$\textquotesingle{}} \KeywordTok{|} \FunctionTok{sed} \StringTok{\textquotesingle{}s/.*\textbackslash{}.//\textquotesingle{}} \KeywordTok{|} \FunctionTok{sort} \KeywordTok{|} \FunctionTok{uniq} \AttributeTok{{-}c} \KeywordTok{|} \FunctionTok{sort} \AttributeTok{{-}nr}
    
    \CommentTok{\# TODOs y FIXMEs}
    \BuiltInTok{echo} \AttributeTok{{-}e} \StringTok{"\textbackslash{}n=== TAREAS PENDIENTES ==="}
    \ExtensionTok{rg} \AttributeTok{{-}C}\NormalTok{ 1 }\StringTok{"(TODO|FIXME|HACK)"} \AttributeTok{{-}{-}type{-}add} \StringTok{\textquotesingle{}code:*.\{js,ts,py,go,rs,java\}\textquotesingle{}} \AttributeTok{{-}t}\NormalTok{ code}
    
    \CommentTok{\# Dependencias no utilizadas (Node.js)}
    \ControlFlowTok{if} \BuiltInTok{[} \OtherTok{{-}f} \StringTok{"package.json"} \BuiltInTok{]}\KeywordTok{;} \ControlFlowTok{then}
        \BuiltInTok{echo} \AttributeTok{{-}e} \StringTok{"\textbackslash{}n=== DEPENDENCIAS ==="}
        \ExtensionTok{npm}\NormalTok{ list }\AttributeTok{{-}{-}depth}\OperatorTok{=}\NormalTok{0 }\DecValTok{2}\OperatorTok{\textgreater{}}\NormalTok{/dev/null }\KeywordTok{|} \FunctionTok{tail} \AttributeTok{{-}n}\NormalTok{ +2}
    \ControlFlowTok{fi}
\KeywordTok{\}}

\FunctionTok{lint\_and\_format()} \KeywordTok{\{}
    \BuiltInTok{echo} \StringTok{"🧹 Limpiando código..."}
    
    \CommentTok{\# JavaScript/TypeScript}
    \ControlFlowTok{if} \BuiltInTok{[} \OtherTok{{-}f} \StringTok{"package.json"} \BuiltInTok{]}\KeywordTok{;} \ControlFlowTok{then}
        \ExtensionTok{npm}\NormalTok{ run lint:fix }\DecValTok{2}\OperatorTok{\textgreater{}}\NormalTok{/dev/null }\KeywordTok{||} \BuiltInTok{echo} \StringTok{"No lint script found"}
        \ExtensionTok{npm}\NormalTok{ run format }\DecValTok{2}\OperatorTok{\textgreater{}}\NormalTok{/dev/null }\KeywordTok{||} \BuiltInTok{echo} \StringTok{"No format script found"}
    \ControlFlowTok{fi}
    
    \CommentTok{\# Python}
    \ControlFlowTok{if} \FunctionTok{find}\NormalTok{ . }\AttributeTok{{-}name} \StringTok{"*.py"} \KeywordTok{|} \FunctionTok{head} \AttributeTok{{-}1} \OperatorTok{\textgreater{}}\NormalTok{/dev/null}\KeywordTok{;} \ControlFlowTok{then}
        \ExtensionTok{black}\NormalTok{ . }\DecValTok{2}\OperatorTok{\textgreater{}}\NormalTok{/dev/null }\KeywordTok{||} \BuiltInTok{echo} \StringTok{"black not installed"}
        \ExtensionTok{isort}\NormalTok{ . }\DecValTok{2}\OperatorTok{\textgreater{}}\NormalTok{/dev/null }\KeywordTok{||} \BuiltInTok{echo} \StringTok{"isort not installed"}
    \ControlFlowTok{fi}
    
    \CommentTok{\# Go}
    \ControlFlowTok{if} \FunctionTok{find}\NormalTok{ . }\AttributeTok{{-}name} \StringTok{"*.go"} \KeywordTok{|} \FunctionTok{head} \AttributeTok{{-}1} \OperatorTok{\textgreater{}}\NormalTok{/dev/null}\KeywordTok{;} \ControlFlowTok{then}
        \ExtensionTok{go}\NormalTok{ fmt ./...}
        \ExtensionTok{goimports} \AttributeTok{{-}w}\NormalTok{ . }\DecValTok{2}\OperatorTok{\textgreater{}}\NormalTok{/dev/null }\KeywordTok{||} \BuiltInTok{echo} \StringTok{"goimports not installed"}
    \ControlFlowTok{fi}
\KeywordTok{\}}

\FunctionTok{run\_tests()} \KeywordTok{\{}
    \BuiltInTok{echo} \StringTok{"🧪 Ejecutando tests..."}
    
    \ControlFlowTok{if} \BuiltInTok{[} \OtherTok{{-}f} \StringTok{"package.json"} \BuiltInTok{]}\KeywordTok{;} \ControlFlowTok{then}
        \ExtensionTok{npm}\NormalTok{ test}
    \ControlFlowTok{elif} \BuiltInTok{[} \OtherTok{{-}f} \StringTok{"requirements.txt"} \BuiltInTok{]} \KeywordTok{||} \BuiltInTok{[} \OtherTok{{-}f} \StringTok{"pyproject.toml"} \BuiltInTok{]}\KeywordTok{;} \ControlFlowTok{then}
        \ExtensionTok{python} \AttributeTok{{-}m}\NormalTok{ pytest}
    \ControlFlowTok{elif} \BuiltInTok{[} \OtherTok{{-}f} \StringTok{"go.mod"} \BuiltInTok{]}\KeywordTok{;} \ControlFlowTok{then}
        \ExtensionTok{go}\NormalTok{ test ./...}
    \ControlFlowTok{elif} \BuiltInTok{[} \OtherTok{{-}f} \StringTok{"Cargo.toml"} \BuiltInTok{]}\KeywordTok{;} \ControlFlowTok{then}
        \ExtensionTok{cargo}\NormalTok{ test}
    \ControlFlowTok{fi}
\KeywordTok{\}}

\FunctionTok{security\_scan()} \KeywordTok{\{}
    \BuiltInTok{echo} \StringTok{"🔒 Escaneando seguridad..."}
    
    \CommentTok{\# Node.js}
    \ControlFlowTok{if} \BuiltInTok{[} \OtherTok{{-}f} \StringTok{"package.json"} \BuiltInTok{]}\KeywordTok{;} \ControlFlowTok{then}
        \ExtensionTok{npm}\NormalTok{ audit}
    \ControlFlowTok{fi}
    
    \CommentTok{\# Python}
    \ControlFlowTok{if} \BuiltInTok{[} \OtherTok{{-}f} \StringTok{"requirements.txt"} \BuiltInTok{]}\KeywordTok{;} \ControlFlowTok{then}
        \ExtensionTok{safety}\NormalTok{ check }\DecValTok{2}\OperatorTok{\textgreater{}}\NormalTok{/dev/null }\KeywordTok{||} \BuiltInTok{echo} \StringTok{"safety not installed"}
    \ControlFlowTok{fi}
    
    \CommentTok{\# Secretos hardcodeados}
    \BuiltInTok{echo} \StringTok{"Buscando secretos potenciales:"}
    \ExtensionTok{rg} \AttributeTok{{-}i} \StringTok{"(password|secret|key|token)\textbackslash{}s*[:=]\textbackslash{}s*[\textquotesingle{}}\DataTypeTok{\textbackslash{}"}\StringTok{][\^{}\textquotesingle{}}\DataTypeTok{\textbackslash{}"}\StringTok{]\{8,\}"} \AttributeTok{{-}{-}type{-}add} \StringTok{\textquotesingle{}code:*.\{js,ts,py,go,rs\}\textquotesingle{}} \AttributeTok{{-}t}\NormalTok{ code}
\KeywordTok{\}}

\CommentTok{\# Ejecutar pipeline completo}
\FunctionTok{main()} \KeywordTok{\{}
    \BuiltInTok{echo} \StringTok{"🚀 Iniciando pipeline de desarrollo..."}
    
    \ExtensionTok{analyze\_project}
    \ExtensionTok{lint\_and\_format}
    \ExtensionTok{run\_tests}
    \ExtensionTok{security\_scan}
    
    \BuiltInTok{echo} \StringTok{"✅ Pipeline completado"}
\KeywordTok{\}}

\ExtensionTok{main} \StringTok{"}\VariableTok{$@}\StringTok{"}
\end{Highlighting}
\end{Shaded}

\subsection{Workflow de Git avanzado}\label{workflow-de-git-avanzado}

\begin{Shaded}
\begin{Highlighting}[]
\CommentTok{\#!/bin/bash}
\CommentTok{\# git{-}workflow.sh {-} Workflow avanzado de Git}

\FunctionTok{smart\_commit()} \KeywordTok{\{}
    \BuiltInTok{local} \VariableTok{message}\OperatorTok{=}\StringTok{"}\VariableTok{$1}\StringTok{"}
    
    \ControlFlowTok{if} \BuiltInTok{[} \OtherTok{{-}z} \StringTok{"}\VariableTok{$message}\StringTok{"} \BuiltInTok{]}\KeywordTok{;} \ControlFlowTok{then}
        \BuiltInTok{echo} \StringTok{"Uso: smart\_commit \textless{}mensaje\textgreater{}"}
        \ControlFlowTok{return} \DecValTok{1}
    \ControlFlowTok{fi}
    
    \CommentTok{\# Verificar que hay cambios}
    \ControlFlowTok{if} \FunctionTok{git}\NormalTok{ diff }\AttributeTok{{-}{-}quiet} \KeywordTok{\&\&} \FunctionTok{git}\NormalTok{ diff }\AttributeTok{{-}{-}cached} \AttributeTok{{-}{-}quiet}\KeywordTok{;} \ControlFlowTok{then}
        \BuiltInTok{echo} \StringTok{"No hay cambios para commitear"}
        \ControlFlowTok{return} \DecValTok{1}
    \ControlFlowTok{fi}
    
    \CommentTok{\# Mostrar qué se va a commitear}
    \BuiltInTok{echo} \StringTok{"Cambios a commitear:"}
    \FunctionTok{git}\NormalTok{ status }\AttributeTok{{-}{-}short}
    
    \CommentTok{\# Verificar tests antes de commit}
    \ControlFlowTok{if} \BuiltInTok{[} \OtherTok{{-}f} \StringTok{"package.json"} \BuiltInTok{]}\KeywordTok{;} \ControlFlowTok{then}
        \BuiltInTok{echo} \StringTok{"Ejecutando tests..."}
        \ExtensionTok{npm}\NormalTok{ test }\KeywordTok{||} \KeywordTok{\{}
            \BuiltInTok{echo} \StringTok{"Tests fallan. ¿Continuar? (y/n)"}
            \BuiltInTok{read} \AttributeTok{{-}r} \VariableTok{response}
            \BuiltInTok{[} \StringTok{"}\VariableTok{$response}\StringTok{"} \OtherTok{!=} \StringTok{"y"} \BuiltInTok{]} \KeywordTok{\&\&} \ControlFlowTok{return} \DecValTok{1}
        \KeywordTok{\}}
    \ControlFlowTok{fi}
    
    \CommentTok{\# Agregar archivos y commitear}
    \FunctionTok{git}\NormalTok{ add .}
    \FunctionTok{git}\NormalTok{ commit }\AttributeTok{{-}m} \StringTok{"}\VariableTok{$message}\StringTok{"}
    
    \CommentTok{\# Sugerir push si hay commits pendientes}
    \ControlFlowTok{if} \BuiltInTok{[} \StringTok{"}\VariableTok{$(}\FunctionTok{git}\NormalTok{ rev{-}list @\{u\}..HEAD }\DecValTok{2}\OperatorTok{\textgreater{}}\NormalTok{/dev/null }\KeywordTok{|} \FunctionTok{wc} \AttributeTok{{-}l}\VariableTok{)}\StringTok{"} \OtherTok{{-}gt}\NormalTok{ 0 }\BuiltInTok{]}\KeywordTok{;} \ControlFlowTok{then}
        \BuiltInTok{echo} \StringTok{"¿Push a origin? (y/n)"}
        \BuiltInTok{read} \AttributeTok{{-}r} \VariableTok{response}
        \BuiltInTok{[} \StringTok{"}\VariableTok{$response}\StringTok{"} \OtherTok{=} \StringTok{"y"} \BuiltInTok{]} \KeywordTok{\&\&} \FunctionTok{git}\NormalTok{ push}
    \ControlFlowTok{fi}
\KeywordTok{\}}

\FunctionTok{interactive\_rebase()} \KeywordTok{\{}
    \BuiltInTok{local} \VariableTok{commits}\OperatorTok{=}\StringTok{"}\VariableTok{$\{1}\OperatorTok{:{-}}\NormalTok{5}\VariableTok{\}}\StringTok{"}
    
    \BuiltInTok{echo} \StringTok{"Commits recientes:"}
    \FunctionTok{git}\NormalTok{ log }\AttributeTok{{-}{-}oneline} \AttributeTok{{-}n} \StringTok{"}\VariableTok{$commits}\StringTok{"}
    
    \BuiltInTok{echo} \AttributeTok{{-}e} \StringTok{"\textbackslash{}n¿Hacer rebase interactivo de últimos }\VariableTok{$commits}\StringTok{ commits? (y/n)"}
    \BuiltInTok{read} \AttributeTok{{-}r} \VariableTok{response}
    
    \ControlFlowTok{if} \BuiltInTok{[} \StringTok{"}\VariableTok{$response}\StringTok{"} \OtherTok{=} \StringTok{"y"} \BuiltInTok{]}\KeywordTok{;} \ControlFlowTok{then}
        \FunctionTok{git}\NormalTok{ rebase }\AttributeTok{{-}i} \StringTok{"HEAD\textasciitilde{}}\VariableTok{$commits}\StringTok{"}
    \ControlFlowTok{fi}
\KeywordTok{\}}

\FunctionTok{cleanup\_branches()} \KeywordTok{\{}
    \BuiltInTok{echo} \StringTok{"Branches locales:"}
    \FunctionTok{git}\NormalTok{ branch}
    
    \BuiltInTok{echo} \AttributeTok{{-}e} \StringTok{"\textbackslash{}nBranches remotos ya mergeados:"}
    \FunctionTok{git}\NormalTok{ branch }\AttributeTok{{-}r} \AttributeTok{{-}{-}merged} \KeywordTok{|} \FunctionTok{grep} \AttributeTok{{-}v} \StringTok{\textquotesingle{}\textbackslash{}{-}\textgreater{}\textquotesingle{}} \KeywordTok{|} \FunctionTok{grep} \AttributeTok{{-}v}\NormalTok{ main }\KeywordTok{|} \FunctionTok{grep} \AttributeTok{{-}v}\NormalTok{ master}
    
    \BuiltInTok{echo} \AttributeTok{{-}e} \StringTok{"\textbackslash{}n¿Limpiar branches mergeados? (y/n)"}
    \BuiltInTok{read} \AttributeTok{{-}r} \VariableTok{response}
    
    \ControlFlowTok{if} \BuiltInTok{[} \StringTok{"}\VariableTok{$response}\StringTok{"} \OtherTok{=} \StringTok{"y"} \BuiltInTok{]}\KeywordTok{;} \ControlFlowTok{then}
        \CommentTok{\# Limpiar branches locales mergeados}
        \FunctionTok{git}\NormalTok{ branch }\AttributeTok{{-}{-}merged} \KeywordTok{|} \FunctionTok{grep} \AttributeTok{{-}v} \StringTok{\textquotesingle{}\textbackslash{}*\textbackslash{}|main\textbackslash{}|master\textquotesingle{}} \KeywordTok{|} \FunctionTok{xargs} \AttributeTok{{-}n}\NormalTok{ 1 git branch }\AttributeTok{{-}d}
        
        \CommentTok{\# Limpiar referencias remotas}
        \FunctionTok{git}\NormalTok{ remote prune origin}
    \ControlFlowTok{fi}
\KeywordTok{\}}

\FunctionTok{release\_workflow()} \KeywordTok{\{}
    \BuiltInTok{local} \VariableTok{version}\OperatorTok{=}\StringTok{"}\VariableTok{$1}\StringTok{"}
    
    \ControlFlowTok{if} \BuiltInTok{[} \OtherTok{{-}z} \StringTok{"}\VariableTok{$version}\StringTok{"} \BuiltInTok{]}\KeywordTok{;} \ControlFlowTok{then}
        \BuiltInTok{echo} \StringTok{"Uso: release\_workflow \textless{}version\textgreater{}"}
        \ControlFlowTok{return} \DecValTok{1}
    \ControlFlowTok{fi}
    
    \CommentTok{\# Verificar que estamos en main/master}
    \VariableTok{current\_branch}\OperatorTok{=}\VariableTok{$(}\FunctionTok{git}\NormalTok{ branch }\AttributeTok{{-}{-}show{-}current}\VariableTok{)}
    \ControlFlowTok{if} \BuiltInTok{[} \StringTok{"}\VariableTok{$current\_branch}\StringTok{"} \OtherTok{!=} \StringTok{"main"} \BuiltInTok{]} \KeywordTok{\&\&} \BuiltInTok{[} \StringTok{"}\VariableTok{$current\_branch}\StringTok{"} \OtherTok{!=} \StringTok{"master"} \BuiltInTok{]}\KeywordTok{;} \ControlFlowTok{then}
        \BuiltInTok{echo} \StringTok{"Cambia a main/master antes de crear release"}
        \ControlFlowTok{return} \DecValTok{1}
    \ControlFlowTok{fi}
    
    \CommentTok{\# Actualizar desde origin}
    \FunctionTok{git}\NormalTok{ pull origin }\StringTok{"}\VariableTok{$current\_branch}\StringTok{"}
    
    \CommentTok{\# Crear tag}
    \FunctionTok{git}\NormalTok{ tag }\AttributeTok{{-}a} \StringTok{"v}\VariableTok{$version}\StringTok{"} \AttributeTok{{-}m} \StringTok{"Release v}\VariableTok{$version}\StringTok{"}
    
    \CommentTok{\# Push tag}
    \FunctionTok{git}\NormalTok{ push origin }\StringTok{"v}\VariableTok{$version}\StringTok{"}
    
    \CommentTok{\# Crear release en GitHub (si gh está disponible)}
    \ControlFlowTok{if} \BuiltInTok{command} \AttributeTok{{-}v}\NormalTok{ gh }\OperatorTok{\textgreater{}}\NormalTok{/dev/null}\KeywordTok{;} \ControlFlowTok{then}
        \ExtensionTok{gh}\NormalTok{ release create }\StringTok{"v}\VariableTok{$version}\StringTok{"} \AttributeTok{{-}{-}generate{-}notes}
    \ControlFlowTok{fi}
    
    \BuiltInTok{echo} \StringTok{"Release v}\VariableTok{$version}\StringTok{ creado exitosamente"}
\KeywordTok{\}}

\CommentTok{\# Función principal con menú}
\ControlFlowTok{case} \StringTok{"}\VariableTok{$\{1}\OperatorTok{:{-}}\NormalTok{menu}\VariableTok{\}}\StringTok{"} \KeywordTok{in}
    \SpecialStringTok{commit}\KeywordTok{)}
        \BuiltInTok{shift}
        \ExtensionTok{smart\_commit} \StringTok{"}\VariableTok{$*}\StringTok{"}
        \ControlFlowTok{;;}
    \SpecialStringTok{rebase}\KeywordTok{)}
        \ExtensionTok{interactive\_rebase} \StringTok{"}\VariableTok{$2}\StringTok{"}
        \ControlFlowTok{;;}
    \SpecialStringTok{cleanup}\KeywordTok{)}
        \ExtensionTok{cleanup\_branches}
        \ControlFlowTok{;;}
    \SpecialStringTok{release}\KeywordTok{)}
        \ExtensionTok{release\_workflow} \StringTok{"}\VariableTok{$2}\StringTok{"}
        \ControlFlowTok{;;}
    \PreprocessorTok{*}\KeywordTok{)}
        \BuiltInTok{echo} \StringTok{"Git Workflow Tool"}
        \BuiltInTok{echo} \StringTok{"Uso: }\VariableTok{$0}\StringTok{ \textless{}comando\textgreater{} [argumentos]"}
        \BuiltInTok{echo} \StringTok{""}
        \BuiltInTok{echo} \StringTok{"Comandos:"}
        \BuiltInTok{echo} \StringTok{"  commit \textless{}mensaje\textgreater{}  {-} Commit inteligente con verificaciones"}
        \BuiltInTok{echo} \StringTok{"  rebase [n]        {-} Rebase interactivo de últimos n commits"}
        \BuiltInTok{echo} \StringTok{"  cleanup           {-} Limpiar branches mergeados"}
        \BuiltInTok{echo} \StringTok{"  release \textless{}version\textgreater{} {-} Crear release con tag"}
        \ControlFlowTok{;;}
\ControlFlowTok{esac}
\end{Highlighting}
\end{Shaded}

\section{Análisis de datos con CLI}\label{anuxe1lisis-de-datos-con-cli}

\subsection{Pipeline de análisis de
logs}\label{pipeline-de-anuxe1lisis-de-logs}

\begin{Shaded}
\begin{Highlighting}[]
\CommentTok{\#!/bin/bash}
\CommentTok{\# log{-}analyzer.sh {-} Análisis completo de logs}

\VariableTok{LOG\_FILE}\OperatorTok{=}\StringTok{"}\VariableTok{$1}\StringTok{"}
\VariableTok{OUTPUT\_DIR}\OperatorTok{=}\StringTok{"./analysis{-}}\VariableTok{$(}\FunctionTok{date}\NormalTok{ +\%Y\%m\%d}\VariableTok{)}\StringTok{"}

\ControlFlowTok{if} \BuiltInTok{[} \OtherTok{!} \OtherTok{{-}f} \StringTok{"}\VariableTok{$LOG\_FILE}\StringTok{"} \BuiltInTok{]}\KeywordTok{;} \ControlFlowTok{then}
    \BuiltInTok{echo} \StringTok{"Uso: }\VariableTok{$0}\StringTok{ \textless{}archivo\_log\textgreater{}"}
    \BuiltInTok{exit}\NormalTok{ 1}
\ControlFlowTok{fi}

\FunctionTok{mkdir} \AttributeTok{{-}p} \StringTok{"}\VariableTok{$OUTPUT\_DIR}\StringTok{"}

\BuiltInTok{echo} \StringTok{"📊 Analizando: }\VariableTok{$LOG\_FILE}\StringTok{"}
\BuiltInTok{echo} \StringTok{"📁 Resultados en: }\VariableTok{$OUTPUT\_DIR}\StringTok{"}

\CommentTok{\# 1. Estadísticas generales}
\FunctionTok{analyze\_general()} \KeywordTok{\{}
    \BuiltInTok{echo} \StringTok{"=== ESTADÍSTICAS GENERALES ==="} \OperatorTok{\textgreater{}} \StringTok{"}\VariableTok{$OUTPUT\_DIR}\StringTok{/general.txt"}
    \BuiltInTok{echo} \StringTok{"Total de líneas: }\VariableTok{$(}\FunctionTok{wc} \AttributeTok{{-}l} \OperatorTok{\textless{}} \StringTok{"}\VariableTok{$LOG\_FILE}\StringTok{"}\VariableTok{)}\StringTok{"} \OperatorTok{\textgreater{}\textgreater{}} \StringTok{"}\VariableTok{$OUTPUT\_DIR}\StringTok{/general.txt"}
    \BuiltInTok{echo} \StringTok{"Tamaño del archivo: }\VariableTok{$(}\FunctionTok{du} \AttributeTok{{-}h} \StringTok{"}\VariableTok{$LOG\_FILE}\StringTok{"} \KeywordTok{|} \FunctionTok{cut} \AttributeTok{{-}f1}\VariableTok{)}\StringTok{"} \OperatorTok{\textgreater{}\textgreater{}} \StringTok{"}\VariableTok{$OUTPUT\_DIR}\StringTok{/general.txt"}
    \BuiltInTok{echo} \StringTok{"Primer entrada: }\VariableTok{$(}\FunctionTok{head} \AttributeTok{{-}1} \StringTok{"}\VariableTok{$LOG\_FILE}\StringTok{"} \KeywordTok{|} \FunctionTok{cut} \AttributeTok{{-}d}\StringTok{\textquotesingle{} \textquotesingle{}} \AttributeTok{{-}f1{-}2}\VariableTok{)}\StringTok{"} \OperatorTok{\textgreater{}\textgreater{}} \StringTok{"}\VariableTok{$OUTPUT\_DIR}\StringTok{/general.txt"}
    \BuiltInTok{echo} \StringTok{"Última entrada: }\VariableTok{$(}\FunctionTok{tail} \AttributeTok{{-}1} \StringTok{"}\VariableTok{$LOG\_FILE}\StringTok{"} \KeywordTok{|} \FunctionTok{cut} \AttributeTok{{-}d}\StringTok{\textquotesingle{} \textquotesingle{}} \AttributeTok{{-}f1{-}2}\VariableTok{)}\StringTok{"} \OperatorTok{\textgreater{}\textgreater{}} \StringTok{"}\VariableTok{$OUTPUT\_DIR}\StringTok{/general.txt"}
\KeywordTok{\}}

\CommentTok{\# 2. Análisis de errores}
\FunctionTok{analyze\_errors()} \KeywordTok{\{}
    \BuiltInTok{echo} \StringTok{"🚨 Analizando errores..."}
    
    \ExtensionTok{rg} \AttributeTok{{-}i} \StringTok{"(error|exception|fatal|critical)"} \StringTok{"}\VariableTok{$LOG\_FILE}\StringTok{"} \OperatorTok{\textgreater{}} \StringTok{"}\VariableTok{$OUTPUT\_DIR}\StringTok{/errors.txt"}
    
    \BuiltInTok{echo} \StringTok{"=== TOP ERRORES ==="} \OperatorTok{\textgreater{}} \StringTok{"}\VariableTok{$OUTPUT\_DIR}\StringTok{/error\_summary.txt"}
    \ExtensionTok{rg} \AttributeTok{{-}i} \AttributeTok{{-}o} \StringTok{"(error|exception|fatal|critical).*"} \StringTok{"}\VariableTok{$LOG\_FILE}\StringTok{"} \KeywordTok{|} \DataTypeTok{\textbackslash{}}
    \FunctionTok{sort} \KeywordTok{|} \FunctionTok{uniq} \AttributeTok{{-}c} \KeywordTok{|} \FunctionTok{sort} \AttributeTok{{-}nr} \KeywordTok{|} \FunctionTok{head} \AttributeTok{{-}20} \OperatorTok{\textgreater{}\textgreater{}} \StringTok{"}\VariableTok{$OUTPUT\_DIR}\StringTok{/error\_summary.txt"}
\KeywordTok{\}}

\CommentTok{\# 3. Análisis de IPs (si es log web)}
\FunctionTok{analyze\_ips()} \KeywordTok{\{}
    \BuiltInTok{echo} \StringTok{"🌐 Analizando IPs..."}
    
    \CommentTok{\# Top IPs}
    \ExtensionTok{rg} \AttributeTok{{-}o} \StringTok{\textquotesingle{}\^{}[0{-}9.]+\textquotesingle{}} \StringTok{"}\VariableTok{$LOG\_FILE}\StringTok{"} \KeywordTok{|} \FunctionTok{sort} \KeywordTok{|} \FunctionTok{uniq} \AttributeTok{{-}c} \KeywordTok{|} \FunctionTok{sort} \AttributeTok{{-}nr} \KeywordTok{|} \FunctionTok{head} \AttributeTok{{-}20} \OperatorTok{\textgreater{}} \StringTok{"}\VariableTok{$OUTPUT\_DIR}\StringTok{/top\_ips.txt"}
    
    \CommentTok{\# IPs sospechosas (muchas peticiones)}
    \ExtensionTok{rg} \AttributeTok{{-}o} \StringTok{\textquotesingle{}\^{}[0{-}9.]+\textquotesingle{}} \StringTok{"}\VariableTok{$LOG\_FILE}\StringTok{"} \KeywordTok{|} \FunctionTok{sort} \KeywordTok{|} \FunctionTok{uniq} \AttributeTok{{-}c} \KeywordTok{|} \FunctionTok{sort} \AttributeTok{{-}nr} \KeywordTok{|} \DataTypeTok{\textbackslash{}}
    \FunctionTok{awk} \StringTok{\textquotesingle{}$1 \textgreater{} 1000 \{print $2 " (" $1 " requests)"\}\textquotesingle{}} \OperatorTok{\textgreater{}} \StringTok{"}\VariableTok{$OUTPUT\_DIR}\StringTok{/suspicious\_ips.txt"}
\KeywordTok{\}}

\CommentTok{\# 4. Análisis temporal}
\FunctionTok{analyze\_timeline()} \KeywordTok{\{}
    \BuiltInTok{echo} \StringTok{"⏰ Analizando timeline..."}
    
    \CommentTok{\# Actividad por hora}
    \ExtensionTok{rg} \AttributeTok{{-}o} \StringTok{\textquotesingle{}\^{}\textbackslash{}S+ \textbackslash{}S+\textquotesingle{}} \StringTok{"}\VariableTok{$LOG\_FILE}\StringTok{"} \KeywordTok{|} \FunctionTok{cut} \AttributeTok{{-}d:} \AttributeTok{{-}f1{-}2} \KeywordTok{|} \FunctionTok{sort} \KeywordTok{|} \FunctionTok{uniq} \AttributeTok{{-}c} \KeywordTok{|} \FunctionTok{sort} \AttributeTok{{-}nr} \OperatorTok{\textgreater{}} \StringTok{"}\VariableTok{$OUTPUT\_DIR}\StringTok{/hourly\_activity.txt"}
    
    \CommentTok{\# Actividad por día}
    \ExtensionTok{rg} \AttributeTok{{-}o} \StringTok{\textquotesingle{}\^{}\textbackslash{}S+\textquotesingle{}} \StringTok{"}\VariableTok{$LOG\_FILE}\StringTok{"} \KeywordTok{|} \FunctionTok{sort} \KeywordTok{|} \FunctionTok{uniq} \AttributeTok{{-}c} \KeywordTok{|} \FunctionTok{sort} \AttributeTok{{-}nr} \OperatorTok{\textgreater{}} \StringTok{"}\VariableTok{$OUTPUT\_DIR}\StringTok{/daily\_activity.txt"}
\KeywordTok{\}}

\CommentTok{\# 5. Generar dashboard HTML}
\FunctionTok{generate\_dashboard()} \KeywordTok{\{}
    \BuiltInTok{echo} \StringTok{"📈 Generando dashboard..."}
    
    \FunctionTok{cat} \OperatorTok{\textgreater{}} \StringTok{"}\VariableTok{$OUTPUT\_DIR}\StringTok{/dashboard.html"} \OperatorTok{\textless{}\textless{} \textquotesingle{}EOF\textquotesingle{}}
\StringTok{\textless{}!DOCTYPE html\textgreater{}}
\StringTok{\textless{}html\textgreater{}}
\StringTok{\textless{}head\textgreater{}}
\StringTok{    \textless{}title\textgreater{}Log Analysis Dashboard\textless{}/title\textgreater{}}
\StringTok{    \textless{}style\textgreater{}}
\StringTok{        body \{ font{-}family: Arial, sans{-}serif; margin: 20px; \}}
\StringTok{        .section \{ margin: 20px 0; padding: 15px; border: 1px solid \#ddd; \}}
\StringTok{        .error \{ color: red; \}}
\StringTok{        .warning \{ color: orange; \}}
\StringTok{        .info \{ color: blue; \}}
\StringTok{        pre \{ background: \#f5f5f5; padding: 10px; overflow{-}x: auto; \}}
\StringTok{    \textless{}/style\textgreater{}}
\StringTok{\textless{}/head\textgreater{}}
\StringTok{\textless{}body\textgreater{}}
\StringTok{    \textless{}h1\textgreater{}Log Analysis Dashboard\textless{}/h1\textgreater{}}
\StringTok{    }
\StringTok{    \textless{}div class="section"\textgreater{}}
\StringTok{        \textless{}h2\textgreater{}General Statistics\textless{}/h2\textgreater{}}
\StringTok{        \textless{}pre id="general"\textgreater{}\textless{}/pre\textgreater{}}
\StringTok{    \textless{}/div\textgreater{}}
\StringTok{    }
\StringTok{    \textless{}div class="section"\textgreater{}}
\StringTok{        \textless{}h2\textgreater{}Top Errors\textless{}/h2\textgreater{}}
\StringTok{        \textless{}pre id="errors"\textgreater{}\textless{}/pre\textgreater{}}
\StringTok{    \textless{}/div\textgreater{}}
\StringTok{    }
\StringTok{    \textless{}div class="section"\textgreater{}}
\StringTok{        \textless{}h2\textgreater{}Top IPs\textless{}/h2\textgreater{}}
\StringTok{        \textless{}pre id="ips"\textgreater{}\textless{}/pre\textgreater{}}
\StringTok{    \textless{}/div\textgreater{}}
\StringTok{    }
\StringTok{    \textless{}div class="section"\textgreater{}}
\StringTok{        \textless{}h2\textgreater{}Timeline Analysis\textless{}/h2\textgreater{}}
\StringTok{        \textless{}pre id="timeline"\textgreater{}\textless{}/pre\textgreater{}}
\StringTok{    \textless{}/div\textgreater{}}
\StringTok{    }
\StringTok{    \textless{}script\textgreater{}}
\StringTok{        // Cargar datos desde archivos de texto}
\StringTok{        fetch(\textquotesingle{}./general.txt\textquotesingle{}).then(r =\textgreater{} r.text()).then(data =\textgreater{} \{}
\StringTok{            document.getElementById(\textquotesingle{}general\textquotesingle{}).textContent = data;}
\StringTok{        \});}
\StringTok{        // Similar para otros archivos...}
\StringTok{    \textless{}/script\textgreater{}}
\StringTok{\textless{}/body\textgreater{}}
\StringTok{\textless{}/html\textgreater{}}
\OperatorTok{EOF}
\KeywordTok{\}}

\CommentTok{\# Ejecutar análisis completo}
\FunctionTok{main()} \KeywordTok{\{}
    \ExtensionTok{analyze\_general}
    \ExtensionTok{analyze\_errors}
    \ExtensionTok{analyze\_ips}
    \ExtensionTok{analyze\_timeline}
    \ExtensionTok{generate\_dashboard}
    
    \BuiltInTok{echo} \StringTok{"✅ Análisis completado"}
    \BuiltInTok{echo} \StringTok{"📊 Dashboard: }\VariableTok{$OUTPUT\_DIR}\StringTok{/dashboard.html"}
    \BuiltInTok{echo} \StringTok{"📁 Archivos: }\VariableTok{$OUTPUT\_DIR}\StringTok{/"}
\KeywordTok{\}}

\ExtensionTok{main}
\end{Highlighting}
\end{Shaded}

\section{Automatización de
sistemas}\label{automatizaciuxf3n-de-sistemas}

\subsection{Monitor y alertas
inteligentes}\label{monitor-y-alertas-inteligentes}

\begin{Shaded}
\begin{Highlighting}[]
\CommentTok{\#!/bin/bash}
\CommentTok{\# smart{-}monitor.sh {-} Sistema de monitoreo inteligente}

\VariableTok{CONFIG\_FILE}\OperatorTok{=}\StringTok{"}\VariableTok{$HOME}\StringTok{/.config/smart{-}monitor/config.json"}
\VariableTok{ALERT\_LOG}\OperatorTok{=}\StringTok{"}\VariableTok{$HOME}\StringTok{/.local/log/alerts.log"}

\CommentTok{\# Crear configuración por defecto}
\FunctionTok{init\_config()} \KeywordTok{\{}
    \FunctionTok{mkdir} \AttributeTok{{-}p} \StringTok{"}\VariableTok{$(}\FunctionTok{dirname} \StringTok{"}\VariableTok{$CONFIG\_FILE}\StringTok{"}\VariableTok{)}\StringTok{"}
    \FunctionTok{mkdir} \AttributeTok{{-}p} \StringTok{"}\VariableTok{$(}\FunctionTok{dirname} \StringTok{"}\VariableTok{$ALERT\_LOG}\StringTok{"}\VariableTok{)}\StringTok{"}
    
    \FunctionTok{cat} \OperatorTok{\textgreater{}} \StringTok{"}\VariableTok{$CONFIG\_FILE}\StringTok{"} \OperatorTok{\textless{}\textless{} \textquotesingle{}EOF\textquotesingle{}}
\StringTok{\{}
\StringTok{    "thresholds": \{}
\StringTok{        "cpu": 80,}
\StringTok{        "memory": 85,}
\StringTok{        "disk": 90,}
\StringTok{        "load": 5.0}
\StringTok{    \},}
\StringTok{    "checks": \{}
\StringTok{        "services": ["nginx", "postgresql", "redis"],}
\StringTok{        "ports": [80, 443, 5432, 6379],}
\StringTok{        "urls": ["https://example.com", "https://api.example.com/health"]}
\StringTok{    \},}
\StringTok{    "alerts": \{}
\StringTok{        "webhook": "https://hooks.slack.com/your/webhook",}
\StringTok{        "email": "admin@example.com"}
\StringTok{    \}}
\StringTok{\}}
\OperatorTok{EOF}
    
    \BuiltInTok{echo} \StringTok{"Configuración creada en: }\VariableTok{$CONFIG\_FILE}\StringTok{"}
\KeywordTok{\}}

\CommentTok{\# Función de logging}
\FunctionTok{log\_alert()} \KeywordTok{\{}
    \BuiltInTok{local} \VariableTok{level}\OperatorTok{=}\StringTok{"}\VariableTok{$1}\StringTok{"}
    \BuiltInTok{local} \VariableTok{message}\OperatorTok{=}\StringTok{"}\VariableTok{$2}\StringTok{"}
    \BuiltInTok{echo} \StringTok{"[}\VariableTok{$(}\FunctionTok{date} \StringTok{\textquotesingle{}+\%Y{-}\%m{-}\%d \%H:\%M:\%S\textquotesingle{}}\VariableTok{)}\StringTok{] [}\VariableTok{$level}\StringTok{] }\VariableTok{$message}\StringTok{"} \KeywordTok{|} \FunctionTok{tee} \AttributeTok{{-}a} \StringTok{"}\VariableTok{$ALERT\_LOG}\StringTok{"}
\KeywordTok{\}}

\CommentTok{\# Verificaciones del sistema}
\FunctionTok{check\_system\_resources()} \KeywordTok{\{}
    \BuiltInTok{local} \VariableTok{cpu\_threshold}\OperatorTok{=}\VariableTok{$(}\ExtensionTok{jq} \AttributeTok{{-}r} \StringTok{\textquotesingle{}.thresholds.cpu\textquotesingle{}} \StringTok{"}\VariableTok{$CONFIG\_FILE}\StringTok{"}\VariableTok{)}
    \BuiltInTok{local} \VariableTok{mem\_threshold}\OperatorTok{=}\VariableTok{$(}\ExtensionTok{jq} \AttributeTok{{-}r} \StringTok{\textquotesingle{}.thresholds.memory\textquotesingle{}} \StringTok{"}\VariableTok{$CONFIG\_FILE}\StringTok{"}\VariableTok{)}
    \BuiltInTok{local} \VariableTok{disk\_threshold}\OperatorTok{=}\VariableTok{$(}\ExtensionTok{jq} \AttributeTok{{-}r} \StringTok{\textquotesingle{}.thresholds.disk\textquotesingle{}} \StringTok{"}\VariableTok{$CONFIG\_FILE}\StringTok{"}\VariableTok{)}
    
    \CommentTok{\# CPU}
    \VariableTok{cpu\_usage}\OperatorTok{=}\VariableTok{$(}\ExtensionTok{top} \AttributeTok{{-}bn1} \KeywordTok{|} \FunctionTok{grep} \StringTok{"Cpu(s)"} \KeywordTok{|} \FunctionTok{awk} \StringTok{\textquotesingle{}\{print $2\}\textquotesingle{}} \KeywordTok{|} \FunctionTok{cut} \AttributeTok{{-}d}\StringTok{\textquotesingle{}\%\textquotesingle{}} \AttributeTok{{-}f1}\VariableTok{)}
    \ControlFlowTok{if} \KeywordTok{((} \VariableTok{$(}\BuiltInTok{echo} \StringTok{"}\VariableTok{$cpu\_usage}\StringTok{ \textgreater{} }\VariableTok{$cpu\_threshold}\StringTok{"} \KeywordTok{|} \FunctionTok{bc} \AttributeTok{{-}l}\VariableTok{)} \KeywordTok{));} \ControlFlowTok{then}
        \ExtensionTok{log\_alert} \StringTok{"CRITICAL"} \StringTok{"CPU usage is }\VariableTok{$\{cpu\_usage\}}\StringTok{\%"}
        \ExtensionTok{send\_alert} \StringTok{"🚨 High CPU Usage"} \StringTok{"CPU usage: }\VariableTok{$\{cpu\_usage\}}\StringTok{\%"}
    \ControlFlowTok{fi}
    
    \CommentTok{\# Memory}
    \VariableTok{mem\_usage}\OperatorTok{=}\VariableTok{$(}\FunctionTok{free} \KeywordTok{|} \FunctionTok{grep}\NormalTok{ Mem }\KeywordTok{|} \FunctionTok{awk} \StringTok{\textquotesingle{}\{printf("\%.1f", $3/$2 * 100.0)\}\textquotesingle{}}\VariableTok{)}
    \ControlFlowTok{if} \KeywordTok{((} \VariableTok{$(}\BuiltInTok{echo} \StringTok{"}\VariableTok{$mem\_usage}\StringTok{ \textgreater{} }\VariableTok{$mem\_threshold}\StringTok{"} \KeywordTok{|} \FunctionTok{bc} \AttributeTok{{-}l}\VariableTok{)} \KeywordTok{));} \ControlFlowTok{then}
        \ExtensionTok{log\_alert} \StringTok{"CRITICAL"} \StringTok{"Memory usage is }\VariableTok{$\{mem\_usage\}}\StringTok{\%"}
        \ExtensionTok{send\_alert} \StringTok{"🚨 High Memory Usage"} \StringTok{"Memory usage: }\VariableTok{$\{mem\_usage\}}\StringTok{\%"}
    \ControlFlowTok{fi}
    
    \CommentTok{\# Disk}
    \VariableTok{disk\_usage}\OperatorTok{=}\VariableTok{$(}\FunctionTok{df}\NormalTok{ / }\KeywordTok{|} \FunctionTok{tail} \AttributeTok{{-}1} \KeywordTok{|} \FunctionTok{awk} \StringTok{\textquotesingle{}\{print $5\}\textquotesingle{}} \KeywordTok{|} \FunctionTok{sed} \StringTok{\textquotesingle{}s/\%//\textquotesingle{}}\VariableTok{)}
    \ControlFlowTok{if} \BuiltInTok{[} \StringTok{"}\VariableTok{$disk\_usage}\StringTok{"} \OtherTok{{-}gt} \StringTok{"}\VariableTok{$disk\_threshold}\StringTok{"} \BuiltInTok{]}\KeywordTok{;} \ControlFlowTok{then}
        \ExtensionTok{log\_alert} \StringTok{"CRITICAL"} \StringTok{"Disk usage is }\VariableTok{$\{disk\_usage\}}\StringTok{\%"}
        \ExtensionTok{send\_alert} \StringTok{"🚨 High Disk Usage"} \StringTok{"Disk usage: }\VariableTok{$\{disk\_usage\}}\StringTok{\%"}
    \ControlFlowTok{fi}
\KeywordTok{\}}

\FunctionTok{check\_services()} \KeywordTok{\{}
    \ExtensionTok{jq} \AttributeTok{{-}r} \StringTok{\textquotesingle{}.checks.services[]\textquotesingle{}} \StringTok{"}\VariableTok{$CONFIG\_FILE}\StringTok{"} \KeywordTok{|} \ControlFlowTok{while} \BuiltInTok{read} \AttributeTok{{-}r} \VariableTok{service}\KeywordTok{;} \ControlFlowTok{do}
        \ControlFlowTok{if} \OtherTok{! }\ExtensionTok{systemctl}\NormalTok{ is{-}active }\StringTok{"}\VariableTok{$service}\StringTok{"} \OperatorTok{\textgreater{}}\NormalTok{/dev/null }\DecValTok{2}\OperatorTok{\textgreater{}\&}\DecValTok{1}\KeywordTok{;} \ControlFlowTok{then}
            \ExtensionTok{log\_alert} \StringTok{"ERROR"} \StringTok{"Service }\VariableTok{$service}\StringTok{ is not running"}
            \ExtensionTok{send\_alert} \StringTok{"❌ Service Down"} \StringTok{"}\VariableTok{$service}\StringTok{ is not running"}
        \ControlFlowTok{fi}
    \ControlFlowTok{done}
\KeywordTok{\}}

\FunctionTok{check\_urls()} \KeywordTok{\{}
    \ExtensionTok{jq} \AttributeTok{{-}r} \StringTok{\textquotesingle{}.checks.urls[]\textquotesingle{}} \StringTok{"}\VariableTok{$CONFIG\_FILE}\StringTok{"} \KeywordTok{|} \ControlFlowTok{while} \BuiltInTok{read} \AttributeTok{{-}r} \VariableTok{url}\KeywordTok{;} \ControlFlowTok{do}
        \VariableTok{response\_code}\OperatorTok{=}\VariableTok{$(}\ExtensionTok{curl} \AttributeTok{{-}o}\NormalTok{ /dev/null }\AttributeTok{{-}s} \AttributeTok{{-}w} \StringTok{"\%\{http\_code\}"} \StringTok{"}\VariableTok{$url}\StringTok{"}\VariableTok{)}
        \VariableTok{response\_time}\OperatorTok{=}\VariableTok{$(}\ExtensionTok{curl} \AttributeTok{{-}o}\NormalTok{ /dev/null }\AttributeTok{{-}s} \AttributeTok{{-}w} \StringTok{"\%\{time\_total\}"} \StringTok{"}\VariableTok{$url}\StringTok{"}\VariableTok{)}
        
        \ControlFlowTok{if} \BuiltInTok{[} \StringTok{"}\VariableTok{$response\_code}\StringTok{"} \OtherTok{!=} \StringTok{"200"} \BuiltInTok{]}\KeywordTok{;} \ControlFlowTok{then}
            \ExtensionTok{log\_alert} \StringTok{"ERROR"} \StringTok{"}\VariableTok{$url}\StringTok{ returned }\VariableTok{$response\_code}\StringTok{"}
            \ExtensionTok{send\_alert} \StringTok{"🌐 Website Issue"} \StringTok{"}\VariableTok{$url}\StringTok{ returned }\VariableTok{$response\_code}\StringTok{"}
        \ControlFlowTok{elif} \KeywordTok{((} \VariableTok{$(}\BuiltInTok{echo} \StringTok{"}\VariableTok{$response\_time}\StringTok{ \textgreater{} 5"} \KeywordTok{|} \FunctionTok{bc} \AttributeTok{{-}l}\VariableTok{)} \KeywordTok{));} \ControlFlowTok{then}
            \ExtensionTok{log\_alert} \StringTok{"WARNING"} \StringTok{"}\VariableTok{$url}\StringTok{ is slow (}\VariableTok{$\{response\_time\}}\StringTok{s)"}
        \ControlFlowTok{fi}
    \ControlFlowTok{done}
\KeywordTok{\}}

\FunctionTok{send\_alert()} \KeywordTok{\{}
    \BuiltInTok{local} \VariableTok{title}\OperatorTok{=}\StringTok{"}\VariableTok{$1}\StringTok{"}
    \BuiltInTok{local} \VariableTok{message}\OperatorTok{=}\StringTok{"}\VariableTok{$2}\StringTok{"}
    \BuiltInTok{local} \VariableTok{webhook}\OperatorTok{=}\VariableTok{$(}\ExtensionTok{jq} \AttributeTok{{-}r} \StringTok{\textquotesingle{}.alerts.webhook\textquotesingle{}} \StringTok{"}\VariableTok{$CONFIG\_FILE}\StringTok{"}\VariableTok{)}
    
    \ControlFlowTok{if} \BuiltInTok{[} \StringTok{"}\VariableTok{$webhook}\StringTok{"} \OtherTok{!=} \StringTok{"null"} \BuiltInTok{]}\KeywordTok{;} \ControlFlowTok{then}
        \ExtensionTok{curl} \AttributeTok{{-}X}\NormalTok{ POST }\AttributeTok{{-}H} \StringTok{\textquotesingle{}Content{-}type: application/json\textquotesingle{}} \DataTypeTok{\textbackslash{}}
             \AttributeTok{{-}{-}data} \StringTok{"\{}\DataTypeTok{\textbackslash{}"}\StringTok{text}\DataTypeTok{\textbackslash{}"}\StringTok{:}\DataTypeTok{\textbackslash{}"}\VariableTok{$title}\StringTok{: }\VariableTok{$message}\DataTypeTok{\textbackslash{}"}\StringTok{\}"} \DataTypeTok{\textbackslash{}}
             \StringTok{"}\VariableTok{$webhook}\StringTok{"}
    \ControlFlowTok{fi}
\KeywordTok{\}}

\CommentTok{\# Función principal}
\FunctionTok{monitor()} \KeywordTok{\{}
    \ExtensionTok{log\_alert} \StringTok{"INFO"} \StringTok{"Starting system monitoring"}
    
    \ExtensionTok{check\_system\_resources}
    \ExtensionTok{check\_services}
    \ExtensionTok{check\_urls}
    
    \ExtensionTok{log\_alert} \StringTok{"INFO"} \StringTok{"Monitoring cycle completed"}
\KeywordTok{\}}

\CommentTok{\# Menú principal}
\ControlFlowTok{case} \StringTok{"}\VariableTok{$\{1}\OperatorTok{:{-}}\NormalTok{monitor}\VariableTok{\}}\StringTok{"} \KeywordTok{in}
    \SpecialStringTok{init}\KeywordTok{)}
        \ExtensionTok{init\_config}
        \ControlFlowTok{;;}
    \SpecialStringTok{monitor}\KeywordTok{)}
        \BuiltInTok{[} \OtherTok{!} \OtherTok{{-}f} \StringTok{"}\VariableTok{$CONFIG\_FILE}\StringTok{"} \BuiltInTok{]} \KeywordTok{\&\&} \ExtensionTok{init\_config}
        \ExtensionTok{monitor}
        \ControlFlowTok{;;}
    \SpecialStringTok{logs}\KeywordTok{)}
        \FunctionTok{tail} \AttributeTok{{-}f} \StringTok{"}\VariableTok{$ALERT\_LOG}\StringTok{"}
        \ControlFlowTok{;;}
    \PreprocessorTok{*}\KeywordTok{)}
        \BuiltInTok{echo} \StringTok{"Smart Monitor"}
        \BuiltInTok{echo} \StringTok{"Uso: }\VariableTok{$0}\StringTok{ \textless{}comando\textgreater{}"}
        \BuiltInTok{echo} \StringTok{""}
        \BuiltInTok{echo} \StringTok{"Comandos:"}
        \BuiltInTok{echo} \StringTok{"  init     {-} Crear configuración inicial"}
        \BuiltInTok{echo} \StringTok{"  monitor  {-} Ejecutar verificaciones"}
        \BuiltInTok{echo} \StringTok{"  logs     {-} Ver logs de alertas"}
        \ControlFlowTok{;;}
\ControlFlowTok{esac}
\end{Highlighting}
\end{Shaded}

\section{Workflows creativos}\label{workflows-creativos}

\subsection{Generador de reportes
automático}\label{generador-de-reportes-automuxe1tico}

\begin{Shaded}
\begin{Highlighting}[]
\CommentTok{\#!/bin/bash}
\CommentTok{\# report{-}generator.sh {-} Generador automático de reportes}

\VariableTok{PROJECT\_DIR}\OperatorTok{=}\StringTok{"}\VariableTok{$\{1}\OperatorTok{:{-}}\NormalTok{.}\VariableTok{\}}\StringTok{"}
\VariableTok{REPORT\_DIR}\OperatorTok{=}\StringTok{"}\VariableTok{$PROJECT\_DIR}\StringTok{/reports/}\VariableTok{$(}\FunctionTok{date}\NormalTok{ +\%Y\%m\%d}\VariableTok{)}\StringTok{"}

\FunctionTok{generate\_project\_report()} \KeywordTok{\{}
    \FunctionTok{mkdir} \AttributeTok{{-}p} \StringTok{"}\VariableTok{$REPORT\_DIR}\StringTok{"}
    
    \BuiltInTok{echo} \StringTok{"📋 Generando reporte del proyecto..."}
    
    \CommentTok{\# Reporte principal en Markdown}
    \FunctionTok{cat} \OperatorTok{\textgreater{}} \StringTok{"}\VariableTok{$REPORT\_DIR}\StringTok{/project\_report.md"} \OperatorTok{\textless{}\textless{} EOF}
\StringTok{\# Project Report {-} }\VariableTok{$(}\FunctionTok{date} \StringTok{\textquotesingle{}+\%Y{-}\%m{-}\%d\textquotesingle{}}\VariableTok{)}

\StringTok{\#\# Overview}

\StringTok{**Project:** }\VariableTok{$(}\FunctionTok{basename} \StringTok{"}\VariableTok{$PROJECT\_DIR}\StringTok{"}\VariableTok{)}\StringTok{  }
\StringTok{**Generated:** }\VariableTok{$(}\FunctionTok{date}\VariableTok{)}\StringTok{  }
\StringTok{**Directory:** }\VariableTok{$PROJECT\_DIR}

\StringTok{\#\# Code Statistics}

\DataTypeTok{\textbackslash{}\textasciigrave{}\textbackslash{}\textasciigrave{}\textbackslash{}\textasciigrave{}}
\VariableTok{$(}\FunctionTok{find} \StringTok{"}\VariableTok{$PROJECT\_DIR}\StringTok{"} \AttributeTok{{-}type}\NormalTok{ f }\AttributeTok{{-}name} \StringTok{"*.py"} \AttributeTok{{-}o} \AttributeTok{{-}name} \StringTok{"*.js"} \AttributeTok{{-}o} \AttributeTok{{-}name} \StringTok{"*.ts"} \AttributeTok{{-}o} \AttributeTok{{-}name} \StringTok{"*.go"} \KeywordTok{|} \FunctionTok{xargs}\NormalTok{ wc }\AttributeTok{{-}l} \KeywordTok{|} \FunctionTok{tail} \AttributeTok{{-}1}\VariableTok{)}
\DataTypeTok{\textbackslash{}\textasciigrave{}\textbackslash{}\textasciigrave{}\textbackslash{}\textasciigrave{}}

\StringTok{\#\#\# Files by Type}
\DataTypeTok{\textbackslash{}\textasciigrave{}\textbackslash{}\textasciigrave{}\textbackslash{}\textasciigrave{}}
\VariableTok{$(}\FunctionTok{find} \StringTok{"}\VariableTok{$PROJECT\_DIR}\StringTok{"} \AttributeTok{{-}type}\NormalTok{ f }\KeywordTok{|} \FunctionTok{grep} \AttributeTok{{-}E} \StringTok{\textquotesingle{}\textbackslash{}.[a{-}z]+$\textquotesingle{}} \KeywordTok{|} \FunctionTok{sed} \StringTok{\textquotesingle{}s/.*\textbackslash{}.//\textquotesingle{}} \KeywordTok{|} \FunctionTok{sort} \KeywordTok{|} \FunctionTok{uniq} \AttributeTok{{-}c} \KeywordTok{|} \FunctionTok{sort} \AttributeTok{{-}nr} \KeywordTok{|} \FunctionTok{head} \AttributeTok{{-}10}\VariableTok{)}
\DataTypeTok{\textbackslash{}\textasciigrave{}\textbackslash{}\textasciigrave{}\textbackslash{}\textasciigrave{}}

\StringTok{\#\# Git Activity}

\StringTok{\#\#\# Recent Commits}
\DataTypeTok{\textbackslash{}\textasciigrave{}\textbackslash{}\textasciigrave{}\textbackslash{}\textasciigrave{}}
\VariableTok{$(}\BuiltInTok{cd} \StringTok{"}\VariableTok{$PROJECT\_DIR}\StringTok{"} \KeywordTok{\&\&} \FunctionTok{git}\NormalTok{ log }\AttributeTok{{-}{-}oneline} \AttributeTok{{-}10}\VariableTok{)}
\DataTypeTok{\textbackslash{}\textasciigrave{}\textbackslash{}\textasciigrave{}\textbackslash{}\textasciigrave{}}

\StringTok{\#\#\# Contributors}
\DataTypeTok{\textbackslash{}\textasciigrave{}\textbackslash{}\textasciigrave{}\textbackslash{}\textasciigrave{}}
\VariableTok{$(}\BuiltInTok{cd} \StringTok{"}\VariableTok{$PROJECT\_DIR}\StringTok{"} \KeywordTok{\&\&} \FunctionTok{git}\NormalTok{ shortlog }\AttributeTok{{-}sn} \KeywordTok{|} \FunctionTok{head} \AttributeTok{{-}10}\VariableTok{)}
\DataTypeTok{\textbackslash{}\textasciigrave{}\textbackslash{}\textasciigrave{}\textbackslash{}\textasciigrave{}}

\StringTok{\#\# Issues and TODOs}

\VariableTok{$(}\ExtensionTok{rg} \AttributeTok{{-}C}\NormalTok{ 1 }\StringTok{"(TODO|FIXME|HACK)"} \StringTok{"}\VariableTok{$PROJECT\_DIR}\StringTok{"} \DecValTok{2}\OperatorTok{\textgreater{}}\NormalTok{/dev/null }\KeywordTok{||} \BuiltInTok{echo} \StringTok{"No issues found"}\VariableTok{)}

\StringTok{\#\# Dependencies}

\OperatorTok{EOF}

    \CommentTok{\# Agregar info específica por tecnología}
    \ControlFlowTok{if} \BuiltInTok{[} \OtherTok{{-}f} \StringTok{"}\VariableTok{$PROJECT\_DIR}\StringTok{/package.json"} \BuiltInTok{]}\KeywordTok{;} \ControlFlowTok{then}
        \BuiltInTok{echo} \StringTok{"\#\#\# Node.js Dependencies"} \OperatorTok{\textgreater{}\textgreater{}} \StringTok{"}\VariableTok{$REPORT\_DIR}\StringTok{/project\_report.md"}
        \BuiltInTok{echo} \StringTok{\textquotesingle{}\textasciigrave{}\textasciigrave{}\textasciigrave{}json\textquotesingle{}} \OperatorTok{\textgreater{}\textgreater{}} \StringTok{"}\VariableTok{$REPORT\_DIR}\StringTok{/project\_report.md"}
        \ExtensionTok{jq} \StringTok{\textquotesingle{}.dependencies // \{\}\textquotesingle{}} \StringTok{"}\VariableTok{$PROJECT\_DIR}\StringTok{/package.json"} \OperatorTok{\textgreater{}\textgreater{}} \StringTok{"}\VariableTok{$REPORT\_DIR}\StringTok{/project\_report.md"}
        \BuiltInTok{echo} \StringTok{\textquotesingle{}\textasciigrave{}\textasciigrave{}\textasciigrave{}\textquotesingle{}} \OperatorTok{\textgreater{}\textgreater{}} \StringTok{"}\VariableTok{$REPORT\_DIR}\StringTok{/project\_report.md"}
    \ControlFlowTok{fi}
    
    \ControlFlowTok{if} \BuiltInTok{[} \OtherTok{{-}f} \StringTok{"}\VariableTok{$PROJECT\_DIR}\StringTok{/requirements.txt"} \BuiltInTok{]}\KeywordTok{;} \ControlFlowTok{then}
        \BuiltInTok{echo} \StringTok{"\#\#\# Python Dependencies"} \OperatorTok{\textgreater{}\textgreater{}} \StringTok{"}\VariableTok{$REPORT\_DIR}\StringTok{/project\_report.md"}
        \BuiltInTok{echo} \StringTok{\textquotesingle{}\textasciigrave{}\textasciigrave{}\textasciigrave{}\textquotesingle{}} \OperatorTok{\textgreater{}\textgreater{}} \StringTok{"}\VariableTok{$REPORT\_DIR}\StringTok{/project\_report.md"}
        \FunctionTok{cat} \StringTok{"}\VariableTok{$PROJECT\_DIR}\StringTok{/requirements.txt"} \OperatorTok{\textgreater{}\textgreater{}} \StringTok{"}\VariableTok{$REPORT\_DIR}\StringTok{/project\_report.md"}
        \BuiltInTok{echo} \StringTok{\textquotesingle{}\textasciigrave{}\textasciigrave{}\textasciigrave{}\textquotesingle{}} \OperatorTok{\textgreater{}\textgreater{}} \StringTok{"}\VariableTok{$REPORT\_DIR}\StringTok{/project\_report.md"}
    \ControlFlowTok{fi}
    
    \CommentTok{\# Convertir a HTML con pandoc}
    \ControlFlowTok{if} \BuiltInTok{command} \AttributeTok{{-}v}\NormalTok{ pandoc }\OperatorTok{\textgreater{}}\NormalTok{/dev/null}\KeywordTok{;} \ControlFlowTok{then}
        \ExtensionTok{pandoc} \AttributeTok{{-}{-}toc} \AttributeTok{{-}c}\NormalTok{ github.css }\StringTok{"}\VariableTok{$REPORT\_DIR}\StringTok{/project\_report.md"} \AttributeTok{{-}o} \StringTok{"}\VariableTok{$REPORT\_DIR}\StringTok{/project\_report.html"}
    \ControlFlowTok{fi}
    
    \CommentTok{\# Generar visualización con glow}
    \ControlFlowTok{if} \BuiltInTok{command} \AttributeTok{{-}v}\NormalTok{ glow }\OperatorTok{\textgreater{}}\NormalTok{/dev/null}\KeywordTok{;} \ControlFlowTok{then}
        \ExtensionTok{glow} \AttributeTok{{-}o} \StringTok{"}\VariableTok{$REPORT\_DIR}\StringTok{/project\_report.pdf"} \StringTok{"}\VariableTok{$REPORT\_DIR}\StringTok{/project\_report.md"} \DecValTok{2}\OperatorTok{\textgreater{}}\NormalTok{/dev/null }\KeywordTok{||} \FunctionTok{true}
    \ControlFlowTok{fi}
    
    \BuiltInTok{echo} \StringTok{"✅ Reporte generado en: }\VariableTok{$REPORT\_DIR}\StringTok{"}
\KeywordTok{\}}

\ExtensionTok{generate\_project\_report}
\end{Highlighting}
\end{Shaded}

\begin{tcolorbox}[enhanced jigsaw, toprule=.15mm, bottomrule=.15mm, opacityback=0, coltitle=black, rightrule=.15mm, colframe=quarto-callout-tip-color-frame, titlerule=0mm, opacitybacktitle=0.6, left=2mm, colback=white, bottomtitle=1mm, arc=.35mm, leftrule=.75mm, title=\textcolor{quarto-callout-tip-color}{\faLightbulb}\hspace{0.5em}{Tips para workflows avanzados}, colbacktitle=quarto-callout-tip-color!10!white, breakable, toptitle=1mm]

\begin{itemize}
\tightlist
\item
  Combina herramientas complementarias para crear pipelines potentes
\item
  Usa JSON para configuraciones complejas que puedas procesar con
  \texttt{jq}
\item
  Implementa logging y error handling en scripts de producción
\item
  Crea interfaces de línea de comandos consistentes para tus scripts
\end{itemize}

\end{tcolorbox}

\begin{tcolorbox}[enhanced jigsaw, toprule=.15mm, bottomrule=.15mm, opacityback=0, coltitle=black, rightrule=.15mm, colframe=quarto-callout-important-color-frame, titlerule=0mm, opacitybacktitle=0.6, left=2mm, colback=white, bottomtitle=1mm, arc=.35mm, leftrule=.75mm, title=\textcolor{quarto-callout-important-color}{\faExclamation}\hspace{0.5em}{Consideraciones para automatización}, colbacktitle=quarto-callout-important-color!10!white, breakable, toptitle=1mm]

\begin{itemize}
\tightlist
\item
  Siempre incluye validación de entrada en scripts automatizados
\item
  Implementa timeouts y rate limiting para operaciones de red
\item
  Mantén logs detallados para debugging y auditoría
\item
  Considera la seguridad al manejar credenciales y datos sensibles
\end{itemize}

\end{tcolorbox}

En el último capítulo exploraremos configuraciones avanzadas y
personalización del entorno de trabajo.

\chapter{Gestión de Homebrew}\label{gestiuxf3n-de-homebrew}

Homebrew es más que un simple instalador de paquetes - es un ecosistema
completo que requiere mantenimiento y gestión adecuada. Esta sección
cubre las mejores prácticas para mantener tu instalación de Homebrew
funcionando de manera óptima.

\section{Comandos esenciales de
Homebrew}\label{comandos-esenciales-de-homebrew}

\subsection{Instalación y
desinstalación}\label{instalaciuxf3n-y-desinstalaciuxf3n}

\begin{Shaded}
\begin{Highlighting}[]
\CommentTok{\# Instalar un paquete}
\ExtensionTok{brew}\NormalTok{ install nombre{-}del{-}paquete}

\CommentTok{\# Instalar una versión específica}
\ExtensionTok{brew}\NormalTok{ install nombre{-}del{-}paquete@version}

\CommentTok{\# Desinstalar un paquete}
\ExtensionTok{brew}\NormalTok{ uninstall nombre{-}del{-}paquete}

\CommentTok{\# Desinstalar completamente incluyendo dependencias no utilizadas}
\ExtensionTok{brew}\NormalTok{ uninstall }\AttributeTok{{-}{-}zap}\NormalTok{ nombre{-}del{-}paquete}
\end{Highlighting}
\end{Shaded}

\subsection{Búsqueda y exploración}\label{buxfasqueda-y-exploraciuxf3n}

\begin{Shaded}
\begin{Highlighting}[]
\CommentTok{\# Buscar paquetes disponibles}
\ExtensionTok{brew}\NormalTok{ search término{-}de{-}búsqueda}

\CommentTok{\# Información detallada de un paquete}
\ExtensionTok{brew}\NormalTok{ info nombre{-}del{-}paquete}

\CommentTok{\# Listar todos los paquetes instalados}
\ExtensionTok{brew}\NormalTok{ list}

\CommentTok{\# Listar solo las fórmulas principales (sin dependencias)}
\ExtensionTok{brew}\NormalTok{ leaves}
\end{Highlighting}
\end{Shaded}

\subsection{Mantenimiento y limpieza}\label{mantenimiento-y-limpieza}

\begin{Shaded}
\begin{Highlighting}[]
\CommentTok{\# Actualizar Homebrew y todos los paquetes}
\ExtensionTok{brew}\NormalTok{ update }\KeywordTok{\&\&} \ExtensionTok{brew}\NormalTok{ upgrade}

\CommentTok{\# Actualizar solo Homebrew (sin actualizar paquetes)}
\ExtensionTok{brew}\NormalTok{ update}

\CommentTok{\# Actualizar un paquete específico}
\ExtensionTok{brew}\NormalTok{ upgrade nombre{-}del{-}paquete}

\CommentTok{\# Limpiar cachés y archivos temporales}
\ExtensionTok{brew}\NormalTok{ cleanup}

\CommentTok{\# Limpiar archivos específicos de un paquete}
\ExtensionTok{brew}\NormalTok{ cleanup nombre{-}del{-}paquete}

\CommentTok{\# Ver qué se va a limpiar sin ejecutar}
\ExtensionTok{brew}\NormalTok{ cleanup }\AttributeTok{{-}{-}dry{-}run}
\end{Highlighting}
\end{Shaded}

\subsection{Diagnóstico y resolución de
problemas}\label{diagnuxf3stico-y-resoluciuxf3n-de-problemas}

\begin{Shaded}
\begin{Highlighting}[]
\CommentTok{\# Verificar la instalación de Homebrew}
\ExtensionTok{brew}\NormalTok{ doctor}

\CommentTok{\# Verificar la integridad de los paquetes instalados}
\ExtensionTok{brew}\NormalTok{ audit }\AttributeTok{{-}{-}installed}

\CommentTok{\# Reinstalar un paquete problemático}
\ExtensionTok{brew}\NormalTok{ reinstall nombre{-}del{-}paquete}

\CommentTok{\# Ver dependencias de un paquete}
\ExtensionTok{brew}\NormalTok{ deps nombre{-}del{-}paquete}

\CommentTok{\# Ver qué paquetes dependen de uno específico}
\ExtensionTok{brew}\NormalTok{ uses nombre{-}del{-}paquete }\AttributeTok{{-}{-}installed}
\end{Highlighting}
\end{Shaded}

\section{Gestión de fórmulas
obsoletas}\label{gestiuxf3n-de-fuxf3rmulas-obsoletas}

Con el tiempo, algunas fórmulas pueden volverse obsoletas o ser
reemplazadas por alternativas mejores. Es importante mantener el sistema
actualizado.

\subsection{Identificar paquetes
obsoletos}\label{identificar-paquetes-obsoletos}

\begin{Shaded}
\begin{Highlighting}[]
\CommentTok{\# Ver fórmulas obsoletas instaladas}
\ExtensionTok{brew}\NormalTok{ list }\AttributeTok{{-}{-}versions} \KeywordTok{|} \FunctionTok{grep} \AttributeTok{{-}E} \StringTok{"(deprecated|obsolete)"}

\CommentTok{\# Verificar el estado de un paquete específico}
\ExtensionTok{brew}\NormalTok{ info nombre{-}del{-}paquete }\KeywordTok{|} \FunctionTok{grep} \AttributeTok{{-}E} \StringTok{"(deprecated|obsolete|unmaintained)"}
\end{Highlighting}
\end{Shaded}

\subsection{Proceso de migración
recomendado}\label{proceso-de-migraciuxf3n-recomendado}

\begin{enumerate}
\def\labelenumi{\arabic{enumi}.}
\tightlist
\item
  \textbf{Identificar la alternativa recomendada}
\item
  \textbf{Instalar la nueva herramienta}
\item
  \textbf{Probar la funcionalidad}
\item
  \textbf{Migrar configuraciones si es necesario}
\item
  \textbf{Desinstalar la versión obsoleta}
\item
  \textbf{Limpiar archivos residuales}
\end{enumerate}

Ejemplo práctico:

\begin{Shaded}
\begin{Highlighting}[]
\CommentTok{\# 1. Verificar estado actual}
\ExtensionTok{brew}\NormalTok{ info herramienta{-}obsoleta}

\CommentTok{\# 2. Instalar alternativa}
\ExtensionTok{brew}\NormalTok{ install nueva{-}herramienta}

\CommentTok{\# 3. Probar funcionalidad}
\ExtensionTok{nueva{-}herramienta} \AttributeTok{{-}{-}version}

\CommentTok{\# 4. Desinstalar obsoleta}
\ExtensionTok{brew}\NormalTok{ uninstall herramienta{-}obsoleta}

\CommentTok{\# 5. Limpiar sistema}
\ExtensionTok{brew}\NormalTok{ cleanup}
\end{Highlighting}
\end{Shaded}

\section{Configuración avanzada}\label{configuraciuxf3n-avanzada-2}

\subsection{Taps (repositorios
adicionales)}\label{taps-repositorios-adicionales}

\begin{Shaded}
\begin{Highlighting}[]
\CommentTok{\# Agregar un tap}
\ExtensionTok{brew}\NormalTok{ tap usuario/repositorio}

\CommentTok{\# Listar taps instalados}
\ExtensionTok{brew}\NormalTok{ tap}

\CommentTok{\# Eliminar un tap}
\ExtensionTok{brew}\NormalTok{ untap usuario/repositorio}
\end{Highlighting}
\end{Shaded}

\subsection{Servicios (daemons)}\label{servicios-daemons}

\begin{Shaded}
\begin{Highlighting}[]
\CommentTok{\# Listar servicios disponibles}
\ExtensionTok{brew}\NormalTok{ services list}

\CommentTok{\# Iniciar un servicio}
\ExtensionTok{brew}\NormalTok{ services start nombre{-}del{-}servicio}

\CommentTok{\# Parar un servicio}
\ExtensionTok{brew}\NormalTok{ services stop nombre{-}del{-}servicio}

\CommentTok{\# Reiniciar un servicio}
\ExtensionTok{brew}\NormalTok{ services restart nombre{-}del{-}servicio}

\CommentTok{\# Limpiar servicios rotos}
\ExtensionTok{brew}\NormalTok{ services cleanup}
\end{Highlighting}
\end{Shaded}

\subsection{Variables de entorno
importantes}\label{variables-de-entorno-importantes}

\begin{Shaded}
\begin{Highlighting}[]
\CommentTok{\# Directorio de instalación de Homebrew}
\BuiltInTok{echo} \VariableTok{$HOMEBREW\_PREFIX}

\CommentTok{\# Directorio de cachés}
\BuiltInTok{echo} \VariableTok{$HOMEBREW\_CACHE}

\CommentTok{\# Configurar directorio de caché personalizado}
\BuiltInTok{export} \VariableTok{HOMEBREW\_CACHE}\OperatorTok{=}\NormalTok{/path/to/custom/cache}
\end{Highlighting}
\end{Shaded}

\section{Mejores prácticas}\label{mejores-pruxe1cticas-2}

\subsection{Mantenimiento regular}\label{mantenimiento-regular}

\begin{Shaded}
\begin{Highlighting}[]
\CommentTok{\#!/bin/bash}
\CommentTok{\# Script de mantenimiento semanal de Homebrew}

\BuiltInTok{echo} \StringTok{"🔄 Actualizando Homebrew..."}
\ExtensionTok{brew}\NormalTok{ update}

\BuiltInTok{echo} \StringTok{"📦 Actualizando paquetes instalados..."}
\ExtensionTok{brew}\NormalTok{ upgrade}

\BuiltInTok{echo} \StringTok{"🧹 Limpiando archivos temporales..."}
\ExtensionTok{brew}\NormalTok{ cleanup}

\BuiltInTok{echo} \StringTok{"🩺 Verificando integridad del sistema..."}
\ExtensionTok{brew}\NormalTok{ doctor}

\BuiltInTok{echo} \StringTok{"✅ Mantenimiento completado"}
\end{Highlighting}
\end{Shaded}

\subsection{Backup de paquetes
instalados}\label{backup-de-paquetes-instalados}

\begin{Shaded}
\begin{Highlighting}[]
\CommentTok{\# Crear lista de paquetes instalados}
\ExtensionTok{brew}\NormalTok{ leaves }\OperatorTok{\textgreater{}}\NormalTok{ \textasciitilde{}/homebrew{-}packages.txt}

\CommentTok{\# Restaurar desde backup}
\FunctionTok{cat}\NormalTok{ \textasciitilde{}/homebrew{-}packages.txt }\KeywordTok{|} \FunctionTok{xargs}\NormalTok{ brew install}
\end{Highlighting}
\end{Shaded}

\subsection{Configuración de aliases
útiles}\label{configuraciuxf3n-de-aliases-uxfatiles}

\begin{Shaded}
\begin{Highlighting}[]
\CommentTok{\# Agregar a \textasciitilde{}/.zshrc o \textasciitilde{}/.bashrc}
\BuiltInTok{alias}\NormalTok{ brewup=}\StringTok{\textquotesingle{}brew update \&\& brew upgrade \&\& brew cleanup\textquotesingle{}}
\BuiltInTok{alias}\NormalTok{ brewdoc=}\StringTok{\textquotesingle{}brew doctor\textquotesingle{}}
\BuiltInTok{alias}\NormalTok{ brewinfo=}\StringTok{\textquotesingle{}brew info\textquotesingle{}}
\BuiltInTok{alias}\NormalTok{ brewsearch=}\StringTok{\textquotesingle{}brew search\textquotesingle{}}
\BuiltInTok{alias}\NormalTok{ brewlist=}\StringTok{\textquotesingle{}brew leaves\textquotesingle{}}
\end{Highlighting}
\end{Shaded}

\section{Troubleshooting común}\label{troubleshooting-comuxfan}

\subsection{Problemas de permisos}\label{problemas-de-permisos}

\begin{Shaded}
\begin{Highlighting}[]
\CommentTok{\# Reparar permisos}
\FunctionTok{sudo}\NormalTok{ chown }\AttributeTok{{-}R} \VariableTok{$(}\FunctionTok{whoami}\VariableTok{)} \VariableTok{$(}\ExtensionTok{brew} \AttributeTok{{-}{-}prefix}\VariableTok{)}\NormalTok{/}\PreprocessorTok{*}

\CommentTok{\# Reinstalar Homebrew si es necesario}
\ExtensionTok{/bin/bash} \AttributeTok{{-}c} \StringTok{"}\VariableTok{$(}\ExtensionTok{curl} \AttributeTok{{-}fsSL}\NormalTok{ https://raw.githubusercontent.com/Homebrew/install/HEAD/uninstall.sh}\VariableTok{)}\StringTok{"}
\ExtensionTok{/bin/bash} \AttributeTok{{-}c} \StringTok{"}\VariableTok{$(}\ExtensionTok{curl} \AttributeTok{{-}fsSL}\NormalTok{ https://raw.githubusercontent.com/Homebrew/install/HEAD/install.sh}\VariableTok{)}\StringTok{"}
\end{Highlighting}
\end{Shaded}

\subsection{Problemas de dependencias}\label{problemas-de-dependencias}

\begin{Shaded}
\begin{Highlighting}[]
\CommentTok{\# Forzar reinstalación de dependencias}
\ExtensionTok{brew}\NormalTok{ reinstall }\VariableTok{$(}\ExtensionTok{brew}\NormalTok{ deps nombre{-}del{-}paquete}\VariableTok{)}

\CommentTok{\# Verificar dependencias rotas}
\ExtensionTok{brew}\NormalTok{ missing}
\end{Highlighting}
\end{Shaded}

\subsection{Limpieza profunda}\label{limpieza-profunda}

\begin{Shaded}
\begin{Highlighting}[]
\CommentTok{\# Eliminar todas las versiones antiguas}
\ExtensionTok{brew}\NormalTok{ cleanup }\AttributeTok{{-}{-}prune}\OperatorTok{=}\NormalTok{all}

\CommentTok{\# Eliminar caché completo}
\FunctionTok{rm} \AttributeTok{{-}rf} \VariableTok{$(}\ExtensionTok{brew} \AttributeTok{{-}{-}cache}\VariableTok{)}

\CommentTok{\# Reconstruir base de datos}
\ExtensionTok{brew}\NormalTok{ update{-}reset}
\end{Highlighting}
\end{Shaded}

\section{Monitoreo y estadísticas}\label{monitoreo-y-estaduxedsticas}

\subsection{Información del sistema}\label{informaciuxf3n-del-sistema}

\begin{Shaded}
\begin{Highlighting}[]
\CommentTok{\# Tamaño total de la instalación}
\FunctionTok{du} \AttributeTok{{-}sh} \VariableTok{$(}\ExtensionTok{brew} \AttributeTok{{-}{-}prefix}\VariableTok{)}

\CommentTok{\# Número de paquetes instalados}
\ExtensionTok{brew}\NormalTok{ list }\KeywordTok{|} \FunctionTok{wc} \AttributeTok{{-}l}

\CommentTok{\# Espacio usado por cachés}
\FunctionTok{du} \AttributeTok{{-}sh} \VariableTok{$(}\ExtensionTok{brew} \AttributeTok{{-}{-}cache}\VariableTok{)}

\CommentTok{\# Últimas actualizaciones}
\ExtensionTok{brew}\NormalTok{ log }\AttributeTok{{-}{-}oneline} \AttributeTok{{-}10}
\end{Highlighting}
\end{Shaded}

\subsection{Análisis de uso}\label{anuxe1lisis-de-uso}

\begin{Shaded}
\begin{Highlighting}[]
\CommentTok{\# Paquetes más grandes instalados}
\ExtensionTok{brew}\NormalTok{ list }\AttributeTok{{-}{-}formula} \KeywordTok{|} \FunctionTok{xargs} \AttributeTok{{-}n1} \AttributeTok{{-}I\{\}}\NormalTok{ sh }\AttributeTok{{-}c} \StringTok{\textquotesingle{}echo $(brew {-}{-}prefix \{\}): $(du {-}sh $(brew {-}{-}prefix \{\}) 2\textgreater{}/dev/null | cut {-}f1)\textquotesingle{}} \KeywordTok{|} \FunctionTok{sort} \AttributeTok{{-}hr} \KeywordTok{|} \FunctionTok{head} \AttributeTok{{-}10}

\CommentTok{\# Dependencias más utilizadas}
\ExtensionTok{brew}\NormalTok{ uses }\AttributeTok{{-}{-}installed} \AttributeTok{{-}{-}recursive} \KeywordTok{|} \FunctionTok{sort} \KeywordTok{|} \FunctionTok{uniq} \AttributeTok{{-}c} \KeywordTok{|} \FunctionTok{sort} \AttributeTok{{-}nr} \KeywordTok{|} \FunctionTok{head} \AttributeTok{{-}10}
\end{Highlighting}
\end{Shaded}

\begin{center}\rule{0.5\linewidth}{0.5pt}\end{center}

\begin{tcolorbox}[enhanced jigsaw, toprule=.15mm, bottomrule=.15mm, opacityback=0, coltitle=black, rightrule=.15mm, colframe=quarto-callout-important-color-frame, titlerule=0mm, opacitybacktitle=0.6, left=2mm, colback=white, bottomtitle=1mm, arc=.35mm, leftrule=.75mm, title=\textcolor{quarto-callout-important-color}{\faExclamation}\hspace{0.5em}{Recordatorio importante}, colbacktitle=quarto-callout-important-color!10!white, breakable, toptitle=1mm]

Siempre realiza un backup de tus configuraciones importantes antes de
realizar cambios mayores en tu instalación de Homebrew. El comando
\texttt{brew\ doctor} es tu mejor amigo para diagnosticar problemas.

\end{tcolorbox}

\chapter{Configuración y
Personalización}\label{configuraciuxf3n-y-personalizaciuxf3n}

Este capítulo final cubre la configuración avanzada del entorno de
trabajo, personalización profunda de herramientas y creación de un
sistema altamente productivo y eficiente.

\section{Configuración del Shell}\label{configuraciuxf3n-del-shell}

\subsection{Configuración de zsh
avanzada}\label{configuraciuxf3n-de-zsh-avanzada}

Archivo \texttt{\textasciitilde{}/.zshrc} optimizado:

\begin{Shaded}
\begin{Highlighting}[]
\CommentTok{\# \textasciitilde{}/.zshrc {-} Configuración avanzada}

\CommentTok{\# History}
\VariableTok{HISTSIZE}\OperatorTok{=}\NormalTok{50000}
\VariableTok{SAVEHIST}\OperatorTok{=}\NormalTok{50000}
\VariableTok{HISTFILE}\OperatorTok{=}\NormalTok{\textasciitilde{}/.zsh\_history}
\ExtensionTok{setopt}\NormalTok{ HIST\_VERIFY}
\ExtensionTok{setopt}\NormalTok{ SHARE\_HISTORY}
\ExtensionTok{setopt}\NormalTok{ APPEND\_HISTORY}
\ExtensionTok{setopt}\NormalTok{ INC\_APPEND\_HISTORY}
\ExtensionTok{setopt}\NormalTok{ HIST\_IGNORE\_DUPS}
\ExtensionTok{setopt}\NormalTok{ HIST\_IGNORE\_ALL\_DUPS}
\ExtensionTok{setopt}\NormalTok{ HIST\_IGNORE\_SPACE}

\CommentTok{\# Options}
\ExtensionTok{setopt}\NormalTok{ AUTO\_CD}
\ExtensionTok{setopt}\NormalTok{ CORRECT}
\ExtensionTok{setopt}\NormalTok{ CORRECT\_ALL}

\CommentTok{\# Completions}
\ExtensionTok{autoload} \AttributeTok{{-}Uz}\NormalTok{ compinit}
\ExtensionTok{compinit}

\CommentTok{\# Case insensitive completion}
\ExtensionTok{zstyle} \StringTok{\textquotesingle{}:completion:*\textquotesingle{}}\NormalTok{ matcher{-}list }\StringTok{\textquotesingle{}m:\{a{-}zA{-}Z\}=\{A{-}Za{-}z\}\textquotesingle{}}

\CommentTok{\# Initialize tools}
\BuiltInTok{eval} \StringTok{"}\VariableTok{$(}\ExtensionTok{starship}\NormalTok{ init zsh}\VariableTok{)}\StringTok{"}
\BuiltInTok{eval} \StringTok{"}\VariableTok{$(}\ExtensionTok{direnv}\NormalTok{ hook zsh}\VariableTok{)}\StringTok{"}
\BuiltInTok{eval} \StringTok{"}\VariableTok{$(}\ExtensionTok{zoxide}\NormalTok{ init zsh}\VariableTok{)}\StringTok{"}
\BuiltInTok{eval} \VariableTok{$(}\ExtensionTok{thefuck} \AttributeTok{{-}{-}alias}\NormalTok{ fix}\VariableTok{)}

\CommentTok{\# Path additions}
\BuiltInTok{export} \VariableTok{PATH}\OperatorTok{=}\StringTok{"/opt/homebrew/bin:}\VariableTok{$PATH}\StringTok{"}
\BuiltInTok{export} \VariableTok{PATH}\OperatorTok{=}\StringTok{"}\VariableTok{$HOME}\StringTok{/.local/bin:}\VariableTok{$PATH}\StringTok{"}

\CommentTok{\# Environment variables}
\BuiltInTok{export} \VariableTok{EDITOR}\OperatorTok{=}\StringTok{"code {-}{-}wait"}
\BuiltInTok{export} \VariableTok{VISUAL}\OperatorTok{=}\StringTok{"}\VariableTok{$EDITOR}\StringTok{"}
\BuiltInTok{export} \VariableTok{PAGER}\OperatorTok{=}\StringTok{"bat"}
\BuiltInTok{export} \VariableTok{MANPAGER}\OperatorTok{=}\StringTok{"sh {-}c \textquotesingle{}col {-}bx | bat {-}l man {-}p\textquotesingle{}"}

\CommentTok{\# FZF configuration}
\BuiltInTok{export} \VariableTok{FZF\_DEFAULT\_COMMAND}\OperatorTok{=}\StringTok{\textquotesingle{}rg {-}{-}files {-}{-}hidden {-}{-}follow {-}{-}glob "!.git/*"\textquotesingle{}}
\BuiltInTok{export} \VariableTok{FZF\_DEFAULT\_OPTS}\OperatorTok{=}\StringTok{\textquotesingle{}}
\StringTok{    {-}{-}height 40\% }
\StringTok{    {-}{-}layout=reverse }
\StringTok{    {-}{-}border}
\StringTok{    {-}{-}preview "bat {-}{-}color=always {-}{-}style=header,grid {-}{-}line{-}range :300 \{\}"}
\StringTok{\textquotesingle{}}

\CommentTok{\# Aliases}
\BuiltInTok{alias}\NormalTok{ ls=}\StringTok{\textquotesingle{}eza\textquotesingle{}}
\BuiltInTok{alias}\NormalTok{ ll=}\StringTok{\textquotesingle{}eza {-}la {-}{-}git\textquotesingle{}}
\BuiltInTok{alias}\NormalTok{ lt=}\StringTok{\textquotesingle{}eza {-}{-}tree\textquotesingle{}}
\BuiltInTok{alias}\NormalTok{ cat=}\StringTok{\textquotesingle{}bat\textquotesingle{}}
\BuiltInTok{alias}\NormalTok{ grep=}\StringTok{\textquotesingle{}rg\textquotesingle{}}
\BuiltInTok{alias}\NormalTok{ find=}\StringTok{\textquotesingle{}fd\textquotesingle{}}
\BuiltInTok{alias}\NormalTok{ du=}\StringTok{\textquotesingle{}dust\textquotesingle{}}
\BuiltInTok{alias}\NormalTok{ df=}\StringTok{\textquotesingle{}duf\textquotesingle{}}
\BuiltInTok{alias}\NormalTok{ ps=}\StringTok{\textquotesingle{}procs\textquotesingle{}}
\BuiltInTok{alias}\NormalTok{ top=}\StringTok{\textquotesingle{}htop\textquotesingle{}}

\CommentTok{\# Git aliases}
\BuiltInTok{alias}\NormalTok{ g=}\StringTok{\textquotesingle{}git\textquotesingle{}}
\BuiltInTok{alias}\NormalTok{ gs=}\StringTok{\textquotesingle{}git status\textquotesingle{}}
\BuiltInTok{alias}\NormalTok{ ga=}\StringTok{\textquotesingle{}git add\textquotesingle{}}
\BuiltInTok{alias}\NormalTok{ gc=}\StringTok{\textquotesingle{}git commit\textquotesingle{}}
\BuiltInTok{alias}\NormalTok{ gp=}\StringTok{\textquotesingle{}git push\textquotesingle{}}
\BuiltInTok{alias}\NormalTok{ gl=}\StringTok{\textquotesingle{}git pull\textquotesingle{}}
\BuiltInTok{alias}\NormalTok{ gco=}\StringTok{\textquotesingle{}git checkout\textquotesingle{}}
\BuiltInTok{alias}\NormalTok{ gb=}\StringTok{\textquotesingle{}git branch\textquotesingle{}}
\BuiltInTok{alias}\NormalTok{ gm=}\StringTok{\textquotesingle{}git merge\textquotesingle{}}
\BuiltInTok{alias}\NormalTok{ gd=}\StringTok{\textquotesingle{}git diff\textquotesingle{}}
\BuiltInTok{alias}\NormalTok{ glog=}\StringTok{\textquotesingle{}git log {-}{-}oneline {-}{-}graph\textquotesingle{}}

\CommentTok{\# Functions}
\CommentTok{\# Quick CD and list}
\FunctionTok{cl()} \KeywordTok{\{}
    \BuiltInTok{cd} \StringTok{"}\VariableTok{$1}\StringTok{"} \KeywordTok{\&\&} \ExtensionTok{ll}
\KeywordTok{\}}

\CommentTok{\# Make directory and CD into it}
\FunctionTok{mkcd()} \KeywordTok{\{}
    \FunctionTok{mkdir} \AttributeTok{{-}p} \StringTok{"}\VariableTok{$1}\StringTok{"} \KeywordTok{\&\&} \BuiltInTok{cd} \StringTok{"}\VariableTok{$1}\StringTok{"}
\KeywordTok{\}}

\CommentTok{\# Find and edit files}
\FunctionTok{fe()} \KeywordTok{\{}
    \BuiltInTok{local} \VariableTok{files}
    \VariableTok{files}\OperatorTok{=}\VariableTok{$(}\ExtensionTok{fzf} \AttributeTok{{-}{-}multi} \AttributeTok{{-}{-}preview} \StringTok{\textquotesingle{}bat {-}{-}color=always \{\}\textquotesingle{}}\VariableTok{)} \KeywordTok{\&\&} \VariableTok{$\{EDITOR}\OperatorTok{:{-}}\NormalTok{vim}\VariableTok{\}} \StringTok{"}\VariableTok{$\{files}\OperatorTok{[@]}\VariableTok{\}}\StringTok{"}
\KeywordTok{\}}

\CommentTok{\# Kill process by name}
\FunctionTok{fkill()} \KeywordTok{\{}
    \BuiltInTok{local} \VariableTok{pid}
    \VariableTok{pid}\OperatorTok{=}\VariableTok{$(}\FunctionTok{ps} \AttributeTok{{-}ef} \KeywordTok{|} \FunctionTok{sed}\NormalTok{ 1d }\KeywordTok{|} \ExtensionTok{fzf} \AttributeTok{{-}m} \KeywordTok{|} \FunctionTok{awk} \StringTok{\textquotesingle{}\{print $2\}\textquotesingle{}}\VariableTok{)}
    \ControlFlowTok{if} \BuiltInTok{[} \StringTok{"x}\VariableTok{$pid}\StringTok{"} \OtherTok{!=} \StringTok{"x"} \BuiltInTok{]}\KeywordTok{;} \ControlFlowTok{then}
        \BuiltInTok{echo} \VariableTok{$pid} \KeywordTok{|} \FunctionTok{xargs}\NormalTok{ kill }\AttributeTok{{-}}\VariableTok{$\{1}\OperatorTok{:{-}}\NormalTok{9}\VariableTok{\}}
    \ControlFlowTok{fi}
\KeywordTok{\}}

\CommentTok{\# Git commit with conventional commits}
\FunctionTok{gci()} \KeywordTok{\{}
    \BuiltInTok{local} \VariableTok{type}\OperatorTok{=}\StringTok{"}\VariableTok{$1}\StringTok{"}
    \BuiltInTok{local} \VariableTok{scope}\OperatorTok{=}\StringTok{"}\VariableTok{$2}\StringTok{"}
    \BuiltInTok{local} \VariableTok{message}\OperatorTok{=}\StringTok{"}\VariableTok{$3}\StringTok{"}
    
    \ControlFlowTok{if} \BuiltInTok{[} \OtherTok{{-}z} \StringTok{"}\VariableTok{$type}\StringTok{"} \BuiltInTok{]} \KeywordTok{||} \BuiltInTok{[} \OtherTok{{-}z} \StringTok{"}\VariableTok{$message}\StringTok{"} \BuiltInTok{]}\KeywordTok{;} \ControlFlowTok{then}
        \BuiltInTok{echo} \StringTok{"Uso: gci \textless{}type\textgreater{} [scope] \textless{}message\textgreater{}"}
        \BuiltInTok{echo} \StringTok{"Types: feat, fix, docs, style, refactor, test, chore"}
        \ControlFlowTok{return} \DecValTok{1}
    \ControlFlowTok{fi}
    
    \ControlFlowTok{if} \BuiltInTok{[} \OtherTok{{-}n} \StringTok{"}\VariableTok{$scope}\StringTok{"} \BuiltInTok{]}\KeywordTok{;} \ControlFlowTok{then}
        \FunctionTok{git}\NormalTok{ commit }\AttributeTok{{-}m} \StringTok{"}\VariableTok{$\{type\}}\StringTok{(}\VariableTok{$\{scope\}}\StringTok{): }\VariableTok{$\{message\}}\StringTok{"}
    \ControlFlowTok{else}
        \FunctionTok{git}\NormalTok{ commit }\AttributeTok{{-}m} \StringTok{"}\VariableTok{$\{type\}}\StringTok{: }\VariableTok{$\{message\}}\StringTok{"}
    \ControlFlowTok{fi}
\KeywordTok{\}}

\CommentTok{\# Project initialization}
\FunctionTok{init\_project()} \KeywordTok{\{}
    \BuiltInTok{local} \VariableTok{project\_type}\OperatorTok{=}\StringTok{"}\VariableTok{$1}\StringTok{"}
    \BuiltInTok{local} \VariableTok{project\_name}\OperatorTok{=}\StringTok{"}\VariableTok{$2}\StringTok{"}
    
    \ControlFlowTok{if} \BuiltInTok{[} \OtherTok{{-}z} \StringTok{"}\VariableTok{$project\_type}\StringTok{"} \BuiltInTok{]} \KeywordTok{||} \BuiltInTok{[} \OtherTok{{-}z} \StringTok{"}\VariableTok{$project\_name}\StringTok{"} \BuiltInTok{]}\KeywordTok{;} \ControlFlowTok{then}
        \BuiltInTok{echo} \StringTok{"Uso: init\_project \textless{}type\textgreater{} \textless{}name\textgreater{}"}
        \BuiltInTok{echo} \StringTok{"Types: python, node, go, rust"}
        \ControlFlowTok{return} \DecValTok{1}
    \ControlFlowTok{fi}
    
    \FunctionTok{mkdir} \StringTok{"}\VariableTok{$project\_name}\StringTok{"} \KeywordTok{\&\&} \BuiltInTok{cd} \StringTok{"}\VariableTok{$project\_name}\StringTok{"}
    
    \ControlFlowTok{case} \StringTok{"}\VariableTok{$project\_type}\StringTok{"} \KeywordTok{in}
        \SpecialStringTok{python}\KeywordTok{)}
            \ExtensionTok{python3} \AttributeTok{{-}m}\NormalTok{ venv .venv}
            \BuiltInTok{echo} \StringTok{"source .venv/bin/activate"} \OperatorTok{\textgreater{}}\NormalTok{ .envrc}
            \BuiltInTok{echo} \StringTok{".venv/"} \OperatorTok{\textgreater{}}\NormalTok{ .gitignore}
            \BuiltInTok{echo} \StringTok{"*.pyc"} \OperatorTok{\textgreater{}\textgreater{}}\NormalTok{ .gitignore}
            \BuiltInTok{echo} \StringTok{"\_\_pycache\_\_/"} \OperatorTok{\textgreater{}\textgreater{}}\NormalTok{ .gitignore}
            \ExtensionTok{direnv}\NormalTok{ allow}
            \ControlFlowTok{;;}
        \SpecialStringTok{node}\KeywordTok{)}
            \ExtensionTok{npm}\NormalTok{ init }\AttributeTok{{-}y}
            \BuiltInTok{echo} \StringTok{"node\_modules/"} \OperatorTok{\textgreater{}}\NormalTok{ .gitignore}
            \BuiltInTok{echo} \StringTok{".env"} \OperatorTok{\textgreater{}\textgreater{}}\NormalTok{ .gitignore}
            \BuiltInTok{echo} \StringTok{"dist/"} \OperatorTok{\textgreater{}\textgreater{}}\NormalTok{ .gitignore}
            \ControlFlowTok{;;}
        \SpecialStringTok{go}\KeywordTok{)}
            \ExtensionTok{go}\NormalTok{ mod init }\StringTok{"}\VariableTok{$project\_name}\StringTok{"}
            \BuiltInTok{echo} \StringTok{"\# }\VariableTok{$project\_name}\StringTok{"} \OperatorTok{\textgreater{}}\NormalTok{ README.md}
            \ControlFlowTok{;;}
        \SpecialStringTok{rust}\KeywordTok{)}
            \ExtensionTok{cargo}\NormalTok{ init}
            \ControlFlowTok{;;}
    \ControlFlowTok{esac}
    
    \FunctionTok{git}\NormalTok{ init}
    \FunctionTok{touch}\NormalTok{ README.md}
    \FunctionTok{git}\NormalTok{ add .}
    \FunctionTok{git}\NormalTok{ commit }\AttributeTok{{-}m} \StringTok{"chore: initial commit"}
\KeywordTok{\}}

\CommentTok{\# Load local configuration if exists}
\BuiltInTok{[} \OtherTok{{-}f}\NormalTok{ \textasciitilde{}/.zshrc.local }\BuiltInTok{]} \KeywordTok{\&\&} \BuiltInTok{source}\NormalTok{ \textasciitilde{}/.zshrc.local}
\end{Highlighting}
\end{Shaded}

\subsection{Configuración de
Starship}\label{configuraciuxf3n-de-starship}

Archivo \texttt{\textasciitilde{}/.config/starship.toml}:

\begin{Shaded}
\begin{Highlighting}[]
\CommentTok{\# Starship configuration}

\DataTypeTok{format} \OperatorTok{=} \StringTok{"""}
\StringTok{$username}\SpecialCharTok{\textbackslash{}}
\StringTok{$hostname}\SpecialCharTok{\textbackslash{}}
\StringTok{$directory}\SpecialCharTok{\textbackslash{}}
\StringTok{$git\_branch}\SpecialCharTok{\textbackslash{}}
\StringTok{$git\_state}\SpecialCharTok{\textbackslash{}}
\StringTok{$git\_status}\SpecialCharTok{\textbackslash{}}
\StringTok{$git\_metrics}\SpecialCharTok{\textbackslash{}}
\StringTok{$fill}\SpecialCharTok{\textbackslash{}}
\StringTok{$nodejs}\SpecialCharTok{\textbackslash{}}
\StringTok{$python}\SpecialCharTok{\textbackslash{}}
\StringTok{$rust}\SpecialCharTok{\textbackslash{}}
\StringTok{$golang}\SpecialCharTok{\textbackslash{}}
\StringTok{$package}\SpecialCharTok{\textbackslash{}}
\StringTok{$docker\_context}\SpecialCharTok{\textbackslash{}}
\StringTok{$time}\SpecialCharTok{\textbackslash{}}
\StringTok{$line\_break}\SpecialCharTok{\textbackslash{}}
\StringTok{$character"""}

\KeywordTok{[fill]}
\DataTypeTok{symbol} \OperatorTok{=} \StringTok{" "}

\KeywordTok{[directory]}
\DataTypeTok{style} \OperatorTok{=} \StringTok{"blue"}
\DataTypeTok{truncation\_length} \OperatorTok{=} \DecValTok{4}
\DataTypeTok{truncation\_symbol} \OperatorTok{=} \StringTok{"…/"}

\KeywordTok{[character]}
\DataTypeTok{success\_symbol} \OperatorTok{=} \StringTok{"[❯](purple)"}
\DataTypeTok{error\_symbol} \OperatorTok{=} \StringTok{"[❯](red)"}
\DataTypeTok{vicmd\_symbol} \OperatorTok{=} \StringTok{"[❮](green)"}

\KeywordTok{[git\_branch]}
\DataTypeTok{symbol} \OperatorTok{=} \StringTok{"🌱 "}
\DataTypeTok{truncation\_length} \OperatorTok{=} \DecValTok{15}
\DataTypeTok{truncation\_symbol} \OperatorTok{=} \StringTok{"…"}

\KeywordTok{[git\_status]}
\DataTypeTok{ahead} \OperatorTok{=} \StringTok{"⇡$\{count\}"}
\DataTypeTok{diverged} \OperatorTok{=} \StringTok{"⇕⇡$\{ahead\_count\}⇣$\{behind\_count\}"}
\DataTypeTok{behind} \OperatorTok{=} \StringTok{"⇣$\{count\}"}
\DataTypeTok{conflicted} \OperatorTok{=} \StringTok{"🏳"}
\DataTypeTok{untracked} \OperatorTok{=} \StringTok{"🤷"}
\DataTypeTok{stashed} \OperatorTok{=} \StringTok{"📦"}
\DataTypeTok{modified} \OperatorTok{=} \StringTok{"📝"}
\DataTypeTok{staged} \OperatorTok{=} \StringTok{\textquotesingle{}}\VerbatimStringTok{[++\textbackslash{}($count\textbackslash{})](green)}\StringTok{\textquotesingle{}}
\DataTypeTok{renamed} \OperatorTok{=} \StringTok{"👅"}
\DataTypeTok{deleted} \OperatorTok{=} \StringTok{"🗑"}

\KeywordTok{[nodejs]}
\DataTypeTok{symbol} \OperatorTok{=} \StringTok{"⬢ "}
\DataTypeTok{detect\_files} \OperatorTok{=} \OperatorTok{[}\StringTok{"package.json"}\OperatorTok{,} \StringTok{".nvmrc"}\OperatorTok{]}
\DataTypeTok{detect\_folders} \OperatorTok{=} \OperatorTok{[}\StringTok{"node\_modules"}\OperatorTok{]}

\KeywordTok{[python]}
\DataTypeTok{symbol} \OperatorTok{=} \StringTok{"🐍 "}
\DataTypeTok{detect\_extensions} \OperatorTok{=} \OperatorTok{[}\StringTok{"py"}\OperatorTok{]}
\DataTypeTok{detect\_files} \OperatorTok{=} \OperatorTok{[}\StringTok{"requirements.txt"}\OperatorTok{,} \StringTok{".python{-}version"}\OperatorTok{,} \StringTok{"pyproject.toml"}\OperatorTok{]}

\KeywordTok{[rust]}
\DataTypeTok{symbol} \OperatorTok{=} \StringTok{"🦀 "}
\DataTypeTok{detect\_extensions} \OperatorTok{=} \OperatorTok{[}\StringTok{"rs"}\OperatorTok{]}
\DataTypeTok{detect\_files} \OperatorTok{=} \OperatorTok{[}\StringTok{"Cargo.toml"}\OperatorTok{]}

\KeywordTok{[golang]}
\DataTypeTok{symbol} \OperatorTok{=} \StringTok{"🐹 "}
\DataTypeTok{detect\_extensions} \OperatorTok{=} \OperatorTok{[}\StringTok{"go"}\OperatorTok{]}
\DataTypeTok{detect\_files} \OperatorTok{=} \OperatorTok{[}\StringTok{"go.mod"}\OperatorTok{,} \StringTok{"go.sum"}\OperatorTok{,} \StringTok{"glide.yaml"}\OperatorTok{]}

\KeywordTok{[docker\_context]}
\DataTypeTok{symbol} \OperatorTok{=} \StringTok{"🐳 "}
\DataTypeTok{detect\_files} \OperatorTok{=} \OperatorTok{[}\StringTok{"docker{-}compose.yml"}\OperatorTok{,} \StringTok{"docker{-}compose.yaml"}\OperatorTok{,} \StringTok{"Dockerfile"}\OperatorTok{]}

\KeywordTok{[time]}
\DataTypeTok{disabled} \OperatorTok{=} \ConstantTok{false}
\DataTypeTok{format} \OperatorTok{=} \StringTok{\textquotesingle{}}\VerbatimStringTok{🕙[\textbackslash{}[ $time \textbackslash{}]]($style) }\StringTok{\textquotesingle{}}
\DataTypeTok{time\_format} \OperatorTok{=} \StringTok{"\%T"}
\DataTypeTok{utc\_time\_offset} \OperatorTok{=} \StringTok{"local"}

\KeywordTok{[package]}
\DataTypeTok{symbol} \OperatorTok{=} \StringTok{"📦 "}
\end{Highlighting}
\end{Shaded}

\section{Configuración de herramientas
específicas}\label{configuraciuxf3n-de-herramientas-especuxedficas}

\subsection{Git configuración
avanzada}\label{git-configuraciuxf3n-avanzada}

\begin{Shaded}
\begin{Highlighting}[]
\CommentTok{\#!/bin/bash}
\CommentTok{\# setup{-}git.sh {-} Configuración avanzada de Git}

\FunctionTok{setup\_git\_config()} \KeywordTok{\{}
    \BuiltInTok{echo} \StringTok{"🔧 Configurando Git..."}
    
    \CommentTok{\# Configuración básica}
    \FunctionTok{git}\NormalTok{ config }\AttributeTok{{-}{-}global}\NormalTok{ init.defaultBranch main}
    \FunctionTok{git}\NormalTok{ config }\AttributeTok{{-}{-}global}\NormalTok{ pull.rebase false}
    \FunctionTok{git}\NormalTok{ config }\AttributeTok{{-}{-}global}\NormalTok{ push.default simple}
    \FunctionTok{git}\NormalTok{ config }\AttributeTok{{-}{-}global}\NormalTok{ core.autocrlf input}
    \FunctionTok{git}\NormalTok{ config }\AttributeTok{{-}{-}global}\NormalTok{ core.editor }\StringTok{"code {-}{-}wait"}
    
    \CommentTok{\# Aliases útiles}
    \FunctionTok{git}\NormalTok{ config }\AttributeTok{{-}{-}global}\NormalTok{ alias.st status}
    \FunctionTok{git}\NormalTok{ config }\AttributeTok{{-}{-}global}\NormalTok{ alias.co checkout}
    \FunctionTok{git}\NormalTok{ config }\AttributeTok{{-}{-}global}\NormalTok{ alias.br branch}
    \FunctionTok{git}\NormalTok{ config }\AttributeTok{{-}{-}global}\NormalTok{ alias.ci commit}
    \FunctionTok{git}\NormalTok{ config }\AttributeTok{{-}{-}global}\NormalTok{ alias.unstage }\StringTok{\textquotesingle{}reset HEAD {-}{-}\textquotesingle{}}
    \FunctionTok{git}\NormalTok{ config }\AttributeTok{{-}{-}global}\NormalTok{ alias.last }\StringTok{\textquotesingle{}log {-}1 HEAD\textquotesingle{}}
    \FunctionTok{git}\NormalTok{ config }\AttributeTok{{-}{-}global}\NormalTok{ alias.visual }\StringTok{\textquotesingle{}!gitk\textquotesingle{}}
    \FunctionTok{git}\NormalTok{ config }\AttributeTok{{-}{-}global}\NormalTok{ alias.graph }\StringTok{\textquotesingle{}log {-}{-}graph {-}{-}pretty=format:"\%h {-}\%d \%s (\%cr) \textless{}\%an\textgreater{}" {-}{-}abbrev{-}commit\textquotesingle{}}
    \FunctionTok{git}\NormalTok{ config }\AttributeTok{{-}{-}global}\NormalTok{ alias.conflicts }\StringTok{\textquotesingle{}diff {-}{-}name{-}only {-}{-}diff{-}filter=U\textquotesingle{}}
    
    \CommentTok{\# Configuración de merge}
    \FunctionTok{git}\NormalTok{ config }\AttributeTok{{-}{-}global}\NormalTok{ merge.tool vscode}
    \FunctionTok{git}\NormalTok{ config }\AttributeTok{{-}{-}global}\NormalTok{ mergetool.vscode.cmd }\StringTok{\textquotesingle{}code {-}{-}wait $MERGED\textquotesingle{}}
    \FunctionTok{git}\NormalTok{ config }\AttributeTok{{-}{-}global}\NormalTok{ diff.tool vscode}
    \FunctionTok{git}\NormalTok{ config }\AttributeTok{{-}{-}global}\NormalTok{ difftool.vscode.cmd }\StringTok{\textquotesingle{}code {-}{-}wait {-}{-}diff $LOCAL $REMOTE\textquotesingle{}}
    
    \CommentTok{\# Configuración de colores}
    \FunctionTok{git}\NormalTok{ config }\AttributeTok{{-}{-}global}\NormalTok{ color.ui auto}
    \FunctionTok{git}\NormalTok{ config }\AttributeTok{{-}{-}global}\NormalTok{ color.branch.current }\StringTok{"yellow reverse"}
    \FunctionTok{git}\NormalTok{ config }\AttributeTok{{-}{-}global}\NormalTok{ color.branch.local yellow}
    \FunctionTok{git}\NormalTok{ config }\AttributeTok{{-}{-}global}\NormalTok{ color.branch.remote green}
    \FunctionTok{git}\NormalTok{ config }\AttributeTok{{-}{-}global}\NormalTok{ color.diff.meta }\StringTok{"yellow bold"}
    \FunctionTok{git}\NormalTok{ config }\AttributeTok{{-}{-}global}\NormalTok{ color.diff.frag }\StringTok{"magenta bold"}
    \FunctionTok{git}\NormalTok{ config }\AttributeTok{{-}{-}global}\NormalTok{ color.diff.old }\StringTok{"red bold"}
    \FunctionTok{git}\NormalTok{ config }\AttributeTok{{-}{-}global}\NormalTok{ color.diff.new }\StringTok{"green bold"}
    \FunctionTok{git}\NormalTok{ config }\AttributeTok{{-}{-}global}\NormalTok{ color.status.added yellow}
    \FunctionTok{git}\NormalTok{ config }\AttributeTok{{-}{-}global}\NormalTok{ color.status.changed green}
    \FunctionTok{git}\NormalTok{ config }\AttributeTok{{-}{-}global}\NormalTok{ color.status.untracked cyan}
\KeywordTok{\}}

\FunctionTok{setup\_git\_hooks()} \KeywordTok{\{}
    \BuiltInTok{echo} \StringTok{"🪝 Configurando Git hooks..."}
    
    \CommentTok{\# Hook de pre{-}commit}
    \FunctionTok{cat} \OperatorTok{\textgreater{}}\NormalTok{ .git/hooks/pre{-}commit }\OperatorTok{\textless{}\textless{} \textquotesingle{}EOF\textquotesingle{}}
\StringTok{\#!/bin/bash}
\StringTok{\# Pre{-}commit hook}

\StringTok{\# Verificar que no hay debuggers}
\StringTok{if grep {-}r "debugger\textbackslash{}|console\textbackslash{}.log\textbackslash{}|pdb\textbackslash{}.set\_trace" {-}{-}include="*.js" {-}{-}include="*.py" .; then}
\StringTok{    echo "❌ Debuggers encontrados. Remueve antes de commitear."}
\StringTok{    exit 1}
\StringTok{fi}

\StringTok{\# Ejecutar linting si existe}
\StringTok{if [ {-}f "package.json" ]; then}
\StringTok{    npm run lint 2\textgreater{}/dev/null || true}
\StringTok{fi}

\StringTok{\# Ejecutar tests si existen}
\StringTok{if [ {-}f "package.json" ]; then}
\StringTok{    npm test 2\textgreater{}/dev/null || true}
\StringTok{fi}
\OperatorTok{EOF}

    \FunctionTok{chmod}\NormalTok{ +x .git/hooks/pre{-}commit}
\KeywordTok{\}}

\ExtensionTok{setup\_git\_config}
\BuiltInTok{[} \OtherTok{{-}d} \StringTok{".git"} \BuiltInTok{]} \KeywordTok{\&\&} \ExtensionTok{setup\_git\_hooks}
\end{Highlighting}
\end{Shaded}

\subsection{Configuración de herramientas de
desarrollo}\label{configuraciuxf3n-de-herramientas-de-desarrollo}

\begin{Shaded}
\begin{Highlighting}[]
\CommentTok{\#!/bin/bash}
\CommentTok{\# setup{-}dev{-}tools.sh}

\FunctionTok{setup\_bat()} \KeywordTok{\{}
    \BuiltInTok{echo} \StringTok{"🦇 Configurando bat..."}
    
    \FunctionTok{mkdir} \AttributeTok{{-}p}\NormalTok{ \textasciitilde{}/.config/bat}
    \FunctionTok{cat} \OperatorTok{\textgreater{}}\NormalTok{ \textasciitilde{}/.config/bat/config }\OperatorTok{\textless{}\textless{} \textquotesingle{}EOF\textquotesingle{}}
\StringTok{{-}{-}theme="gruvbox{-}dark"}
\StringTok{{-}{-}style="numbers,changes,header"}
\StringTok{{-}{-}paging=auto}
\StringTok{{-}{-}wrap=never}
\OperatorTok{EOF}
\KeywordTok{\}}

\FunctionTok{setup\_ripgrep()} \KeywordTok{\{}
    \BuiltInTok{echo} \StringTok{"🔍 Configurando ripgrep..."}
    
    \FunctionTok{cat} \OperatorTok{\textgreater{}}\NormalTok{ \textasciitilde{}/.ripgreprc }\OperatorTok{\textless{}\textless{} \textquotesingle{}EOF\textquotesingle{}}
\StringTok{{-}{-}max{-}columns=150}
\StringTok{{-}{-}max{-}columns{-}preview}
\StringTok{{-}{-}smart{-}case}
\StringTok{{-}{-}hidden}
\StringTok{{-}{-}glob=!.git/*}
\StringTok{{-}{-}glob=!node\_modules/*}
\StringTok{{-}{-}glob=!dist/*}
\StringTok{{-}{-}glob=!build/*}
\StringTok{{-}{-}colors=line:none}
\StringTok{{-}{-}colors=line:style:bold}
\OperatorTok{EOF}
    
    \BuiltInTok{export} \VariableTok{RIPGREP\_CONFIG\_PATH}\OperatorTok{=}\StringTok{"}\VariableTok{$HOME}\StringTok{/.ripgreprc"}
\KeywordTok{\}}

\FunctionTok{setup\_fzf()} \KeywordTok{\{}
    \BuiltInTok{echo} \StringTok{"🔭 Configurando fzf..."}
    
    \BuiltInTok{export} \VariableTok{FZF\_DEFAULT\_COMMAND}\OperatorTok{=}\StringTok{\textquotesingle{}rg {-}{-}files {-}{-}hidden {-}{-}follow {-}{-}glob "!.git/*"\textquotesingle{}}
    \BuiltInTok{export} \VariableTok{FZF\_DEFAULT\_OPTS}\OperatorTok{=}\StringTok{\textquotesingle{}}
\StringTok{        {-}{-}height 40\% }
\StringTok{        {-}{-}layout=reverse }
\StringTok{        {-}{-}border}
\StringTok{        {-}{-}preview "bat {-}{-}color=always {-}{-}style=header,grid {-}{-}line{-}range :300 \{\}"}
\StringTok{        {-}{-}bind "ctrl{-}/:change{-}preview{-}window(down|hidden|)"}
\StringTok{        {-}{-}bind "ctrl{-}y:execute{-}silent(echo \{\} | pbcopy)+abort"}
\StringTok{        {-}{-}color=bg+:\#414559,bg:\#303446,spinner:\#f2d5cf,hl:\#e78284}
\StringTok{        {-}{-}color=fg:\#c6d0f5,header:\#e78284,info:\#ca9ee6,pointer:\#f2d5cf}
\StringTok{        {-}{-}color=marker:\#f2d5cf,fg+:\#c6d0f5,prompt:\#ca9ee6,hl+:\#e78284}
\StringTok{    \textquotesingle{}}
\KeywordTok{\}}

\FunctionTok{setup\_all()} \KeywordTok{\{}
    \ExtensionTok{setup\_bat}
    \ExtensionTok{setup\_ripgrep}
    \ExtensionTok{setup\_fzf}
    
    \BuiltInTok{echo} \StringTok{"✅ Configuración de herramientas completada"}
\KeywordTok{\}}

\ExtensionTok{setup\_all}
\end{Highlighting}
\end{Shaded}

\section{Scripts de productividad}\label{scripts-de-productividad}

\subsection{Sistema de templates de
proyecto}\label{sistema-de-templates-de-proyecto}

\begin{Shaded}
\begin{Highlighting}[]
\CommentTok{\#!/bin/bash}
\CommentTok{\# project{-}templates.sh}

\VariableTok{TEMPLATES\_DIR}\OperatorTok{=}\StringTok{"}\VariableTok{$HOME}\StringTok{/.project{-}templates"}

\FunctionTok{create\_templates()} \KeywordTok{\{}
    \FunctionTok{mkdir} \AttributeTok{{-}p} \StringTok{"}\VariableTok{$TEMPLATES\_DIR}\StringTok{"}
    
    \CommentTok{\# Template Python}
    \FunctionTok{mkdir} \AttributeTok{{-}p} \StringTok{"}\VariableTok{$TEMPLATES\_DIR}\StringTok{/python"}
    \FunctionTok{cat} \OperatorTok{\textgreater{}} \StringTok{"}\VariableTok{$TEMPLATES\_DIR}\StringTok{/python/.envrc"} \OperatorTok{\textless{}\textless{} \textquotesingle{}EOF\textquotesingle{}}
\StringTok{export PYTHONPATH="$PWD/src:$PYTHONPATH"}
\StringTok{export VIRTUAL\_ENV="$PWD/.venv"}
\StringTok{export PATH="$VIRTUAL\_ENV/bin:$PATH"}

\StringTok{if [ {-}d "$VIRTUAL\_ENV" ]; then}
\StringTok{    source "$VIRTUAL\_ENV/bin/activate"}
\StringTok{fi}
\OperatorTok{EOF}

    \FunctionTok{cat} \OperatorTok{\textgreater{}} \StringTok{"}\VariableTok{$TEMPLATES\_DIR}\StringTok{/python/pyproject.toml"} \OperatorTok{\textless{}\textless{} \textquotesingle{}EOF\textquotesingle{}}
\StringTok{[build{-}system]}
\StringTok{requires = ["hatchling"]}
\StringTok{build{-}backend = "hatchling.build"}

\StringTok{[project]}
\StringTok{name = "PROJECT\_NAME"}
\StringTok{version = "0.1.0"}
\StringTok{description = ""}
\StringTok{authors = [\{name = "Your Name", email = "your@email.com"\}]}
\StringTok{license = \{text = "MIT"\}}
\StringTok{readme = "README.md"}
\StringTok{requires{-}python = "\textgreater{}=3.8"}
\StringTok{dependencies = []}

\StringTok{[project.optional{-}dependencies]}
\StringTok{dev = [}
\StringTok{    "pytest",}
\StringTok{    "black",}
\StringTok{    "isort",}
\StringTok{    "flake8",}
\StringTok{    "mypy"}
\StringTok{]}

\StringTok{[tool.black]}
\StringTok{line{-}length = 88}
\StringTok{target{-}version = [\textquotesingle{}py38\textquotesingle{}]}

\StringTok{[tool.isort]}
\StringTok{profile = "black"}
\StringTok{line\_length = 88}
\OperatorTok{EOF}

    \CommentTok{\# Template Node.js}
    \FunctionTok{mkdir} \AttributeTok{{-}p} \StringTok{"}\VariableTok{$TEMPLATES\_DIR}\StringTok{/nodejs"}
    \FunctionTok{cat} \OperatorTok{\textgreater{}} \StringTok{"}\VariableTok{$TEMPLATES\_DIR}\StringTok{/nodejs/.envrc"} \OperatorTok{\textless{}\textless{} \textquotesingle{}EOF\textquotesingle{}}
\StringTok{export NODE\_ENV=development}
\StringTok{export PATH="$PWD/node\_modules/.bin:$PATH"}
\OperatorTok{EOF}

    \FunctionTok{cat} \OperatorTok{\textgreater{}} \StringTok{"}\VariableTok{$TEMPLATES\_DIR}\StringTok{/nodejs/package.json"} \OperatorTok{\textless{}\textless{} \textquotesingle{}EOF\textquotesingle{}}
\StringTok{\{}
\StringTok{  "name": "PROJECT\_NAME",}
\StringTok{  "version": "1.0.0",}
\StringTok{  "description": "",}
\StringTok{  "main": "index.js",}
\StringTok{  "scripts": \{}
\StringTok{    "start": "node index.js",}
\StringTok{    "dev": "nodemon index.js",}
\StringTok{    "test": "jest",}
\StringTok{    "lint": "eslint .",}
\StringTok{    "lint:fix": "eslint . {-}{-}fix",}
\StringTok{    "format": "prettier {-}{-}write ."}
\StringTok{  \},}
\StringTok{  "keywords": [],}
\StringTok{  "author": "Your Name",}
\StringTok{  "license": "MIT",}
\StringTok{  "devDependencies": \{}
\StringTok{    "eslint": "\^{}8.0.0",}
\StringTok{    "prettier": "\^{}2.0.0",}
\StringTok{    "nodemon": "\^{}2.0.0",}
\StringTok{    "jest": "\^{}28.0.0"}
\StringTok{  \}}
\StringTok{\}}
\OperatorTok{EOF}

    \BuiltInTok{echo} \StringTok{"📁 Templates creados en }\VariableTok{$TEMPLATES\_DIR}\StringTok{"}
\KeywordTok{\}}

\FunctionTok{new\_project()} \KeywordTok{\{}
    \BuiltInTok{local} \VariableTok{template}\OperatorTok{=}\StringTok{"}\VariableTok{$1}\StringTok{"}
    \BuiltInTok{local} \VariableTok{name}\OperatorTok{=}\StringTok{"}\VariableTok{$2}\StringTok{"}
    
    \ControlFlowTok{if} \BuiltInTok{[} \OtherTok{{-}z} \StringTok{"}\VariableTok{$template}\StringTok{"} \BuiltInTok{]} \KeywordTok{||} \BuiltInTok{[} \OtherTok{{-}z} \StringTok{"}\VariableTok{$name}\StringTok{"} \BuiltInTok{]}\KeywordTok{;} \ControlFlowTok{then}
        \BuiltInTok{echo} \StringTok{"Uso: new\_project \textless{}template\textgreater{} \textless{}name\textgreater{}"}
        \BuiltInTok{echo} \StringTok{"Templates disponibles:"}
        \FunctionTok{ls} \StringTok{"}\VariableTok{$TEMPLATES\_DIR}\StringTok{"} \DecValTok{2}\OperatorTok{\textgreater{}}\NormalTok{/dev/null }\KeywordTok{||} \BuiltInTok{echo} \StringTok{"No hay templates (ejecuta: create\_templates)"}
        \ControlFlowTok{return} \DecValTok{1}
    \ControlFlowTok{fi}
    
    \ControlFlowTok{if} \BuiltInTok{[} \OtherTok{!} \OtherTok{{-}d} \StringTok{"}\VariableTok{$TEMPLATES\_DIR}\StringTok{/}\VariableTok{$template}\StringTok{"} \BuiltInTok{]}\KeywordTok{;} \ControlFlowTok{then}
        \BuiltInTok{echo} \StringTok{"❌ Template \textquotesingle{}}\VariableTok{$template}\StringTok{\textquotesingle{} no encontrado"}
        \ControlFlowTok{return} \DecValTok{1}
    \ControlFlowTok{fi}
    
    \ControlFlowTok{if} \BuiltInTok{[} \OtherTok{{-}d} \StringTok{"}\VariableTok{$name}\StringTok{"} \BuiltInTok{]}\KeywordTok{;} \ControlFlowTok{then}
        \BuiltInTok{echo} \StringTok{"❌ Directorio \textquotesingle{}}\VariableTok{$name}\StringTok{\textquotesingle{} ya existe"}
        \ControlFlowTok{return} \DecValTok{1}
    \ControlFlowTok{fi}
    
    \BuiltInTok{echo} \StringTok{"🚀 Creando proyecto \textquotesingle{}}\VariableTok{$name}\StringTok{\textquotesingle{} desde template \textquotesingle{}}\VariableTok{$template}\StringTok{\textquotesingle{}..."}
    
    \CommentTok{\# Copiar template}
    \FunctionTok{cp} \AttributeTok{{-}r} \StringTok{"}\VariableTok{$TEMPLATES\_DIR}\StringTok{/}\VariableTok{$template}\StringTok{"} \StringTok{"}\VariableTok{$name}\StringTok{"}
    \BuiltInTok{cd} \StringTok{"}\VariableTok{$name}\StringTok{"}
    
    \CommentTok{\# Reemplazar PROJECT\_NAME en archivos}
    \ControlFlowTok{if} \BuiltInTok{command} \AttributeTok{{-}v}\NormalTok{ sed }\OperatorTok{\textgreater{}}\NormalTok{/dev/null}\KeywordTok{;} \ControlFlowTok{then}
        \FunctionTok{find}\NormalTok{ . }\AttributeTok{{-}type}\NormalTok{ f }\AttributeTok{{-}name} \StringTok{"*.json"} \AttributeTok{{-}o} \AttributeTok{{-}name} \StringTok{"*.toml"} \AttributeTok{{-}o} \AttributeTok{{-}name} \StringTok{"*.md"} \KeywordTok{|} \DataTypeTok{\textbackslash{}}
        \FunctionTok{xargs}\NormalTok{ sed }\AttributeTok{{-}i.bak} \StringTok{"s/PROJECT\_NAME/}\VariableTok{$name}\StringTok{/g"}
        \FunctionTok{find}\NormalTok{ . }\AttributeTok{{-}name} \StringTok{"*.bak"} \AttributeTok{{-}delete}
    \ControlFlowTok{fi}
    
    \CommentTok{\# Inicializar git}
    \FunctionTok{git}\NormalTok{ init}
    \FunctionTok{git}\NormalTok{ add .}
    \FunctionTok{git}\NormalTok{ commit }\AttributeTok{{-}m} \StringTok{"chore: initial commit from template"}
    
    \CommentTok{\# Configurar según tipo de proyecto}
    \ControlFlowTok{case} \StringTok{"}\VariableTok{$template}\StringTok{"} \KeywordTok{in}
        \SpecialStringTok{python}\KeywordTok{)}
            \ExtensionTok{python3} \AttributeTok{{-}m}\NormalTok{ venv .venv}
            \ExtensionTok{direnv}\NormalTok{ allow}
            \ControlFlowTok{;;}
        \SpecialStringTok{nodejs}\KeywordTok{)}
            \ExtensionTok{npm}\NormalTok{ install}
            \ExtensionTok{direnv}\NormalTok{ allow}
            \ControlFlowTok{;;}
    \ControlFlowTok{esac}
    
    \BuiltInTok{echo} \StringTok{"✅ Proyecto \textquotesingle{}}\VariableTok{$name}\StringTok{\textquotesingle{} creado exitosamente"}
    \BuiltInTok{echo} \StringTok{"📁 Ubicación: }\VariableTok{$(}\BuiltInTok{pwd}\VariableTok{)}\StringTok{"}
\KeywordTok{\}}

\ControlFlowTok{case} \StringTok{"}\VariableTok{$\{1}\OperatorTok{:{-}}\NormalTok{help}\VariableTok{\}}\StringTok{"} \KeywordTok{in}
    \SpecialStringTok{create{-}templates}\KeywordTok{)}
        \ExtensionTok{create\_templates}
        \ControlFlowTok{;;}
    \SpecialStringTok{new}\KeywordTok{)}
        \ExtensionTok{new\_project} \StringTok{"}\VariableTok{$2}\StringTok{"} \StringTok{"}\VariableTok{$3}\StringTok{"}
        \ControlFlowTok{;;}
    \SpecialStringTok{list}\KeywordTok{)}
        \BuiltInTok{echo} \StringTok{"Templates disponibles:"}
        \FunctionTok{ls} \StringTok{"}\VariableTok{$TEMPLATES\_DIR}\StringTok{"} \DecValTok{2}\OperatorTok{\textgreater{}}\NormalTok{/dev/null }\KeywordTok{||} \BuiltInTok{echo} \StringTok{"No hay templates"}
        \ControlFlowTok{;;}
    \PreprocessorTok{*}\KeywordTok{)}
        \BuiltInTok{echo} \StringTok{"Project Templates Manager"}
        \BuiltInTok{echo} \StringTok{"Uso: }\VariableTok{$0}\StringTok{ \textless{}comando\textgreater{} [argumentos]"}
        \BuiltInTok{echo} \StringTok{""}
        \BuiltInTok{echo} \StringTok{"Comandos:"}
        \BuiltInTok{echo} \StringTok{"  create{-}templates     {-} Crear templates básicos"}
        \BuiltInTok{echo} \StringTok{"  new \textless{}template\textgreater{} \textless{}name\textgreater{} {-} Crear proyecto desde template"}
        \BuiltInTok{echo} \StringTok{"  list                 {-} Listar templates disponibles"}
        \ControlFlowTok{;;}
\ControlFlowTok{esac}
\end{Highlighting}
\end{Shaded}

\subsection{Backup automatizado}\label{backup-automatizado}

\begin{Shaded}
\begin{Highlighting}[]
\CommentTok{\#!/bin/bash}
\CommentTok{\# auto{-}backup.sh {-} Sistema de backup automatizado}

\VariableTok{BACKUP\_CONFIG}\OperatorTok{=}\StringTok{"}\VariableTok{$HOME}\StringTok{/.config/auto{-}backup/config.json"}
\VariableTok{BACKUP\_LOG}\OperatorTok{=}\StringTok{"}\VariableTok{$HOME}\StringTok{/.local/log/backup.log"}

\FunctionTok{init\_backup()} \KeywordTok{\{}
    \FunctionTok{mkdir} \AttributeTok{{-}p} \StringTok{"}\VariableTok{$(}\FunctionTok{dirname} \StringTok{"}\VariableTok{$BACKUP\_CONFIG}\StringTok{"}\VariableTok{)}\StringTok{"}
    \FunctionTok{mkdir} \AttributeTok{{-}p} \StringTok{"}\VariableTok{$(}\FunctionTok{dirname} \StringTok{"}\VariableTok{$BACKUP\_LOG}\StringTok{"}\VariableTok{)}\StringTok{"}
    
    \FunctionTok{cat} \OperatorTok{\textgreater{}} \StringTok{"}\VariableTok{$BACKUP\_CONFIG}\StringTok{"} \OperatorTok{\textless{}\textless{} \textquotesingle{}EOF\textquotesingle{}}
\StringTok{\{}
\StringTok{    "directories": [}
\StringTok{        "\textasciitilde{}/Documents",}
\StringTok{        "\textasciitilde{}/Projects",}
\StringTok{        "\textasciitilde{}/.config"}
\StringTok{    ],}
\StringTok{    "exclude": [}
\StringTok{        "*.tmp",}
\StringTok{        "*.log",}
\StringTok{        "node\_modules",}
\StringTok{        ".git",}
\StringTok{        "dist",}
\StringTok{        "build"}
\StringTok{    ],}
\StringTok{    "destinations": \{}
\StringTok{        "local": "\textasciitilde{}/Backups",}
\StringTok{        "remote": "user@server.com:/backups"}
\StringTok{    \},}
\StringTok{    "retention": \{}
\StringTok{        "daily": 7,}
\StringTok{        "weekly": 4,}
\StringTok{        "monthly": 12}
\StringTok{    \}}
\StringTok{\}}
\OperatorTok{EOF}
    
    \BuiltInTok{echo} \StringTok{"Configuración de backup creada en: }\VariableTok{$BACKUP\_CONFIG}\StringTok{"}
\KeywordTok{\}}

\FunctionTok{perform\_backup()} \KeywordTok{\{}
    \BuiltInTok{local} \VariableTok{destination\_type}\OperatorTok{=}\StringTok{"}\VariableTok{$\{1}\OperatorTok{:{-}}\NormalTok{local}\VariableTok{\}}\StringTok{"}
    \BuiltInTok{local} \VariableTok{timestamp}\OperatorTok{=}\VariableTok{$(}\FunctionTok{date}\NormalTok{ +\%Y\%m\%d\_\%H\%M\%S}\VariableTok{)}
    
    \FunctionTok{log\_message()} \KeywordTok{\{}
        \BuiltInTok{echo} \StringTok{"[}\VariableTok{$(}\FunctionTok{date}\VariableTok{)}\StringTok{] }\VariableTok{$1}\StringTok{"} \KeywordTok{|} \FunctionTok{tee} \AttributeTok{{-}a} \StringTok{"}\VariableTok{$BACKUP\_LOG}\StringTok{"}
    \KeywordTok{\}}
    
    \ExtensionTok{log\_message} \StringTok{"Iniciando backup (}\VariableTok{$destination\_type}\StringTok{)..."}
    
    \CommentTok{\# Leer configuración}
    \BuiltInTok{local} \VariableTok{directories}\OperatorTok{=}\VariableTok{$(}\ExtensionTok{jq} \AttributeTok{{-}r} \StringTok{\textquotesingle{}.directories[]\textquotesingle{}} \StringTok{"}\VariableTok{$BACKUP\_CONFIG}\StringTok{"}\VariableTok{)}
    \BuiltInTok{local} \VariableTok{excludes}\OperatorTok{=}\VariableTok{$(}\ExtensionTok{jq} \AttributeTok{{-}r} \StringTok{\textquotesingle{}.exclude[]\textquotesingle{}} \StringTok{"}\VariableTok{$BACKUP\_CONFIG}\StringTok{"} \KeywordTok{|} \FunctionTok{sed} \StringTok{\textquotesingle{}s/\^{}/{-}{-}exclude=/\textquotesingle{}}\VariableTok{)}
    \BuiltInTok{local} \VariableTok{destination}\OperatorTok{=}\VariableTok{$(}\ExtensionTok{jq} \AttributeTok{{-}r} \StringTok{".destinations.}\VariableTok{$destination\_type}\StringTok{"} \StringTok{"}\VariableTok{$BACKUP\_CONFIG}\StringTok{"}\VariableTok{)}
    
    \CommentTok{\# Expandir tilde en destination}
    \VariableTok{destination}\OperatorTok{=}\VariableTok{$(}\BuiltInTok{eval}\NormalTok{ echo }\StringTok{"}\VariableTok{$destination}\StringTok{"}\VariableTok{)}
    
    \CommentTok{\# Crear directorio de backup}
    \BuiltInTok{local} \VariableTok{backup\_dir}\OperatorTok{=}\StringTok{"}\VariableTok{$destination}\StringTok{/backup\_}\VariableTok{$timestamp}\StringTok{"}
    \FunctionTok{mkdir} \AttributeTok{{-}p} \StringTok{"}\VariableTok{$backup\_dir}\StringTok{"}
    
    \CommentTok{\# Realizar backup de cada directorio}
    \BuiltInTok{echo} \StringTok{"}\VariableTok{$directories}\StringTok{"} \KeywordTok{|} \ControlFlowTok{while} \BuiltInTok{read} \AttributeTok{{-}r} \VariableTok{dir}\KeywordTok{;} \ControlFlowTok{do}
        \ControlFlowTok{if} \BuiltInTok{[} \OtherTok{{-}n} \StringTok{"}\VariableTok{$dir}\StringTok{"} \BuiltInTok{]}\KeywordTok{;} \ControlFlowTok{then}
            \VariableTok{expanded\_dir}\OperatorTok{=}\VariableTok{$(}\BuiltInTok{eval}\NormalTok{ echo }\StringTok{"}\VariableTok{$dir}\StringTok{"}\VariableTok{)}
            \ControlFlowTok{if} \BuiltInTok{[} \OtherTok{{-}d} \StringTok{"}\VariableTok{$expanded\_dir}\StringTok{"} \BuiltInTok{]}\KeywordTok{;} \ControlFlowTok{then}
                \ExtensionTok{log\_message} \StringTok{"Backing up: }\VariableTok{$expanded\_dir}\StringTok{"}
                
                \FunctionTok{rsync} \AttributeTok{{-}av} \AttributeTok{{-}{-}delete} \DataTypeTok{\textbackslash{}}
                      \VariableTok{$excludes} \DataTypeTok{\textbackslash{}}
                      \StringTok{"}\VariableTok{$expanded\_dir}\StringTok{/"} \DataTypeTok{\textbackslash{}}
                      \StringTok{"}\VariableTok{$backup\_dir}\StringTok{/}\VariableTok{$(}\FunctionTok{basename} \StringTok{"}\VariableTok{$expanded\_dir}\StringTok{"}\VariableTok{)}\StringTok{/"}
                      
                \ControlFlowTok{if} \BuiltInTok{[} \VariableTok{$?} \OtherTok{{-}eq}\NormalTok{ 0 }\BuiltInTok{]}\KeywordTok{;} \ControlFlowTok{then}
                    \ExtensionTok{log\_message} \StringTok{"✅ Success: }\VariableTok{$expanded\_dir}\StringTok{"}
                \ControlFlowTok{else}
                    \ExtensionTok{log\_message} \StringTok{"❌ Error: }\VariableTok{$expanded\_dir}\StringTok{"}
                \ControlFlowTok{fi}
            \ControlFlowTok{fi}
        \ControlFlowTok{fi}
    \ControlFlowTok{done}
    
    \CommentTok{\# Comprimir backup}
    \ExtensionTok{log\_message} \StringTok{"Comprimiendo backup..."}
    \BuiltInTok{cd} \StringTok{"}\VariableTok{$(}\FunctionTok{dirname} \StringTok{"}\VariableTok{$backup\_dir}\StringTok{"}\VariableTok{)}\StringTok{"}
    \FunctionTok{tar} \AttributeTok{{-}czf} \StringTok{"backup\_}\VariableTok{$timestamp}\StringTok{.tar.gz"} \StringTok{"backup\_}\VariableTok{$timestamp}\StringTok{"}
    \FunctionTok{rm} \AttributeTok{{-}rf} \StringTok{"backup\_}\VariableTok{$timestamp}\StringTok{"}
    
    \CommentTok{\# Limpieza de backups antiguos}
    \ExtensionTok{cleanup\_old\_backups} \StringTok{"}\VariableTok{$destination}\StringTok{"}
    
    \ExtensionTok{log\_message} \StringTok{"Backup completado: backup\_}\VariableTok{$timestamp}\StringTok{.tar.gz"}
\KeywordTok{\}}

\FunctionTok{cleanup\_old\_backups()} \KeywordTok{\{}
    \BuiltInTok{local} \VariableTok{backup\_dir}\OperatorTok{=}\StringTok{"}\VariableTok{$1}\StringTok{"}
    \BuiltInTok{local} \VariableTok{daily\_retention}\OperatorTok{=}\VariableTok{$(}\ExtensionTok{jq} \AttributeTok{{-}r} \StringTok{\textquotesingle{}.retention.daily\textquotesingle{}} \StringTok{"}\VariableTok{$BACKUP\_CONFIG}\StringTok{"}\VariableTok{)}
    
    \CommentTok{\# Eliminar backups antiguos (mantener solo los últimos N)}
    \BuiltInTok{cd} \StringTok{"}\VariableTok{$backup\_dir}\StringTok{"}
    \FunctionTok{ls} \AttributeTok{{-}t}\NormalTok{ backup\_}\PreprocessorTok{*}\NormalTok{.tar.gz }\KeywordTok{|} \FunctionTok{tail} \AttributeTok{{-}n}\NormalTok{ +}\VariableTok{$((daily\_retention} \OperatorTok{+} \DecValTok{1}\VariableTok{))} \KeywordTok{|} \FunctionTok{xargs}\NormalTok{ rm }\AttributeTok{{-}f}
\KeywordTok{\}}

\ControlFlowTok{case} \StringTok{"}\VariableTok{$\{1}\OperatorTok{:{-}}\NormalTok{help}\VariableTok{\}}\StringTok{"} \KeywordTok{in}
    \SpecialStringTok{init}\KeywordTok{)}
        \ExtensionTok{init\_backup}
        \ControlFlowTok{;;}
    \SpecialStringTok{backup}\KeywordTok{)}
        \BuiltInTok{[} \OtherTok{!} \OtherTok{{-}f} \StringTok{"}\VariableTok{$BACKUP\_CONFIG}\StringTok{"} \BuiltInTok{]} \KeywordTok{\&\&} \ExtensionTok{init\_backup}
        \ExtensionTok{perform\_backup} \StringTok{"}\VariableTok{$\{2}\OperatorTok{:{-}}\NormalTok{local}\VariableTok{\}}\StringTok{"}
        \ControlFlowTok{;;}
    \SpecialStringTok{logs}\KeywordTok{)}
        \FunctionTok{tail} \AttributeTok{{-}f} \StringTok{"}\VariableTok{$BACKUP\_LOG}\StringTok{"}
        \ControlFlowTok{;;}
    \PreprocessorTok{*}\KeywordTok{)}
        \BuiltInTok{echo} \StringTok{"Auto Backup System"}
        \BuiltInTok{echo} \StringTok{"Uso: }\VariableTok{$0}\StringTok{ \textless{}comando\textgreater{} [tipo]"}
        \BuiltInTok{echo} \StringTok{""}
        \BuiltInTok{echo} \StringTok{"Comandos:"}
        \BuiltInTok{echo} \StringTok{"  init            {-} Crear configuración inicial"}
        \BuiltInTok{echo} \StringTok{"  backup [local]  {-} Realizar backup local"}
        \BuiltInTok{echo} \StringTok{"  backup remote   {-} Realizar backup remoto"}
        \BuiltInTok{echo} \StringTok{"  logs            {-} Ver logs de backup"}
        \ControlFlowTok{;;}
\ControlFlowTok{esac}
\end{Highlighting}
\end{Shaded}

\section{Instalación completa del
entorno}\label{instalaciuxf3n-completa-del-entorno}

\begin{Shaded}
\begin{Highlighting}[]
\CommentTok{\#!/bin/bash}
\CommentTok{\# setup{-}complete{-}environment.sh {-} Instalación completa}

\FunctionTok{setup\_homebrew\_tools()} \KeywordTok{\{}
    \BuiltInTok{echo} \StringTok{"🍺 Verificando herramientas de Homebrew..."}
    
    \VariableTok{tools}\OperatorTok{=}\VariableTok{(}
        \StringTok{"bat"} \StringTok{"eza"} \StringTok{"ripgrep"} \StringTok{"fzf"} \StringTok{"zoxide"} \StringTok{"starship"}
        \StringTok{"direnv"} \StringTok{"tealdeer"} \StringTok{"thefuck"} \StringTok{"git"} \StringTok{"gh"} \StringTok{"node"}
        \StringTok{"ffmpeg"} \StringTok{"imagemagick"} \StringTok{"pandoc"} \StringTok{"glow"} \StringTok{"htop"}
        \StringTok{"fastfetch"} \StringTok{"cowsay"} \StringTok{"jq"} \StringTok{"curl"} \StringTok{"wget"} \StringTok{"aria2"}
    \VariableTok{)}
    
    \ControlFlowTok{for}\NormalTok{ tool }\KeywordTok{in} \StringTok{"}\VariableTok{$\{tools}\OperatorTok{[@]}\VariableTok{\}}\StringTok{"}\KeywordTok{;} \ControlFlowTok{do}
        \ControlFlowTok{if} \OtherTok{! }\BuiltInTok{command} \AttributeTok{{-}v} \StringTok{"}\VariableTok{$tool}\StringTok{"} \OperatorTok{\textgreater{}}\NormalTok{/dev/null}\KeywordTok{;} \ControlFlowTok{then}
            \BuiltInTok{echo} \StringTok{"Instalando }\VariableTok{$tool}\StringTok{..."}
            \ExtensionTok{brew}\NormalTok{ install }\StringTok{"}\VariableTok{$tool}\StringTok{"}
        \ControlFlowTok{fi}
    \ControlFlowTok{done}
\KeywordTok{\}}

\FunctionTok{setup\_dotfiles()} \KeywordTok{\{}
    \BuiltInTok{echo} \StringTok{"📁 Configurando dotfiles..."}
    
    \CommentTok{\# Backup de configuraciones existentes}
    \BuiltInTok{[} \OtherTok{{-}f}\NormalTok{ \textasciitilde{}/.zshrc }\BuiltInTok{]} \KeywordTok{\&\&} \FunctionTok{cp}\NormalTok{ \textasciitilde{}/.zshrc \textasciitilde{}/.zshrc.backup}
    \BuiltInTok{[} \OtherTok{{-}f}\NormalTok{ \textasciitilde{}/.gitconfig }\BuiltInTok{]} \KeywordTok{\&\&} \FunctionTok{cp}\NormalTok{ \textasciitilde{}/.gitconfig \textasciitilde{}/.gitconfig.backup}
    
    \CommentTok{\# Aplicar configuraciones}
    \ExtensionTok{./setup{-}git.sh}
    \ExtensionTok{./setup{-}dev{-}tools.sh}
    
    \CommentTok{\# Copiar archivos de configuración}
    \FunctionTok{cp}\NormalTok{ .zshrc \textasciitilde{}/.zshrc}
    \FunctionTok{cp}\NormalTok{ starship.toml \textasciitilde{}/.config/starship.toml}
\KeywordTok{\}}

\FunctionTok{setup\_directories()} \KeywordTok{\{}
    \BuiltInTok{echo} \StringTok{"📂 Creando estructura de directorios..."}
    
    \FunctionTok{mkdir} \AttributeTok{{-}p}\NormalTok{ \textasciitilde{}/Projects/}\DataTypeTok{\{personal}\OperatorTok{,}\DataTypeTok{work}\OperatorTok{,}\DataTypeTok{forks\}}
    \FunctionTok{mkdir} \AttributeTok{{-}p}\NormalTok{ \textasciitilde{}/Scripts}
    \FunctionTok{mkdir} \AttributeTok{{-}p}\NormalTok{ \textasciitilde{}/.local/}\DataTypeTok{\{bin}\OperatorTok{,}\DataTypeTok{log\}}
    \FunctionTok{mkdir} \AttributeTok{{-}p}\NormalTok{ \textasciitilde{}/.config/}\DataTypeTok{\{auto{-}backup}\OperatorTok{,}\DataTypeTok{smart{-}monitor\}}
\KeywordTok{\}}

\FunctionTok{setup\_scripts()} \KeywordTok{\{}
    \BuiltInTok{echo} \StringTok{"📜 Instalando scripts útiles..."}
    
    \CommentTok{\# Copiar scripts a \textasciitilde{}/.local/bin}
    \FunctionTok{cp} \PreprocessorTok{*}\NormalTok{.sh \textasciitilde{}/.local/bin/}
    \FunctionTok{chmod}\NormalTok{ +x \textasciitilde{}/.local/bin/}\PreprocessorTok{*}\NormalTok{.sh}
    
    \CommentTok{\# Crear enlaces simbólicos para facilidad de uso}
    \FunctionTok{ln} \AttributeTok{{-}sf}\NormalTok{ \textasciitilde{}/.local/bin/project{-}templates.sh \textasciitilde{}/.local/bin/new{-}project}
    \FunctionTok{ln} \AttributeTok{{-}sf}\NormalTok{ \textasciitilde{}/.local/bin/auto{-}backup.sh \textasciitilde{}/.local/bin/backup}
    \FunctionTok{ln} \AttributeTok{{-}sf}\NormalTok{ \textasciitilde{}/.local/bin/git{-}workflow.sh \textasciitilde{}/.local/bin/git{-}flow}
\KeywordTok{\}}

\FunctionTok{finalize\_setup()} \KeywordTok{\{}
    \BuiltInTok{echo} \StringTok{"🎯 Finalizando configuración..."}
    
    \CommentTok{\# Actualizar tealdeer}
    \ExtensionTok{tldr} \AttributeTok{{-}{-}update}
    
    \CommentTok{\# Configurar direnv para que funcione inmediatamente}
    \BuiltInTok{eval} \StringTok{"}\VariableTok{$(}\ExtensionTok{direnv}\NormalTok{ hook zsh}\VariableTok{)}\StringTok{"}
    
    \CommentTok{\# Mensaje final}
    \FunctionTok{cat} \OperatorTok{\textless{}\textless{} \textquotesingle{}EOF\textquotesingle{}}

\StringTok{🎉 ¡Configuración completada!}

\StringTok{Pasos siguientes:}
\StringTok{1. Reinicia tu terminal o ejecuta: source \textasciitilde{}/.zshrc}
\StringTok{2. Configura Git con tu información: git config {-}{-}global user.name "Tu Nombre"}
\StringTok{3. Ejecuta: gh auth login para configurar GitHub CLI}
\StringTok{4. Crea tu primer proyecto: new{-}project python mi{-}proyecto}

\StringTok{Comandos útiles instalados:}
\StringTok{{-} new{-}project \textless{}tipo\textgreater{} \textless{}nombre\textgreater{} {-} Crear proyecto desde template}
\StringTok{{-} backup [local|remote]       {-} Realizar backup}
\StringTok{{-} git{-}flow \textless{}comando\textgreater{}          {-} Workflow avanzado de Git}

\StringTok{¡Disfruta tu nuevo entorno de desarrollo! 🚀}
\OperatorTok{EOF}
\KeywordTok{\}}

\FunctionTok{main()} \KeywordTok{\{}
    \BuiltInTok{echo} \StringTok{"🔧 Configurando entorno completo de desarrollo..."}
    
    \ExtensionTok{setup\_homebrew\_tools}
    \ExtensionTok{setup\_dotfiles}
    \ExtensionTok{setup\_directories}
    \ExtensionTok{setup\_scripts}
    \ExtensionTok{finalize\_setup}
\KeywordTok{\}}

\ExtensionTok{main} \StringTok{"}\VariableTok{$@}\StringTok{"}
\end{Highlighting}
\end{Shaded}

\begin{tcolorbox}[enhanced jigsaw, toprule=.15mm, bottomrule=.15mm, opacityback=0, coltitle=black, rightrule=.15mm, colframe=quarto-callout-tip-color-frame, titlerule=0mm, opacitybacktitle=0.6, left=2mm, colback=white, bottomtitle=1mm, arc=.35mm, leftrule=.75mm, title=\textcolor{quarto-callout-tip-color}{\faLightbulb}\hspace{0.5em}{Tips finales de configuración}, colbacktitle=quarto-callout-tip-color!10!white, breakable, toptitle=1mm]

\begin{itemize}
\tightlist
\item
  Mantén un repositorio Git con tus dotfiles para sincronizar entre
  máquinas
\item
  Documenta tus configuraciones personalizadas en un README
\item
  Haz backups regulares de tus configuraciones importantes
\item
  Experimenta con nuevas herramientas pero mantén tu setup estable para
  trabajo
\end{itemize}

\end{tcolorbox}

\begin{tcolorbox}[enhanced jigsaw, toprule=.15mm, bottomrule=.15mm, opacityback=0, coltitle=black, rightrule=.15mm, colframe=quarto-callout-important-color-frame, titlerule=0mm, opacitybacktitle=0.6, left=2mm, colback=white, bottomtitle=1mm, arc=.35mm, leftrule=.75mm, title=\textcolor{quarto-callout-important-color}{\faExclamation}\hspace{0.5em}{Mantenimiento del entorno}, colbacktitle=quarto-callout-important-color!10!white, breakable, toptitle=1mm]

\begin{itemize}
\tightlist
\item
  Actualiza regularmente tus herramientas:
  \texttt{brew\ update\ \&\&\ brew\ upgrade}
\item
  Revisa y limpia configuraciones que ya no uses
\item
  Mantén scripts actualizados y funcionales
\item
  Considera usar herramientas como \texttt{mackup} para backup de
  configuraciones
\end{itemize}

\end{tcolorbox}

\begin{center}\rule{0.5\linewidth}{0.5pt}\end{center}

\section{Conclusión}\label{conclusiuxf3n}

Este libro ha cubierto un ecosistema completo de herramientas CLI
modernas que transformarán tu productividad en terminal. Desde
navegación básica hasta workflows complejos de desarrollo, tienes ahora
una base sólida para construir un entorno de trabajo eficiente y
personalizado.

\subsection{Lo que has aprendido:}\label{lo-que-has-aprendido}

✅ \textbf{Navegación eficiente} con \texttt{eza}, \texttt{tree},
\texttt{ranger} y \texttt{zoxide}\\
✅ \textbf{Gestión avanzada de archivos} con \texttt{rsync} y
herramientas de renombrado\\
✅ \textbf{Búsqueda ultrarrápida} con \texttt{ripgrep}, \texttt{fzf} y
\texttt{jq}\\
✅ \textbf{Desarrollo moderno} con \texttt{git}, \texttt{gh} y
\texttt{node}\\
✅ \textbf{Procesamiento multimedia} con \texttt{ffmpeg},
\texttt{yt-dlp} e \texttt{imagemagick}\\
✅ \textbf{Networking} con \texttt{curl}, \texttt{httpie} y herramientas
de descarga\\
✅ \textbf{Monitoreo} con \texttt{htop} y \texttt{fastfetch}\\
✅ \textbf{Documentación} con \texttt{bat}, \texttt{pandoc} y
\texttt{glow}\\
✅ \textbf{Utilidades} como \texttt{tealdeer}, \texttt{thefuck} y
\texttt{starship}\\
✅ \textbf{Workflows complejos} combinando múltiples herramientas\\
✅ \textbf{Configuración completa} de un entorno productivo

\subsection{Próximos pasos:}\label{pruxf3ximos-pasos-1}

\begin{enumerate}
\def\labelenumi{\arabic{enumi}.}
\tightlist
\item
  \textbf{Practica regularmente} - La muscle memory es clave
\item
  \textbf{Personaliza} según tus necesidades específicas
\item
  \textbf{Automatiza} tareas repetitivas con scripts
\item
  \textbf{Comparte} tu conocimiento con otros desarrolladores
\item
  \textbf{Mantente actualizado} con nuevas herramientas y versiones
\end{enumerate}

¡El dominio de estas herramientas CLI te convertirá en un desarrollador
más eficiente y productivo! 🚀

\chapter{🌐 Integración del
Ecosistema}\label{integraciuxf3n-del-ecosistema-1}

\begin{tcolorbox}[enhanced jigsaw, toprule=.15mm, bottomrule=.15mm, opacityback=0, coltitle=black, rightrule=.15mm, colframe=quarto-callout-tip-color-frame, titlerule=0mm, opacitybacktitle=0.6, left=2mm, colback=white, bottomtitle=1mm, arc=.35mm, leftrule=.75mm, title=\textcolor{quarto-callout-tip-color}{\faLightbulb}\hspace{0.5em}{Un Ecosistema Completo y Conectado}, colbacktitle=quarto-callout-tip-color!10!white, breakable, toptitle=1mm]

La \textbf{Guía CLI de Homebrew} es más que un libro - es un ecosistema
completo de herramientas interconectadas que puedes usar de múltiples
formas según tus necesidades.

\end{tcolorbox}

\section{🎯 Métodos de Acceso}\label{muxe9todos-de-acceso}

\subsection{1. 📖 Libro Interactivo (Web)}\label{libro-interactivo-web}

\textbf{Acceso:} \url{https://laguileracl.github.io/homebrew-cli-guide/}

\textbf{Características:} - ✅ Navegación intuitiva con búsqueda - ✅
Código editable en tiempo real\\
- ✅ Ejecución simulada de comandos - ✅ Gestión de snippets
personalizados - ✅ Temas claro/oscuro - ✅ Responsive design

\textbf{Casos de uso:} - Aprendizaje interactivo - Referencia rápida en
el navegador - Experimentación con código - Documentación colaborativa

\subsection{2. 📱 Dashboard Interactivo}\label{dashboard-interactivo}

\textbf{Acceso:} Abrir \texttt{tools-explorer.html} en el navegador

\textbf{Características:} - 🔍 Búsqueda en tiempo real - 🏷️ Filtrado por
categorías y dificultad - 📊 Estadísticas dinámicas - 📋 Copia directa
de comandos - 🎨 Interfaz moderna y responsive

\textbf{Casos de uso:} - Exploración rápida de herramientas -
Descubrimiento de nuevas CLI - Búsqueda específica por características

\begin{Shaded}
\begin{Highlighting}[]
\CommentTok{\# Abrir dashboard}
\ExtensionTok{open}\NormalTok{ tools{-}explorer.html}
\CommentTok{\# O en Linux/Windows}
\FunctionTok{xdg{-}open}\NormalTok{ tools{-}explorer.html}
\end{Highlighting}
\end{Shaded}

\subsection{3. 💻 CLI Offline}\label{cli-offline-1}

\textbf{Acceso:} \texttt{./scripts/cli-guide}

\textbf{Características:} - ⚡ Búsqueda ultrarrápida con
\texttt{ripgrep} - 🔍 Modo interactivo con \texttt{fzf} - 📄 Múltiples
formatos de salida - 🚀 Funciona sin conexión a internet

\textbf{Casos de uso:} - Consultas desde terminal - Integración en
scripts - Uso en servidores remotos

\begin{Shaded}
\begin{Highlighting}[]
\CommentTok{\# Búsqueda básica}
\ExtensionTok{./scripts/cli{-}guide}\NormalTok{ search git}

\CommentTok{\# Modo interactivo}
\ExtensionTok{./scripts/cli{-}guide}\NormalTok{ interactive}

\CommentTok{\# Listar por categoría}
\ExtensionTok{./scripts/cli{-}guide}\NormalTok{ list desarrollo}

\CommentTok{\# Ver información específica}
\ExtensionTok{./scripts/cli{-}guide}\NormalTok{ info ripgrep}
\end{Highlighting}
\end{Shaded}

\subsection{4. 🌐 API REST}\label{api-rest}

\textbf{Acceso:} \texttt{http://localhost:3000} (requiere servidor)

\textbf{Características:} - 🔗 8 endpoints RESTful - 🔍 Búsqueda con
puntuación de relevancia - 📄 Paginación automática - 🚀 Respuestas
optimizadas - 📊 Estadísticas de uso

\textbf{Casos de uso:} - Integración en aplicaciones - Desarrollo de
extensiones - Automatización y scripts

\begin{Shaded}
\begin{Highlighting}[]
\CommentTok{\# Iniciar servidor API}
\BuiltInTok{cd}\NormalTok{ api{-}server}
\ExtensionTok{npm}\NormalTok{ install}
\ExtensionTok{npm}\NormalTok{ start}

\CommentTok{\# Ejemplos de uso}
\ExtensionTok{curl} \StringTok{"http://localhost:3000/tools"}
\ExtensionTok{curl} \StringTok{"http://localhost:3000/search?q=git"}
\ExtensionTok{curl} \StringTok{"http://localhost:3000/categories"}
\end{Highlighting}
\end{Shaded}

\subsection{5. 🛠️ Extensión VS Code}\label{extensiuxf3n-vs-code}

\textbf{Acceso:} Carpeta \texttt{vscode-extension/}

\textbf{Características:} - 🎯 Comandos integrados en VS Code - 📋
Snippets de código - 🔍 Panel de búsqueda - ⚡ Acceso rápido desde
paleta de comandos

\textbf{Casos de uso:} - Desarrollo integrado - Snippets rápidos -
Documentación contextual

\begin{Shaded}
\begin{Highlighting}[]
\CommentTok{\# Instalar extensión en desarrollo}
\BuiltInTok{cd}\NormalTok{ vscode{-}extension}
\ExtensionTok{npm}\NormalTok{ install}
\ExtensionTok{npm}\NormalTok{ run compile}
\ExtensionTok{code} \AttributeTok{{-}{-}install{-}extension}\NormalTok{ .}
\end{Highlighting}
\end{Shaded}

\subsection{6. 🐳 Contenedores Docker}\label{contenedores-docker}

\textbf{Acceso:} \texttt{docker-compose\ up}

\textbf{Características:} - 🌐 Stack completo (API + Nginx + Redis) - 🔧
Configuración lista para producción - 📊 Monitoreo y métricas - 🚀
Escalable y portable

\textbf{Casos de uso:} - Despliegue en producción - Desarrollo en equipo
- Demostraciones

\begin{Shaded}
\begin{Highlighting}[]
\CommentTok{\# Desplegar stack completo}
\ExtensionTok{docker{-}compose}\NormalTok{ up }\AttributeTok{{-}d}

\CommentTok{\# Solo API}
\ExtensionTok{docker}\NormalTok{ run }\AttributeTok{{-}p}\NormalTok{ 3000:3000 cli{-}tools{-}api}

\CommentTok{\# Con balanceador}
\ExtensionTok{docker}\NormalTok{ run }\AttributeTok{{-}p}\NormalTok{ 80:80 cli{-}tools{-}web}
\end{Highlighting}
\end{Shaded}

\section{🔄 Flujos de Trabajo
Integrados}\label{flujos-de-trabajo-integrados}

\subsection{Para Desarrolladores}\label{para-desarrolladores}

\begin{Shaded}
\begin{Highlighting}[]
\NormalTok{graph LR}
\NormalTok{    A[VS Code] {-}{-}\textgreater{}|Snippet| B[Editar código]}
\NormalTok{    B {-}{-}\textgreater{} C[Probar en Terminal]}
\NormalTok{    C {-}{-}\textgreater{}|Consultar| D[CLI Offline]}
\NormalTok{    D {-}{-}\textgreater{} E[Actualizar en Libro]}
\NormalTok{    E {-}{-}\textgreater{}|Commit| F[GitHub]}
\end{Highlighting}
\end{Shaded}

\subsection{Para Usuarios Finales}\label{para-usuarios-finales}

\begin{Shaded}
\begin{Highlighting}[]
\NormalTok{graph LR}
\NormalTok{    A[Dashboard Web] {-}{-}\textgreater{}|Descubrir| B[Herramienta]}
\NormalTok{    B {-}{-}\textgreater{}|Probar| C[Libro Interactivo]}
\NormalTok{    C {-}{-}\textgreater{}|Usar| D[Terminal Local]}
\NormalTok{    D {-}{-}\textgreater{}|Guardar| E[Snippets]}
\end{Highlighting}
\end{Shaded}

\subsection{Para Contribuidores}\label{para-contribuidores}

\begin{Shaded}
\begin{Highlighting}[]
\NormalTok{graph LR}
\NormalTok{    A[API Explorar] {-}{-}\textgreater{}|Encontrar gaps| B[Añadir herramienta]}
\NormalTok{    B {-}{-}\textgreater{}|Pull Request| C[GitHub]}
\NormalTok{    C {-}{-}\textgreater{}|CI/CD| D[Validación]}
\NormalTok{    D {-}{-}\textgreater{}|Deploy| E[Libro actualizado]}
\end{Highlighting}
\end{Shaded}

\section{🎯 Configuración
Recomendada}\label{configuraciuxf3n-recomendada}

\subsection{Setup Básico (Solo
lectura)}\label{setup-buxe1sico-solo-lectura}

\begin{Shaded}
\begin{Highlighting}[]
\CommentTok{\# Acceso web directo}
\ExtensionTok{open}\NormalTok{ https://laguileracl.github.io/homebrew{-}cli{-}guide/}
\end{Highlighting}
\end{Shaded}

\subsection{Setup Intermedio (Con
dashboard)}\label{setup-intermedio-con-dashboard}

\begin{Shaded}
\begin{Highlighting}[]
\CommentTok{\# Clonar repositorio}
\FunctionTok{git}\NormalTok{ clone https://github.com/laguileracl/homebrew{-}cli{-}guide.git}
\BuiltInTok{cd}\NormalTok{ homebrew{-}cli{-}guide}

\CommentTok{\# Usar dashboard y CLI}
\ExtensionTok{open}\NormalTok{ tools{-}explorer.html}
\ExtensionTok{./scripts/cli{-}guide}\NormalTok{ interactive}
\end{Highlighting}
\end{Shaded}

\subsection{Setup Avanzado (Ecosistema
completo)}\label{setup-avanzado-ecosistema-completo}

\begin{Shaded}
\begin{Highlighting}[]
\CommentTok{\# 1. Clonar y configurar}
\FunctionTok{git}\NormalTok{ clone https://github.com/laguileracl/homebrew{-}cli{-}guide.git}
\BuiltInTok{cd}\NormalTok{ homebrew{-}cli{-}guide}

\CommentTok{\# 2. Instalar dependencias API}
\BuiltInTok{cd}\NormalTok{ api{-}server }\KeywordTok{\&\&} \ExtensionTok{npm}\NormalTok{ install }\KeywordTok{\&\&} \BuiltInTok{cd}\NormalTok{ ..}

\CommentTok{\# 3. Configurar VS Code extension}
\BuiltInTok{cd}\NormalTok{ vscode{-}extension }\KeywordTok{\&\&} \ExtensionTok{npm}\NormalTok{ install }\KeywordTok{\&\&} \BuiltInTok{cd}\NormalTok{ ..}

\CommentTok{\# 4. Iniciar servicios}
\ExtensionTok{docker{-}compose}\NormalTok{ up }\AttributeTok{{-}d}

\CommentTok{\# 5. Verificar funcionamiento}
\ExtensionTok{curl}\NormalTok{ http://localhost:3000/health}
\end{Highlighting}
\end{Shaded}

\section{📊 Comparación de Métodos}\label{comparaciuxf3n-de-muxe9todos}

\begin{longtable}[]{@{}llllll@{}}
\toprule\noalign{}
Método & Offline & Interactivo & Búsqueda & Edición & API \\
\midrule\noalign{}
\endhead
\bottomrule\noalign{}
\endlastfoot
\textbf{Libro Web} & ❌ & ✅ & ✅ & ✅ & Opcional \\
\textbf{Dashboard} & ✅ & ✅ & ✅ & ❌ & No \\
\textbf{CLI} & ✅ & ✅ & ✅ & ❌ & No \\
\textbf{API} & ❌ & ❌ & ✅ & ❌ & ✅ \\
\textbf{VS Code} & ✅ & ✅ & ✅ & ✅ & Opcional \\
\textbf{Docker} & ❌ & ✅ & ✅ & ❌ & ✅ \\
\end{longtable}

\section{🔧 Personalización}\label{personalizaciuxf3n-2}

\subsection{Variables de Entorno}\label{variables-de-entorno}

\begin{Shaded}
\begin{Highlighting}[]
\CommentTok{\# API Configuration}
\BuiltInTok{export} \VariableTok{CLI\_API\_PORT}\OperatorTok{=}\NormalTok{3000}
\BuiltInTok{export} \VariableTok{CLI\_API\_HOST}\OperatorTok{=}\NormalTok{localhost}
\BuiltInTok{export} \VariableTok{CLI\_DATA\_PATH}\OperatorTok{=}\NormalTok{./tools{-}index.json}

\CommentTok{\# Quarto Configuration  }
\BuiltInTok{export} \VariableTok{QUARTO\_PYTHON}\OperatorTok{=}\VariableTok{$(}\FunctionTok{which}\NormalTok{ python3}\VariableTok{)}
\BuiltInTok{export} \VariableTok{QUARTO\_RENDER\_FORMAT}\OperatorTok{=}\NormalTok{html}
\end{Highlighting}
\end{Shaded}

\subsection{Configuración API}\label{configuraciuxf3n-api}

\begin{Shaded}
\begin{Highlighting}[]
\CommentTok{// api{-}server/config.js}
\NormalTok{module}\OperatorTok{.}\AttributeTok{exports} \OperatorTok{=}\NormalTok{ \{}
  \DataTypeTok{port}\OperatorTok{:} \BuiltInTok{process}\OperatorTok{.}\AttributeTok{env}\OperatorTok{.}\AttributeTok{PORT} \OperatorTok{||} \DecValTok{3000}\OperatorTok{,}
  \DataTypeTok{cors}\OperatorTok{:}\NormalTok{ \{}
    \DataTypeTok{origin}\OperatorTok{:}\NormalTok{ [}\StringTok{\textquotesingle{}http://localhost:4200\textquotesingle{}}\OperatorTok{,} \StringTok{\textquotesingle{}https://laguileracl.github.io\textquotesingle{}}\NormalTok{]}\OperatorTok{,}
    \DataTypeTok{methods}\OperatorTok{:}\NormalTok{ [}\StringTok{\textquotesingle{}GET\textquotesingle{}}\OperatorTok{,} \StringTok{\textquotesingle{}POST\textquotesingle{}}\NormalTok{]}
\NormalTok{  \}}\OperatorTok{,}
  \DataTypeTok{rateLimit}\OperatorTok{:}\NormalTok{ \{}
    \DataTypeTok{windowMs}\OperatorTok{:} \DecValTok{15} \OperatorTok{*} \DecValTok{60} \OperatorTok{*} \DecValTok{1000}\OperatorTok{,} \CommentTok{// 15 minutos}
    \DataTypeTok{max}\OperatorTok{:} \DecValTok{100} \CommentTok{// máximo 100 requests por ventana}
\NormalTok{  \}}
\NormalTok{\}}\OperatorTok{;}
\end{Highlighting}
\end{Shaded}

\section{🚀 Próximas
Características}\label{pruxf3ximas-caracteruxedsticas}

\subsection{En Desarrollo}\label{en-desarrollo}

\begin{itemize}
\tightlist
\item[$\square$]
  \textbf{PWA} - Aplicación web progresiva offline
\item[$\square$]
  \textbf{CLI con AI} - Recomendaciones inteligentes\\
\item[$\square$]
  \textbf{Plugin Vim/Neovim} - Integración con editores
\item[$\square$]
  \textbf{Alfred Workflow} - Acceso rápido en macOS
\item[$\square$]
  \textbf{Raycast Extension} - Búsqueda nativa
\end{itemize}

\subsection{Contribuciones
Bienvenidas}\label{contribuciones-bienvenidas}

\begin{itemize}
\tightlist
\item
  🎨 \textbf{Temas personalizados} para el libro
\item
  🔌 \textbf{Integraciones} con otros editores
\item
  📱 \textbf{Apps móviles} nativas
\item
  🤖 \textbf{Bots} para Discord/Slack
\item
  📊 \textbf{Analytics} y métricas de uso
\end{itemize}

\begin{center}\rule{0.5\linewidth}{0.5pt}\end{center}

\begin{tcolorbox}[enhanced jigsaw, toprule=.15mm, bottomrule=.15mm, opacityback=0, coltitle=black, rightrule=.15mm, colframe=quarto-callout-important-color-frame, titlerule=0mm, opacitybacktitle=0.6, left=2mm, colback=white, bottomtitle=1mm, arc=.35mm, leftrule=.75mm, title=\textcolor{quarto-callout-important-color}{\faExclamation}\hspace{0.5em}{¡Importante!}, colbacktitle=quarto-callout-important-color!10!white, breakable, toptitle=1mm]

Para la \textbf{mejor experiencia posible}, recomendamos usar:

\begin{enumerate}
\def\labelenumi{\arabic{enumi}.}
\tightlist
\item
  \textbf{Libro interactivo} para aprendizaje
\item
  \textbf{Dashboard} para exploración
\item
  \textbf{CLI} para uso diario
\item
  \textbf{API} para integración
\end{enumerate}

¡Cada método complementa a los otros! 🌟

\end{tcolorbox}

\section{🤝 Comunidad y Soporte}\label{comunidad-y-soporte}

\subsection{Canales de Comunicación}\label{canales-de-comunicaciuxf3n}

\begin{itemize}
\tightlist
\item
  🐛 \textbf{Issues}:
  \href{https://github.com/laguileracl/homebrew-cli-guide/issues}{GitHub
  Issues}
\item
  💬 \textbf{Discusiones}:
  \href{https://github.com/laguileracl/homebrew-cli-guide/discussions}{GitHub
  Discussions}
\item
  📧 \textbf{Email}:
  \href{mailto:maintainer@example.com}{\nolinkurl{maintainer@example.com}}
\item
  🐦 \textbf{Twitter}:
  \href{https://twitter.com/cli_tools_guide}{(\textbf{cli\_tools\_guide?})}
\end{itemize}

\subsection{Contribuir}\label{contribuir}

\begin{enumerate}
\def\labelenumi{\arabic{enumi}.}
\tightlist
\item
  \textbf{Fork} el repositorio
\item
  \textbf{Crea} una rama para tu feature
\item
  \textbf{Añade} herramientas al \texttt{tools-index.json}
\item
  \textbf{Actualiza} la documentación
\item
  \textbf{Envía} un Pull Request
\end{enumerate}

¡Tu contribución hace que este ecosistema sea mejor para todos! 🚀


\backmatter


\end{document}
